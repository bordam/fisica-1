%!TEX root = ../main.tex
\section*{Prefazione}

Questo libro raccoglie gli appunti del corso di \emph{Fisica Sperimentale 1} per alunni ingegneri matematici, tenuto al Politecnico di Milano nell'anno accademico 2018-2019.

Seguendo l'esperienza di altri ragazzi abbiamo deciso di realizzare questo manuale per venire incontro alle esigenze degli studenti, che talvolta non riescono a seguire perfettamente il docente e non trovano una corrispondenza appropriata sul libro di testo, il quale è spesso sovrabbondante di argomenti.

Scrivendo questo libro abbiamo avuto modo di capire meglio alcuni aspetti della materia, cercando di renderli più chiari possibile con gli strumenti che l'ambiente \latex ci ha messo a disposizione. Speriamo che questo stimoli lo studio e l'interesse nei lettori.

Il manuale si articola in vari capitoli che seguono il programma del corso e si conclude con una serie di appendici con alcuni riferimenti utili durante lo studio dei vari argomenti (derivazione, integrazione, costanti fisiche, trigonometria, ecc.).

Ringraziamo in special modo tutti i forum e gli utenti che inizialmente ci hanno aiutato a farci strada nei meandri di questo nuovo ambiente di scrittura. È stato faticoso e abbiamo sbattuto la testa davanti a molti ostacoli, ma è stato anche estremamente istruttivo e formativo.\\
Ringraziamo infine il docente per i suoi insegnamenti e la chiarezza con cui ha portato avanti uno dei corsi più interessanti e fondamentali per ogni ingegnere.

\begin{flushright}
	\emph{Gli autori} \hspace*{2cm}
\end{flushright}