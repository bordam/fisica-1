\documentclass[10pt,a4paper]{book}
\usepackage{amsfonts}
\usepackage[none]{hyphenat} % PER NON FAR ANDARE A CAPO LE PAROLE CON IL TRATTINO
\usepackage[nottoc,notlot,notlof]{tocbibind} % boh
\usepackage{graphicx}
\usepackage{float}
\usepackage{centernot} % serve per il 'A NON implica B' nel comando \centernot
\usepackage{wrapfig}
\pdfsuppresswarningpagegroup=1



% FIGURE MATHCHA.IO
\usepackage{amsmath}
\usepackage{tikz}
\usepackage{mathdots}
\usepackage{yhmath}
\usepackage{cancel}
\usepackage{color}
\usepackage{siunitx}
\usepackage{array}
\usepackage{multirow}
\usepackage{amssymb}
\usepackage{gensymb}
\usepackage{tabularx}
\usepackage{booktabs}
\usepackage{caption}\captionsetup{belowskip=12pt,aboveskip=4pt}
\usetikzlibrary{fadings}
\usetikzlibrary{patterns}
\usetikzlibrary{shadows.blur}
\usepackage{placeins} % The placeins package gives the command \FloatBarrier, which will make sure any floats will be put in before this point
\usepackage{flafter}  % The flafter package ensures that floats don't appear until after they appear in the code.



% PER INCLUDERE FIGURE
\usepackage{import}
\usepackage{pdfpages}
\usepackage{transparent}
\usepackage{xcolor}

% QUESTO NUOVO COMANDO SERVE PER INCLDERE LE FIGURE FATTE IN INKSCAPE, CHE SI DEVONO TROVARE NELLA STESSA DIRECTORY DENTRO LA CARTELLA figures
\newcommand{\incfig}[2][1]{%
	\def\svgwidth{#1\columnwidth}
	\import{./figures/}{#2.pdf_tex}
}



% RISCRITTURA DI COMANDI
%\renewcommand{\epsilon}{\varepsilon} % NON USATO
%\renewcommand{\theta}{\vartheta} % NON USATO
%\renewcommand{\rho}{\varrho}
%\renewcommand{\phi}{\varphi} % NON USATO
\renewcommand{\degree}{^\circ\text{C}} % SIMBOLO GRADI
\newcommand{\notimplies}{\mathrel{{\ooalign{\hidewidth$\not\phantom{=}$\hidewidth\cr$\implies$}}}}

\usepackage{mathtools} % SERVE PER I DUE COMANDI DOPO
\DeclarePairedDelimiter{\abs}{\lvert}{\rvert} % CREA UN COMANDO abs()
\DeclarePairedDelimiter{\norma}{\lVert}{\rVert} % CREA UN COMANDO norma()



% INDICE
\setcounter{secnumdepth}{3} % DI DEFAULT LE SUBSUBSECTION NON SONO NUMERATE, COSÌ SÌ
\setcounter{tocdepth}{1} % FISSA LA PROFONDITÀ DELLE COSE MOSTRATE NELL'INDICE
\usepackage{tocstyle}
\usetocstyle{standard}
\usepackage[hidelinks]{hyperref} % RENDE L'INDICE INTERATTIVO E hidelinks NASCONDE IL BORDO ROSSO DAI RIFERIMENTI


% APPENDICI
\usepackage[toc,page]{appendix}
\newcommand{\nocontentsline}[3]{} % QUESTO COMANDO E QUELLO DOPO SERVONO PER AVERE IL COMANDO \tocless DA METTERE PRIMA DI UNA SEZIONE CHE NON VOGLIO FAR APPARIRE NELL'INDICE
\newcommand{\tocless}[2]{\bgroup\let\addcontentsline=\nocontentsline#1{#2}\egroup}



% FONT
\usepackage[T1]{fontenc}
\usepackage[utf8]{inputenc}
\usepackage[italian]{babel}
\usepackage{latexsym}
\usepackage{textcomp}
\usepackage{siunitx}



% SCRITTA LaTeX
\newcommand{\latex}{\LaTeX\xspace}
\newcommand{\tex}{\TeX\xspace}



% PARAGRAFI, INTERLINEA E MARGINE
\emergencystretch 3em % PER EVITARE CHE IL TESTO VADA OLTRE I MARGINI
\parindent 0ex % TOGLIE INTENDAMENTO PARAGRAFI
% \setlength{\parindent}{4em} % CAMBIA INDENTAMENTO PARAGRAFI
\setlength{\parskip}{\baselineskip} % CAMBIA SPAZIO TRA PARAGRAFI (POSSO METTERE ANCHE 1em)
% \renewcommand{\baselinestretch}{1.5} % CAMBIA INTERLINEA
% \usepackage[margin=1in]{geometry} % CAMBIA MARGINE DEL DOCUMENTO



% HEADER
\usepackage{fancyhdr}
\pagestyle{fancy}
\fancyhead{} % PULISCI HEADER
\fancyfoot{} % PULISCI FOOTER

% TOGLI I PROSISMI DUE COMMENTI PER CAMBIARE DA 'Capitolo X. Blabla' A 'X. Blabla'
%\renewcommand{\chaptermark}[1]{\markboth{#1}{}}
\fancyhead[RE]{\nouppercase{\leftmark}} %\fancyhead[RE]{\thechapter.  \leftmark}
\fancyhead[LO]{\nouppercase{\rightmark}}
\fancyhead[LE,RO]{\thepage}

%\fancyhead[L]{\slshape \MakeUppercase{FISICA SPERIMENTALE 2}}
%\fancyhead[R]{\slshape A cura di Teo Bucci e Giulia Di Giusto}
%\fancyfoot[C]{\thepage}
%\renewcommand{\headrulewidth}{0pt} % CANCELLA LINEA ORIZZONTALE HEAD



% CENTRARE VERTICALMENTE IL TITOLO DELLA PAGINA PRINCIPALE
\usepackage{titling}
\renewcommand\maketitlehooka{\null\mbox{}\vfill}
\renewcommand\maketitlehookd{\vfill\null}



% BOX DI VARIO TIPO

% TEOREMA TIPO 1
\usepackage{mdframed}
\newmdtheoremenv{mdtheo}{Teorema}

% BOX DEFINIZIONI/FORMULE IMPORANTI/TEOREMA TIPO 2
\usepackage{tcolorbox}
\usepackage{blindtext}
\usepackage{tikz,tkz-tab,amsmath}
\tcbuselibrary{theorems}

% FORMULE
%\newtcolorbox{formula}{
%	colback=red!5!white,
%	colframe=red!75!black
%	}
\newtcolorbox{formula}{
	colback=black!5!white,
	colframe=black
	}

% TEOREMA TIPO 2
\newtcbtheorem
	[number within=section]
	{theo}
	{Teorema}
	{
		colback=green!5,
		colframe=green!35!black,
		fonttitle=\bfseries,
		separator sign none % toglie i :
	}
	{th}

% DEFINIZIONI
\newtcbtheorem
	[number within=section] % init options
	{definition} % nome da scrivere in LaTeX
	{Definizione}% Titolo che verrà visualizzazione
	{
		colback=blue!5,
		colframe=blue!45!black,
		fonttitle=\bfseries,
	} % options
	{def} % PREFISSO/LABEL PER LE REF



% CHECKLIST
% perchèé èé ' "
% $1$ $ 1$ $1 $ $ 1 $
% pedici fin, in con \text{}
% \[ \]
% spazio prima di ,:; 
% spazio equation spazio
% spazio section spazio
% spazio a inizio riga
% spazio }
% itemize
% _{} ^{} togliere graffe se possibile
% spazio prima di |
% i-esima $i$-esima
% \vec{ spazio }
% ^\circ\text{C} gradi \degree o ^o
% minuscole dopo il .
% Terra/terra


%%%%%%%%%%%%%%%%%%%%%%%%%%%%%%%%%%%%%%%%%%%%%%%
%%%%%%%%%%%%%%%%%%%%%%%%%%%%%%%%%%%%%%%%%%%%%%%



\begin{document}




%%%%%%%%%%%%%%%%%%%%%%%%%%%%%%%%%%%%%%%%%%%%%%%
%%%%%%%%%%%%%%%%%%%%%%%%%%%%%%%%%%%%%%%%%%%%%%%


% INFORMAZIONI PER LA PAGINA INIZIALE
%\title{Fisica Sperimentale 1}
%\author{Teo Bucci, Giulia Di Giusto}
%\date{\today}

% PAGINA INIZIALE
%\begin{titlingpage} % PER AVERLO CENTRATO
%	\maketitle
%	\thispagestyle{empty} % TOGLI STILE HEADER E FOOTER
%	\clearpage % PASSA ALLA PAGINA DOPO
%\end{titlingpage}


%%%%%%%%%%%%%%%%%%%%%%%%%%%%%%%%%%%%%%%%%%%%%%%
%%%%%%%%%%%%%%%%%%%%%%%%%%%%%%%%%%%%%%%%%%%%%%%


% COPERTINA

\pagestyle{empty} % SWITCHA PER NON AVERE NUMERO PAGINA
\vspace*{\fill}
Teo Bucci \\
Giulia Di Giusto \\
\begin{center}
	{\large \textsc{Appunti di}}\\
	\vspace*{0.4cm}
	{\Huge \textsc{Fisica Sperimentale 1}}
	%{\large \textsc{Dalle lezioni della prof. Daniela Comelli}}
\end{center}
\vspace*{\fill}
\newpage


%%%%%%%%%%%%%%%%%%%%%%%%%%%%%%%%%%%%%%%%%%%%%%%
%%%%%%%%%%%%%%%%%%%%%%%%%%%%%%%%%%%%%%%%%%%%%%%


% SECONDA PAGINA

% {\Large \textit{Appunti di Fisica Sperimentale 1}}

\vspace*{\fill}

\textcopyright \ Gli autori, tutti i diritti riservati

Sono proibite tutte le riproduzioni senza autorizzazione scritta degli autori.

Revisione del \today

Developed by\\
Teo Bucci - \texttt{teobucci8@gmail.com}\\
Giulia Di Giusto - \texttt{digiusto99@gmail.com}

Compiled with \ensuremath\heartsuit \\
%Foto di copertina by Matt Pett (\emph{@mattpunsplash, unsplash.com})

Per segnalare eventuali errori o suggerimenti potete contattare gli autori.

\newpage


%%%%%%%%%%%%%%%%%%%%%%%%%%%%%%%%%%%%%%%%%%%%%%%
%%%%%%%%%%%%%%%%%%%%%%%%%%%%%%%%%%%%%%%%%%%%%%%


% PREFAZIONE

{\Huge \textbf{Prefazione}}

Questo libro raccoglie gli appunti del corso di \emph{Fisica Sperimentale 1} per alunni ingegneri matematici, tenuto al Politecnico di Milano nell'anno accademico 2018-2019.% dalla prof. Daniela Comelli.

Seguendo l'esperienza di altri ragazzi abbiamo deciso di realizzare questo manuale per venire incontro alle esigenze degli studenti, che talvolta non riescono a seguire perfettamente il docente e non trovano una corrispondenza appropriata sul libro di testo, il quale è spesso sovrabbondante di argomenti.

Scrivendo questo libro abbiamo avuto modo di capire meglio alcuni aspetti della materia, cercando di renderli più chiari possibile con gli strumenti che l'ambiente \latex ci ha messo a disposizione. Speriamo che questo stimoli lo studio e l'interesse nei lettori.

Il manuale si articola in vari capitoli che seguono il programma del corso e si conclude con una serie di appendici con alcuni riferimenti utili durante lo studio dei vari argomenti (derivazione, integrazione, costanti fisiche, trigonometria, ecc.).

Ringraziamo tutti coloro che ci hanno supportato.\\
Ringraziamo in special modo tutti i forum e gli utenti che inizialmente ci hanno aiutato a farci strada nei meandri di questo nuovo ambiente di scrittura. È stato faticoso e abbiamo sbattuto la testa davanti a molti ostacoli, ma è stato anche estremamente istruttivo e formativo.\\
Ringraziamo infine il docente per i suoi insegnamenti e la chiarezza con cui ha portato avanti uno dei corsi più interessanti e fondamentali per ogni ingegnere.

\begin{flushright}
\emph{Gli autori} \hspace*{2cm}
\end{flushright}

\newpage


%%%%%%%%%%%%%%%%%%%%%%%%%%%%%%%%%%%%%%%%%%%%%%%
%%%%%%%%%%%%%%%%%%%%%%%%%%%%%%%%%%%%%%%%%%%%%%%


% INDICE

\addtocontents{toc}{\protect\thispagestyle{empty}} % MI ASSICURA CHE NON CI SIA STILE NELL'INDICE, COME ULTERIORE METODO DI SICUREZZA
\tableofcontents % PRODUCE L'INDICE
\newpage % PER TERMINARE SUBITO LA PAGINA DOPO L'INDICE

% FACOLTATIVA PAGINA VUOTA
%$ $\\
%\newpage

\pagestyle{fancy} % RISWITCHA PER RIAVERE IL NUMERO PAGINA
\setcounter{page}{1} % FA RIPARTIRE IL CONTATORE PAGINA DA 1


%%%%%%%%%%%%%%%%%%%%%%%%%%%%%%%%%%%%%%%%%%%%%%%
%%%%%%%%%%%%%%%%%%%%%%%%%%%%%%%%%%%%%%%%%%%%%%%
%%%%%%%%%%%%%%%%%%%%%%%%%%%%%%%%%%%%%%%%%%%%%%%
%%%%%%%%%%%%%%%%%%%%%%%%%%%%%%%%%%%%%%%%%%%%%%%























































































































\chapter{Introduzione allo studio della fisica}

\section{Le grandezze fisiche}

Concetto fondamentale al fine di affrontare lo studio dei fenomeni fisici è quello di \textbf{grandezza fisica}, termine riferito a tutte quelle proprietà di un corpo, di una sostanza o appunto di un fenomeno che possono essere misurate.
Vi sono alcune grandezze fisiche che possono essere completamente definite una volta assegnatone il valore corredato dall'unità di misura; esse per questo si dicono \emph{scalari}. Possono tuttavia sussistere casi in cui, oltre all'informazione data dal valore e dall'unità di misura, sia necessaria la specificazione di una direzione e di un verso; tali grandezze vengono chiamate \emph{vettoriali}. Ne è un esempio la forza, completamente definita dalla sua intensità, dalla sua direzione e dal suo verso.

Le unità di misura devono essere scelte in modo tale che siano facilmente riproducibili e precise. In particolare modo, il più diffuso sistema di unità di misura, noto come \textbf{Sistema Internazionale}, SI, definisce sette grandezze fisiche fondamentali dalle quali, sfruttando le leggi fisiche che le legano, possono essere ricavate tutte le altre grandezze. Esse sono riportate in seguito:
\begin{center}
	\begin{tabular}{lll}
	\toprule
	Grandezza & Unità di misura &Simbolo \\
	\midrule
	lunghezza & metri & $m$ \\
	massa  & chilogrammi & $kg$ \\
	tempo & secondi  & $s$\\
	temperatura &gradi kelvin & $K$\\
	quantità di materia &moli  & $mol$\\
	intensità luminosa  &candele& $cd$ \\
	intensità di corrente elettrica  &ampère & $A$ \\
	\bottomrule
	\end{tabular}
\end{center}

\paragraph{Osservazione} Le equazioni fisiche sono omogenee dal punto di vista dimensionale, ciò significa che in un'equazione l'unità di misura è la stessa sia a sinistra che a destra.







































\section{Introduzione al calcolo vettoriale}

\subsection{Definizione di un vettore e delle sue componenti}

Il \textbf{vettore} è l'ente matematico adatto alla rappresentazione delle grandezze vettoriali precedentemente definite e per le quali vale un'algebra che differisce da quella scalare.

\begin{figure}[htpb]
	\centering

	\tikzset{every picture/.style={line width=0.75pt}} %set default line width to 0.75pt

	\begin{tikzpicture}[x=0.75pt,y=0.75pt,yscale=-1,xscale=1]
	%uncomment if require: \path (0,300); %set diagram left start at 0, and has height of 300

	%Straight Lines [id:da4464327382646005]
	\draw  [dash pattern={on 0.84pt off 2.51pt}]  (300,80) -- (120,190) ;
	%Straight Lines [id:da6798121572011886]
	\draw [line width=1.5]    (170,160) -- (246.61,112.12) ;
	\draw [shift={(250,110)}, rotate = 507.99] [fill={rgb, 255:red, 0; green, 0; blue, 0 }  ][line width=0.08]  [draw opacity=0] (13.4,-6.43) -- (0,0) -- (13.4,6.44) -- (8.9,0) -- cycle    ;

	% Text Node
	\draw (197,121) node    {$\vec{v}$};
	% Text Node
	\draw (222.5,156.5) node    {$\vec{u}_v$};

	\end{tikzpicture}
\end{figure}

Matematicamente un vettore ha infinite rappresentazioni: infatti tutti i segmenti orientati di eguale lunghezza, paralleli ed equiversi sono rappresentazioni dello stesso vettore e quindi equivalenti, si dice che essi formano un sistema di vettori \textbf{equipollenti}. Fissata una direzione orientata $\vec{r}$ vi sono infiniti vettori che hanno quella direzione e differiscono solo nel modulo e nel verso. In particolare fra di essi esiste un vettore $\vec{u}$ che è concorde in verso a $\vec{r}$ e ha modulo unitario:
\[
	\vec{v}=\abs{\vec{v}} \,\vec{u}_v \qquad \text{dove $\vec{v}=$ modulo del vettore $\vec{v}$}
\]
Più in generale, un qualsiasi vettore di modulo unitario si definisce \textbf{versore}.

Si consideri il sistema di riferimento cartesiano: esso può essere rappresentato dai versori $\vec{u}_x$, $\vec{u}_y$, $\vec{u}_z$.
\begin{figure}[htpb]
	\centering
	

	\tikzset{every picture/.style={line width=0.75pt}} %set default line width to 0.75pt        

	\begin{tikzpicture}[x=0.75pt,y=0.75pt,yscale=-1,xscale=1]
	%uncomment if require: \path (0,300); %set diagram left start at 0, and has height of 300

	%Shape: Pie [id:dp6201472441222928] 
	\draw  [fill={rgb, 255:red, 207; green, 207; blue, 207 }  ,fill opacity=1 ] (219.17,91.78) .. controls (221.64,97.7) and (223,104.19) .. (223,111) -- (173,111) -- cycle ;
	%Straight Lines [id:da5356624891507755] 
	\draw    (111,165) -- (405.5,165) ;
	\draw [shift={(408.5,165)}, rotate = 180] [fill={rgb, 255:red, 0; green, 0; blue, 0 }  ][line width=0.08]  [draw opacity=0] (10.72,-5.15) -- (0,0) -- (10.72,5.15) -- (7.12,0) -- cycle    ;
	%Straight Lines [id:da993433031949847] 
	\draw [line width=1.5]    (173,111) -- (317.8,51.52) ;
	\draw [shift={(321.5,50)}, rotate = 517.6700000000001] [fill={rgb, 255:red, 0; green, 0; blue, 0 }  ][line width=0.08]  [draw opacity=0] (13.4,-6.43) -- (0,0) -- (13.4,6.44) -- (8.9,0) -- cycle    ;
	%Straight Lines [id:da38435392810852576] 
	\draw  [dash pattern={on 0.84pt off 2.51pt}]  (173,111) -- (314.5,111) ;
	%Straight Lines [id:da6456289887316091] 
	\draw  [dash pattern={on 0.84pt off 2.51pt}]  (173,111) -- (173,165) ;
	%Straight Lines [id:da4993284795246429] 
	\draw  [dash pattern={on 0.84pt off 2.51pt}]  (314.5,51) -- (314.5,165) ;
	%Straight Lines [id:da9011995931714469] 
	\draw [line width=1.5]    (173,165) -- (310.5,165) ;
	\draw [shift={(314.5,165)}, rotate = 180] [fill={rgb, 255:red, 0; green, 0; blue, 0 }  ][line width=0.08]  [draw opacity=0] (13.4,-6.43) -- (0,0) -- (13.4,6.44) -- (8.9,0) -- cycle    ;
	%Straight Lines [id:da693284506691856] 
	\draw [line width=1.5]    (348.5,140) -- (386,140) ;
	\draw [shift={(390,140)}, rotate = 180] [fill={rgb, 255:red, 0; green, 0; blue, 0 }  ][line width=0.08]  [draw opacity=0] (13.4,-6.43) -- (0,0) -- (13.4,6.44) -- (8.9,0) -- cycle    ;

	% Text Node
	\draw (236,68) node    {$\vec{v}$};
	% Text Node
	\draw (237,97) node    {$\alpha $};
	% Text Node
	\draw (243,181) node    {$\vec{v}_{x}$};
	% Text Node
	\draw (365,123.5) node    {$\vec{u}_{x}$};


	\end{tikzpicture}
\end{figure}
Proiettare un qualunque vettore $\vec{v}$ su uno dei tre assi significa andare a valutare quanto vale $\vec{v}$ lungo di esso e la quantità risultante dalla proiezione è detta \emph{componente} del vettore $\vec{v}$ lungo il suddetto asse. Ad esempio, per quanto riguarda l'asse $x$, essa è data da:
\[
	\vec{v}_x=v\cos\alpha\, \vec{u}_x
\]
Ovviamente l'espressione è analoga considerando l'asse delle $y$ o delle $z$.

\begin{figure}[htpb]
	\centering
	

	\tikzset{every picture/.style={line width=0.75pt}} %set default line width to 0.75pt

	\begin{tikzpicture}[x=0.75pt,y=0.75pt,yscale=-1,xscale=1]
	%uncomment if require: \path (0,339); %set diagram left start at 0, and has height of 339

	%Shape: Pie [id:dp9667084827284058]
	\draw  [draw opacity=0][fill={rgb, 255:red, 207; green, 207; blue, 207 }  ,fill opacity=1 ] (253.86,219.15) .. controls (253.84,224.43) and (252.79,229.47) .. (250.91,234.08) -- (214,219) -- cycle ;
	%Shape: Pie [id:dp8404828655017231]
	\draw  [draw opacity=0][fill={rgb, 255:red, 207; green, 207; blue, 207 }  ,fill opacity=1 ] (243.79,231.84) .. controls (238.81,243.36) and (227.35,251.43) .. (214,251.43) .. controls (205.05,251.43) and (196.94,247.8) .. (191.07,241.93) -- (214,219) -- cycle ;
	%Shape: Pie [id:dp1921798512080477]
	\draw  [draw opacity=0][fill={rgb, 255:red, 207; green, 207; blue, 207 }  ,fill opacity=1 ] (212.88,88) .. controls (212.92,88) and (212.96,88) .. (213,88) .. controls (229.58,88) and (243.97,97.38) .. (251.14,111.13) -- (213,131) -- cycle ;
	%Straight Lines [id:da40010362980616754]
	\draw    (151,185) -- (445.5,185) ;
	\draw [shift={(448.5,185)}, rotate = 180] [fill={rgb, 255:red, 0; green, 0; blue, 0 }  ][line width=0.08]  [draw opacity=0] (10.72,-5.15) -- (0,0) -- (10.72,5.15) -- (7.12,0) -- cycle    ;
	%Straight Lines [id:da7488969456901324]
	\draw [line width=1.5]    (213,131) -- (309.96,79.87) ;
	\draw [shift={(313.5,78)}, rotate = 512.19] [fill={rgb, 255:red, 0; green, 0; blue, 0 }  ][line width=0.08]  [draw opacity=0] (13.4,-6.43) -- (0,0) -- (13.4,6.44) -- (8.9,0) -- cycle    ;
	%Straight Lines [id:da6439329921032626]
	\draw  [dash pattern={on 0.84pt off 2.51pt}]  (313.5,78) -- (150.5,78) ;
	%Straight Lines [id:da06318067450836828]
	\draw    (151,185) -- (151,27) ;
	\draw [shift={(151,24)}, rotate = 450] [fill={rgb, 255:red, 0; green, 0; blue, 0 }  ][line width=0.08]  [draw opacity=0] (10.72,-5.15) -- (0,0) -- (10.72,5.15) -- (7.12,0) -- cycle    ;
	%Straight Lines [id:da3735738788431415]
	\draw    (151,185) -- (49.12,286.88) ;
	\draw [shift={(47,289)}, rotate = 315] [fill={rgb, 255:red, 0; green, 0; blue, 0 }  ][line width=0.08]  [draw opacity=0] (10.72,-5.15) -- (0,0) -- (10.72,5.15) -- (7.12,0) -- cycle    ;
	%Straight Lines [id:da4212922877969385]
	\draw  [dash pattern={on 0.84pt off 2.51pt}]  (213,220) -- (213,78) ;
	%Straight Lines [id:da8129583034692129]
	\draw  [dash pattern={on 0.84pt off 2.51pt}]  (173,260) -- (248,185) ;
	%Straight Lines [id:da5309983983054607]
	\draw  [dash pattern={on 0.84pt off 2.51pt}]  (117.5,219) -- (348,219) ;
	%Straight Lines [id:da3180503246327162]
	\draw [line width=0.75]    (214,219) -- (308.24,258.83) ;
	\draw [shift={(311,260)}, rotate = 202.91] [fill={rgb, 255:red, 0; green, 0; blue, 0 }  ][line width=0.08]  [draw opacity=0] (10.72,-5.15) -- (0,0) -- (10.72,5.15) -- (7.12,0) -- cycle    ;
	%Straight Lines [id:da4455044711565159]
	\draw  [dash pattern={on 0.84pt off 2.51pt}]  (311,260) -- (311,79) ;
	%Straight Lines [id:da26365916114812626]
	\draw  [dash pattern={on 0.84pt off 2.51pt}]  (312,260) -- (387,185) ;
	%Straight Lines [id:da02453639685058273]
	\draw  [dash pattern={on 0.84pt off 2.51pt}]  (77.5,259) -- (308,259) ;
	%Straight Lines [id:da9177326517219728]
	\draw  [dash pattern={on 0.84pt off 2.51pt}]  (213,131) -- (151,131) ;
	%Straight Lines [id:da8000425396378845]
	\draw [line width=1.5]    (248,185) -- (383.5,185) ;
	\draw [shift={(387.5,185)}, rotate = 180] [fill={rgb, 255:red, 0; green, 0; blue, 0 }  ][line width=0.08]  [draw opacity=0] (13.4,-6.43) -- (0,0) -- (13.4,6.44) -- (8.9,0) -- cycle    ;
	%Straight Lines [id:da135170186754614]
	\draw [line width=1.5]    (117.5,219) -- (79.33,257.17) ;
	\draw [shift={(76.5,260)}, rotate = 315] [fill={rgb, 255:red, 0; green, 0; blue, 0 }  ][line width=0.08]  [draw opacity=0] (13.4,-6.43) -- (0,0) -- (13.4,6.44) -- (8.9,0) -- cycle    ;
	%Straight Lines [id:da5870054208298388]
	\draw [line width=1.5]    (151,131) -- (151,82) ;
	\draw [shift={(151,78)}, rotate = 450] [fill={rgb, 255:red, 0; green, 0; blue, 0 }  ][line width=0.08]  [draw opacity=0] (13.4,-6.43) -- (0,0) -- (13.4,6.44) -- (8.9,0) -- cycle    ;
	%Shape: Circle [id:dp2419013582112337]
	\draw  [fill={rgb, 255:red, 0; green, 0; blue, 0 }  ,fill opacity=1 ] (212.29,218.71) .. controls (212.29,217.61) and (213.18,216.71) .. (214.29,216.71) .. controls (215.39,216.71) and (216.29,217.61) .. (216.29,218.71) .. controls (216.29,219.82) and (215.39,220.71) .. (214.29,220.71) .. controls (213.18,220.71) and (212.29,219.82) .. (212.29,218.71) -- cycle ;
	%Shape: Circle [id:dp012579141877352207]
	\draw  [fill={rgb, 255:red, 0; green, 0; blue, 0 }  ,fill opacity=1 ] (211,131) .. controls (211,129.9) and (211.9,129) .. (213,129) .. controls (214.1,129) and (215,129.9) .. (215,131) .. controls (215,132.1) and (214.1,133) .. (213,133) .. controls (211.9,133) and (211,132.1) .. (211,131) -- cycle ;

	% Text Node
	\draw (311,60) node    {$\vec{v}$};
	% Text Node
	\draw (41,275) node    {$x$};
	% Text Node
	\draw (135,24) node    {$z$};
	% Text Node
	\draw (137,106) node    {$\vec{v}_z$};
	% Text Node
	\draw (87.67,219.33) node    {$\vec{v}_x$};
	% Text Node
	\draw (349,168) node    {$\vec{v}_y$};
	% Text Node
	\draw (255,92.67) node    {$\gamma $};
	% Text Node
	\draw (245.67,246) node    {$\alpha $};
	% Text Node
	\draw (274.33,231.33) node    {$\beta $};
	% Text Node
	\draw (465,180) node    {$y$};


	\end{tikzpicture}
\end{figure}


\subsection{Operazioni fra vettori}

\paragraph{Prodotto di un vettore per uno scalare} Il risultato è un vettore che ha la stessa direzione del vettore di partenza $\vec{v}$, modulo pari a $\abs{\vec{v}}$ moltiplicato per lo scalare $m$, stesso verso di $\vec{v}$ se $m>0$, opposto se $m<0$.
\[
	\vec{v}\,m
\]

\paragraph{Somma vettoriale} Dati due vettori $\vec{a}$ e $\vec{b}$ la somma è un vettore che si ottiene graficamente tramite il cosiddetto \emph{metodo del parallelogramma}: si applica nell'origine di $\vec{a}$ anche il vettore $\vec{b}$ spostandolo parallelamente a se stesso; il vettore somma coincide con la diagonale del parallelogramma che ha come lati $\vec{a}$ e $\vec{b}$.
\begin{figure}[htpb]
	\centering
		

	\tikzset{every picture/.style={line width=0.75pt}} %set default line width to 0.75pt

	\begin{tikzpicture}[x=0.75pt,y=0.75pt,yscale=-1,xscale=1]
	%uncomment if require: \path (0,300); %set diagram left start at 0, and has height of 300

	%Straight Lines [id:da2440730162800544]
	\draw [line width=1.5]    (160,180) -- (296,180) ;
	\draw [shift={(300,180)}, rotate = 180] [fill={rgb, 255:red, 0; green, 0; blue, 0 }  ][line width=0.08]  [draw opacity=0] (13.4,-6.43) -- (0,0) -- (13.4,6.44) -- (8.9,0) -- cycle    ;
	%Straight Lines [id:da4892608957102378]
	\draw [line width=1.5]    (160,180) -- (217.4,113.04) ;
	\draw [shift={(220,110)}, rotate = 490.6] [fill={rgb, 255:red, 0; green, 0; blue, 0 }  ][line width=0.08]  [draw opacity=0] (13.4,-6.43) -- (0,0) -- (13.4,6.44) -- (8.9,0) -- cycle    ;
	%Straight Lines [id:da3719981116584463]
	\draw  [dash pattern={on 0.84pt off 2.51pt}]  (220,110) -- (360,110) ;
	%Straight Lines [id:da6616324461946157]
	\draw  [dash pattern={on 0.84pt off 2.51pt}]  (360,110) -- (300,180) ;
	%Straight Lines [id:da8126730528636246]
	\draw [line width=1.5]    (160,180) -- (356.22,111.32) ;
	\draw [shift={(360,110)}, rotate = 520.71] [fill={rgb, 255:red, 0; green, 0; blue, 0 }  ][line width=0.08]  [draw opacity=0] (13.4,-6.43) -- (0,0) -- (13.4,6.44) -- (8.9,0) -- cycle    ;

	% Text Node
	\draw (173,139) node    {$\vec{a}$};
	% Text Node
	\draw (230,196) node    {$\vec{b}$};
	% Text Node
	\draw (244,131) node    {$\vec{c}$};

	\end{tikzpicture}
\end{figure}
Riprendendo il concetto di scomposizione dei vettori sugli assi cartesiani, si osservi che, dal momento che i tre versori $\vec{u}_x, \vec{u}_y, \vec{u}_z$ sono mutuamente ortogonali fra di loro:
\[
	\vec{v}=\vec{v}_x + \vec{v}_y + \vec{v}_z
\]
Questo permette di affermare che le operazioni fra vettori possono essere eseguite più facilmente sfruttandone la scomposizione lungo gli assi cartesiani. Il vettore somma infatti ha come componenti la somma delle componenti dei singoli vettori di partenza.
È infine evidente che il risultato della somma di più vettori non dipende dal loro ordine: vale la \emph{proprietà associativa}.

\paragraph{Differenza fra due vettori} Tale operazione può essere facilmente definita come la somma fra il primo vettore e l'opposto del secondo, ossia il secondo moltiplicato per $-1$.
\\
\\
\\
\\
Si definiscono inoltre due diverse operazioni di prodotto che danno rispettivamente come risultato una quantità scalare o vettoriale.

\paragraph{Prodotto scalare} Dati due vettori $\vec{a}$ e $\vec{b}$ si definisce prodotto scalare la quantità:
\[
	\vec{a}\cdot \vec{b}= \abs{\vec{a}}\abs{\vec{b}}\cos\alpha
\]
Indicando con $\alpha$ l'angolo formato dai due vettori, il risultato del prodotto scalare è una quantità scalare data dal prodotto dei moduli di $\vec{a}$ e di $\vec{b}$ per il coseno dell'angolo fra di essi compreso. Tale quantità è anche pari alla somma dei prodotti delle componenti omologhe dei singoli vettori. Il prodotto scalare è nullo non solo se uno dei due vettori è nullo, ma anche se essi sono ortogonali. Valgono inoltre le proprietà associativa e distributiva. Un altro modo, fisicamente molto significativo, di rappresentare il prodotto scalare è il seguente:
\[
	\vec{a}\cdot \vec{b}= \abs{\vec{a}}\,(\abs{\vec{b}}\cos\alpha)
\]
\begin{figure}[htpb]
	\centering
	

	\tikzset{every picture/.style={line width=0.75pt}} %set default line width to 0.75pt        

	\begin{tikzpicture}[x=0.75pt,y=0.75pt,yscale=-0.9,xscale=0.9]
	%uncomment if require: \path (0,300); %set diagram left start at 0, and has height of 300

	%Straight Lines [id:da27354838547691696] 
	\draw [line width=1.5]    (60,180) -- (286,180) ;
	\draw [shift={(290,180)}, rotate = 180] [fill={rgb, 255:red, 0; green, 0; blue, 0 }  ][line width=0.08]  [draw opacity=0] (13.4,-6.43) -- (0,0) -- (13.4,6.44) -- (8.9,0) -- cycle    ;
	%Straight Lines [id:da5678151431421725] 
	\draw [line width=1.5]    (60,180) -- (137.5,83.12) ;
	\draw [shift={(140,80)}, rotate = 488.66] [fill={rgb, 255:red, 0; green, 0; blue, 0 }  ][line width=0.08]  [draw opacity=0] (13.4,-6.43) -- (0,0) -- (13.4,6.44) -- (8.9,0) -- cycle    ;
	%Straight Lines [id:da3953822489943375] 
	\draw  [dash pattern={on 0.84pt off 2.51pt}]  (140,80) -- (140,180) ;
	%Shape: Arc [id:dp8373824492896644] 
	\draw  [draw opacity=0] (77.98,155.98) .. controls (85.28,161.46) and (90,170.18) .. (90,180) .. controls (90,180.2) and (90,180.41) .. (89.99,180.61) -- (60,180) -- cycle ; \draw   (77.98,155.98) .. controls (85.28,161.46) and (90,170.18) .. (90,180) .. controls (90,180.2) and (90,180.41) .. (89.99,180.61) ;
	%Straight Lines [id:da9058029829917178] 
	\draw    (130,170) -- (130,180) ;
	%Straight Lines [id:da715168339128929] 
	\draw    (140,170) -- (130,170) ;
	%Straight Lines [id:da4600264437767905] 
	\draw    (140,200) -- (60,200) ;
	\draw [shift={(60,200)}, rotate = 360] [color={rgb, 255:red, 0; green, 0; blue, 0 }  ][line width=0.75]    (0,5.59) -- (0,-5.59)   ;
	\draw [shift={(140,200)}, rotate = 360] [color={rgb, 255:red, 0; green, 0; blue, 0 }  ][line width=0.75]    (0,5.59) -- (0,-5.59)   ;
	%Straight Lines [id:da9351143374864943] 
	\draw [line width=1.5]    (340,180) -- (566,180) ;
	\draw [shift={(570,180)}, rotate = 180] [fill={rgb, 255:red, 0; green, 0; blue, 0 }  ][line width=0.08]  [draw opacity=0] (13.4,-6.43) -- (0,0) -- (13.4,6.44) -- (8.9,0) -- cycle    ;
	%Straight Lines [id:da5952375166706461] 
	\draw [line width=1.5]    (340,180) -- (417.5,83.12) ;
	\draw [shift={(420,80)}, rotate = 488.66] [fill={rgb, 255:red, 0; green, 0; blue, 0 }  ][line width=0.08]  [draw opacity=0] (13.4,-6.43) -- (0,0) -- (13.4,6.44) -- (8.9,0) -- cycle    ;
	%Straight Lines [id:da15985064204341093] 
	\draw  [dash pattern={on 0.84pt off 2.51pt}]  (430,67.5) -- (340,180) ;
	%Shape: Arc [id:dp6036328273138647] 
	\draw  [draw opacity=0] (357.98,155.98) .. controls (365.28,161.46) and (370,170.18) .. (370,180) .. controls (370,180.2) and (370,180.41) .. (369.99,180.61) -- (340,180) -- cycle ; \draw   (357.98,155.98) .. controls (365.28,161.46) and (370,170.18) .. (370,180) .. controls (370,180.2) and (370,180.41) .. (369.99,180.61) ;
	%Straight Lines [id:da17422970123026738] 
	\draw  [dash pattern={on 0.84pt off 2.51pt}]  (570,180) -- (430,68) ;
	%Straight Lines [id:da09367412711699274] 
	\draw    (437.5,73.5) -- (431.4,81.13) ;
	%Straight Lines [id:da7293699633872541] 
	\draw    (431.4,81.13) -- (423.78,75.03) ;

	%Straight Lines [id:da22095363277223878] 
	\draw    (420,57.5) -- (330,170) ;
	\draw [shift={(330,170)}, rotate = 308.65999999999997] [color={rgb, 255:red, 0; green, 0; blue, 0 }  ][line width=0.75]    (0,5.59) -- (0,-5.59)   ;
	\draw [shift={(420,57.5)}, rotate = 308.65999999999997] [color={rgb, 255:red, 0; green, 0; blue, 0 }  ][line width=0.75]    (0,5.59) -- (0,-5.59)   ;

	% Text Node
	\draw (83,121) node    {$\vec{a}$};
	% Text Node
	\draw (237,197) node    {$\vec{b}$};
	% Text Node
	\draw (99,211) node    {$|\vec{a} |\cos \alpha $};
	% Text Node
	\draw (97.5,160) node    {$\alpha $};
	% Text Node
	\draw (406.33,115) node    {$\vec{a}$};
	% Text Node
	\draw (517,197) node    {$\vec{b}$};
	% Text Node
	\draw (359,83) node    {$|\vec{b} |\cos \alpha $};
	% Text Node
	\draw (377.5,160) node    {$\alpha $};


	\end{tikzpicture}
\end{figure}
\FloatBarrier
Ossia: il prodotto scalare fra due vettori è eguale al prodotto del modulo di uno dei vettori per la proiezione su di esso dell'altro vettore.

\paragraph{Prodotto vettoriale} Dati due vettori $\vec{a}$ e $\vec{b}$ si definisce prodotto vettoriale il vettore $\vec{c}$, che si indica con il simbolo:
\[
	\vec{c}=\vec{a} \times \vec{b}
\]
Tale espressione si legge `\textit{a vettor b}'. Esso ha:
\begin{itemize}
	\item direzione perpendicolare al piano individuato da $\vec{a}$ e $\vec{b}$;
	\item verso di una normale vite destrogira (ruotando da $\vec{a}$ a $\vec{b}$ nel verso di una vite il verso di $\vec{c}$ è indicato dalla punta della vite);
	\item modulo pari a $\abs{\vec{a}}\abs{\vec{b}}\sin\alpha$ che coincide con l'area del parallelogramma di lati $\vec{a}$ e $\vec{b}$.
\end{itemize}
\begin{figure}[htpb]
	\centering
	

	% Pattern Info
	 
	\tikzset{
	pattern size/.store in=\mcSize, 
	pattern size = 5pt,
	pattern thickness/.store in=\mcThickness, 
	pattern thickness = 0.3pt,
	pattern radius/.store in=\mcRadius, 
	pattern radius = 1pt}
	\makeatletter
	\pgfutil@ifundefined{pgf@pattern@name@_uar2b48nb}{
	\pgfdeclarepatternformonly[\mcThickness,\mcSize]{_uar2b48nb}
	{\pgfqpoint{0pt}{-\mcThickness}}
	{\pgfpoint{\mcSize}{\mcSize}}
	{\pgfpoint{\mcSize}{\mcSize}}
	{
	\pgfsetcolor{\tikz@pattern@color}
	\pgfsetlinewidth{\mcThickness}
	\pgfpathmoveto{\pgfqpoint{0pt}{\mcSize}}
	\pgfpathlineto{\pgfpoint{\mcSize+\mcThickness}{-\mcThickness}}
	\pgfusepath{stroke}
	}}
	\makeatother
	\tikzset{every picture/.style={line width=0.75pt}} %set default line width to 0.75pt        

	\begin{tikzpicture}[x=0.75pt,y=0.75pt,yscale=-1,xscale=1]
	%uncomment if require: \path (0,300); %set diagram left start at 0, and has height of 300

	%Shape: Parallelogram [id:dp4647818089148623] 
	\draw  [draw opacity=0][pattern=_uar2b48nb,pattern size=3.75pt,pattern thickness=0.75pt,pattern radius=0pt, pattern color={rgb, 255:red, 222; green, 222; blue, 222}] (170,120) -- (310,120) -- (250,190) -- (110,190) -- cycle ;
	%Straight Lines [id:da07829591594899687] 
	\draw [line width=1.5]    (110,190) -- (246,190) ;
	\draw [shift={(250,190)}, rotate = 180] [fill={rgb, 255:red, 0; green, 0; blue, 0 }  ][line width=0.08]  [draw opacity=0] (13.4,-6.43) -- (0,0) -- (13.4,6.44) -- (8.9,0) -- cycle    ;
	%Straight Lines [id:da3908497787206091] 
	\draw [line width=1.5]    (110,190) -- (167.4,123.04) ;
	\draw [shift={(170,120)}, rotate = 490.6] [fill={rgb, 255:red, 0; green, 0; blue, 0 }  ][line width=0.08]  [draw opacity=0] (13.4,-6.43) -- (0,0) -- (13.4,6.44) -- (8.9,0) -- cycle    ;
	%Straight Lines [id:da5010366366593619] 
	\draw  [dash pattern={on 0.84pt off 2.51pt}]  (170,120) -- (310,120) ;
	%Straight Lines [id:da3791967059336212] 
	\draw  [dash pattern={on 0.84pt off 2.51pt}]  (310,120) -- (250,190) ;
	%Straight Lines [id:da18657789312970685] 
	\draw [line width=1.5]    (110,190) -- (110,44) ;
	\draw [shift={(110,40)}, rotate = 450] [fill={rgb, 255:red, 0; green, 0; blue, 0 }  ][line width=0.08]  [draw opacity=0] (13.4,-6.43) -- (0,0) -- (13.4,6.44) -- (8.9,0) -- cycle    ;
	%Shape: Arc [id:dp11097324545322373] 
	\draw  [draw opacity=0] (124.11,172.11) .. controls (133.57,175.14) and (140,181.12) .. (140,188) .. controls (140,188.4) and (139.98,188.79) .. (139.94,189.18) -- (110,188) -- cycle ; \draw   (124.11,172.11) .. controls (133.57,175.14) and (140,181.12) .. (140,188) .. controls (140,188.4) and (139.98,188.79) .. (139.94,189.18) ;
	%Shape: Free Drawing [id:dp43190093896936044] 
	\draw  [color={rgb, 255:red, 0; green, 0; blue, 0 }  ,draw opacity=1 ][line width=1.5] [line join = round][line cap = round] (497,53.5) .. controls (497,58.51) and (495.62,63.5) .. (496,68.5) .. controls (497.03,81.92) and (514.53,115.3) .. (524,125.5) .. controls (531.26,133.32) and (540.84,142.34) .. (547,148.5) .. controls (549.31,150.81) and (562.21,153.29) .. (561,154.5) ;
	%Shape: Free Drawing [id:dp2469297143503948] 
	\draw  [color={rgb, 255:red, 0; green, 0; blue, 0 }  ,draw opacity=1 ][line width=1.5] [line join = round][line cap = round] (497,53.5) .. controls (495.57,53.5) and (492.53,50.05) .. (490,51.5) .. controls (469.32,63.32) and (495.46,100.04) .. (482,113.5) .. controls (476.53,118.97) and (465.75,113.42) .. (460,111.5) .. controls (445.65,106.72) and (410.75,96.13) .. (398,102.5) .. controls (390.34,106.33) and (400.23,113.34) .. (406,115.5) .. controls (415.9,119.21) and (443.5,123.76) .. (447,132.5) .. controls (448.96,137.41) and (436.2,141.66) .. (432,142.5) .. controls (419.35,145.03) and (406.92,147.84) .. (409,164.5) .. controls (409.34,167.2) and (420.48,166.55) .. (421,166.5) .. controls (423.36,166.29) and (428.74,162.13) .. (430,161.5) .. controls (436.52,158.24) and (447.08,156.5) .. (454,156.5) ;
	%Shape: Free Drawing [id:dp2699755526754224] 
	\draw  [color={rgb, 255:red, 0; green, 0; blue, 0 }  ,draw opacity=1 ][line width=1.5] [line join = round][line cap = round] (458,157.5) .. controls (458,160.85) and (456.4,164.2) .. (457,167.5) .. controls (459.38,180.58) and (481.86,171.69) .. (489,170.5) .. controls (495.92,169.35) and (499.73,168.13) .. (505,165.5) .. controls (505.3,165.35) and (505.92,165.82) .. (506,165.5) .. controls (508.11,157.07) and (479.9,156.93) .. (473,156.5) .. controls (467.31,156.14) and (460.03,154.47) .. (456,158.5) ;
	%Shape: Free Drawing [id:dp0065945962483409115] 
	\draw  [color={rgb, 255:red, 0; green, 0; blue, 0 }  ,draw opacity=1 ][line width=1.5] [line join = round][line cap = round] (461,178.5) .. controls (461,199.46) and (482.17,191.41) .. (496,184.5) .. controls (499.72,182.64) and (510.02,180.44) .. (511,176.5) .. controls (512.56,170.28) and (506.23,170.5) .. (502,170.5) ;
	%Shape: Free Drawing [id:dp7784500736262567] 
	\draw  [color={rgb, 255:red, 0; green, 0; blue, 0 }  ,draw opacity=1 ][line width=1.5] [line join = round][line cap = round] (477,192.5) .. controls (490.14,201.26) and (503.9,194.34) .. (519,195.5) .. controls (529.05,196.27) and (536.53,209.5) .. (547,209.5) ;
	%Shape: Free Drawing [id:dp5945113146611334] 
	\draw  [color={rgb, 255:red, 155; green, 155; blue, 155 }  ,draw opacity=1 ][line width=1.5] [line join = round][line cap = round] (489,81.5) .. controls (490.8,81.5) and (492.73,80.77) .. (494,79.5) ;
	%Shape: Free Drawing [id:dp863292993267458] 
	\draw  [color={rgb, 255:red, 155; green, 155; blue, 155 }  ,draw opacity=1 ][line width=1.5] [line join = round][line cap = round] (407,111.5) .. controls (408.41,108.69) and (409.78,105.72) .. (412,103.5) ;
	%Shape: Free Drawing [id:dp8451572197613626] 
	\draw  [color={rgb, 255:red, 155; green, 155; blue, 155 }  ,draw opacity=1 ][line width=1.5] [line join = round][line cap = round] (428,117.5) .. controls (428,114.67) and (428.82,110.5) .. (432,110.5) ;
	%Shape: Free Drawing [id:dp5382674395061409] 
	\draw  [color={rgb, 255:red, 155; green, 155; blue, 155 }  ,draw opacity=1 ][line width=1.5] [line join = round][line cap = round] (445,126.5) .. controls (446.87,119.94) and (447.46,118.04) .. (452,113.5) ;
	%Shape: Free Drawing [id:dp887365066791401] 
	\draw  [color={rgb, 255:red, 155; green, 155; blue, 155 }  ,draw opacity=1 ][line width=1.5] [line join = round][line cap = round] (425,151.5) .. controls (426.75,154.41) and (427,158.1) .. (427,161.5) ;
	%Shape: Free Drawing [id:dp773009858564268] 
	\draw  [color={rgb, 255:red, 155; green, 155; blue, 155 }  ,draw opacity=1 ][line width=1.5] [line join = round][line cap = round] (436,146.5) .. controls (437.63,149.21) and (439,152.34) .. (439,155.5) ;
	%Shape: Free Drawing [id:dp4112761415080526] 
	\draw  [color={rgb, 255:red, 155; green, 155; blue, 155 }  ,draw opacity=1 ][line width=1.5] [line join = round][line cap = round] (448,142.5) .. controls (450.79,142.5) and (451.66,151.84) .. (450,153.5) ;
	%Shape: Free Drawing [id:dp6484449716929179] 
	\draw  [color={rgb, 255:red, 155; green, 155; blue, 155 }  ,draw opacity=1 ][line width=1.5] [line join = round][line cap = round] (454,136.5) .. controls (460.79,136.5) and (474,145.82) .. (474,153.5) ;
	%Shape: Free Drawing [id:dp6067471895852834] 
	\draw  [color={rgb, 255:red, 155; green, 155; blue, 155 }  ,draw opacity=1 ][line width=1.5] [line join = round][line cap = round] (479,120.5) .. controls (476.4,123.1) and (478,127.82) .. (478,131.5) ;
	%Shape: Free Drawing [id:dp3287032141639632] 
	\draw  [color={rgb, 255:red, 155; green, 155; blue, 155 }  ,draw opacity=1 ][line width=1.5] [line join = round][line cap = round] (485,123.5) .. controls (485,134.19) and (492.94,147.06) .. (502,152.5) ;
	%Shape: Free Drawing [id:dp2903477573462456] 
	\draw  [color={rgb, 255:red, 155; green, 155; blue, 155 }  ,draw opacity=1 ][line width=1.5] [line join = round][line cap = round] (528,145.5) .. controls (528,149.96) and (527.22,152.39) .. (523,154.5) ;
	%Shape: Free Drawing [id:dp21087767721795547] 
	\draw  [color={rgb, 255:red, 155; green, 155; blue, 155 }  ,draw opacity=1 ][line width=1.5] [line join = round][line cap = round] (524,185.5) .. controls (524,180.61) and (532,160.97) .. (532,158.5) ;
	%Shape: Free Drawing [id:dp7518276869665781] 
	\draw  [color={rgb, 255:red, 155; green, 155; blue, 155 }  ,draw opacity=1 ][line width=1.5] [line join = round][line cap = round] (470,161.5) .. controls (470,166.13) and (471,167.93) .. (471,171.5) ;
	%Shape: Free Drawing [id:dp4308694134358355] 
	\draw  [color={rgb, 255:red, 155; green, 155; blue, 155 }  ,draw opacity=1 ][line width=1.5] [line join = round][line cap = round] (485,160.5) .. controls (485,164.02) and (486,165.61) .. (486,168.5) ;
	%Shape: Free Drawing [id:dp03487892767445544] 
	\draw  [color={rgb, 255:red, 155; green, 155; blue, 155 }  ,draw opacity=1 ][line width=1.5] [line join = round][line cap = round] (474,179.5) .. controls (474.62,181.96) and (475.59,184.39) .. (477,186.5) ;
	%Shape: Free Drawing [id:dp7220964275690744] 
	\draw  [color={rgb, 255:red, 155; green, 155; blue, 155 }  ,draw opacity=1 ][line width=1.5] [line join = round][line cap = round] (489,175.5) .. controls (489,177.68) and (489.72,182.5) .. (492,182.5) ;

	%Straight Lines [id:da980106951249955] 
	\draw [line width=1.5]    (405,107) -- (368.79,94.78) ;
	\draw [shift={(365,93.5)}, rotate = 378.65] [fill={rgb, 255:red, 0; green, 0; blue, 0 }  ][line width=0.08]  [draw opacity=0] (13.4,-6.43) -- (0,0) -- (13.4,6.44) -- (8.9,0) -- cycle    ;
	%Straight Lines [id:da9191850982532173] 
	\draw [line width=1.5]    (416,159) -- (384.54,175.63) ;
	\draw [shift={(381,177.5)}, rotate = 332.14] [fill={rgb, 255:red, 0; green, 0; blue, 0 }  ][line width=0.08]  [draw opacity=0] (13.4,-6.43) -- (0,0) -- (13.4,6.44) -- (8.9,0) -- cycle    ;
	%Straight Lines [id:da5262132613493635] 
	\draw [line width=1.5]    (489,64) -- (489,30.5) ;
	\draw [shift={(489,26.5)}, rotate = 450] [fill={rgb, 255:red, 0; green, 0; blue, 0 }  ][line width=0.08]  [draw opacity=0] (13.4,-6.43) -- (0,0) -- (13.4,6.44) -- (8.9,0) -- cycle    ;

	% Text Node
	\draw (147,121) node    {$\vec{b}$};
	% Text Node
	\draw (213,207) node    {$\vec{a}$};
	% Text Node
	\draw (154.5,57) node    {$\vec{c} =\vec{a} \times \vec{b}$};
	% Text Node
	\draw (216.5,148) node   [align=left] {area = $\displaystyle |\vec{c} |$};
	% Text Node
	\draw (147.5,170) node    {$\alpha $};
	% Text Node
	\draw (375,75) node    {$\vec{a}$};
	% Text Node
	\draw (378,157) node    {$\vec{b}$};
	% Text Node
	\draw (518.5,28) node    {$\vec{a} \times \vec{b}$};


	\end{tikzpicture}
\end{figure}
\FloatBarrier
È bene osservare che il prodotto vettoriale non gode della proprietà commutativa. Infatti eseguendo il prodotto $\vec{b}\times \vec{a}$ si ottiene il vettore $-\vec{c}$, opposto a $\vec{c}$. Si dice dunque che il prodotto vettoriale è \emph{anticommutativo}. Esso può essere iterato, ma va specificato l'ordine delle diverse operazioni, perché non vale la proprietà associativa.

Calcolare il prodotto vettoriale $\vec{a} \times \vec{b}$ equivale a calcolare il determinante della seguente matrice:
\[
	\begin{bmatrix}
		\vec{u}_x & \vec{u}_y & \vec{u}_z \\
		a_x & a_y & a_z \\
		b_x & b_y & b_z
	\end{bmatrix}
\]
Ossia
\[
	\vec{a} \times \vec{b} = (a_y b_z - a_z b_y)\vec{u}_x + (a_z b_x - a_x b_z) \vec{u}_y + (a_x b_y - a_y b_x) \vec{u}_z
\]























































































































\chapter{Cinematica del punto materiale}

\section{La meccanica}

La meccanica è quella teoria che si occupa dello studio del moto di un corpo: essa spiega le relazioni che sussistono fra le cause che generano il moto e le sue caratteristiche, esprimendole con leggi quantitative. Se il corpo è esteso, come lo sono tutti i corpi materiali, il moto può risultare notevolmente complicato. Per questa ragione si inizia a trattare il suo studio partendo dal più semplice corpo, quello puntiforme, detto \textbf{punto materiale}. Si tratta di un corpo che presenta dimensioni trascurabili rispetto a quelle dello spazio in cui può muoversi o degli altri corpi con cui può interagire. Uno studio più generale come questo permette di definire più facilmente alcune grandezze meccaniche fondamentali e di comprenderne di conseguenza il significato con immediatezza, in assenza delle complicazioni che deriverebbero dalla struttura estesa del corpo. D'altra parte un corpo esteso solo eccezionalmente si comporta come un punto materiale: esso può compiere contemporaneamente altri tipi di moto, come rotazioni, vibrazioni ecc.

L'analisi completa del moto riguarda due diversi aspetti:
\begin{itemize}
	\item la \textbf{dinamica}: ramo della meccanica che studia le cause del moto;
	\item la \textbf{cinematica}: branca della maccenica che si occupa della descrizione del moto di un corpo indipendentemente dalle cause che lo determinano.
\end{itemize}
In questa sezione ci si occupa della cinematica fisica. Esistono due differenti modalità attraverso le quali essa può essere studiata: la \textbf{trattazione scalare}, nell'ambito della quale tutte le grandezze fisiche rilevanti possono essere espresse servendosi solo di numeri con segno, a cui si contrappone la \textbf{trattazione vettoriale}, dove i numeri vengono affiancati da informazioni geometriche.







































\section{Trattazione scalare della cinematica fisica}

Studiare il moto di un punto significa conoscerne la posizione in ogni istante.

\paragraph{Osservazione} La posizione di un punto viene sempre assegnata rispetto a un \textbf{sistema di riferimento}. Esso è definito completamente da un punto nello spazio detto \emph{origine} e da tre direzioni preferenziali. Il più noto è quello cartesiano (la terna $x, y, z$ è detta destrorsa perché le tre direzioni nello spazio possono essere indicate correttamente grazie all'uso della mano destra).

\begin{figure}[htpb]
	\centering
		

	\tikzset{every picture/.style={line width=0.75pt}} %set default line width to 0.75pt

	\begin{tikzpicture}[x=0.75pt,y=0.75pt,yscale=-1,xscale=1]
	%uncomment if require: \path (0,300); %set diagram left start at 0, and has height of 300

	%Curve Lines [id:da2877491069433016]
	\draw    (93.5,127) .. controls (140.5,9) and (226.5,210) .. (291.5,128) ;
	%Curve Lines [id:da7157799350393301]
	\draw [line width=1.5]    (107,103) .. controls (135,69) and (169.67,105.67) .. (198.33,122.33) ;
	%Shape: Circle [id:dp8125790931296473]
	\draw  [fill={rgb, 255:red, 0; green, 0; blue, 0 }  ,fill opacity=1 ] (104,103) .. controls (104,101.34) and (105.34,100) .. (107,100) .. controls (108.66,100) and (110,101.34) .. (110,103) .. controls (110,104.66) and (108.66,106) .. (107,106) .. controls (105.34,106) and (104,104.66) .. (104,103) -- cycle ;
	%Shape: Circle [id:dp18864730482947079]
	\draw  [fill={rgb, 255:red, 0; green, 0; blue, 0 }  ,fill opacity=1 ] (194.33,121.33) .. controls (194.33,119.68) and (195.68,118.33) .. (197.33,118.33) .. controls (198.99,118.33) and (200.33,119.68) .. (200.33,121.33) .. controls (200.33,122.99) and (198.99,124.33) .. (197.33,124.33) .. controls (195.68,124.33) and (194.33,122.99) .. (194.33,121.33) -- cycle ;

	% Text Node
	\draw (94.67,90.67) node    {$\Omega $};
	% Text Node
	\draw (214.67,104.67) node    {$p( t)$};
	% Text Node
	\draw (60,92) node   [align=left] {origine};


	\end{tikzpicture}
\end{figure}
Il luogo dei punti successivi occupati dal punto in movimento nello spazio dà luogo a una linea continua aperta o chiusa che prende il nome di \textbf{traiettoria}. Nota la traiettoria $\Gamma$, è necessario conoscere come la posizione del punto materiale evolve su di essa nel tempo. Per fare ciò, si fissa un'origine $O$ su $\Gamma$ e si individua la posizione $P$ del punto con la lunghezza della curva $OP$ sulla traiettoria: questa viene detta \textbf{ascissa curvilinea}, $s(t)$. Si costruisce quindi la funzione scalare $s(t)$, nota come \textbf{legge oraria}, per ogni istante di tempo, definendo così il moto in modo completo, permettendo di conoscere quale è la traiettoria seguita dal corpo e come essa venga tracciata.
								
Si noti che l'ascissa curvilinea può essere un valore positivo o negativo. Nel primo caso il punto si trova a destra dell'origine, nel secondo a sinistra. La funzione $s(t)$ può essere rappresentata in un grafico, il \emph{diagramma orario}, che presenta sull'asse delle ascisse il tempo $t$, mentre sull'asse delle ordinate la posizione $s$ all'istante $t$. Ovviamente ha un senso utilizzare la trattazione scalare quando la traiettoria è nota a priori.

\begin{figure}[htpb]
	\centering
		

	\tikzset{every picture/.style={line width=0.75pt}} %set default line width to 0.75pt

	\begin{tikzpicture}[x=0.75pt,y=0.75pt,yscale=-1,xscale=1]
	%uncomment if require: \path (0,300); %set diagram left start at 0, and has height of 300

	%Shape: Axis 2D [id:dp1464076526716247]
	\draw  (50,204.4) -- (356.5,204.4)(80.65,55) -- (80.65,221) (349.5,199.4) -- (356.5,204.4) -- (349.5,209.4) (75.65,62) -- (80.65,55) -- (85.65,62)  ;
	%Curve Lines [id:da9345995628596111]
	\draw    (114,123) .. controls (154,93) and (174,153) .. (214,123) .. controls (254,93) and (239,158) .. (264,133) .. controls (289,108) and (289,168) .. (314,143) ;

	% Text Node
	\draw (68,54) node    {$s$};
	% Text Node
	\draw (367,208) node    {$t$};


	\end{tikzpicture}
\end{figure}

È possibile a questo punto aggiungere la definizione di grandezze cinematiche che rendono più semplice la trattazione. Lo studio delle variazioni di posizione lungo la traiettoria infatti porta a definire il concetto di \textit{velocità}, mentre lo studio delle variazioni di velocità nel tempo introdurrà la grandezza \textit{accelerazione}. Le grandezze fondamentali in cinematica sono pertanto lo spazio, la velocità, l'accelerazione e il tempo. Quest'ultimo molto spesso viene usato come variabile indipendente, in funzione del quale si esprimono le altre grandezze.

Si comincia introducendo il concetto di \textbf{velocità media}. Si ha che:
\begin{equation}
	\boxed{v_\text{media} (t_1,t_2)= \frac{s(t_2)-s(t_1)}{t_2 - t_1}= \frac{\Delta s}{\Delta t} \quad \biggl(\frac{m}{s}\biggr)}
\end{equation}
Nel grafico $v=\frac{\Delta s}{\Delta t}$ individua la pendenza della retta che passa per $s_1$ e $s_2$. La velocità media tuttavia fornisce una informazione complessiva, ma quasi nessuna indicazione sulle caratteristiche effettive del moto: non è possibile comprendere grazie a essa che cosa accade in ogni istante. Ad esempio, quando essa è pari a zero, può significare che il punto si è mosso per poi tornare indietro avendo così $s_1=s_2$, non necessariamente che esso sia rimasto immobile. Si noti inoltre che la velocità media può essere positiva, negativa o nulla.

Per avere un'informazione più precisa si rende l'intervallo di osservazione il più piccolo possibile. Se $\Delta s$ risulta suddiviso in un numero elevatissimo di intervallini $ds$, ciascuno percorso nell'istante di tempo $dt$, si può definire la \textbf{velocità istantanea} come il rapporto $v=\frac{ds}{dt}$ calcolato in quel determinanto istante. Il metodo appena descritto in modo relativamente semplice consiste matematicamente nel calcolare il limite per $\Delta t \to 0$ del rapporto incrementale $\frac{\Delta s}{\Delta t}$.
Pertanto la velocità istantanea di un punto nel moto rettilineo è data dalla derivata dello spazio rispetto al tempo.
\begin{equation}
	\boxed{v_\textup{istantanea}=v(t)=\lim_{\Delta t \to 0} \frac{\Delta s} {\Delta t}= \frac{ds}{dt}=s'(t)}
\end{equation}
Il segno della velocità indica il verso del moto sull'asse $x$: se $v>0$ la coordinata $x$ cresce, se $v<0$, essa diminuisce e il moto avviene nel senso opposto.

\begin{figure}[htpb]
	\centering
		

	\tikzset{every picture/.style={line width=0.75pt}} %set default line width to 0.75pt

	\begin{tikzpicture}[x=0.75pt,y=0.75pt,yscale=-1,xscale=1]
	%uncomment if require: \path (0,300); %set diagram left start at 0, and has height of 300

	%Shape: Axis 2D [id:dp5986556027961727]
	\draw  (51,155) -- (259.5,155)(69.88,81) -- (69.88,229) (252.5,150) -- (259.5,155) -- (252.5,160) (64.88,88) -- (69.88,81) -- (74.88,88)  ;
	%Shape: Axis 2D [id:dp5662695542333875]
	\draw  (321,155) -- (517.11,155)(338.76,81) -- (338.76,229) (510.11,150) -- (517.11,155) -- (510.11,160) (333.76,88) -- (338.76,81) -- (343.76,88)  ;
	%Straight Lines [id:da6779633587954197]
	\draw [line width=1.5]    (70.38,129) -- (120.5,129) ;
	%Straight Lines [id:da9167817945203716]
	\draw [line width=1.5]    (338.76,155) -- (388.88,155) ;
	%Straight Lines [id:da25933988363695937]
	\draw [line width=1.5]    (120.5,129) -- (150.5,90.25) ;
	%Straight Lines [id:da8168651142384409]
	\draw [line width=1.5]    (388.88,121) -- (418.88,121.25) ;
	%Straight Lines [id:da30217738665424876]
	\draw [line width=1.5]    (150.5,90.25) -- (200.62,90.25) ;
	%Straight Lines [id:da10190075508153096]
	\draw [line width=1.5]    (420.26,155) -- (470.38,155) ;
	%Straight Lines [id:da9005064343267386]
	\draw [line width=1.5]    (200.62,90.25) -- (236.5,179.75) ;
	%Straight Lines [id:da3440663507084676]
	\draw [line width=1.5]    (470.38,175) -- (506.26,175.5) ;

	% Text Node
	\draw (58,78) node    {$s$};
	% Text Node
	\draw (274,155) node    {$t$};
	% Text Node
	\draw (328,78) node    {$v$};
	% Text Node
	\draw (526.98,155) node    {$t$};


	\end{tikzpicture}
\end{figure}

In maniera del tutto analoga è possibile ottenere dal grafico della velocità l'accelerazione scalare media e istantanea.
\begin{gather}
	\boxed{a_\textup{media} (t_1, t_2)= \frac{v(t_2)-v(t_1)}{t_2 - t_1}} \\
	\boxed{a_\textup{istantanea}=a(t)=\lim_{\Delta t \to 0} \frac{\Delta v} {\Delta t} =\frac{dv}{dt}=\frac{d^2s}{dt^2}=s''(t)}
\end{gather}
Risulta evidente che se dall'ascissa curvilinea è possibile arrivare alla velocità e all'accelerazione, può essere attuato anche il procedimento inverso: ossia ricavare le legge oraria $s(t)$ nota la dipendenza dal tempo dell'accelerazione istantanea, $a(t)$. Questo problema è conosciuto come \textbf{problema inverso}.
\begin{gather*}
	a(t)=\frac{dv}{dt} \implies dv=a(t) dt \implies \int^{v(t)}_{v(t_0)} dv = \int^t_{t_0} a(t)\,dt \\
	\implies v(t)-v(t_0)=\int^t_{t_0} a(t)\,dt \implies v(t) = v(t_0)+\int^t_{t_0} a(t)\,dt
\end{gather*}
Si ha nella soluzione una costante $v(t_0)$, che rappresenta la velocità del punto all'istante $t_0$. Per calcolare esplicitamente $v(t)$ si devono conoscere la forma analitica di $a(t)$ e la velocità iniziale $v_0$.
Possiamo ora andare a ricavare l'ascissa curvilinea:
\begin{gather*}
	v(t)=\frac{ds}{dt} \implies ds=v(t) dt \implies \int^{s(t)}_{s(t_0)} ds = \int^t_{t_0} v(t)\,dt \\
	\implies s(t)=s(t_0)+\int^t_{t_0} v(t) \,dt
\end{gather*}
Il termine $s(t_0)$ rappresenta la posizione iniziale del punto, occupata nell'istante iniziale $t_0$. Pertanto per calcolare $s(t)$, nota $v(t)$, è necessario conoscere tale posizione iniziale.

Nello studio del moto di un punto materiale, si pone particolare attenzione a due specifiche tipologie di moto, quello uniforme e quello uniformemente accelerato.

Si parla di \textbf{moto uniforme} quando la legge oraria del moto è tale per cui la velocità scalare è costante nel tempo. In questo caso $a(t)=0$. Si ottiene una funzione lineare nel tempo.
\begin{gather*}
	a(t)=0 \implies v(t)=\text{costante} \\
	s(t)=s(t_0)+ \int^t_{t_0} v(t_0) \,dt \implies s(t)=s_0 + v_0 (t - t_0)
\end{gather*}

\begin{figure}[htpb]
	\centering

	\tikzset{every picture/.style={line width=0.75pt}} %set default line width to 0.75pt

	\begin{tikzpicture}[x=0.75pt,y=0.75pt,yscale=-1,xscale=1]
	%uncomment if require: \path (0,300); %set diagram left start at 0, and has height of 300

	%Shape: Axis 2D [id:dp35300894905623537]
	\draw  (51,155) -- (259.5,155)(69.88,81) -- (69.88,229) (252.5,150) -- (259.5,155) -- (252.5,160) (64.88,88) -- (69.88,81) -- (74.88,88)  ;
	%Straight Lines [id:da6687997804192232]
	\draw [line width=1.5]    (69.88,200) -- (250.27,200) ;
	%Shape: Axis 2D [id:dp10601630570504228]
	\draw  (321,155) -- (517.11,155)(338.76,81) -- (338.76,229) (510.11,150) -- (517.11,155) -- (510.11,160) (333.76,88) -- (338.76,81) -- (343.76,88)  ;
	%Straight Lines [id:da791609347568192]
	\draw [line width=1.5]    (338.76,100) -- (508.43,200) ;

	% Text Node
	\draw (58,78) node    {$v$};
	% Text Node
	\draw (274,155) node    {$t$};
	% Text Node
	\draw (58,195) node    {$v_0$};
	% Text Node
	\draw (328,78) node    {$s$};
	% Text Node
	\draw (526.98,155) node    {$t$};
	% Text Node
	\draw (328,99) node    {$s_0$};
	% Text Node
	\draw (490,100) node   [align=left] {la pendenza (negativa)\\rappresenta la velocità};


	\end{tikzpicture}
\end{figure}

Si parla di \textbf{moto uniformemente accelerato} quando la legge oraria del moto è tale per cui l'accelerazione scalare è costante nel tempo.
\[
	a(t)=a_0 \implies v(t)=a_0 + v_0 (t-t_0)
\]

\begin{figure}[htpb]
	\centering
		

	\tikzset{every picture/.style={line width=0.75pt}} %set default line width to 0.75pt

	\begin{tikzpicture}[x=0.75pt,y=0.75pt,yscale=-1,xscale=1]
	%uncomment if require: \path (0,300); %set diagram left start at 0, and has height of 300

	%Shape: Axis 2D [id:dp1770714194583043]
	\draw  (51,155) -- (259.5,155)(69.88,81) -- (69.88,229) (252.5,150) -- (259.5,155) -- (252.5,160) (64.88,88) -- (69.88,81) -- (74.88,88)  ;
	%Straight Lines [id:da2762035008236656]
	\draw [line width=1.5]    (69.88,125) -- (250.27,125) ;
	%Shape: Axis 2D [id:dp3462618828150654]
	\draw  (321,155) -- (517.11,155)(338.76,81) -- (338.76,229) (510.11,150) -- (517.11,155) -- (510.11,160) (333.76,88) -- (338.76,81) -- (343.76,88)  ;
	%Straight Lines [id:da906495089646081]
	\draw [line width=1.5]    (338.76,140) -- (508.43,100) ;
	%Straight Lines [id:da5656057195303843]
	\draw [line width=0.75]  [dash pattern={on 0.84pt off 2.51pt}]  (338.76,141) -- (508.43,141) ;
	%Shape: Arc [id:dp4803004788697469]
	\draw  [draw opacity=0] (415.99,123.34) .. controls (417.14,128.71) and (417.75,134.29) .. (417.75,140) .. controls (417.75,140.45) and (417.75,140.9) .. (417.74,141.35) -- (338.76,140) -- cycle ; \draw   (415.99,123.34) .. controls (417.14,128.71) and (417.75,134.29) .. (417.75,140) .. controls (417.75,140.45) and (417.75,140.9) .. (417.74,141.35) ;

	% Text Node
	\draw (58,78) node    {$a$};
	% Text Node
	\draw (274,155) node    {$t$};
	% Text Node
	\draw (58,120) node    {$a_0$};
	% Text Node
	\draw (328,78) node    {$v$};
	% Text Node
	\draw (526.98,155) node    {$t$};
	% Text Node
	\draw (328,135.5) node    {$v_0$};
	% Text Node
	\draw (409.5,85) node    {$\tan \vartheta =a_0$};
	% Text Node
	\draw (447.5,127) node    {$\vartheta $};


	\end{tikzpicture}
\end{figure}

Si vede come l'accelerazione abbia l'effetto di aumentare linearmente la velocità.
\[
	\implies s(t)=s_0+ \int^t_{t_0} v(t)\,dt = \int^t_{t_0} a_0 + v_0 (t-t_0)\,dt
\]
Si trova allora:
\begin{equation}
	\boxed{s(t)=s_0+v_0 t+ \frac{1}{2}a_0 t^2}
\end{equation}
La legge oraria aumenta quadraticamente nel tempo.

Riportiamo in sintesi le leggi caratteristiche del moto uniformemente accelerato:
\[
	\begin{cases}
		a(t)=a_0 \\
		v(t) = v(t_0)+\int^t_{t_0} a(t)\,dt \\
		s(t)=s_0+v_0 t+ \frac{1}{2}a_0 t^2
	\end{cases}
\]

\begin{figure}[htpb]
	\centering
		

	\tikzset{every picture/.style={line width=0.75pt}} %set default line width to 0.75pt

	\begin{tikzpicture}[x=0.75pt,y=0.75pt,yscale=-1,xscale=1]
	%uncomment if require: \path (0,300); %set diagram left start at 0, and has height of 300

	%Curve Lines [id:da6971889067942005]
	\draw [line width=1.5]    (126.5,75) .. controls (231.5,295) and (313.5,298) .. (418.5,76) ;
	%Shape: Axis 2D [id:dp7960519414735971]
	\draw  (75.5,183) -- (503.5,183)(272.5,33) -- (272.5,270) (496.5,178) -- (503.5,183) -- (496.5,188) (267.5,40) -- (272.5,33) -- (277.5,40)  ;

	% Text Node
	\draw (259,33) node    {$s$};
	% Text Node
	\draw (518,183) node    {$t$};


	\end{tikzpicture}
\end{figure}







































\section{Trattazione vettoriale della cinematica fisica}

I concetti di traiettoria e di legge oraria che la trattazione scalare separa, sono invece combinati in quella che prende il nome di cinematica vettoriale. Dal momento che la traiettoria di un punto in generale è una linea curva, la direzione e il verso dello spostamento sono informazioni istantanee e variano generalmente in ogni punto di essa. Non è quindi sufficiente specificare il valore numerico dello spostamento, ma occorre precisare in quale direzione e verso esso stia avvenendo. Ecco perché la trattazione vettoriale individua la posizione del punto materiale tramite un vettore posizione. Esso è definito univocamente dalle tre coordinate $x(t), y(t), z(t)$. Infatti si ha: $\vec{r}(t)=\vec{x}(t)+\vec{y}(t)+\vec{z}(t)$. Conoscere come evolve la traiettoria nello spazio equivale a conoscere le tre leggi scalari $x=x(t), y=y(t), z=z(t)$.

\begin{figure}[htpb]
	\centering
		

	\tikzset{every picture/.style={line width=0.75pt}} %set default line width to 0.75pt

	\begin{tikzpicture}[x=0.75pt,y=0.75pt,yscale=-1,xscale=1]
	%uncomment if require: \path (0,368); %set diagram left start at 0, and has height of 368

	%Straight Lines [id:da05709963918727534]
	\draw    (191,206) -- (392.5,206) ;
	\draw [shift={(395.5,206)}, rotate = 180] [fill={rgb, 255:red, 0; green, 0; blue, 0 }  ][line width=0.08]  [draw opacity=0] (10.72,-5.15) -- (0,0) -- (10.72,5.15) -- (7.12,0) -- cycle    ;
	%Straight Lines [id:da6045238370319512]
	\draw [line width=1.5]    (191,206) -- (293.24,133.32) ;
	\draw [shift={(296.5,131)}, rotate = 504.59] [fill={rgb, 255:red, 0; green, 0; blue, 0 }  ][line width=0.08]  [draw opacity=0] (13.4,-6.43) -- (0,0) -- (13.4,6.44) -- (8.9,0) -- cycle    ;
	%Straight Lines [id:da8043019054159646]
	\draw    (191,206) -- (191,90) ;
	\draw [shift={(191,87)}, rotate = 450] [fill={rgb, 255:red, 0; green, 0; blue, 0 }  ][line width=0.08]  [draw opacity=0] (10.72,-5.15) -- (0,0) -- (10.72,5.15) -- (7.12,0) -- cycle    ;
	%Straight Lines [id:da1934286126219129]
	\draw    (191,206) -- (138.37,258.63) ;
	\draw [shift={(136.25,260.75)}, rotate = 315] [fill={rgb, 255:red, 0; green, 0; blue, 0 }  ][line width=0.08]  [draw opacity=0] (10.72,-5.15) -- (0,0) -- (10.72,5.15) -- (7.12,0) -- cycle    ;
	%Straight Lines [id:da9724944386541952]
	\draw  [dash pattern={on 0.84pt off 2.51pt}]  (296.5,244) -- (296.5,131) ;
	%Straight Lines [id:da9950812379998912]
	\draw  [dash pattern={on 0.84pt off 2.51pt}]  (296.5,244) -- (334,206.5) ;
	%Straight Lines [id:da05573249238067435]
	\draw  [dash pattern={on 0.84pt off 2.51pt}]  (154.5,244) -- (296.5,244) ;
	%Shape: Circle [id:dp5872605017768655]
	\draw  [fill={rgb, 255:red, 0; green, 0; blue, 0 }  ,fill opacity=1 ] (189,206) .. controls (189,204.9) and (189.9,204) .. (191,204) .. controls (192.1,204) and (193,204.9) .. (193,206) .. controls (193,207.1) and (192.1,208) .. (191,208) .. controls (189.9,208) and (189,207.1) .. (189,206) -- cycle ;
	%Straight Lines [id:da62906602305837]
	\draw  [dash pattern={on 0.84pt off 2.51pt}]  (296.5,244) -- (191,208) ;
	%Shape: Arc [id:dp9794837241454388]
	\draw  [draw opacity=0] (190.69,170.75) .. controls (190.79,170.75) and (190.9,170.75) .. (191,170.75) .. controls (202.77,170.75) and (213.2,176.52) .. (219.6,185.39) -- (191,206) -- cycle ; \draw   (190.69,170.75) .. controls (190.79,170.75) and (190.9,170.75) .. (191,170.75) .. controls (202.77,170.75) and (213.2,176.52) .. (219.6,185.39) ;
	%Shape: Arc [id:dp49268886952965163]
	\draw  [draw opacity=0] (218.01,217.02) .. controls (213.66,227.67) and (203.21,235.17) .. (191,235.17) .. controls (183,235.17) and (175.75,231.94) .. (170.48,226.73) -- (191,206) -- cycle ; \draw   (218.01,217.02) .. controls (213.66,227.67) and (203.21,235.17) .. (191,235.17) .. controls (183,235.17) and (175.75,231.94) .. (170.48,226.73) ;
	%Shape: Arc [id:dp8952397992417833]
	\draw  [draw opacity=0] (227.83,206.57) .. controls (227.76,211.33) and (226.78,215.87) .. (225.07,220.03) -- (191,206) -- cycle ; \draw   (227.83,206.57) .. controls (227.76,211.33) and (226.78,215.87) .. (225.07,220.03) ;

	% Text Node
	\draw (253.5,136.5) node    {$\vec{r}( t)$};
	% Text Node
	\draw (127,254.5) node    {$x$};
	% Text Node
	\draw (180,87) node    {$z$};
	% Text Node
	\draw (211,160.17) node    {$\gamma $};
	% Text Node
	\draw (223.17,233) node    {$\alpha $};
	% Text Node
	\draw (261.33,215.33) node    {$\beta $};
	% Text Node
	\draw (408,205) node    {$y$};
	% Text Node
	\draw (314,122.67) node    {$p( t)$};


	\end{tikzpicture}
\end{figure}

\paragraph{Esempio:}
\[
	\vec{r}(t)=
		\begin{cases}
			x(t)=t^2+1 \\
			y(t)=4t \\
			z(t)=2
		\end{cases}
\]
Si isola $t$, eliminando la dipendenza da esso:
\[
	\begin{cases}
		t=\frac{y}{4} \\
		x=\frac{y^2}{16}+1 \\
		z=2
	\end{cases}
\]
Una volta individuata la traiettoria, bisogna capire come la posizione evolve nel tempo. A tal fine, si introduce un \textbf{vettore velocità} che è definito come la derivata del vettore posizione nel tempo.
\[
	\vec{v}(t)=\frac{d\vec{s}}{dt}
\]
Fare la derivata di un vettore significa derivare le tre funzioni scalari $x(t), y(t), z(t)$, che lo definiscono.	
\[
	\vec{v}(t)
		\begin{cases}
			v_x(t)=\frac{dx}{dt}=2t \\
			v_y(t)=\frac{dy}{dt}=4 \\
			v_z(t)=\frac{dz}{dt}=0
		\end{cases}
\]
La velocità scalare introdotta nell'ambito della trattazione scalare non è altro che la lunghezza di questo vettore velocità.
\begin{gather*}
	\norma{\vec{v}(t)}=\sqrt{v_x^2+v_y^2+v_z^2}=\sqrt{4t^2+16} \\
	s(t)=\underbrace{s(t_0)}_{=0}+\int^t_{t_0} 2\sqrt{t^2+4}\,dt
\end{gather*}
Mentre in precedenza è stata introdotta l'ascissa curvilinea per descrivere lo spostamento del punto lungo la traiettoria, la grandezza vettoriale analoga che viene ivi definita prende il nome di \textbf{raggio vettore}, o \textbf{vettore spostamento}. La sua direzione coincide con la corda che congiunge due punti considerati sulla traiettoria. Bisogna prestare molta attenzione a non confondere $\Delta\vec{r}$ con lo spazio effettivamente percorso lungo la curva, due concetti ben diversi.

\begin{figure}[htpb]
	\centering
		

	\tikzset{every picture/.style={line width=0.75pt}} %set default line width to 0.75pt

	\begin{tikzpicture}[x=0.75pt,y=0.75pt,yscale=-1,xscale=1]
	%uncomment if require: \path (0,300); %set diagram left start at 0, and has height of 300

	%Straight Lines [id:da4221153943908629]
	\draw    (212,171) -- (413.5,171) ;
	\draw [shift={(416.5,171)}, rotate = 180] [fill={rgb, 255:red, 0; green, 0; blue, 0 }  ][line width=0.08]  [draw opacity=0] (10.72,-5.15) -- (0,0) -- (10.72,5.15) -- (7.12,0) -- cycle    ;
	%Straight Lines [id:da6791623240738924]
	\draw [line width=1.5]    (212,171) -- (255.86,73.65) ;
	\draw [shift={(257.5,70)}, rotate = 474.25] [fill={rgb, 255:red, 0; green, 0; blue, 0 }  ][line width=0.08]  [draw opacity=0] (13.4,-6.43) -- (0,0) -- (13.4,6.44) -- (8.9,0) -- cycle    ;
	%Straight Lines [id:da14897452664630717]
	\draw    (212,171) -- (212,55) ;
	\draw [shift={(212,52)}, rotate = 450] [fill={rgb, 255:red, 0; green, 0; blue, 0 }  ][line width=0.08]  [draw opacity=0] (10.72,-5.15) -- (0,0) -- (10.72,5.15) -- (7.12,0) -- cycle    ;
	%Straight Lines [id:da5843316458197236]
	\draw    (212,171) -- (179.37,203.63) ;
	\draw [shift={(177.25,205.75)}, rotate = 315] [fill={rgb, 255:red, 0; green, 0; blue, 0 }  ][line width=0.08]  [draw opacity=0] (10.72,-5.15) -- (0,0) -- (10.72,5.15) -- (7.12,0) -- cycle    ;
	%Shape: Circle [id:dp3769141587673239]
	\draw  [fill={rgb, 255:red, 0; green, 0; blue, 0 }  ,fill opacity=1 ] (210,171) .. controls (210,169.9) and (210.9,169) .. (212,169) .. controls (213.1,169) and (214,169.9) .. (214,171) .. controls (214,172.1) and (213.1,173) .. (212,173) .. controls (210.9,173) and (210,172.1) .. (210,171) -- cycle ;
	%Curve Lines [id:da20792216723267476]
	\draw    (229.5,87) .. controls (269.5,57) and (294,50) .. (317.5,96) .. controls (341,142) and (392.5,141) .. (417.5,116) ;
	%Straight Lines [id:da3898407246017126]
	\draw [line width=1.5]    (212,171) -- (326.87,117.68) ;
	\draw [shift={(330.5,116)}, rotate = 515.1] [fill={rgb, 255:red, 0; green, 0; blue, 0 }  ][line width=0.08]  [draw opacity=0] (13.4,-6.43) -- (0,0) -- (13.4,6.44) -- (8.9,0) -- cycle    ;
	%Straight Lines [id:da62028222338974]
	\draw [color={rgb, 255:red, 74; green, 74; blue, 74 }  ,draw opacity=1 ][line width=0.75]    (257.5,70) -- (327.96,114.4) ;
	\draw [shift={(330.5,116)}, rotate = 212.22] [fill={rgb, 255:red, 74; green, 74; blue, 74 }  ,fill opacity=1 ][line width=0.08]  [draw opacity=0] (10.72,-5.15) -- (0,0) -- (10.72,5.15) -- (7.12,0) -- cycle    ;

	% Text Node
	\draw (253,127) node    {$\vec{r}( t_1)$};
	% Text Node
	\draw (170,200.5) node    {$x$};
	% Text Node
	\draw (201,52) node    {$z$};
	% Text Node
	\draw (429,170) node    {$y$};
	% Text Node
	\draw (248,52.67) node    {$p( t_1)$};
	% Text Node
	\draw (353,103.17) node    {$p( t_2)$};
	% Text Node
	\draw (295.5,151.5) node    {$\vec{r}( t_2)$};
	% Text Node
	\draw (280.5,101) node    {$\Delta \vec{r}$};


	\end{tikzpicture}
\end{figure}

Si va ora a definire la \textbf{velocità media vettoriale} come:
\[
	\vec{v}_{\text{media}}=\frac{\Delta\vec{r}}{\Delta t}
\]
$\vec{v}_{\text{media}}$ avrà stessa direzione e verso di $\Delta\vec{r}$, l'intervallo di tempo infatti è uno scalare sempre positivo.
\[
	\vec{v}_{\text{media}}\implies \begin{cases} v_{\text{media} \,x} =\frac{\Delta x}{\Delta t} \\ v_{\text{media} \,y} =\frac{\Delta y}{\Delta t} \\ v_{\text{media} \,z} =\frac{\Delta z}{\Delta t} \end{cases}
\]
Per definire la \textbf{velocità vettoriale istantanea}, dobbiamo far tendere a $0$ l'intervallo di tempo, passaggio che equivale a ricondursi al limite del rapporto incrementale:
\[
	\vec{v}(t)=\lim_{\Delta t \to 0} \frac{\Delta \vec{r}}{\Delta t}=\frac{d\vec{r}}{dt}=\vec{r'}(t)
\]
Conoscere la velocità istantanea vettoriale significa conoscere tre funzioni che informano su come varia la velocità lungo gli assi $x, y, z$. Quando $t \to t_0$ il vettore $\Delta\vec{r}$ tende in modulo a diventare esattamente pari allo spazio percorso sulla traiettoria. In particolare l'incremento $d\vec{r}$ del raggio vettore risulta in direzione tangente alla traiettoria, per cui possiamo scrivere
\begin{equation}
	\label{velocita}
	\boxed{\vec{v}(t)=\frac{d\vec{r}}{dt}=\frac{ds}{dt}\,\vec{u}_t}
\end{equation}
dove $\vec{u}_t$ è il versore della tangente alla curva. $\vec{v}_\text{istantanea}$ infatti ha lo stesso valore della velocità scalare e stesso verso e direzione del versore intrinseco $\vec{u}_t$.
Questi ultimi due fattori sono caratteristiche intrinseche alla traiettoria, che non dipendono ovvero dalla scelta del sistema di riferimento. Ecco perché tale scrittura ~\eqref{velocita} è nota come \textit{scomposizione del vettore velocità in componenti intrinseche alla traiettoria}.
Si può spostare l'origine $O$ in un'altra posizione, si possono ruotare gli assi, ma la direzione, il verso, il modulo della velocità e la curva restano gli stessi. Si parla di invarianza delle relazioni vettoriali rispetto alla scelta del sistema di riferimento.

Definito il vettore velocità è possibile poi andare a definire un vettore accelerazione che, come al solito, sarà:
\[
	\vec{a}_{\text{media}}=\frac{\vec{v}_{t_1}-\vec{v}_{t_0}}{t_1-t_0}
\]
Riducendo gli intervalli di tempo, si otterrà la derivata del vettore velocità, che corrisponde al vettore accelerazione al generico istante $t$:
\[
	\vec{a}_{\text{ist}}=\frac{d\vec{v}}{dt} \implies
		\begin{cases}
		a_x=\frac{dv_x}{dt}=\frac{d^2s_x}{dt^2} \\
		a_y=\frac{dv_y}{dt}=\frac{d^2s_y}{dt^2}\\
		a_z=\frac{dv_z}{dt}=\frac{d^2s_z}{dt^2}
		\end{cases}
\]
$\vec{a}(t)$ è scomponibile in componenti cartesiane, che informano su come varia nel tempo le componenti in $x, y, z$ dell'accelerazione.

Si noti come fondamentalmente l'approccio della cinematica vettoriale sia quello di andare a studiare tre leggi scalari: il problema inverso si ripete tre volte. La trattazione ha un difetto perché nasconde il comportamento effettivo del moto del punto materiale. Dare una descrizione di questo tipo non permette infatti di visualizzare direttamente come esso si muove: combinare i tre moti infatti non è così semplice.

\subsection{Rappresentazione del vettore accelerazione in componenti intrinseche alla traiettoria}

Mentre il vettore velocità ha direzione tangente alla traiettoria e può quindi sempre essere espresso come:
\[
	\vec{v}(t)=\frac{ds}{dt}\,\vec{u}_t=v(t)\,\vec{u}_t \quad \text{Espressione intrinseca del vettore velocità}
\]
Il vettore accelerazione ammette una scomposizione in due componenti intrinseche alla traiettoria.

\begin{figure}[htpb]
	\centering
		

	\tikzset{every picture/.style={line width=0.75pt}} %set default line width to 0.75pt

	\begin{tikzpicture}[x=0.75pt,y=0.75pt,yscale=-1,xscale=1]
	%uncomment if require: \path (0,300); %set diagram left start at 0, and has height of 300

	%Shape: Axis 2D [id:dp3400766029212272]
	\draw  (82,197.45) -- (348.5,197.45)(108.65,53) -- (108.65,213.5) (341.5,192.45) -- (348.5,197.45) -- (341.5,202.45) (103.65,60) -- (108.65,53) -- (113.65,60)  ;
	%Curve Lines [id:da5974370210452586]
	\draw    (146,116) .. controls (221.5,17) and (243.5,242) .. (302.5,101) ;
	%Straight Lines [id:da8529673806643632]
	\draw [line width=1.5]    (181,90) -- (233.5,89.07) ;
	\draw [shift={(237.5,89)}, rotate = 538.99] [fill={rgb, 255:red, 0; green, 0; blue, 0 }  ][line width=0.08]  [draw opacity=0] (13.4,-6.43) -- (0,0) -- (13.4,6.44) -- (8.9,0) -- cycle    ;
	%Shape: Circle [id:dp6915133943146423]
	\draw  [fill={rgb, 255:red, 0; green, 0; blue, 0 }  ,fill opacity=1 ] (177.5,90) .. controls (177.5,88.62) and (178.62,87.5) .. (180,87.5) .. controls (181.38,87.5) and (182.5,88.62) .. (182.5,90) .. controls (182.5,91.38) and (181.38,92.5) .. (180,92.5) .. controls (178.62,92.5) and (177.5,91.38) .. (177.5,90) -- cycle ;
	%Straight Lines [id:da3714954615127908]
	\draw [line width=1.5]    (180,90) -- (180,121) ;
	\draw [shift={(180,125)}, rotate = 270] [fill={rgb, 255:red, 0; green, 0; blue, 0 }  ][line width=0.08]  [draw opacity=0] (13.4,-6.43) -- (0,0) -- (13.4,6.44) -- (8.9,0) -- cycle    ;
	%Straight Lines [id:da6433144150732559]
	\draw [line width=1.5]    (273.5,146) -- (257.56,114.57) ;
	\draw [shift={(255.75,111)}, rotate = 423.11] [fill={rgb, 255:red, 0; green, 0; blue, 0 }  ][line width=0.08]  [draw opacity=0] (13.4,-6.43) -- (0,0) -- (13.4,6.44) -- (8.9,0) -- cycle    ;
	%Straight Lines [id:da8505210239426837]
	\draw [line width=1.5]    (273.5,146) -- (322.37,115.14) ;
	\draw [shift={(325.75,113)}, rotate = 507.72] [fill={rgb, 255:red, 0; green, 0; blue, 0 }  ][line width=0.08]  [draw opacity=0] (13.4,-6.43) -- (0,0) -- (13.4,6.44) -- (8.9,0) -- cycle    ;
	%Shape: Circle [id:dp8706507040610654]
	\draw  [fill={rgb, 255:red, 0; green, 0; blue, 0 }  ,fill opacity=1 ] (271,146) .. controls (271,144.62) and (272.12,143.5) .. (273.5,143.5) .. controls (274.88,143.5) and (276,144.62) .. (276,146) .. controls (276,147.38) and (274.88,148.5) .. (273.5,148.5) .. controls (272.12,148.5) and (271,147.38) .. (271,146) -- cycle ;

	% Text Node
	\draw (175.5,135) node    {$\vec{u}_n$};
	% Text Node
	\draw (207,70.5) node    {$\vec{u}_t$};
	% Text Node
	\draw (341,124.5) node    {$\vec{u}_t$};
	% Text Node
	\draw (273,102.5) node    {$\vec{u}_n$};


	\end{tikzpicture}
\end{figure}

Data una traiettoria qualunque, è possibile definire in ogni suo punto due versori particolari: quello tangente, $\vec{u_t}$, con verso concorde a $d\vec{r}$, e quello normale (o ortogonale), $\vec{u_n}$. Esso sarà perpendicolare al vettore tangente e punterà verso l'interno della concavità della traiettoria. Questi due versori individuano un piano che prende il nome di \textbf{piano osculatore}. Essi ovviamente non mantengono direzione, ad eccezione del \emph{moto rettilineo}.
Per ottenere la scomposizione dell'accelerazione nelle due componenti, la si ridefinisce come:
\begin{equation}
	\label{eqn:scomposizione}
	\boxed{\vec{a}(t)=\frac{d\vec{v}}{dt}=\frac{d(\vec{u}_t\,v)}{dt}=\vec{u}_t\,\frac{dv}{dt}+v\,\frac{d\vec{u}_t}{dt}}
\end{equation}
II primo membro di ~\eqref{eqn:scomposizione} fornisce un'informazione legata alla variazione della velocità scalare (alla quale infatti viene applicato l'operatore derivata) ed esiste quindi solo se essa varia. Ha direzione tangente alla traiettoria e per tale motivo prende il nome di \textbf{accelerazione tangenziale}. È evidente che non è pari al modulo dell'accelerazione vettoriale per via del secondo componente comparso derivando il prodotto. In esso l'operatore derivata è applicato al versore tangente $\vec{u}_t$. Il fatto che stia variando nel tempo è indice di un cambiamento di direzione sulla traiettoria. Ogni volta che quest'ultima è curvilinea esiste il secondo termine nella somma. Calcoliamolo in maniera più esplicita.

\paragraph{Derivata di un versore} Si può dimostrare che la derivata di un versore dà luogo a un vettore che ha direzione ortogonale al versore di partenza.
\[
	\vec{u}_t\cdot\vec{u}_t=1 \implies \frac{d\vec{u}_t}{dt}\cdot \vec{u}_t+ \vec{u}_t\cdot \frac{d\vec{u}_t}{dt}=0
\]
Il prodotto scalare gode della proprietà commutativa, perciò i due termini ottenuti tramite la derivata del prodotto sono uguali. Segue allora che:
\[
	2\vec{u}_t\cdot\frac{d\vec{u}_t}{dt}=0 \implies \vec{u}_t\perp \frac{d\vec{u}_t}{dt} \implies \frac{d\vec{u}_t}{dt} \parallel \vec{u}_n
\]
La variazione angolare $d\vartheta$ si può ricondurre alle grandezze cinematiche che sono state introdotte, essendo legata alla variazione infinitesima di ascissa curvilinea:
\[
	ds=\rho\,d\vartheta
\]
In particolare, l'angolo $d\vartheta$ individuato dai due versori, è equivalente all'angolo individuato nel cerchio di raggio unitario rappresentato in figura.
\[
	\norma{d\vec{u}_t}=1\,d\vartheta
\]
Fatta questa premessa, è possibile affermare che:
\[
	\frac{d\vec{u}_t}{dt}=\frac{d\vartheta}{dt}\,\vec{u}_n
\]
e, di conseguenza:
\[
	v\,\frac{d\vec{u}_t}{dt}= v\,\frac{d\vartheta}{dt}\,\vec{u}_n=\frac{v}{\rho}\,\frac{ds}{dt}\,\vec{u}_n=\frac{v^2}{\rho}\,\vec{u}_n
\]
\begin{equation}
	\boxed{\vec{a}_n=\frac{v^2}{\rho}\,\vec{u}_n}
\end{equation}
Il fatto che all'aumentare della curvatura il raggio del cerchio osculatore diminuisca, porta a definire la \textbf{curvatura} della traiettoria come il reciproco di tale raggio:
\[
	k:=\frac{1}{\rho}
\]
si ha allora:
\[
	\vec{a}_n=k\,{v^2}\,\vec{u}_n
\]

\begin{figure}[htb!]
	\vspace*{-50mm}
	\centering
	

	\tikzset{every picture/.style={line width=0.75pt}} %set default line width to 0.75pt

	\begin{tikzpicture}[x=0.75pt,y=0.75pt,yscale=-1,xscale=1]
	%uncomment if require: \path (0,225); %set diagram left start at 0, and has height of 225

	%Shape: Circle [id:dp5208500816458348]
	\draw   (127.5,128) .. controls (127.5,78.57) and (167.57,38.5) .. (217,38.5) .. controls (266.43,38.5) and (306.5,78.57) .. (306.5,128) .. controls (306.5,177.43) and (266.43,217.5) .. (217,217.5) .. controls (167.57,217.5) and (127.5,177.43) .. (127.5,128) -- cycle ;
	%Curve Lines [id:da09806334628889624]
	\draw    (67,149) .. controls (320.5,-168) and (308.5,325) .. (457.5,40) ;
	%Straight Lines [id:da6623793670478895]
	\draw  [dash pattern={on 0.84pt off 2.51pt}]  (217,38.5) -- (217,128) ;
	%Straight Lines [id:da9470005639819916]
	\draw  [dash pattern={on 0.84pt off 2.51pt}]  (278.33,61.92) -- (217,128) ;
	%Shape: Arc [id:dp21315146015323272]
	\draw  [draw opacity=0] (217.37,90.67) .. controls (226.82,90.76) and (235.43,94.36) .. (241.96,100.23) -- (217,128) -- cycle ; \draw   (217.37,90.67) .. controls (226.82,90.76) and (235.43,94.36) .. (241.96,100.23) ;
	%Straight Lines [id:da09170287984211711]
	\draw [line width=1.5]    (217,38.5) -- (290.38,38.5) ;
	\draw [shift={(294.38,38.5)}, rotate = 180] [fill={rgb, 255:red, 0; green, 0; blue, 0 }  ][line width=0.08]  [draw opacity=0] (13.4,-6.43) -- (0,0) -- (13.4,6.44) -- (8.9,0) -- cycle    ;
	%Straight Lines [id:da8134422742349374]
	\draw [line width=1.5]    (278.33,61.92) -- (325.89,104.34) ;
	\draw [shift={(328.88,107)}, rotate = 221.73] [fill={rgb, 255:red, 0; green, 0; blue, 0 }  ][line width=0.08]  [draw opacity=0] (13.4,-6.43) -- (0,0) -- (13.4,6.44) -- (8.9,0) -- cycle    ;

	% Text Node
	\draw (206.67,71) node    {$\rho $};
	% Text Node
	\draw (214.67,138.33) node    {$C$};
	% Text Node
	\draw (236,78.33) node    {$d\vartheta $};
	% Text Node
	\draw (252.5,19.33) node    {$\vec{u}_t( t_1)$};
	% Text Node
	\draw (341,69.83) node    {$\vec{u}_t( t_1 +dt)$};
	% Text Node
	\draw (375,156.83) node    {$ds\ =\rho \ d\vartheta $};


	\end{tikzpicture}
	\vspace*{-30mm}
\end{figure}

Quindi il raggio di curvatura compare al denominatore perché comunica il fatto che l'accelerazione normale aumenta tanto più la curvatura è elevata. Ovviamente il valore dell'accelerazione normale dipende anche dal quadrato della velocità. Quando la traiettoria è rettilinea, l'accelerazione normale è pari a zero e $\rho$ tende a infinito, quindi il vettore accelerazione coincide con il vettore accelerazione tangenziale.

\subsection{Esempi di moto}

Questa sezione si concentra su alcune particolari tipologie di moto che ricorrono con frequenza.

\subsubsection{Moto parabolico}

Il moto parabolico è quello caratteristico di corpi pesanti che si muovono in prossimità della superficie terrestre, soggetti alla sola accelerazione di gravità. In questo ambito si trascura l'eventuale attrito creato dall'aria.
Nella descrizione di tale moto, è preferibile utilizzare un riferimento cartesiano in cui una delle due direzioni è parallela a $\vec{g}$. Si tratta di un moto piano poiché la velocità sta sempre sul piano individuato dai vettori costanti $\vec{v}_0$ e $\vec{g}$. Dal momento che ciò che è noto è l'accelerazione lungo le due componenti ($\vec{g}$ lungo le $y$ e nulla lungo le $x$) il problema si riduce alla risoluzione di due problemi inversi.

\begin{figure}[htpb]
	\centering
		

	\tikzset{every picture/.style={line width=0.75pt}} %set default line width to 0.75pt

	\begin{tikzpicture}[x=0.75pt,y=0.75pt,yscale=-1,xscale=1]
	%uncomment if require: \path (0,300); %set diagram left start at 0, and has height of 300

	%Shape: Axis 2D [id:dp21763212568332824]
	\draw  (70,217) -- (473.5,217)(92.5,53) -- (92.5,231) (466.5,212) -- (473.5,217) -- (466.5,222) (87.5,60) -- (92.5,53) -- (97.5,60)  ;
	%Straight Lines [id:da05203031601857888]
	\draw [line width=1.5]    (92.5,217) -- (163.86,136) ;
	\draw [shift={(166.5,133)}, rotate = 491.38] [fill={rgb, 255:red, 0; green, 0; blue, 0 }  ][line width=0.08]  [draw opacity=0] (13.4,-6.43) -- (0,0) -- (13.4,6.44) -- (8.9,0) -- cycle    ;
	%Straight Lines [id:da509522474347498]
	\draw [line width=1.5]    (343.83,67.67) -- (343.83,118.33) ;
	\draw [shift={(343.83,122.33)}, rotate = 270] [fill={rgb, 255:red, 0; green, 0; blue, 0 }  ][line width=0.08]  [draw opacity=0] (13.4,-6.43) -- (0,0) -- (13.4,6.44) -- (8.9,0) -- cycle    ;
	%Shape: Arc [id:dp904536839291062]
	\draw  [draw opacity=0] (120.98,184.45) .. controls (129.89,192.25) and (135.56,203.65) .. (135.75,216.37) -- (92.5,217) -- cycle ; \draw   (120.98,184.45) .. controls (129.89,192.25) and (135.56,203.65) .. (135.75,216.37) ;
	%Curve Lines [id:da9850378824791992]
	\draw [line width=1.5]    (92.5,217) .. controls (214.5,83) and (303.5,83) .. (412.5,216) ;

	% Text Node
	\draw (80,52) node    {$y$};
	% Text Node
	\draw (82,228.67) node    {$O$};
	% Text Node
	\draw (418,231.33) node    {$G$};
	% Text Node
	\draw (480.67,228.67) node    {$x$};
	% Text Node
	\draw (120,158) node    {$\vec{v}_0$};
	% Text Node
	\draw (356,88) node    {$\vec{g}$};
	% Text Node
	\draw (144.5,196.5) node    {$\vartheta $};


	\end{tikzpicture}
	\vspace*{-10mm}
\end{figure}
\[
	\begin{cases}
		a_x=0 \\
		a_y=-g
	\end{cases}
\]
Sono necessarie le condizioni iniziali del punto:
\[
	\vec{r}(t_0=0)= \begin{cases} x(t_0)=0 \\ y(t_0)=0 \end{cases} \vec{v}(t_0)=\begin{cases} v_{0x}=v_0\cos\vartheta \\ v_{0y}=v_0\sin\vartheta \end{cases}
\]
È possibile quindi ricavare le leggi orarie dei moti proiettati, che sono:
\begin{gather*}
	\vec{v}(t) =
		\begin{cases}
			v_x(t)=v_0\cos\vartheta+\int a_0(x)\,dt=v_0\cos\vartheta \\
			v_y(t)=v_0\sin\vartheta+\int (-g)\,dt=v_0\sin\vartheta-gt
		\end{cases} \\
	\vec{r}(t) =
		\begin{cases}
			x(t)=x_0+\int v_x\,dt=v_0\cos\vartheta\,t \quad \text{moto uniforme} \\
			y(t)=y_0+\int v_y\,dt=v_0\sin\vartheta\,t-\frac{1}{2}gt^2 \quad \text{moto unif. accelerato}
		\end{cases}
\end{gather*}
Il moto parabolico è la composizione di un moto uniforme lungo l'asse orizzontale, uniformemente accelerato lungo l'asse ortogonale. La traiettoria è quella di una parabola. Calcoliamo $y(x)$ per poterla visualizzare:
\[
	t=\frac{x(t)}{v_0\cos\vartheta} \implies y(x)=\tan\vartheta x-\frac{g}{2} \,\frac{x^2}{v_0^2\,\cos^2\vartheta}
\]
La traiettoria infatti viene sempre ricavata eliminando il tempo tra $x(t)$ e $y(t)$ e ottenendo così la funzione $y(x)$, che è l'equazione di una parabola. La distanza orizzontale fra il punto di partenza e quello di arrivo è detta \textbf{gittata del lancio}. Per calcolarla si impone $y(x)=0$ e si ottiene:
\begin{gather*}
	y=v_0\sin\vartheta t-\frac{1}{2}gt^2=0 \implies v_0\sin\vartheta-\frac{1}{2}gt=0 \implies t=\frac{2v_0\sin\vartheta}{g} \\
	x_G=v_{0x}t=\underbrace{v_0\cos\vartheta}_{v_{0x}}\,\underbrace{\frac{2v_0\sin\vartheta}{g}}_{t}=2x_M \quad \text{per la simmetria della parabola}
\end{gather*}
Conoscendo $x_M$ è anche possibile calcolare l'ascissa del punto di massima altezza raggiunta, che è, pertanto:
\begin{align*}
	y(x_M)&=\tan\vartheta\,\frac{v_0^2\cos\vartheta \sin\vartheta}{g}-\frac{g}{2}\,\left(\frac{v_0^2\cos\vartheta \sin\vartheta}{g}\right)^2\,\frac{1}{v_0^2\cos^2\vartheta} \\
	y(x_M)&=\frac{v_0^2 \sin^2\vartheta}{g}-\frac{v_0^2}{2g}\,\sin^2\vartheta=\frac{v_0^2\sin^2\vartheta}{2g}
\end{align*}
L'altezza massima raggiunta nel moto parabolico può essere calcolata anche in altri modi, ad esempio imponendo $y'(x)=0$, oppure considerando il fatto che la componente verticale della velocità è nulla al vertice della parabola e che quindi in tale istante $v_x(t)=0$. Il tempo totale di volo è pari al tempo impiegato a percorre $OG$ con velocità costante $v(x)=v(t_0 )\cos\vartheta$. Evidentemente questo tempo è pari al tempo necessario per salire all'altezza $y_{\text{max}}$ e ritornare al suolo. Notiamo infine che nella posizione $G$ la velocità è la stessa in modulo che alla partenza, ma è posta simmetricamente rispetto all'asse $x$. Si noti come le caratteristiche geometriche del moto parabolico di un corpo vicino alla superficie terrestre possano effettivamente essere comprese in modo semplice nel sistema cartesiano adottato, che è in definitiva il più naturale per questo problema in cui vi è una direzione di particolare importanza, quella di $\vec{g}$, a 90 gradi con una direzione di uso pratico molto comune, quella orizzontale.

\subsubsection{Moto rettilineo}

Il moto rettilineo si svolge lungo una retta sulla quale vengono fissati arbitrariamente un'origine e un verso (asse $x$). Il moto del punto è descrivibile tramite una sola coordinata $x(t)$.

\subsubsection{Moto circolare}

Si chiama moto circolare un moto piano la cui traiettoria è rappresentata da una circonferenza.\\

\begin{figure}[htpb]
	\centering
	\tikzset{every picture/.style={line width=0.75pt}} %set default line width to 0.75pt

	\begin{tikzpicture}[x=0.75pt,y=0.75pt,yscale=-1,xscale=1]
	%uncomment if require: \path (0,300); %set diagram left start at 0, and has height of 300

	%Shape: Axis 2D [id:dp6452357499911294]
	\draw  (50,161) -- (335.5,161)(192.5,64) -- (192.5,255) (328.5,156) -- (335.5,161) -- (328.5,166) (187.5,71) -- (192.5,64) -- (197.5,71)  ;
	%Shape: Circle [id:dp4813437894170338]
	\draw  [dash pattern={on 0.84pt off 2.51pt}] (114.5,161) .. controls (114.5,117.92) and (149.42,83) .. (192.5,83) .. controls (235.58,83) and (270.5,117.92) .. (270.5,161) .. controls (270.5,204.08) and (235.58,239) .. (192.5,239) .. controls (149.42,239) and (114.5,204.08) .. (114.5,161) -- cycle ;
	%Shape: Arc [id:dp7325164530727912]
	\draw  [draw opacity=0][line width=1.5]  (219.45,87.6) .. controls (249.15,98.51) and (270.4,126.94) .. (270.66,160.37) -- (192.5,161) -- cycle ; \draw  [line width=1.5]  (219.45,87.6) .. controls (249.15,98.51) and (270.4,126.94) .. (270.66,160.37) ;
	%Shape: Circle [id:dp743830313409608]
	\draw  [fill={rgb, 255:red, 0; green, 0; blue, 0 }  ,fill opacity=1 ] (218,88) .. controls (218,86.62) and (219.12,85.5) .. (220.5,85.5) .. controls (221.88,85.5) and (223,86.62) .. (223,88) .. controls (223,89.38) and (221.88,90.5) .. (220.5,90.5) .. controls (219.12,90.5) and (218,89.38) .. (218,88) -- cycle ;
	%Straight Lines [id:da3632039280212016]
	\draw    (219.45,87.6) -- (192.5,161) ;

	% Text Node
	\draw (284.67,148.33) node    {$\Omega $};
	% Text Node
	\draw (274.67,104.33) node    {$s( t)$};
	% Text Node
	\draw (236,71.67) node    {$p( t)$};
	% Text Node
	\draw (214.67,128.33) node    {$R$};


	\end{tikzpicture}
\end{figure}

Il moto circolare può essere descritto facendo riferimento allo spazio percorso sulla circonferenza $s(t)$ oppure utilizzando l'angolo $\vartheta(t)$ sotteso dall'arco $s(t)$, con $\vartheta (t)=\frac{s(t)}{R}$. L'assumere come variabile l'angolo $\vartheta(t)$ significa in pratica porsi in un sistema di coordinate polari di centro $O$ in cui il moto avviene con raggio di curvatura $\norma{\vec{r}(t)}=R=\text{costante}$ e $\vartheta (t)$ variabile. Si è naturalmente interessati alle variazioni dell'angolo nel tempo e pertanto definiamo la \textbf{velocità angolare} come la derivata dell'angolo rispetto al tempo:
\begin{gather*}
	\omega=\frac{d\vartheta}{dt}=\frac{ds}{dt}\,\frac{1}{R}=\frac{v}{R}\\
	1 \text{ rad}=\frac{s}{R} \qquad \text{angolo quando $s=R$}
\end{gather*}
Risulta che la velocità angolare è proporzionale alla velocità con cui è descritta la circonferenza, se $v$ è variabile, anche $\omega$ lo è.
\[
	\Gamma=
	\begin{cases}
		(x-x_0)^2+(y-y_0)^2=R^2 \\
		z=z_0
	\end{cases}
\]
Il moto circolare più semplice è quello \emph{uniforme}, $v$ e $\omega$ sono costanti e le leggi orarie, con riferimento alle due variabili utilizzate, si scrivono:
\[
	s(t)=s_0+vt \quad \vartheta(t)=\vartheta_0+\int^t_{t_0} \omega\,dt
\]
Il moto circolare uniforme è un moto accelerato con accelerazione costante, ortogonale alla traiettoria:
\[
	a=a_n=\frac{v^2}{R}=\omega^2 R
\]
Nel caso del moto circolare \emph{non uniforme}, oltre all'accelerazione centripeta, che è variabile perché la velocità varia in modulo, bisogna considerare anche l'accelerazione tangenziale $a = \frac{dv}{dt}$. Siccome è variabile $\omega$, si definisce l'\textbf{accelerazione angolare}:
\begin{gather*}
	\alpha(t)=\frac{d\omega}{dt}=\frac{dv}{dt}\,\frac{1}{R}= \frac{a(t)}{R} \\
	d\omega=\alpha(t) \,dt \implies \int^{\omega(t)}_{\omega(t_0)} d\omega=\int^t_{t_0} \alpha(t)\,dt \\
	\omega(t)=\omega(t_0)+\int^t_{t_0}\alpha(t)\,dt \\
	d\vartheta=\omega\,dt \implies \int^{\vartheta(t)}_{\vartheta(t_0)} d\vartheta =\int^t_{t_0}\omega\,dt \\
\end{gather*}
\begin{equation}
	\boxed{\vartheta(t)=\vartheta(t_0)+\omega_0(t-t_0)+\frac{1}{2}a_0(t-t_0)^2}
\end{equation}

\subsubsection{Moto armonico}

La descrizione di tale moto è affrontata in dettaglio nel capitolo 5.







































\section{Cinematica relativa}

Quando il sistema di riferimento in generale si muove di moto qualunque, è necessario sottoporre le leggi affrontate finora ad alcune modifiche. Sperimentalmente è provato che le leggi fisiche non dipendono dalla scelta del sistema di riferimento. Fissato questo e stabilita una certa proprietà, essa resta vera anche se cambiano l'origine e l'orientamento degli assi coordinati, ovvero se ci si riferisce ad un altro sistema ottenuto con una rotazione, una traslazione o con un'operazione combinata. Non esiste pertanto un punto di riferimento privilegiato dello spazio e nemmeno un'orientazione privilegiata: lo spazio appare omogeneo e isotropo. La caratteristica sostanziale di invarianza acquista un aspetto formale se le leggi fisiche vengono espresse come relazione tra entità che godono anch'esse delle suddette proprietà di invarianza, come le grandezze scalari o quelle vettoriali.

La situazione si presenta diversa quando un fenomeno viene osservato da due sistemi di riferimento in moto l'uno rispetto all'altro, poiché in tal caso non sussiste invarianza delle leggi fisiche e lo spostamento viene descritto in modi differenti.

L'osservatore finora considerato è sempre stato posto in un \emph{sistema di riferimento assoluto}, dato cioè da un'origine fissa nello spazio e da tre direzioni preferenziali. Tuttavia si possono presentare situazioni in cui l'osservatore è in movimento. Si definisce in tali casi un sistema di riferimento mobile che si identifica sempre come un sistema di riferimento cartesiano. Esso tuttavia può:
\begin{itemize}
	\item \emph{traslare} nello spazio con una velocità di traslazione. Si ricordi che una traslazione nello spazio è un movimento rigido attraverso il quale i tre versori che identificano il sistema di riferimento non ruotano mai (non è necessariamente un moto lungo la traiettoria rettilinea).
	\item \emph{ruotare} nello spazio con una certa velocità angolare.
\end{itemize}
L'obbiettivo della cinematica relativa è quello di studiare il moto del punto materiale rispetto all'osservatore fisso e quello mobile e trovare delle relazioni che permettono di passare facilmente dall'una all'altra descrizione del fenomeno.

\begin{figure}[htpb]
	\centering
	
	\tikzset{every picture/.style={line width=0.75pt}} %set default line width to 0.75pt

	\begin{tikzpicture}[x=0.75pt,y=0.75pt,yscale=-1,xscale=1]
	%uncomment if require: \path (0,300); %set diagram left start at 0, and has height of 300

	%Straight Lines [id:da9471165549419143]
	\draw  [dash pattern={on 0.84pt off 2.51pt}]  (214,77) -- (214,206) ;
	%Shape: Ellipse [id:dp6549012336948079]
	\draw  [draw opacity=0][fill={rgb, 255:red, 184; green, 184; blue, 184 }  ,fill opacity=1 ] (150.5,153) .. controls (150.5,132.96) and (178.93,116.71) .. (214,116.71) .. controls (249.07,116.71) and (277.5,132.96) .. (277.5,153) .. controls (277.5,173.04) and (249.07,189.29) .. (214,189.29) .. controls (178.93,189.29) and (150.5,173.04) .. (150.5,153) -- cycle ;
	%Straight Lines [id:da2622224811431515]
	\draw    (214,93) -- (214,153) ;
	\draw [shift={(214,90)}, rotate = 90] [fill={rgb, 255:red, 0; green, 0; blue, 0 }  ][line width=0.08]  [draw opacity=0] (10.72,-5.15) -- (0,0) -- (10.72,5.15) -- (7.12,0) -- cycle    ;
	%Straight Lines [id:da23746773885354222]
	\draw  [dash pattern={on 0.84pt off 2.51pt}]  (343.5,113) -- (214,153) ;
	%Straight Lines [id:da5943734544225967]
	\draw    (241.36,108.28) -- (270,135.8) ;
	\draw [shift={(239.2,106.2)}, rotate = 43.86] [fill={rgb, 255:red, 0; green, 0; blue, 0 }  ][line width=0.08]  [draw opacity=0] (10.72,-5.15) -- (0,0) -- (10.72,5.15) -- (7.12,0) -- cycle    ;
	%Straight Lines [id:da7598081460438388]
	\draw    (267.13,136.68) -- (214,153) ;
	\draw [shift={(270,135.8)}, rotate = 162.93] [fill={rgb, 255:red, 0; green, 0; blue, 0 }  ][line width=0.08]  [draw opacity=0] (10.72,-5.15) -- (0,0) -- (10.72,5.15) -- (7.12,0) -- cycle    ;
	%Shape: Circle [id:dp006380434010266001]
	\draw  [fill={rgb, 255:red, 0; green, 0; blue, 0 }  ,fill opacity=1 ] (239.8,144.4) .. controls (239.8,143.18) and (240.78,142.2) .. (242,142.2) .. controls (243.22,142.2) and (244.2,143.18) .. (244.2,144.4) .. controls (244.2,145.62) and (243.22,146.6) .. (242,146.6) .. controls (240.78,146.6) and (239.8,145.62) .. (239.8,144.4) -- cycle ;

	% Text Node
	\draw (196.8,93.6) node    {$\vec{\omega }$};
	% Text Node
	\draw (259.6,156.4) node    {$\vec{r}$};
	% Text Node
	\draw (260.4,97.6) node    {$\vec{v}$};
	% Text Node
	\draw (204.4,160.8) node    {$O$};
	% Text Node
	\draw (234.4,133.2) node    {$P$};


	\end{tikzpicture}
\end{figure}

Per definire un oggetto che ruota era stata introdotta precedentemente la variazione angolare e ad essa erano state associate una velocità e un'accelerazione angolari: quantità scalari.
\[
	v=\omega r \qquad a_t=\alpha r \qquad a_n=\omega^2 r=\frac{v^2}{r}
\]
Queste informazioni possono essere espresse anche in termini vettoriali. Si definisce \textbf{vettore velocità angolare} $\vec{\omega}$ con le seguenti proprietà: il modulo è $\omega=\frac{d\vartheta}{dt}$, la direzione è perpendicolare al piano in cui avviene il moto e il verso è tale per cui dall'estremo del vettore $\vec{\omega}$ il moto appaia antiorario. Si ha:
\[
	\vec{v}=\vec{\omega} \times \vec{r}
\]
Si tratta di una relazione completa che lega la velocità (sempre tangente) al vettore velocità angolare. Di norma, si pensa $\vec{\omega}$ applicato nel centro della circonferenza. La formula soprastante resta comunque valida se $\vec{\omega}$ è applicato in un qualsiasi altro punto dell'asse di rotazione, la cui direzione è individuata da $\vec{u}_z$. Noto $\vec{\omega}$, sono individuati pertanto l'asse di rotazione, il piano del moto, il verso con cui è percorsa la circonferenza e come varia l'angolo nel tempo.
\[
	\vec{\omega}=\omega\, \vec{u}_z
\]
Tutti i punti del disco possiedono un'unica velocità angolare definita come in figura. Da $\vec{\omega}$, per derivazione rispetto al tempo, si ottiene il \textbf{vettore accelerazione angolare} $\vec{\alpha}$ che risulta parallelo a $\vec{\omega}$ (dato che questo ha direzione costante) verso determinato dalla variazione del modulo di $\vec{\omega}$ e modulo $\alpha=\frac{d\omega}{dt}$.
\[
	\vec{\alpha}=\alpha \vec{u}_z
\]
Essa si esprime come segue:
\[
	\vec{\alpha}=\frac{d\vec{v}}{dt}=\frac{d(\vec{\omega} \times \vec{r})}{dt}=\frac{d\vec{\omega}}{dt} \times \vec{r}+\vec{\omega} \times \frac{d\vec{r}}{dt}=\vec{\alpha} \times \vec{r} + \vec{\omega} \times \vec{v}
\]
Da questa relazione si vede che il vettore accelerazione tangente è pari al prodotto vettoriale fra l'accelerazione angolare e il raggio della circonferenza su cui si sta muovendo il punto. L'accelerazione normale è invece frutto di un doppio prodotto vettoriale:
\[
	\vec{a}_n=\vec{\omega} \times \vec{v}=\vec{\omega} \times ( \vec{r} \times \vec{\omega})
\]
Si noti che che $\vec{\omega}$ e $\vec{r}$ sono ortogonali, quindi è sufficiente moltiplicare i loro moduli. Il risultato $\vec{\omega} \times \vec{r}$ è un vettore tangente alla circonferenza che va moltiplicato per $\vec{\omega}$, ortogonale al piano. Il vettore risultante è dunque ortogonale a quella tangente e a $\vec{u}_z$, puntante verso l'asse di rotazione. Il suo modulo è semplicemente il prodotto dei moduli, perché entra sempre in gioco $\sin\frac{\pi}{2}$.

Si supponga di avere un disco che ruota su sé stesso. Tutti i suoi punti si muovono con la stessa velocità angolare, perché devono compiere nello stesso tempo un giro, essendo un corpo che ruota rigidamente, senza deformarsi. Tuttavia tali punti non hanno tutti la stessa velocità $\vec{v}$ perché, scritto scalarmente, $v=\omega r$, quindi essi avranno velocità maggiore man mano che ci si sposta verso l'esterno. In generale quindi si ha stessa velocità angolare ma velocità lineari differenti. Analogo discorso si applica per l'accelerazione tangente e per quella normale.







































\section{Legge di composizione delle velocità}

L'osservatore fisso identifica la posizione del punto $P$ tramite il vettore posizione $\vec{r}_{\text{ass}}$, scomponibile in coordinate cartesiane come segue:
\[
	\vec{r}(t)=x(t) \vec{u}_x+y(t)\vec{u}_y+z(t)\vec{u}_z
\]
Nel corso del tempo $x(t)$ cambia, ma i tre versori rimangono invariati perché il sistema di riferimento è fermo nello spazio.

\begin{figure}[htpb]
	\centering
	

	\tikzset{every picture/.style={line width=0.75pt}} %set default line width to 0.75pt

	\begin{tikzpicture}[x=0.75pt,y=0.75pt,yscale=-1,xscale=1]
	%uncomment if require: \path (0,300); %set diagram left start at 0, and has height of 300

	%Straight Lines [id:da25086700744423784]
	\draw    (108,133) -- (108,193) ;
	\draw [shift={(108,130)}, rotate = 90] [fill={rgb, 255:red, 0; green, 0; blue, 0 }  ][line width=0.08]  [draw opacity=0] (10.72,-5.15) -- (0,0) -- (10.72,5.15) -- (7.12,0) -- cycle    ;
	%Straight Lines [id:da19724660264049287]
	\draw    (48.86,239.15) -- (108,193) ;
	\draw [shift={(46.5,241)}, rotate = 322.03] [fill={rgb, 255:red, 0; green, 0; blue, 0 }  ][line width=0.08]  [draw opacity=0] (10.72,-5.15) -- (0,0) -- (10.72,5.15) -- (7.12,0) -- cycle    ;
	%Straight Lines [id:da7967137770227075]
	\draw    (177.8,226.7) -- (108,193) ;
	\draw [shift={(180.5,228)}, rotate = 205.77] [fill={rgb, 255:red, 0; green, 0; blue, 0 }  ][line width=0.08]  [draw opacity=0] (10.72,-5.15) -- (0,0) -- (10.72,5.15) -- (7.12,0) -- cycle    ;
	%Straight Lines [id:da6556661966529236]
	\draw    (437.51,168.9) -- (378.18,177.85) ;
	\draw [shift={(440.47,168.46)}, rotate = 171.43] [fill={rgb, 255:red, 0; green, 0; blue, 0 }  ][line width=0.08]  [draw opacity=0] (10.72,-5.15) -- (0,0) -- (10.72,5.15) -- (7.12,0) -- cycle    ;
	%Straight Lines [id:da9702780467154479]
	\draw    (323.73,126.25) -- (378.18,177.85) ;
	\draw [shift={(321.55,124.19)}, rotate = 43.46] [fill={rgb, 255:red, 0; green, 0; blue, 0 }  ][line width=0.08]  [draw opacity=0] (10.72,-5.15) -- (0,0) -- (10.72,5.15) -- (7.12,0) -- cycle    ;
	%Straight Lines [id:da16665976731656107]
	\draw    (355.26,251.89) -- (378.18,177.85) ;
	\draw [shift={(354.37,254.75)}, rotate = 287.2] [fill={rgb, 255:red, 0; green, 0; blue, 0 }  ][line width=0.08]  [draw opacity=0] (10.72,-5.15) -- (0,0) -- (10.72,5.15) -- (7.12,0) -- cycle    ;
	%Straight Lines [id:da8237388779050301]
	\draw [color={rgb, 255:red, 155; green, 155; blue, 155 }  ,draw opacity=1 ]   (108,193) -- (375.18,178.01) ;
	\draw [shift={(378.18,177.85)}, rotate = 536.79] [fill={rgb, 255:red, 155; green, 155; blue, 155 }  ,fill opacity=1 ][line width=0.08]  [draw opacity=0] (10.72,-5.15) -- (0,0) -- (10.72,5.15) -- (7.12,0) -- cycle    ;
	%Shape: Boxed Bezier Curve [id:dp8912748108905522]
	\draw    (143.18,117.91) .. controls (172.76,77.6) and (209.3,129.22) .. (238.88,88.9) .. controls (268.46,48.59) and (272.96,115.15) .. (289.63,83.97) ;
	%Shape: Circle [id:dp9836014349883189]
	\draw  [fill={rgb, 255:red, 0; green, 0; blue, 0 }  ,fill opacity=1 ] (207.6,105) .. controls (207.6,103.4) and (208.9,102.1) .. (210.5,102.1) .. controls (212.1,102.1) and (213.4,103.4) .. (213.4,105) .. controls (213.4,106.6) and (212.1,107.9) .. (210.5,107.9) .. controls (208.9,107.9) and (207.6,106.6) .. (207.6,105) -- cycle ;
	%Straight Lines [id:da2664802092852583]
	\draw [color={rgb, 255:red, 155; green, 155; blue, 155 }  ,draw opacity=1 ]   (108,193) -- (208.22,106.95) ;
	\draw [shift={(210.5,105)}, rotate = 499.35] [fill={rgb, 255:red, 155; green, 155; blue, 155 }  ,fill opacity=1 ][line width=0.08]  [draw opacity=0] (10.72,-5.15) -- (0,0) -- (10.72,5.15) -- (7.12,0) -- cycle    ;
	%Straight Lines [id:da46149717327452744]
	\draw [color={rgb, 255:red, 155; green, 155; blue, 155 }  ,draw opacity=1 ]   (378.18,177.85) -- (213.25,106.2) ;
	\draw [shift={(210.5,105)}, rotate = 383.48] [fill={rgb, 255:red, 155; green, 155; blue, 155 }  ,fill opacity=1 ][line width=0.08]  [draw opacity=0] (10.72,-5.15) -- (0,0) -- (10.72,5.15) -- (7.12,0) -- cycle    ;

	% Text Node
	\draw (44.4,252.2) node    {$x$};
	% Text Node
	\draw (188.4,237.2) node    {$y$};
	% Text Node
	\draw (110.4,114.2) node    {$z$};
	% Text Node
	\draw (312.24,119.53) node    {$x'$};
	% Text Node
	\draw (356.66,264.44) node    {$y'$};
	% Text Node
	\draw (454.06,165.69) node    {$z'$};
	% Text Node
	\draw (210.4,87.2) node    {$P$};
	% Text Node
	\draw (96.9,184.2) node    {$O$};
	% Text Node
	\draw (386.4,165.2) node    {$O'$};
	% Text Node
	\draw (146.4,142.2) node  [color={rgb, 255:red, 155; green, 155; blue, 155 }  ,opacity=1 ]  {$\vec{r}$};
	% Text Node
	\draw (261.4,146.2) node  [color={rgb, 255:red, 155; green, 155; blue, 155 }  ,opacity=1 ]  {$\overrightarrow{r'}$};
	% Text Node
	\draw (214.4,170.2) node  [color={rgb, 255:red, 155; green, 155; blue, 155 }  ,opacity=1 ]  {$\vec{r}_{OO'}$};


	\end{tikzpicture}
\end{figure}

L'osservatore relativo descrive il moto del punto $P$ definendo un vettore posizione $\vec{r'}(t)$ che identifica la posizione di $P$ dal sistema di riferimento $Oxy'$. Anche $\vec{r'}(t)$ può essere scomposto in componenti cartesiane rispetto al sistema di riferimento relativo. Otterremo:
\[
	\vec{r'}(t)=x'(t) \vec{u}_{x'}+y'(t)\vec{u}_{y'}+z'(t)\vec{u}_{z'}
\]
La differenza è che in questa espressione di vettore di posizione, oltre alle coordinate del punto materiale, anche i versori $\vec{u}_{x'}$, $\vec{u}_{y'}$, $\vec{u}_{z'}$ variano nel tempo nel momento in cui il sistema ruota (se esso invece subisce la sola traslazione ciò non accade).

Successivamente, si definisce un terzo versore che congiunge il sistema di riferimento fisso con il sistema di riferimento mobile che viene indicato come $\vec{r}_{oo'}$. Esso individua la posizione dell'origine del sistema di riferimento mobile rispetto al sistema di riferimento fisso, rappresenta quindi le coordinate di $O'$ rispetto ai versori $\vec{u}_x, \vec{u}_y, \vec{u}_z$.
\[
	\vec{r}_{oo'}(t)=x_{oo'}(t)\vec{u}_x+y_{oo'}(t)\vec{u}_y+z_{oo'}(t)\vec{u}_z
\]
Definiti questi tre vettori è possibile osservare che vi è una relazione semplice che li lega:
\begin{equation}
	\label{relativo}
	\boxed{\vec{r}(t)=\vec{r'}(t)+\vec{r}_{oo'}(t)}
\end{equation}
Derivando membro a membro si ottiene l'informazione che fornisce il legame fra le velocità. La velocità del punto $P$ rispetto al sistema fisso, che viene chiamata \emph{velocità assoluta}, è data da:
\[
	\vec{v}=\frac{d\vec{r}}{dt}= \frac{dx}{dt}\vec{u}_x + \frac{dy}{dt}\vec{u}_y + \frac{dz}{dt}\vec{u}_z
\]
Derivando il vettore posizione $\vec{r'}$ si ottiene:
\[
	\frac{d\vec{r'}}{dt}=\underbrace{\frac{dx'}{dt}\vec{u}_{x'}+\frac{dy'}{dt}\vec{u}_{y'}+\frac{dz'}{dt}\vec{u}_{z'}}_A+ \underbrace{x'(t) \frac{d\vec{u}_{x'}}{dt}+ y'(t) \frac{d\vec{u}_{y'}}{dt}+ z'(t) \frac{d\vec{u}_{z'}}{dt}}_B
\]
Il termine $A$ rappresenta la velocità del punto $P$ misurata dal sistema di riferimento mobile, viene chiamata \emph{velocità relativa}. Il termine $B$ nasce invece quando i versori variano direzione nel tempo, ossia quando il sistema di riferimento relativo ruota.

\paragraph{Formula di Poisson} La derivata di un versore che varia nel tempo, è un vettore ortogonale a quello di partenza, il cui modulo dà informazione su quanto rapidamente varia la sua direzione nel tempo. Si ha:
\[
	\frac{d\vec{u}_x}{dt}=\vec{\omega} \times \vec{u}_x
\]
Dove $\vec{\omega}$ è la velocità angolare con cui ruota il versore.

I tre versori sono rigidamente legati l'uno all'altro, nel senso che le loro mutue orientazioni non possono cambiare: alla rotazione di uno, con velocità angolare $\omega$, corrisponde la rotazione degli altri due con la stessa velocità angolare, come se essi fossero parte di un unico corpo indeformabile. Si ottiene quindi:
\begin{equation*}
	\begin{aligned}
		\frac{d\vec{r'}}{dt} &= \frac{dx'}{dt} \vec{u}_{x'}+x'(t)(\vec{\omega} \times \vec{u}_{x'})+\frac{dy'}{dt} \vec{u}_{y'} +y'(t)(\vec{\omega} \times \vec{u}_{y'})+\frac{dz'}{dt} \vec{u}_{z'} +z'(t)(\vec{\omega} \times \vec{u}_{z'} ) \\
		&= \frac{dx'}{dt}\vec{u}_{x'}+ \frac{dy'}{dt}\vec{u}_{y'}+ \frac{dz'}{dt}\vec{u}_{z'} + \underbrace{\vec{\omega} \times (x'(t)\vec{u}_{x'}+y'(t)\vec{u}_{y'}+z'(t)\vec{u}_{z'})}_{\vec{\omega} \times \vec{r'}}
	\end{aligned}
\end{equation*}
La velocità di $O'$ rispetto al sistema di riferimento fisso, quindi la velocità di traslazione del sistema di riferimento mobile, è data da:
\[
	\frac{d\vec{r}_{oo'}}{dt}=\frac{dx_{oo'}}{dt}\vec{u}_x+\frac{dy_{oo'}}{dt}\vec{u}_y+\frac{dz_{oo'}}{dt}\vec{u}_z=\vec{v}_{oo'}
\]
In sintesi derivando la ~\eqref{relativo} si ottiene:
\[
	\vec{v}_{\text{ass}}(t)=\vec{v}_{\text{rel}}+\underbrace{(\vec{\omega} \times \vec{r'}+\vec{v}_{oo'})}_{\text{velocità di trascinamento}}=\vec{v}_{\text{rel}}+ \vec{v}_{ \text{trasc} }
\]
I termini evidenziati prendono il nome di \emph{velocità di trascinamento} del punto $P$ e rappresentano la velocità del sistema con cui è trascinato a muoversi quando lo si immagina solidale ad esso. Se il punto $P$ fosse fermo rispetto al sistema di riferimento mobile, la sua velocità misurata dal sistema fisso coinciderebbe con tale velocità, data dalla somma di un termine traslatorio con velocità istantanea $v_{oo'}$ e di un termine relativo con velocità angolare $\vec{\omega}$, variabile in generale sia in modulo che in direzione.

In forma compatta, la legge di composizione delle velocità afferma che un punto materiale in movimento rispetto ad un osservatore relativo, avrà rispetto a un osservatore fisso una velocità assoluta frutto della somma delle velocità relativa e di trascinamento.
\[
	\text{Teorema delle velocità relative} \quad \boxed{\vec{v}_{\text{ass}}=\vec{v}_{\text{rel}}+\vec{v}_{\text{trasc}}}
\]

\paragraph{Esempio} La Terra è un sistema di riferimento mobile nel tempo. In termini di velocità vettoriale si identifica come in figura. Allora un osservatore che si trova in un punto della Terra, a causa della rotazione terrestre è trascinato a ruotare insieme ad essa. La velocità angolare è costante, quella lineare è sempre più piccola verso i poli. Si tratta di un classico esempio io cui si è trascinati insieme al sistema di riferimento in cui ci si muove.

\begin{figure}[htpb]
	\centering
	

	\tikzset{every picture/.style={line width=0.75pt}} %set default line width to 0.75pt

	\begin{tikzpicture}[x=0.75pt,y=0.75pt,yscale=-1,xscale=1]
	%uncomment if require: \path (0,314); %set diagram left start at 0, and has height of 314

	%Shape: Circle [id:dp8683221589787924]
	\draw  [fill={rgb, 255:red, 243; green, 243; blue, 243 }  ,fill opacity=1 ] (200.5,181.75) .. controls (200.5,146.27) and (229.27,117.5) .. (264.75,117.5) .. controls (300.23,117.5) and (329,146.27) .. (329,181.75) .. controls (329,217.23) and (300.23,246) .. (264.75,246) .. controls (229.27,246) and (200.5,217.23) .. (200.5,181.75) -- cycle ;
	%Curve Lines [id:da531178027916203]
	\draw [color={rgb, 255:red, 0; green, 0; blue, 0 }  ][fill={rgb, 255:red, 155; green, 155; blue, 155 }  ,fill opacity=1 ][line width=0.75] [line join = round][line cap = round]   (212.5,161) .. controls (218.08,166.58) and (218.08,174.79) .. (224.5,178) .. controls (229.22,180.36) and (238.71,180.21) .. (242.5,184) .. controls (243.17,184.67) and (237.81,188.44) .. (237.5,190) .. controls (236.4,195.51) and (240.63,197.13) .. (242.5,199) .. controls (247.69,204.19) and (247.7,220.4) .. (250.5,226) .. controls (252.55,230.09) and (253.78,237.24) .. (260.5,235) .. controls (261.64,234.62) and (258.88,233.14) .. (258.5,232) .. controls (257.82,229.97) and (255.81,225.76) .. (256.5,223) .. controls (256.84,221.63) and (261.17,222.33) .. (262.5,221) .. controls (265.6,217.9) and (265.93,209.68) .. (266.5,209) .. controls (267.57,207.72) and (269.9,209.46) .. (271.5,209) .. controls (275.42,207.88) and (276.07,196.85) .. (275.5,194) .. controls (275.03,191.63) and (267.53,190.01) .. (264.5,189) .. controls (262.62,188.37) and (262.71,183.72) .. (261.5,183) .. controls (260.07,182.14) and (258.14,183.27) .. (256.5,183) .. controls (254.64,182.69) and (252.74,179.31) .. (251.5,179) .. controls (248.89,178.35) and (246.14,180.53) .. (243.5,180) .. controls (241.41,179.58) and (241.58,177.52) .. (239.5,177) .. controls (238.53,176.76) and (237.1,177.8) .. (236.5,177) .. controls (235.7,175.93) and (237.74,173.5) .. (236.5,173) .. controls (233.41,171.76) and (229.77,173.65) .. (226.5,173) .. controls (224.87,172.67) and (228.32,170.18) .. (229.5,169) .. controls (232.24,166.26) and (235.77,165.42) .. (240.5,167) .. controls (242.31,167.6) and (251.68,158.82) .. (255.5,155) .. controls (259.94,150.56) and (261.4,153.42) .. (268.5,152) .. controls (268.58,151.98) and (265.6,147.25) .. (265.5,147) .. controls (265.29,146.48) and (263.74,142.06) .. (263.5,142) .. controls (257.62,140.53) and (255.55,141.95) .. (252.5,145) .. controls (251.23,146.27) and (248.17,148.67) .. (247.5,147) .. controls (244.05,138.39) and (259.97,139.21) .. (256.5,134) .. controls (254.38,130.82) and (248.78,133.18) .. (245.5,134) .. controls (245.04,134.11) and (244.71,135.42) .. (244.5,135) .. controls (243.04,132.09) and (237.16,133.42) .. (232.5,134) .. controls (223.5,127) and (202.5,152) .. (211.5,160) ;
	%Curve Lines [id:da11031845277788]
	\draw [color={rgb, 255:red, 0; green, 0; blue, 0 }  ][fill={rgb, 255:red, 155; green, 155; blue, 155 }  ,fill opacity=1 ][line width=0.75] [line join = round][line cap = round]   (314.5,222) .. controls (314.5,219.46) and (305.21,214.56) .. (304.5,211) .. controls (303.43,205.66) and (304.7,190.3) .. (302.5,187) .. controls (301.23,185.09) and (288.47,186.39) .. (287.5,186) .. controls (281.32,183.53) and (282.9,178.59) .. (281.5,173) .. controls (281.27,172.09) and (279.08,171.84) .. (279.5,171) .. controls (280.25,169.5) and (284.9,168.81) .. (285.5,167) .. controls (285.57,166.79) and (286.12,162.11) .. (286.5,162) .. controls (296.58,159.12) and (306.56,165.24) .. (311.5,164) .. controls (313.49,163.5) and (318.85,166.24) .. (320.5,165) .. controls (321.07,164.57) and (320,160.1) .. (319.5,160) .. controls (317.87,159.67) and (315.8,161.04) .. (314.5,160) .. controls (313.46,159.17) and (314.82,157.29) .. (314.5,156) .. controls (314.04,154.17) and (311.1,161.79) .. (310.5,160) .. controls (309.16,155.99) and (308.58,150.97) .. (302.5,153) .. controls (301.61,153.3) and (303.83,154.33) .. (304.5,155) .. controls (305.71,156.21) and (308.65,156.54) .. (307.5,160) .. controls (307.01,161.46) and (302.51,156) .. (302.5,156) .. controls (296.19,154.95) and (296.67,160.54) .. (290.5,159) .. controls (287.77,158.32) and (293.07,153.86) .. (294.5,151) .. controls (296.84,146.33) and (302.37,146.65) .. (309.5,146) .. controls (311.28,145.84) and (310.49,143) .. (312.5,143) .. controls (318.5,138) and (341.5,179) .. (320.5,214) ;
	%Curve Lines [id:da8940953037558799]
	\draw [color={rgb, 255:red, 0; green, 0; blue, 0 }  ][fill={rgb, 255:red, 155; green, 155; blue, 155 }  ,fill opacity=1 ][line width=0.75] [line join = round][line cap = round]   (302.5,138) .. controls (302.5,138.69) and (296.61,142.55) .. (297.5,143) .. controls (300.69,144.6) and (307.5,147.77) .. (307.5,141) ;
	%Curve Lines [id:da5286705701496102]
	\draw [color={rgb, 255:red, 0; green, 0; blue, 0 }  ][fill={rgb, 255:red, 155; green, 155; blue, 155 }  ,fill opacity=1 ][line width=0.75] [line join = round][line cap = round]   (291.5,143) .. controls (291.5,157.74) and (298.55,141) .. (290.5,141) ;
	%Curve Lines [id:da1567416553514649]
	\draw [color={rgb, 255:red, 0; green, 0; blue, 0 }  ][fill={rgb, 255:red, 155; green, 155; blue, 155 }  ,fill opacity=1 ][line width=0.75] [line join = round][line cap = round]   (289.5,148) .. controls (289.09,148.41) and (287.78,151.92) .. (286.5,150) .. controls (283,144.75) and (290.27,148) .. (290.5,148) ;
	%Curve Lines [id:da02198661372958144]
	\draw    (200.5,181.75) .. controls (92.54,216.33) and (443.46,205.61) .. (330.73,182.11) ;
	\draw [shift={(329,181.75)}, rotate = 371.38] [fill={rgb, 255:red, 0; green, 0; blue, 0 }  ][line width=0.08]  [draw opacity=0] (10.72,-5.15) -- (0,0) -- (10.72,5.15) -- (7.12,0) -- cycle    ;
	%Straight Lines [id:da37712678869520455]
	\draw    (264.75,127.5) -- (264.75,97.5) ;
	\draw [shift={(264.75,94.5)}, rotate = 450] [fill={rgb, 255:red, 0; green, 0; blue, 0 }  ][line width=0.08]  [draw opacity=0] (10.72,-5.15) -- (0,0) -- (10.72,5.15) -- (7.12,0) -- cycle    ;

	% Text Node
	\draw (282,97) node    {$\vec{\omega }$};


	\end{tikzpicture}
\end{figure}







































\section{Legge di composizione delle accelerazioni}

È possibile ora ricavare il legame fra le accelerazioni. Rispetto al sistema fisso l'accelerazione assoluta è data da:
\begin{gather*}
	\frac{d\vec{v}_{\text{ass} }}{dt}=\frac{d}{dt}\left(\frac{dx}{dt} \vec{u}_x + \frac{dy}{dt} \vec{u}_y+ \frac{dz}{dt} \vec{u}_z\right)=\frac{d^2x}{dt^2} \vec{u}_x+ \frac{d^2y}{dt^2} \vec{u}_y+\frac{d^2z}{dt^2} \vec{u}_z=\vec{a}_{\text{ass}} (t) \\
	\vec{a}_{\text{ass}}(t)=\frac{d\vec{v}_{\text{rel}}}{dt}+\frac{d}{dt}(\vec{\omega} \times \vec{r}_{\text{rel}})+\frac{d\vec{v}_{oo'}}{dt}
\end{gather*}
Derivando il primo termine si ha:
\[
	\frac{d\vec{v}_{\text{rel}}}{dt}=\frac{d}{dt}\left(\frac{dx'}{dt}\vec{u}_{x'}+\frac{dy'}{dt}\vec{u}_{y'}+\frac{dz'}{dt}\vec{u}_{z'}\right)=\frac{d^2x'}{dt^2} \vec{u}_{x'} + \frac{d^2y'}{dt^2} \vec{u}_{y'}+ \frac{d^2z'}{dt^2} \vec{u}_{z'}
\]
Il termine ottenuto prende il nome di \emph{accelerazione relativa}. Per quanto riguarda il secondo e il terzo termine avremo:
\begin{gather*}
	\frac{d}{dt}(\vec{\omega} \times \vec{r}_{\text{rel}})=\frac{d\vec{\omega}}{dt} \times \vec{r}_{\text{rel}}+\vec{\omega} \times \frac{d\vec{r}_{\text{rel}} }{dt} = \vec{\alpha} \times \vec{r}_{\text{rel}}+\vec{\omega} \times( \vec{v}_{\text{rel}}+\vec{\omega} \times \vec{v}_{\text{rel}}) \\
	\frac{d\vec{v}_{oo'}}{dt}= \frac{d^2x_{oo'}}{dt^2} \vec{u}_x+ \frac{d^2y_{oo'}}{dt^2} \vec{u}_y+ \frac{d^2z_{oo'}}{dt^2} \vec{u}_z+\vec{a}_{oo'}
\end{gather*}
E quindi:
\[
	\vec{a}_{\text{ass}}(t)=\vec{a}_{\text{rel}} +\vec{\omega} \times \vec{v}_{\text{rel}}+\vec{\alpha}\times \vec{r}_{\text{rel}}+\vec{\omega} \times \vec{v}_{\text{rel}}+ \vec{\omega}\times({\vec{\omega}}\times \vec{r}_{\text{rel}})+\vec{a}_{oo'}
\]
\begin{equation}
	\boxed{\vec{a}_{\text{ass}}(t)=\vec{a}_{\text{rel}} +2\,\vec{\omega} \times \vec{v}_{\text{rel}}+\vec{\alpha}\times \vec{r}_{\text{rel}}+ \vec{\omega}\times({\vec{\omega}}\times \vec{r}_{\text{rel}})+\vec{a}_{oo'}}
\end{equation}
Questo risultato è noto come \emph{legge di composizione delle accelerazioni in cinematica relativa}. Il termine:
\[
	\vec{a}_{\text{trasc}}=\underbrace{\vec{\alpha}\times \vec{r}_{\text{rel}}}_A+ \underbrace{\vec{\omega}\times({\vec{\omega}}\times \vec{r}_{\text{rel}})}_B+\vec{a}_{oo'}
\]
prende il nome di \textbf{accelerazione di trascinamento}, e rappresenta l'accelerazione con cui è portato a muoversi il punto $P$ immaginato come congelato nel sistema di riferimento mobile. Il termine:
\[
	\vec{a}_c= 2\,\vec{\omega} \times \vec{v}_{\text{rel}}
\]
prende il nome di \textbf{accelerazione di Coriolis}.
Fondamentalmente quando un punto materiale si muove in un sistema di riferimento mobile, l'accelerazione osservata da un osservatore fermo nello spazio è data dall'accelerazione percepita nel sistema di riferimento relativo, più il termine accelerazione di trascinamento.
I termini $A$ e $B$ sono le due componenti dell'accelerazione che assume il punto $P$ quando il sistema di riferimento mobile ruota. La prima è l'accelerazione normale, la seconda è l'accelerazione tangente. $\vec{a}_{oo'}$ invece è il termine legato alla traslazione del sistema mobile.
Per quanto riguarda l'accelerazione di Coriolis invece, essa esiste se e solo se il sistema di riferimento sta ruotando e il punto materiale ha una certa velocità relativa rispetto ad esso. In generale questa accelerazione esiste ad esempio quando si considera un oggetto che si muove da sud verso nord sulla Terra, che ruota su se stessa. L'effetto che si percepisce è una deviazione verso destra di tale corpo. Allo stesso effetto sono soggetti i venti sulla Terra, che deviano verso est o ad esempio le maree.























































































































\chapter{La dinamica del punto materiale}

Dalla trattazione della cinematica ci si sposta ora alla dinamica del punto materiale, che si preoccupa di capire quali sono le cause fisiche per cui un corpo entra in movimento e descrive un certo tipo di moto. La dinamica del punto materiale, nella sua trattazione classica, deve la sua evoluzione agli studi di Newton.

Le cause del moto si devono vedere come frutto delle interazioni che un punto ha con l'ambiente circostante. Queste interazioni vengono poi identificate in fisica con il concetto di forza, grandezza fisica di carattere vettoriale. Essendo tale, non basta darne l'intensità ma bisogna specificare chiaramente quali sono la sua direzione e il suo verso.  Se per alcuni vettori poi il punto di applicazione non è fondamentale,  per le forze in generale lo è: esse si applicano sul punto materiale (che non ha alcuna dimensione).
Un'importante verifica del fatto che la forza è una grandezza vettoriale si ha quando su un punto materiale ne agiscono contemporaneamente più di una: si constata che il moto del punto ha luogo come se agisse una sola forza data dalla risultante vettoriale delle forze applicate al punto:
\[
	\vec{R}=\vec{F}_1+\vec{F}_2+\dots+\vec{F}_n=\sum_{i=1}^n \vec{F}_i
\]
In effetti l'accelerazione del punto è pari alla somma vettoriale delle accelerazioni che il punto avrebbe se ciascuna forza agisse da sola:
\[
	\vec{a}=\sum_{i=1}^n \vec{a_i}
\]
Questo fondamentale risultato sperimentale fa capire che in presenza di più forze ciascuna agisce indipendentemente dalle altre, comunicando al punto sempre l'accelerazione $a_i$. Si parla a tal proposito di indipendenza delle azioni simultanee. D'altra parte tutto ciò implica che dallo studio del moto di un punto materiale si ottengono informazioni solo sulla risultante delle forze agenti sul punto stesso, $\vec{R}$, e non sulle singole forze che concorrono a formare la forza risultante. In particolare, affermare che la forza agente su un punto è nulla non significa necessariamente che sul punto non agiscono forze, ma spesso indica il fatto che la somma delle forze agenti su di esso è nulla.







































\section{Leggi della dinamica di Newton}

Si va ora a legare ora il concetto di forza con il moto del punto materiale enunciando i tre principi della dinamica newtoniana.

\subsubsection{Prima legge della dinamica}

Il primo principio afferma che:

\noindent\fbox{%
	\parbox{\textwidth}{%
		\emph{In assenza di forze un corpo permane nel suo stato di moto rettilineo uniforme di cui la quiete è un caso particolare.}
	}%
}

Quando la risultante delle forze è nulla si può dire che il corpo permane nel suo stato di moto rettilineo uniforme, moto naturale di un corpo. Ciò significa che esso manterrà una velocità vettoriale costante, ossia costante in modulo, in direzione e in verso. La quiete è il caso in cui questa velocità costante ha modulo pari a $0$. La cosa interessante è che intuitivamente si tende a pensare che lo stato naturale di un corpo quando su di esso non agiscono forze sia la quiete, la quale rappresenta invece solo un caso specifico. La prima legge della dinamica è anche nota come \textbf{principio di inerzia}.

Questo principio è frutto di una serie di sperimentazioni. La più importante è quella del piano inclinato perfettamente levigato che termina con una parte piana. Una volta sceso dal piano inclinato, avendo subito una accelerazione, il corpo non si ferma mai ma prosegue di moto rettilineo uniforme. Questa situazione in realtà rappresenta un caso limite, poiché in natura agisce sempre l'attrito, forza che tende a fermare il corpo. Questo risultato, venne formulato da Galileo a seguito dei suoi esperimenti sul moto dei corpi nel \textit{Dialogo sopra i massimi sistemi}. In esso è implicitamente contenuta l'idea che sarebbe stata esplicitata da Newton e posta sotto forma di legge quantitativa: la variazione di una velocità, in modulo, in direzione o in entrambi, è dovuta all'azione di una forza. Si comprende quindi che un moto accelerato segnala la presenza di una forza agente.

\subsubsection{Seconda legge della dinamica}

La seconda legge è un'estensione della prima e dice che:

\noindent\fbox{%
	\parbox{\textwidth}{%
		\emph{In presenza di forze la cui risultante è diversa da $0$, il vettore velocità varia nel tempo. L'effetto di una forza è quella di generare sul punto materiale in cui sta agendo, un'accelerazione vettoriale, che è sempre proporzionale alla forza agente tramite un coefficiente che prende il nome di \textbf{massa inerziale} del corpo.}
	}%
}

L'accelerazione ha sempre stessa direzione e verso della risultante delle forze perché il coefficiente di proporzionalità (la massa) è una grandezza scalare sempre strettamente positiva che dipende dalle caratteristiche del corpo stesso. Il termine massa inerziale del corpo è legato al fatto che essa esprime l'inerzia del punto, ossia la sua resistenza a variare il proprio stato di moto. Fissata una determinata forza $\vec{F}$, l'effetto dinamico è tanto maggiore quanto minore è la massa. Per un punto materiale che si muove con una determinata accelerazione $\vec{a}$, la forza necessaria a mantenere tale moto è tanto maggiore quanto maggiore è il valore di $m$. Si giustifica così l'uso del termine “punto \emph{materiale}”: per descrivere il comportamento dinamico del punto occorre conoscerne la massa; si può cioè semplificarlo al massimo concependo un corpo privo di struttura, ma non si può rinunciare alla massa, concetto dinamico fondamentale per qualsiasi corpo.  Si può osservare che la prima legge vista in precedenza è un caso particolare della seconda legge:
\begin{equation}
	\vec{F}=m\vec{a}
\end{equation}
È stata fornita una definizione operativa di forza e con questa definizione si può capire quali sono le dimensioni della grandezza in questione: si tratta di una massa per un'accelerazione. Facendo l'analisi dimensionale:
\[
	[F]=[M]\,[L]\,[T]^{-2} \to kg\,m\,s^{-2}
\]
a questa unità di misura si da il nome di Newton.

La seconda legge della dinamica collega direttamente le cause del moto agli effetti di esso. Grazie a questo principio, qualora naturalmente si conoscano la funzione $\vec{F}(t)$ e le condizioni iniziali, vengono infatti ricavate tutte le proprietà relative al moto di un punto materiale e, in particolare, la legge oraria.  Si capisce perché in cinematica si è dato particolare importanza al problema inverso: in generale ciò che nella realtà si può misurare sono le forze, che con la seconda legge della dinamica vengono legate all'accelerazione posseduta dal corpo, punto di partenza per ricavare la legge oraria.

\subsubsection{Terza legge della dinamica}

La terza legge della dinamica afferma che:

\noindent\fbox{%
	\parbox{\textwidth}{%
		\emph{Dati due corpi $A$ e $B$ di massa $m_A$ e $m_B$, che si stanno scambiando una mutua interazione e sono isolati rispetto al resto del mondo, si può affermare che se il corpo $A$ genera sul corpo $B$ una forza allora esisterà una forza generata da $B$ su $A$ tale per cui le due forze $\vec{F}_{AB}$ e $\vec{F}_{BA}$ saranno sempre uguali in intensità, direzione ma con verso opposto.}
	}%
}

La terza legge della dinamica prende il nome di \textbf{principio azione reazione} poiché ogni volta che un corpo subisce l'azione dinamica di un altro corpo, a sua volta genererà su di esso una forza uguale e opposta. Le due forze, azione e reazione, sono applicate in punti ben distinti, una sul corpo $B$ e una sul corpo $A$ e devono necessariamente giacere sulla stessa retta di applicazione. Se tutte e due le forze fossero applicate sullo stesso corpo, questo non potrebbe accelerare perché la risultate delle forze su di esso sarebbe zero. Il fatto che non esista una forza isolata, ma che tutte le forze vadano considerate sempre in coppia, chiarisce che l'interazione tra due corpi è sempre un'azione mutua.







































\section{Quantità di moto}

Questi principi della dinamica sono stati definiti da Newton in una maniera più generale andando a considerare una quantità vettoriale caratteristica dei corpi che si muovono: la \textbf{quantità di moto}. Essa è definita come una grandezza vettoriale che non è altro che il prodotto tra massa inerziale e velocità vettoriale.
\[
	\vec{p}=m\vec{v}
\]

\subsection{I principi della dinamica dal punto di vista della quantità di moto}
È possibile andare a definire i tre principi della dinamica newtoniana in termini di tale quantità.

\paragraph{1} Si può dire che il corpo ha una quantità di moto costante nel tempo se la risultante delle forze su di esso è pari a $0$, tale formulazione del primo principio della dinamica è più generale perché tiene conto anche della situazione in cui il corpo durante il suo moto cambia massa.

\paragraph{2} D'altra parte, nel caso in cui agiscano forze, cambierà la quantità di moto posseduta dal punto materiale nel tempo. In particolar modo, in un istante $t$, esse provocano una variazione infinitesima di $\vec{p}$. Si può esprimere ciò in termini di derivate. Infatti, se la massa è costante:
\[
	\vec{F}=\frac{d\vec{p}}{dt}=\frac{d(m\vec{v})}{dt}=m\vec{a}
\]

\paragraph{3} Affermare che vale il principio di azione reazione dati due corpi $A$ e $B$ che costituiscono un sistema isolato dal resto del mondo, è equivalente a dire che la quantità di moto totale dei due corpi si mantiene costante nel tempo.
\[
	\vec{p}_1+\vec{p}_2=\text{cost} \implies \frac{d\vec{p}_1}{dt}+\frac{d\vec{p}_2}{dt}=0 \implies \frac{d\vec{p}_1}{dt}=-\frac{d\vec{p}_2}{dt} \implies \vec{F}_1=-\vec{F}_2
\]
Si possono sintetizzare le tre leggi della dinamica di Newton come segue:
\begin{center}
	\begin{tabular}{ll}
		\toprule
		\midrule
		prima legge	  & $\vec{R}=0 \implies \vec{p} = \text{cost}$ \\
		seconda legge & $\vec{F}=\frac{d\vec{p}}{dt} = m\vec{a}$ \\
		terza legge   & $\vec{p}_{tot} = $cost \\
		\bottomrule
	\end{tabular}
\end{center}

\subsection{Impulso e teorema dell'impulso}

La relazione locale: $\frac{d\vec{p}}{dt}=m\vec{a}$ può essere trasformata in una relazione integrale che, invece di comunicare cosa accade in un preciso istante, dà informazioni relative a un intervallo di tempo:
\[
	\boxed{\int_{t_0}^{\Delta t +t_0} d\vec{p}=\int_{t_0}^{\Delta t +t_0} \vec{F(t)}\,dt=\Delta \vec{p}}
\]
L'integrale nel tempo di una forza prende il nome di \textbf{impulso} di una forza e il risultato che si ottiene dall'integrazione prende il nome di \textbf{teorema dell'impulso}. È una quantità vettoriale indicata con la lettera $I$, dimensionalmente è una forza per un tempo. Se si conosce l'andamento nel tempo della forza, l'impulso sarà l'area sottesa dalla curva. Il \emph{valore medio} di tale forza nell'intervallo di tempo è quel valore tale per cui l'area del rettangolo che ha altezza pari alla forza media e base pari all'intervallo di tempo, dà esattamente un area pari a quella sottesa dalla curva. È utile da utilizzare nel momento in cui una forza agisce per un breve istante di tempo.
\begin{figure}[htpb]
	\centering
	

	% Pattern Info
	 
	\tikzset{
	pattern size/.store in=\mcSize, 
	pattern size = 5pt,
	pattern thickness/.store in=\mcThickness, 
	pattern thickness = 0.3pt,
	pattern radius/.store in=\mcRadius, 
	pattern radius = 1pt}
	\makeatletter
	\pgfutil@ifundefined{pgf@pattern@name@_2iyabavlg}{
	\pgfdeclarepatternformonly[\mcThickness,\mcSize]{_2iyabavlg}
	{\pgfqpoint{0pt}{-\mcThickness}}
	{\pgfpoint{\mcSize}{\mcSize}}
	{\pgfpoint{\mcSize}{\mcSize}}
	{
	\pgfsetcolor{\tikz@pattern@color}
	\pgfsetlinewidth{\mcThickness}
	\pgfpathmoveto{\pgfqpoint{0pt}{\mcSize}}
	\pgfpathlineto{\pgfpoint{\mcSize+\mcThickness}{-\mcThickness}}
	\pgfusepath{stroke}
	}}
	\makeatother
	\tikzset{every picture/.style={line width=0.75pt}} %set default line width to 0.75pt        

	\begin{tikzpicture}[x=0.75pt,y=0.75pt,yscale=-1,xscale=1]
	%uncomment if require: \path (0,300); %set diagram left start at 0, and has height of 300

	%Shape: Polygon Curved [id:ds941259049534299] 
	\draw  [draw opacity=0][fill={rgb, 255:red, 222; green, 222; blue, 222 }  ,fill opacity=1 ] (173.85,95) .. controls (174.1,104.25) and (174.35,106.08) .. (174.5,113.7) .. controls (162.5,113.63) and (151.65,113.08) .. (138.2,113.6) .. controls (149.25,106.38) and (162.75,99.13) .. (173.85,95) -- cycle ;
	%Shape: Polygon Curved [id:ds8749913091428299] 
	\draw  [draw opacity=0][fill={rgb, 255:red, 222; green, 222; blue, 222 }  ,fill opacity=1 ] (114,134.38) .. controls (113.75,125.13) and (114,121.63) .. (113.85,114) .. controls (124,113.38) and (124.75,114.13) .. (138.2,113.6) .. controls (127.5,120.88) and (125.75,124.38) .. (114,134.38) -- cycle ;
	%Shape: Polygon Curved [id:ds7271949011587744] 
	\draw  [draw opacity=0][pattern=_2iyabavlg,pattern size=3pt,pattern thickness=0.75pt,pattern radius=0pt, pattern color={rgb, 255:red, 222; green, 222; blue, 222}] (114,134.38) .. controls (124.5,124.63) and (129.5,120.38) .. (138.2,113.6) .. controls (154,113.63) and (159.75,113.38) .. (174.5,113.7) .. controls (173.8,159.2) and (173.8,174.4) .. (173.85,208.7) .. controls (149.4,208.8) and (138.75,208.63) .. (113.85,208.7) .. controls (113.75,174.88) and (113.5,151.63) .. (114,134.38) -- cycle ;
	%Shape: Axis 2D [id:dp17772324596718025] 
	\draw  (50,208.7) -- (288.5,208.7)(73.85,53) -- (73.85,226) (281.5,203.7) -- (288.5,208.7) -- (281.5,213.7) (68.85,60) -- (73.85,53) -- (78.85,60)  ;
	%Curve Lines [id:da44816445997436083] 
	\draw    (73.85,188.7) .. controls (119.5,111) and (172.5,82) .. (247.5,81) ;
	%Straight Lines [id:da9863358988402184] 
	\draw  [dash pattern={on 0.84pt off 2.51pt}]  (113.85,208.7) -- (113.85,114) ;
	%Straight Lines [id:da2537758174613314] 
	\draw  [dash pattern={on 0.84pt off 2.51pt}]  (173.85,208.7) -- (173.85,95) ;
	%Straight Lines [id:da6198593738624376] 
	\draw  [dash pattern={on 0.84pt off 2.51pt}]  (174.5,113.7) -- (73.85,113.7) ;

	% Text Node
	\draw (57,111) node    {$M$};
	% Text Node
	\draw (49,50) node    {$F( t)$};
	% Text Node
	\draw (306,209) node    {$t$};
	% Text Node
	\draw (114,223) node    {$t_{0}$};
	% Text Node
	\draw (179,223) node    {$t_{0} +\Delta t$};


	\end{tikzpicture}
\end{figure}
\FloatBarrier

\paragraph{Esempio} Si immagini di lasciar cadere una penna e di fermarla con la mano. Istantaneamente l'oggetto ha ricevuto una variazione netta della quantità di moto che lo ha portato a fermarsi. In pratica essa, nel momento del contatto, ha generato sull'oggetto una reazione normale che l'ha fermato. Rappresentando in un grafico la forza esercitata dalla mano, il suo valore dopo il contatto è molto più piccolo rispetto a quello che ha dovuto esercitare nel momento dell'impatto. L'impulso genera un valore della forza molto elevato in un istante di tempo, il suo effetto è quello di variare molto rapidamente la quantità di moto. Le forze che hanno tale andamento, sono dette \textbf{forze impulsive}.
\begin{gather*}
	\Delta \vec{p}=0-(-m\vec{v})=m\vec{v}=\int_t^{t_0} \vec{F}_\text{mano} (t)\,dt \\
	\int\vec{R}^\text{ext} \,dt=\Delta \vec{p}_\text{tot}
\end{gather*}
\begin{figure}[htpb]
	\centering
	

	% Pattern Info
	 
	\tikzset{
	pattern size/.store in=\mcSize, 
	pattern size = 5pt,
	pattern thickness/.store in=\mcThickness, 
	pattern thickness = 0.3pt,
	pattern radius/.store in=\mcRadius, 
	pattern radius = 1pt}
	\makeatletter
	\pgfutil@ifundefined{pgf@pattern@name@_mjsob4tfc}{
	\pgfdeclarepatternformonly[\mcThickness,\mcSize]{_mjsob4tfc}
	{\pgfqpoint{0pt}{0pt}}
	{\pgfpoint{\mcSize+\mcThickness}{\mcSize+\mcThickness}}
	{\pgfpoint{\mcSize}{\mcSize}}
	{
	\pgfsetcolor{\tikz@pattern@color}
	\pgfsetlinewidth{\mcThickness}
	\pgfpathmoveto{\pgfqpoint{0pt}{0pt}}
	\pgfpathlineto{\pgfpoint{\mcSize+\mcThickness}{\mcSize+\mcThickness}}
	\pgfusepath{stroke}
	}}
	\makeatother
	\tikzset{every picture/.style={line width=0.75pt}} %set default line width to 0.75pt        

	\begin{tikzpicture}[x=0.75pt,y=0.75pt,yscale=-1,xscale=1]
	%uncomment if require: \path (0,300); %set diagram left start at 0, and has height of 300

	%Shape: Polygon Curved [id:ds11145635442832602] 
	\draw  [draw opacity=0][pattern=_mjsob4tfc,pattern size=3pt,pattern thickness=0.75pt,pattern radius=0pt, pattern color={rgb, 255:red, 222; green, 222; blue, 222}] (123.85,228.7) .. controls (136.5,84.25) and (154.5,93.25) .. (193.5,196) .. controls (202,218.25) and (260.5,197) .. (292.5,182) .. controls (292.5,196.25) and (293,205.75) .. (292.5,229) .. controls (242.5,228.25) and (138.68,229.23) .. (123.85,228.7) -- cycle ;
	%Shape: Axis 2D [id:dp7166572401931128] 
	\draw  (70,228.7) -- (308.5,228.7)(93.85,73) -- (93.85,246) (301.5,223.7) -- (308.5,228.7) -- (301.5,233.7) (88.85,80) -- (93.85,73) -- (98.85,80)  ;
	%Curve Lines [id:da23416531821160635] 
	\draw    (123.85,228.7) .. controls (136.5,84.25) and (154.5,93.25) .. (193.5,196) .. controls (202,218.25) and (260.5,197) .. (292.5,182) ;

	% Text Node
	\draw (65,70) node    {$F_{\text{mano}}$};
	% Text Node
	\draw (326,229) node    {$t$};


	\end{tikzpicture}
\end{figure}







































\section{Sistemi di riferimento inerziali}

È lecito chiedersi quale sia il campo di validità di questi tre principi. Essi hanno prima di tutto una valenza nell'ambito della fisica classica, in cui punti e corpi si muovono a velocità inferiori rispetto alla velocità della luce e con dimensioni al di sopra di quelle quantiche. Se così non fosse infatti si dovrebbe lavorare all'interno della meccanica quantistica.

Nell'ambito della fisica classica poi i tre principi della dinamica newtoniana hanno validità solo se si descrive il moto in un particolare sistema di riferimento che prende il nome di \textbf{sistema di riferimento inerziale}.
Un sistema di riferimento inerziale è un sistema di riferimento che risulta muoversi di moto rettilineo uniforme, in cui di conseguenza vale il principio di inerzia.
Ogni sistema di riferimento che si considera consolidale alla Terra non è inerziale, dal momento che essa si muove intorno al Sole e su sé stessa. Un sistema di riferimento inerziale è quello delle \emph{stelle fisse}. Esso è dato da quattro punti nello spazio davvero molto lontani fra di loro, di modo che la loro posizione non cambi mai. Si può dimostrare che qualunque sistema di riferimento che si muove di moto rettilineo uniforme rispetto a quello delle stelle fisse è anch'esso un sistema di riferimento inerziale.
La Terra comunque viene considerata una buona approssimazione del sistema di riferimento inerziale perché il suo moto avviene molto lentamente.
Il fatto che le leggi della meccanica abbiano sempre la stessa forma nei sistemi di riferimento inerziali, fu un risultato enunciato da Galileo nel \textit{Dialogo sopra i due massimi sistemi del mondo}, in particolare modo nell'esperimento del gran naviglio e che prende il nome di \textbf{principio di relatività galileiana}. Ciò che Galileo vuole dimostrare è che il moto, purché uniforme, ovvero privato di rallentamenti o accelerazioni, corrisponde alla stasi. Per dimostrarlo ricorre appunto all'esempio di una grande imbarcazione: sia che la nave sia ferma, sia che la nave compia un moto uniforme, i movimenti degli oggetti non vincolati alla superficie della nave stessa, come possono essere mosche od oggetti lanciati cui lo stesso Galileo fa riferimento, non subiranno mutamenti, bensì resteranno invariati.
Il fatto che il moto uniforme risulti apparentemente equivalente alla stasi pone come questione importante il fatto dell'impossibilità di trovare un sistema di riferimento assoluto; il che implica, inoltre, che sia l'uomo sia la Terra perdano la loro centralità, in quanto non possono più essere considerati i punti di riferimento centrali. Da qui il nome del principio.


Per studiare la dinamica sono quindi preferibili i sistemi di riferimento inerziali, problema che non ci si è posti studiando invece la cinematica del punto.







































\section{Classificazione delle forze}

Si tende a raggruppare le forze in quattro interazioni fondamentali:
\begin{itemize}
	\item Interazione gravitazionale: forza attrattiva che tende ad attrarre fra di loro tutti i corpi dotati di massa. Essa permette di spiegare il moto dei corpi celesti nel sistema solare e diventa considerevole quando essi hanno una massa molto elevata. L'interazione gravitazionale esercitata fra due corpi $A$ e $B$ è inversamente proporzionale al quadrato della distanza fra il centro dei due corpi ed è direttamente proporzionale al prodotto fra le due masse. C'è una costante di proporzionalità che viene indicata come $G$ o $\gamma$ che prende il nome di \emph{costante di gravitazione universale} ed è un numero che ha un valore molto piccolo, dell'ordine di $10^{-11}$. Ecco perché questa forza è trascurabile per corpi di massa piccola.
	Facendo l'analisi dimensionale:
	\[
		G=6.67\cdot 10^{-11} \, N\,m^2\,kg^{-2}
	\]
	Per dare alla forza carattere dimensionale si va a definire un versore radiale, che punta sempre come la congiungente fra i due corpi con verso diretto all'esterno. Si può riscrivere la relazione vettoriale dicendo che la forza è diretta come il versore radiale ma con verso opposto.
	\begin{equation}
		\vec{F}_G=-\frac{GmM}{r^2}\,\vec{u}_r
	\end{equation}
	\begin{figure}[htpb]
		\centering
		
		\tikzset{every picture/.style={line width=0.75pt}} %set default line width to 0.75pt        

		\begin{tikzpicture}[x=0.75pt,y=0.75pt,yscale=-1,xscale=1]
		%uncomment if require: \path (0,300); %set diagram left start at 0, and has height of 300

		%Straight Lines [id:da8632282549610462] 
		\draw    (141.25,180.75) -- (204.75,180.75) ;
		\draw [shift={(207.75,180.75)}, rotate = 180] [fill={rgb, 255:red, 0; green, 0; blue, 0 }  ][line width=0.08]  [draw opacity=0] (10.72,-5.15) -- (0,0) -- (10.72,5.15) -- (7.12,0) -- cycle    ;
		%Shape: Circle [id:dp9617032101256509] 
		\draw  [draw opacity=0][fill={rgb, 255:red, 155; green, 155; blue, 155 }  ,fill opacity=1 ] (122.25,180.75) .. controls (122.25,170.26) and (130.76,161.75) .. (141.25,161.75) .. controls (151.74,161.75) and (160.25,170.26) .. (160.25,180.75) .. controls (160.25,191.24) and (151.74,199.75) .. (141.25,199.75) .. controls (130.76,199.75) and (122.25,191.24) .. (122.25,180.75) -- cycle ;
		%Straight Lines [id:da06892075494490246] 
		\draw    (237.75,180.75) -- (301.25,180.75) ;
		\draw [shift={(234.75,180.75)}, rotate = 0] [fill={rgb, 255:red, 0; green, 0; blue, 0 }  ][line width=0.08]  [draw opacity=0] (10.72,-5.15) -- (0,0) -- (10.72,5.15) -- (7.12,0) -- cycle    ;
		%Shape: Circle [id:dp38068629635870566] 
		\draw  [draw opacity=0][fill={rgb, 255:red, 155; green, 155; blue, 155 }  ,fill opacity=1 ] (282.25,180.75) .. controls (282.25,170.26) and (290.76,161.75) .. (301.25,161.75) .. controls (311.74,161.75) and (320.25,170.26) .. (320.25,180.75) .. controls (320.25,191.24) and (311.74,199.75) .. (301.25,199.75) .. controls (290.76,199.75) and (282.25,191.24) .. (282.25,180.75) -- cycle ;
		%Straight Lines [id:da10419761541948369] 
		\draw    (141.25,143.75) -- (301.25,143.75) ;
		\draw [shift={(301.25,143.75)}, rotate = 180] [color={rgb, 255:red, 0; green, 0; blue, 0 }  ][line width=0.75]    (0,5.59) -- (0,-5.59)   ;
		\draw [shift={(141.25,143.75)}, rotate = 180] [color={rgb, 255:red, 0; green, 0; blue, 0 }  ][line width=0.75]    (0,5.59) -- (0,-5.59)   ;

		% Text Node
		\draw (223,131) node    {$r$};

		\end{tikzpicture}
	\end{figure}
	
	Dove il segno meno sta ad indicare che la forza è attrattiva.
	\item Interazione elettromagnetica: è una forza molto simile alla precedente ma che dipende dalla carica dei corpi. In particolare corpi carichi con lo stesso segno si respingono, corpi carichi con diverso segno si attraggono. È una forza che nel tempo dà luogo alle onde elettromagnetiche e per questo è estremamente importante.
	\item Interazione nucleare forte. Quando due corpi dotati della stessa carica sono vicini si respingono.  Questa forza permette di mantenere la stabilità del nucleo, attrae i protoni fra di loro in modo tale che l'atomo rimanga compatto e ha un raggio di interazione estremamente corto. Quindi fuori dai nuclei praticamente è inesistente.
	\item Interazione nucleare debole: interazione fra gli elettroni che permette di spiegare l'esistenza dei decadimenti radioattivi, quando un atomo instabile emette una radiazione e diventa un altro elemento.
\end{itemize}
Tutte le altre forze sono casi particolari che rientrano in queste categorie. La scoperta che l'apparente grande varietà di tipi di forze è il manifestarsi di poche interazioni fondamentali, assume una grande rilevanza concettuale ed è il risultato di una lunga indagine sperimentale e teorica rivolta all'unificazione delle interazioni.

\subsection{Esempi di forze}

Le forze possono essere classificate in attive e reattive (o vincolari).

\subsubsection{Forze attive}

Si tratta di quelle forze che non sono dovute a un vincolo.

\paragraph{Forza peso} Avendo introdotto l'interazione gravitazionale, è possibile capire perché tutti i corpi sulla superficie terrestre sono accelerati verso il centro della Terra. Essi infatti risentono dell'attrazione di quest'ultima che ha una massa molto grande. Questa forza attrattiva prende il nome di forza peso. Essa si considera costante in modulo per tutti i corpi prossimi alla superficie terrestre perché la loro distanza da essa è molto piccola in confronto al suo raggio. Viene definita una costante $g$ che prende il nome di accelerazione di gravità perché il suo prodotto per la massa da una forza. Si trova che $g=9.81 \frac{m}{s^2}$. Tutti i corpi sono accelerati verso il centro della Terra con intensità proporzionale alla propria massa. Per trasformarla in una relazione vettoriale si definisce $\vec{g}$ come un vettore che punta sempre verso il centro della Terra e che quindi permette di dire che la forza peso vale, per la seconda legge della dinamica:
\[
	\vec{F}_{peso}=m\vec{g}
\]
Un corpo che cade nell'aria in realtà presenta un'accelerazione minore di quella di gravità a causa dell'attrito con l'aria. La proporzionalità fra peso e massa suggerisce che il confronto tra due masse possa essere effettuato confrontando le rispettive forze peso. Tale fatto non deve però portare a confondere i due concetti di massa e forza peso. La prima ha significato dinamico indipendente dalla forza agente, mentre la seconda risulta dall'interazione di un corpo con la Terra. Sulla superficie di un altro pianeta essa sarebbe diversa a causa del diverso valore di $g$.

\paragraph{Forza elettrica} Dati due corpi dotati di carica $q_1$ e $q_2$ si esercita una forza fra di essi inversamente proporzionale al quadrato della loro distanza e proporzionale al prodotto delle loro cariche secondo una costante di proporzionalità $k$. La si esprime tramite il versore radiale. Se i due corpi sono dotati della stessa carica, la forza diventa equiversa a $\vec{u}_r$. Essa è responsabile di molti fenomeni osservati in natura.

\paragraph{Forza elastica} Quando un corpo è in grado di allungarsi, nasce una forza di richiamo che prende il nome di \emph{forza elastica}, la quale vorrebbe riportare il corpo nella sua posizione di riposo. Si può dimostrare che questa forza elastica è direttamente proporzionale all'allungamento $\vec{\Delta L}$ tramite una costante di proporzionalità detta costante elastica della molla.
\[
	\vec{F}=-k\vec{\Delta L}
\]
Il segno meno nasce perché la forza agisce in direzione parallela all'allungamento ma con verso opposto ad esso per richiamare il corpo indietro. Il segno invece non va cambiato se la molla viene accorciata.
\begin{figure}[htpb]
	\centering
	

	\tikzset{every picture/.style={line width=0.75pt}} %set default line width to 0.75pt        

	\begin{tikzpicture}[x=0.75pt,y=0.75pt,yscale=-1,xscale=1]
	%uncomment if require: \path (0,300); %set diagram left start at 0, and has height of 300

	%Shape: Axis 2D [id:dp001535704915357261] 
	\draw  (106,245.4) -- (335.5,245.4)(128.95,51) -- (128.95,267) (328.5,240.4) -- (335.5,245.4) -- (328.5,250.4) (123.95,58) -- (128.95,51) -- (133.95,58)  ;
	%Shape: Spring [id:dp05311315184863874] 
	\draw   (129,114) .. controls (131.25,104) and (137.25,94) .. (149.25,94) .. controls (173.25,94) and (173.25,134) .. (167.25,134) .. controls (161.25,134) and (161.25,94) .. (185.25,94) .. controls (209.25,94) and (209.25,134) .. (203.25,134) .. controls (197.25,134) and (197.25,94) .. (221.25,94) .. controls (245.25,94) and (245.25,134) .. (239.25,134) .. controls (233.25,134) and (233.25,94) .. (257.25,94) .. controls (258.39,94) and (259.47,94.09) .. (260.5,94.26) ;
	%Shape: Spring [id:dp7319432525707565] 
	\draw   (129,214) .. controls (130.13,204) and (133.88,194) .. (141.38,194) .. controls (156.38,194) and (156.38,234) .. (150.38,234) .. controls (144.38,234) and (144.38,194) .. (159.38,194) .. controls (174.38,194) and (174.38,234) .. (168.38,234) .. controls (162.38,234) and (162.38,194) .. (177.38,194) .. controls (192.38,194) and (192.38,234) .. (186.38,234) .. controls (180.38,234) and (180.38,194) .. (195.38,194) .. controls (198.85,194) and (201.52,196.15) .. (203.5,199.45) ;
	%Straight Lines [id:da6026322365253902] 
	\draw    (129.25,78.75) -- (257.5,78.75) ;
	\draw [shift={(260.5,78.75)}, rotate = 180] [fill={rgb, 255:red, 0; green, 0; blue, 0 }  ][line width=0.08]  [draw opacity=0] (10.72,-5.15) -- (0,0) -- (10.72,5.15) -- (7.12,0) -- cycle    ;
	%Straight Lines [id:da9991989473762817] 
	\draw    (129.25,181.75) -- (199.5,181.75) ;
	\draw [shift={(202.5,181.75)}, rotate = 180] [fill={rgb, 255:red, 0; green, 0; blue, 0 }  ][line width=0.08]  [draw opacity=0] (10.72,-5.15) -- (0,0) -- (10.72,5.15) -- (7.12,0) -- cycle    ;
	%Straight Lines [id:da5075934876195309] 
	\draw    (202.5,181.75) -- (256.75,181.75) ;
	\draw [shift={(259.75,181.75)}, rotate = 180] [fill={rgb, 255:red, 0; green, 0; blue, 0 }  ][line width=0.08]  [draw opacity=0] (10.72,-5.15) -- (0,0) -- (10.72,5.15) -- (7.12,0) -- cycle    ;

	% Text Node
	\draw (347,249) node    {$x$};
	% Text Node
	\draw (113,50) node    {$y$};
	% Text Node
	\draw (197,62) node    {$\vec{L}$};
	% Text Node
	\draw (167.06,165) node    {$\vec{L}_{0}$};
	% Text Node
	\draw (228.6,162.6) node    {$\Delta \vec{L}$};


	\end{tikzpicture}
\end{figure}
Si consideri una molla: quando $k$ è alta vuole dire che a pari allungamento essa genererà una forza di richiamo notevole (e che si dovrà applicare una forza notevole per deformarla). In questo caso si parla di \textbf{molla rigida}. Le molle che si deformano facilmente prendono invece il nome di \textbf{molle lasche}.

\paragraph{Tensione dei fili} Tra le forze attive figura anche la tensione, interazione che si manifesta quando si collegano i corpi con funi, che verranno considerate prive di estensione. Una fune è un oggetto che per definizione può sopportare solo forze di trazione; se viene compressa si deforma. La tensione si ha quando la fune è tesa ed è una forza diretta parallelamente alla direzione della fune stessa. In genere viene indicata con $\vec{T}$.  Si considererà in seguito una fune:
\begin{itemize}
	\item di massa trascurabile, approssimabile a $0$. Questa idealizzazione ha un'importante conseguenza:
	\[
		\vec{T}_b-\vec{T}_a=m_{\text{fune}}\,\vec{a}=0 \implies \vec{T}_b=\vec{T}_a
	\]
	Significa che in ogni punto la fune è soggetta alla stessa tensione.
	\item inestensibile, non elastica. Ciò significa che mantiene lunghezza costante e che quindi tutti i suoi punti si muovono con stessa velocità e stessa accelerazione in ogni istante di tempo. Per questo la fune è utile per collegare due corpi e farli muovere alla stessa velocità.
\end{itemize}
\begin{figure}[htpb]
	\centering
	

	\tikzset{every picture/.style={line width=0.75pt}} %set default line width to 0.75pt        

	\begin{tikzpicture}[x=0.75pt,y=0.75pt,yscale=-1,xscale=1]
	%uncomment if require: \path (0,300); %set diagram left start at 0, and has height of 300

	%Straight Lines [id:da4135641760741906] 
	\draw    (198.5,139) -- (413.5,139) ;
	%Straight Lines [id:da41024725268239504] 
	\draw [line width=1.5]    (198.5,139) -- (253.5,139) ;
	\draw [shift={(257.5,139)}, rotate = 180] [fill={rgb, 255:red, 0; green, 0; blue, 0 }  ][line width=0.08]  [draw opacity=0] (13.4,-6.43) -- (0,0) -- (13.4,6.44) -- (8.9,0) -- cycle    ;
	%Straight Lines [id:da6068473195480548] 
	\draw [line width=1.5]    (358.5,139) -- (413.5,139) ;
	\draw [shift={(354.5,139)}, rotate = 0] [fill={rgb, 255:red, 0; green, 0; blue, 0 }  ][line width=0.08]  [draw opacity=0] (13.4,-6.43) -- (0,0) -- (13.4,6.44) -- (8.9,0) -- cycle    ;

	% Text Node
	\draw (230.27,114.56) node    {$\vec{T}_{b}$};
	% Text Node
	\draw (390.27,114.56) node    {$\vec{T}_{a}$};


	\end{tikzpicture}
\end{figure}
Riassumendo, il filo teso esercita agli estremi la tensione $\vec{T}$, il cui valore dipende dalle forze applicate, e che deve essere pensata come la reazione del filo alla forza che lo tende. Per un filo reale la tensione non può superare un valore massimo oltre al quale esso si spezza. Questo valore dipende dalla sostanza con cui è fatto il filo e dalle sue dimensioni geometriche.

Si consideri due corpi collegati ad un filo. Per quanto appena detto, essi presentano la stessa accelerazione. $m_2$ risente anche di una forza esercitata da $m_1$ pari a $\vec{T}$.
\begin{figure}[htpb]
	\centering
	

	\tikzset{every picture/.style={line width=0.75pt}} %set default line width to 0.75pt        

	\begin{tikzpicture}[x=0.75pt,y=0.75pt,yscale=-1,xscale=1]
	%uncomment if require: \path (0,300); %set diagram left start at 0, and has height of 300

	%Shape: Rectangle [id:dp13182710929166452] 
	\draw   (190,110) -- (250,110) -- (250,170) -- (190,170) -- cycle ;
	%Shape: Rectangle [id:dp8276676004174683] 
	\draw   (310,110) -- (370,110) -- (370,170) -- (310,170) -- cycle ;
	%Straight Lines [id:da34170094100934567] 
	\draw [line width=2.25]    (250,140) -- (310,140) ;
	%Straight Lines [id:da48469720672133] 
	\draw [line width=0.75]    (250,130) -- (267,130) ;
	\draw [shift={(270,130)}, rotate = 180] [fill={rgb, 255:red, 0; green, 0; blue, 0 }  ][line width=0.08]  [draw opacity=0] (10.72,-5.15) -- (0,0) -- (10.72,5.15) -- (7.12,0) -- cycle    ;
	%Straight Lines [id:da6470155408923426] 
	\draw [line width=0.75]    (310,130) -- (293,130) ;
	\draw [shift={(290,130)}, rotate = 360] [fill={rgb, 255:red, 0; green, 0; blue, 0 }  ][line width=0.08]  [draw opacity=0] (10.72,-5.15) -- (0,0) -- (10.72,5.15) -- (7.12,0) -- cycle    ;
	%Straight Lines [id:da6637094186690098] 
	\draw [line width=0.75]    (340,140) -- (340,183) ;
	\draw [shift={(340,186)}, rotate = 270] [fill={rgb, 255:red, 0; green, 0; blue, 0 }  ][line width=0.08]  [draw opacity=0] (10.72,-5.15) -- (0,0) -- (10.72,5.15) -- (7.12,0) -- cycle    ;
	%Straight Lines [id:da8958397597628005] 
	\draw [line width=0.75]  [dash pattern={on 0.84pt off 2.51pt}]  (340,140) -- (430,140) ;
	%Straight Lines [id:da2860240635593594] 
	\draw [line width=0.75]    (340,140) -- (427.38,91.46) ;
	\draw [shift={(430,90)}, rotate = 510.95] [fill={rgb, 255:red, 0; green, 0; blue, 0 }  ][line width=0.08]  [draw opacity=0] (10.72,-5.15) -- (0,0) -- (10.72,5.15) -- (7.12,0) -- cycle    ;
	%Straight Lines [id:da12525197401114996] 
	\draw [line width=0.75]    (340,97) -- (340,140) ;
	\draw [shift={(340,94)}, rotate = 90] [fill={rgb, 255:red, 0; green, 0; blue, 0 }  ][line width=0.08]  [draw opacity=0] (10.72,-5.15) -- (0,0) -- (10.72,5.15) -- (7.12,0) -- cycle    ;
	%Straight Lines [id:da264838336587333] 
	\draw [line width=0.75]    (220,117) -- (220,140) ;
	\draw [shift={(220,114)}, rotate = 90] [fill={rgb, 255:red, 0; green, 0; blue, 0 }  ][line width=0.08]  [draw opacity=0] (10.72,-5.15) -- (0,0) -- (10.72,5.15) -- (7.12,0) -- cycle    ;
	%Straight Lines [id:da24672318102456936] 
	\draw [line width=0.75]    (220,140) -- (220,163) ;
	\draw [shift={(220,166)}, rotate = 270] [fill={rgb, 255:red, 0; green, 0; blue, 0 }  ][line width=0.08]  [draw opacity=0] (10.72,-5.15) -- (0,0) -- (10.72,5.15) -- (7.12,0) -- cycle    ;
	%Shape: Circle [id:dp3108503584894222] 
	\draw  [fill={rgb, 255:red, 0; green, 0; blue, 0 }  ,fill opacity=1 ] (217.75,140) .. controls (217.75,138.76) and (218.76,137.75) .. (220,137.75) .. controls (221.24,137.75) and (222.25,138.76) .. (222.25,140) .. controls (222.25,141.24) and (221.24,142.25) .. (220,142.25) .. controls (218.76,142.25) and (217.75,141.24) .. (217.75,140) -- cycle ;
	%Shape: Circle [id:dp06572143906472427] 
	\draw  [fill={rgb, 255:red, 0; green, 0; blue, 0 }  ,fill opacity=1 ] (337.75,140) .. controls (337.75,138.76) and (338.76,137.75) .. (340,137.75) .. controls (341.24,137.75) and (342.25,138.76) .. (342.25,140) .. controls (342.25,141.24) and (341.24,142.25) .. (340,142.25) .. controls (338.76,142.25) and (337.75,141.24) .. (337.75,140) -- cycle ;
	%Shape: Arc [id:dp8957298166828258] 
	\draw  [draw opacity=0] (373.8,121.04) .. controls (376.95,126.64) and (378.75,133.11) .. (378.75,140) -- (340,140) -- cycle ; \draw   (373.8,121.04) .. controls (376.95,126.64) and (378.75,133.11) .. (378.75,140) ;
	%Straight Lines [id:da6733190100538535] 
	\draw [line width=0.75]    (110.5,170) -- (450.5,170) ;

	% Text Node
	\draw (204,136) node    {$m_{2}$};
	% Text Node
	\draw (324.5,138) node    {$m_{1}$};
	% Text Node
	\draw (262.5,109.5) node    {$\vec{T}$};
	% Text Node
	\draw (296.5,109) node    {$\vec{T}$};
	% Text Node
	\draw (352,80) node    {$\vec{R}_{n}$};
	% Text Node
	\draw (232,92) node    {$\vec{R} '_{n}$};
	% Text Node
	\draw (220.5,181) node    {$m_{2}\vec{g}$};
	% Text Node
	\draw (360.5,182.5) node    {$m_{1}\vec{g}$};
	% Text Node
	\draw (446,83.5) node    {$\vec{F}_{0}$};
	% Text Node
	\draw (388.5,124.5) node    {$\vartheta $};


	\end{tikzpicture}
\end{figure}
\begin{gather*}
	F_0\cos\vartheta=(m_1+m_2)\,a \\
	R_n=m_2 g-F_0\sin\vartheta \quad \text{condizione essenziale è che} \quad F_0\sin\vartheta>m_2 g,
\end{gather*}
altrimenti il corpo verrebbe sollevato e non si avrebbe più contatto.

\subsubsection{Reazioni vincolari}

Gli oggetti possono anche interagire essendo soggetti a vincoli: quando il punto materiale è posto ad esempio su un piano, questo impone dei vincoli alla sua posizione. Infatti, se un corpo soggetto all'azione di una forza o della risultante non nulla di un insieme di forze, rimane fermo, bisogna dedurre l'esistenza di una reazione dell'ambiente circostante, detta \textbf{reazione vincolare}, applicata al corpo stesso in modo che rimanga in quiete. I vincoli fanno si che l'oggetto non possa occupare tutte le posizioni possibili nello spazio. Le reazioni vincolari sono quindi forze esercitate dal vincolo sul corpo e con verso sempre opposto al tentativo di spostamento del corpo. Si affrontano in seguito alcuni esempi di reazioni vincolari.

\paragraph{Reazioni normali} Tutte le volte che c'è un vincolo esiste sempre una forza generata da esso al corpo diretta perpendicolarmente al piano d'appoggio: \textbf{reazione normale}. Ad esempio, l'oggetto sarà sempre accelerato verso il centro della Terra, sarà soggetto alla forza peso, ma non sfonderà il piano d'appoggio perché questo si sta comportando da vincolo ed esercita su di esso una forza diretta da se verso l'esterno, ortogonale al piano.
\[
	m\vec{g}-\vec{R}_n=0
\]
\begin{figure}[htpb]
	\centering
	

	\tikzset{every picture/.style={line width=0.75pt}} %set default line width to 0.75pt        

	\begin{tikzpicture}[x=0.75pt,y=0.75pt,yscale=-1,xscale=1]
	%uncomment if require: \path (0,300); %set diagram left start at 0, and has height of 300

	%Shape: Rectangle [id:dp5690079782842519] 
	\draw   (170.5,118) -- (283,118) -- (283,158) -- (170.5,158) -- cycle ;
	%Straight Lines [id:da7512390417384545] 
	\draw    (100,158) -- (353.5,158) ;
	%Straight Lines [id:da45126245871489656] 
	\draw    (226.75,138) -- (226.75,177) ;
	\draw [shift={(226.75,180)}, rotate = 270] [fill={rgb, 255:red, 0; green, 0; blue, 0 }  ][line width=0.08]  [draw opacity=0] (10.72,-5.15) -- (0,0) -- (10.72,5.15) -- (7.12,0) -- cycle    ;
	%Straight Lines [id:da9581308585695909] 
	\draw    (226.75,99) -- (226.75,138) ;
	\draw [shift={(226.75,96)}, rotate = 90] [fill={rgb, 255:red, 0; green, 0; blue, 0 }  ][line width=0.08]  [draw opacity=0] (10.72,-5.15) -- (0,0) -- (10.72,5.15) -- (7.12,0) -- cycle    ;
	%Shape: Circle [id:dp39805459767111984] 
	\draw  [fill={rgb, 255:red, 0; green, 0; blue, 0 }  ,fill opacity=1 ] (224.5,138) .. controls (224.5,136.76) and (225.51,135.75) .. (226.75,135.75) .. controls (227.99,135.75) and (229,136.76) .. (229,138) .. controls (229,139.24) and (227.99,140.25) .. (226.75,140.25) .. controls (225.51,140.25) and (224.5,139.24) .. (224.5,138) -- cycle ;

	% Text Node
	\draw (243.5,97.5) node    {$\vec{R}_n$};
	% Text Node
	\draw (242.5,173) node    {$\vec{P}$};


	\end{tikzpicture}
\end{figure}
\FloatBarrier
Ovviamente esisterà un valore massimo della reazione normale che può essere esercitata dal piano d'appoggio, detto \textbf{carico di rottura}, al di sotto del quale si mantiene $\vec{R}_n$. Si potrebbe pensare che questa reazione normale rappresenti una coppia di forze responsabili del principio azione reazione, ma ciò non può essere vero perché le due forze sono applicate allo stesso corpo.

\subparagraph{Piano inclinato} Si consideri un corpo, assimilabile ad un punto materiale di massa $m$, che possa muoversi sotto l'azione del suo peso, su una superficie piana inclinata di un angolo $\vartheta$ rispetto al piano orizzontale. L'obbiettivo è quello di ricavare la legge oraria del moto. Il corpo sarà sicuramente soggetto all'accelerazione di gravità, ma il vincolo a cui è appoggiato fa si che non cada: esiste sempre una reazione normale generata dal piano d'appoggio. Si supponga che non ci sia attrito. Note queste forze si scrive la seconda legge della dinamica per ricavare la legge oraria. L'oggetto si muove lungo il piano inclinato quindi conviene definire come sistema di riferimento un asse parallelo e uno perpendicolare al piano.
\begin{figure}[htpb]
	\centering
	

	\tikzset{every picture/.style={line width=0.75pt}} %set default line width to 0.75pt        

	\begin{tikzpicture}[x=0.75pt,y=0.75pt,yscale=-1,xscale=1]
	%uncomment if require: \path (0,300); %set diagram left start at 0, and has height of 300

	%Shape: Rectangle [id:dp6463061738178288] 
	\draw  [draw opacity=0][fill={rgb, 255:red, 155; green, 155; blue, 155 }  ,fill opacity=1 ] (311.62,133.16) -- (375.16,162.53) -- (358.38,198.84) -- (294.84,169.47) -- cycle ;
	%Shape: Right Triangle [id:dp6144937690601591] 
	\draw   (100,80) -- (478.98,254) -- (100,254) -- cycle ;
	%Shape: Axis 2D [id:dp5829870105582096] 
	\draw  (161.95,88.82) -- (253.05,130.06)(208.18,10.95) -- (166.94,102.05) (248.74,122.62) -- (253.05,130.06) -- (244.61,131.73) (200.74,15.27) -- (208.18,10.95) -- (209.85,19.39)  ;
	%Straight Lines [id:da9706510721340091] 
	\draw    (335,166) -- (335,230.4) ;
	\draw [shift={(335,233.4)}, rotate = 270] [fill={rgb, 255:red, 0; green, 0; blue, 0 }  ][line width=0.08]  [draw opacity=0] (10.72,-5.15) -- (0,0) -- (10.72,5.15) -- (7.12,0) -- cycle    ;
	%Straight Lines [id:da5886444887800437] 
	\draw    (335,166) -- (310.59,218.82) ;
	\draw [shift={(309.34,221.54)}, rotate = 294.8] [fill={rgb, 255:red, 0; green, 0; blue, 0 }  ][line width=0.08]  [draw opacity=0] (10.72,-5.15) -- (0,0) -- (10.72,5.15) -- (7.12,0) -- cycle    ;
	%Straight Lines [id:da8594044013143456] 
	\draw    (335,166) -- (357.94,176.6) ;
	\draw [shift={(360.66,177.86)}, rotate = 204.8] [fill={rgb, 255:red, 0; green, 0; blue, 0 }  ][line width=0.08]  [draw opacity=0] (10.72,-5.15) -- (0,0) -- (10.72,5.15) -- (7.12,0) -- cycle    ;
	%Straight Lines [id:da05616414654235857] 
	\draw    (359.41,113.18) -- (335,166) ;
	\draw [shift={(360.66,110.46)}, rotate = 114.8] [fill={rgb, 255:red, 0; green, 0; blue, 0 }  ][line width=0.08]  [draw opacity=0] (10.72,-5.15) -- (0,0) -- (10.72,5.15) -- (7.12,0) -- cycle    ;
	%Shape: Arc [id:dp2348034454786576] 
	\draw  [draw opacity=0] (335.06,192.67) .. controls (335.04,192.67) and (335.02,192.67) .. (335,192.67) .. controls (330.94,192.67) and (327.1,191.76) .. (323.66,190.14) -- (335,166) -- cycle ; \draw   (335.06,192.67) .. controls (335.04,192.67) and (335.02,192.67) .. (335,192.67) .. controls (330.94,192.67) and (327.1,191.76) .. (323.66,190.14) ;
	%Shape: Arc [id:dp3867135949818161] 
	\draw  [draw opacity=0] (448.98,253.85) .. controls (449,249.36) and (450.01,245.11) .. (451.79,241.29) -- (478.98,254) -- cycle ; \draw   (448.98,253.85) .. controls (449,249.36) and (450.01,245.11) .. (451.79,241.29) ;
	%Shape: Rectangle [id:dp07831331870274494] 
	\draw  [dash pattern={on 0.84pt off 2.51pt}] (335,166) -- (360.66,177.86) -- (335,233.4) -- (309.34,221.54) -- cycle ;

	% Text Node
	\draw (221.5,21.5) node    {$y$};
	% Text Node
	\draw (378,118) node    {$\vec{R}_{n}$};
	% Text Node
	\draw (379.5,182.67) node    {$\vec{P}_{x}$};
	% Text Node
	\draw (298.33,221.67) node    {$\vec{P}_{y}$};
	% Text Node
	\draw (434.83,243) node    {$\vartheta $};
	% Text Node
	\draw (265.5,130.5) node    {$x$};
	% Text Node
	\draw (327.17,199.33) node    {$\vartheta $};
	% Text Node
	\draw (347,234) node    {$\vec{P}$};
	% Text Node
	\draw (316.5,147.67) node    {$m$};


	\end{tikzpicture}
\end{figure}
\begin{align*}
	x&: \quad P_x=P\sin\vartheta=mg\sin\vartheta \quad a=g\sin\vartheta=\text{costante} \\
	y&: \quad P_y-R_n=mgsin\vartheta-R_n=0 \quad (\text{non c'è moto sull'asse $y$})
\end{align*}
Si noti che si ha $a=g\sin\vartheta$: il corpo scende con moto uniformemente accelerato (infatti $a$ non dipende dal tempo) e l'accelerazione è minore di quella di gravità.
\[
	\vec{F}=m\vec{a} \implies \begin{cases} mg\sin\vartheta=ma \\ mg\cos\vartheta=R_n \end{cases}
\]

\paragraph{Forza d'attrito} Quanto visto in precedenza è valido se i vincoli sono perfettamente lisci. Se invece ciò non accade, la reazione normale continua ad esistere ma oltre ad essa agisce anche un altra reazione vincolare, questa volta tangente al vincolo, che prende il nome di \textbf{forza di attrito}. Ci sono diverse forme di attrito a seconda delle situazioni e dei fenomeni fisici che si possono considerare:
\begin{itemize}
	\item attrito \emph{radente}: generato dallo strisciamento di un corpo su una superficie senza rotolamento;
	\item attrito \emph{volvente}: che si manifesta in presenza di rotolamento e traslazione;
	\item attrito \emph{viscoso}: generato dal moto di un corpo in un fluido.
\end{itemize}
Si comincia studiando la forza di attrito radente. Ne esistono due tipologie:
\begin{itemize}
	\item statico. Si oppone al tentativo di movimento e permette a un corpo soggetto a forze di rimanere in equilibrio. A livello microscopico accade che tra i due corpi, le cariche presenti sulla superficie si attraggono e le zone rugose, grazie all'attrazione elettrostatica, vanno a creare delle microfusioni sulla superficie di contatto. Se il corpo non si muove significa che non si sono spezzate queste microfusioni. Si può dimostrare che la forza di attrito radente statico può assumere tutti valori possibili al i sotto di un valore che corrisponde alla situazione in cui si vanno a rompere le microfusioni che si formano. Questo valore massimo, nel caso dell'attrito statico, non dipende dall'estensione della superficie di contatto ma da quanto sono rugosi i due corpi interessati, aspetto che si caratterizza con il \emph{coefficiente di attrito statico} $\mu_s$, e da quanto il corpo sta scaricando sul piano il suo peso.
	\begin{gather*}
		\vec{R}_{t, max}=-\mu_s\,\norma{\vec{R}_n}\,\vec{u}_t \\
		\norma{\vec{F}_0}<\norma{\vec{R}_{t, max}} \implies \text{il corpo è fermo}
	\end{gather*}
	\item dinamico. Si ha quando si applica una forza maggiore di $\vec{R}_{t, max}$ e il corpo di conseguenza si muove. Anche in questo caso $\vec{R}(t)$ non è un valore noto a priori. In condizioni dinamiche il modulo di $\vec{R}(t)$ è pari a:
	\[
		\norma{\vec{R}_t}=\mu_d\,\norma{\vec{R}_{n}}
	\]
	Dove $\mu_d$ è il coefficiente di attrito dinamico. Si può in generale affermare che il valore del coefficiente di attrito dinamico è inferiore a quello del coefficiente di attrito statico. Questo a indice del fatto che è più facile mantenere in movimento un corpo su un piano scabro che metterlo in moto. Infatti mentre il corpo si muove si continuano a rompere e formare le microfusioni.
\end{itemize}
\begin{figure}[htpb]
	\centering
	

	\tikzset{every picture/.style={line width=0.75pt}} %set default line width to 0.75pt        

	\begin{tikzpicture}[x=0.75pt,y=0.75pt,yscale=-1,xscale=1]
	%uncomment if require: \path (0,300); %set diagram left start at 0, and has height of 300

	%Shape: Rectangle [id:dp6376951996226576] 
	\draw  [draw opacity=0][fill={rgb, 255:red, 155; green, 155; blue, 155 }  ,fill opacity=1 ] (256.5,88) -- (393.5,88) -- (393.5,190) -- (256.5,190) -- cycle ;
	%Straight Lines [id:da8345620016239454] 
	\draw    (322,31) -- (322,139) ;
	\draw [shift={(322,28)}, rotate = 90] [fill={rgb, 255:red, 0; green, 0; blue, 0 }  ][line width=0.08]  [draw opacity=0] (10.72,-5.15) -- (0,0) -- (10.72,5.15) -- (7.12,0) -- cycle    ;
	%Shape: Axis 2D [id:dp5030880839594016] 
	\draw  (126,176.7) -- (179,176.7)(131.3,129) -- (131.3,182) (172,171.7) -- (179,176.7) -- (172,181.7) (126.3,136) -- (131.3,129) -- (136.3,136)  ;
	%Straight Lines [id:da0665208633718779] 
	\draw    (120,190) -- (452.5,190) ;
	%Straight Lines [id:da5325951964673605] 
	\draw    (325,139) -- (325,227) ;
	\draw [shift={(325,230)}, rotate = 270] [fill={rgb, 255:red, 0; green, 0; blue, 0 }  ][line width=0.08]  [draw opacity=0] (10.72,-5.15) -- (0,0) -- (10.72,5.15) -- (7.12,0) -- cycle    ;
	%Straight Lines [id:da2420420382705757] 
	\draw    (325,139) -- (424.5,139) ;
	\draw [shift={(427.5,139)}, rotate = 180] [fill={rgb, 255:red, 0; green, 0; blue, 0 }  ][line width=0.08]  [draw opacity=0] (10.72,-5.15) -- (0,0) -- (10.72,5.15) -- (7.12,0) -- cycle    ;
	%Straight Lines [id:da8646744635567805] 
	\draw    (225.5,139) -- (325,139) ;
	\draw [shift={(222.5,139)}, rotate = 0] [fill={rgb, 255:red, 0; green, 0; blue, 0 }  ][line width=0.08]  [draw opacity=0] (10.72,-5.15) -- (0,0) -- (10.72,5.15) -- (7.12,0) -- cycle    ;
	%Straight Lines [id:da6574678032655001] 
	\draw    (325,139) -- (425.02,70.69) ;
	\draw [shift={(427.5,69)}, rotate = 505.67] [fill={rgb, 255:red, 0; green, 0; blue, 0 }  ][line width=0.08]  [draw opacity=0] (10.72,-5.15) -- (0,0) -- (10.72,5.15) -- (7.12,0) -- cycle    ;
	%Straight Lines [id:da317383377325658] 
	\draw    (325,139) -- (324.52,72) ;
	\draw [shift={(324.5,69)}, rotate = 449.59] [fill={rgb, 255:red, 0; green, 0; blue, 0 }  ][line width=0.08]  [draw opacity=0] (10.72,-5.15) -- (0,0) -- (10.72,5.15) -- (7.12,0) -- cycle    ;
	%Straight Lines [id:da8219185361646106] 
	\draw  [dash pattern={on 0.84pt off 2.51pt}]  (325.5,69) -- (427.5,69) ;
	%Straight Lines [id:da32767482553672544] 
	\draw  [dash pattern={on 0.84pt off 2.51pt}]  (427.5,69) -- (427.5,139) ;

	% Text Node
	\draw (305,29) node    {$\vec{R}_{n}$};
	% Text Node
	\draw (439.5,58.5) node    {$\vec{F}$};
	% Text Node
	\draw (443.5,134.5) node    {$\vec{F}_{x}$};
	% Text Node
	\draw (307.5,68.5) node    {$\vec{F}_{y}$};
	% Text Node
	\draw (210,132.5) node    {$\vec{F}_{d}$};
	% Text Node
	\draw (345.5,217.5) node    {$m\vec{g}$};
	% Text Node
	\draw (190,174.5) node    {$x$};
	% Text Node
	\draw (122.5,118) node    {$y$};


	\end{tikzpicture}
\end{figure}
Il coefficiente $\mu$ è un valore adimensionale perché lega una forza con una forza.

Come già anticipato, un corpo solido può muoversi anche in un fluido, sostanza gassosa o liquida. In questo tentativo di movimento, nasce una forza di reazione vincolare tangente al moto che prende il nome di \textbf{forza di attrito viscoso}.
\begin{figure}[htpb]
	\centering
	

	\tikzset{every picture/.style={line width=0.75pt}} %set default line width to 0.75pt        

	\begin{tikzpicture}[x=0.75pt,y=0.75pt,yscale=-1,xscale=1]
	%uncomment if require: \path (0,300); %set diagram left start at 0, and has height of 300

	%Shape: Rectangle [id:dp8561184232162118] 
	\draw  [draw opacity=0][fill={rgb, 255:red, 212; green, 212; blue, 212 }  ,fill opacity=1 ] (169,89.5) -- (321.5,89.5) -- (321.5,212) -- (169,212) -- cycle ;
	%Straight Lines [id:da7719242425976345] 
	\draw    (169,212) -- (321.5,212) ;
	%Straight Lines [id:da5058462018381182] 
	\draw    (169,212) -- (169,75) ;
	%Straight Lines [id:da6164225160919097] 
	\draw    (321.5,212) -- (321.5,75) ;
	%Shape: Circle [id:dp3166621773670002] 
	\draw  [draw opacity=0][fill={rgb, 255:red, 155; green, 155; blue, 155 }  ,fill opacity=1 ] (232.5,151) .. controls (232.5,143.96) and (238.21,138.25) .. (245.25,138.25) .. controls (252.29,138.25) and (258,143.96) .. (258,151) .. controls (258,158.04) and (252.29,163.75) .. (245.25,163.75) .. controls (238.21,163.75) and (232.5,158.04) .. (232.5,151) -- cycle ;
	%Straight Lines [id:da8089251880144308] 
	\draw    (245.25,151) -- (245.25,194) ;
	\draw [shift={(245.25,197)}, rotate = 270] [fill={rgb, 255:red, 0; green, 0; blue, 0 }  ][line width=0.08]  [draw opacity=0] (10.72,-5.15) -- (0,0) -- (10.72,5.15) -- (7.12,0) -- cycle    ;
	%Straight Lines [id:da7454704616065833] 
	\draw    (245.25,108) -- (245.25,151) ;
	\draw [shift={(245.25,105)}, rotate = 90] [fill={rgb, 255:red, 0; green, 0; blue, 0 }  ][line width=0.08]  [draw opacity=0] (10.72,-5.15) -- (0,0) -- (10.72,5.15) -- (7.12,0) -- cycle    ;
	%Shape: Circle [id:dp7293908222637295] 
	\draw  [draw opacity=0][fill={rgb, 255:red, 0; green, 0; blue, 0 }  ,fill opacity=1 ] (243.17,151) .. controls (243.17,149.85) and (244.1,148.92) .. (245.25,148.92) .. controls (246.4,148.92) and (247.33,149.85) .. (247.33,151) .. controls (247.33,152.15) and (246.4,153.08) .. (245.25,153.08) .. controls (244.1,153.08) and (243.17,152.15) .. (243.17,151) -- cycle ;
	%Straight Lines [id:da16874404501894302] 
	\draw    (194.75,126) -- (194.75,169) ;
	\draw [shift={(194.75,172)}, rotate = 270] [fill={rgb, 255:red, 0; green, 0; blue, 0 }  ][line width=0.08]  [draw opacity=0] (10.72,-5.15) -- (0,0) -- (10.72,5.15) -- (7.12,0) -- cycle    ;

	% Text Node
	\draw (264.5,121.5) node    {$\vec{F}_{2}$};
	% Text Node
	\draw (267.5,179) node    {$m\vec{g}$};
	% Text Node
	\draw (205,143) node    {$\vec{v}$};


	\end{tikzpicture}
\end{figure}
\FloatBarrier
Essa è proporzionale alla velocità del corpo soggetto a tale forza:
\[
	\vec{F}_{AV}=-\beta\vec{v} \implies \vec{a}=-\beta\frac{\vec{v}}{m}
\]
$\beta$ è il \emph{coefficiente di attrito viscoso} e dipende dalla forma del corpo, dalla sua velocità e dalla densità del fluido. Se la caduta del corpo avviene in un fluido non viscoso si avrà un moto rettilineo uniforme; altrimenti,  viene a nascere la forza di attrito viscoso che sottrae velocità; in particolar modo questa decresce in modo esponenziale nel tempo. Applicando infatti la legge di Newton,  posto $\beta=mk$ si ha:
\begin{gather*}
	\vec{F}_1+\vec{F}_2=m\vec{g}-mk\vec{v}=m\vec{a}=m\frac{d\vec{v}}{dt} \\
	\frac{dv}{dt}=g-kv \implies \frac{dv}{g-kv}=dt \implies t=\frac{\ln(g-kv)}{-k} \\
	\implies v(t)=\frac{g}{k}(1-e^{-kt})
\end{gather*}

Partendo da $0$ la velocità cresce, ma sempre più lentamente, per $t\gg\frac{1}{k}$, $v$ assume praticamente il valore costante $\frac{g}{k}$. In effetti si vede che per $v=\frac{g}{k}$ l'accelerazione diventa nulla. Questo risultato asintotico si ottiene anche considerando come varia con la velocità il modulo delle forze agenti: la forza peso è costante, mentre quella di attrito viscoso cresce linearmente con la velocità. Quando $v$ assume il valore $\frac{g}{k}$ si ha l'equilibrio dinamico tra le due forze e la loro risultante si annulla, di conseguenza si instaura un moto uniforme.
\begin{figure}[htpb]
	\centering
	

	\tikzset{every picture/.style={line width=0.75pt}} %set default line width to 0.75pt        

	\begin{tikzpicture}[x=0.75pt,y=0.75pt,yscale=-1,xscale=1]
	%uncomment if require: \path (0,300); %set diagram left start at 0, and has height of 300

	%Straight Lines [id:da984137023687973] 
	\draw [color={rgb, 255:red, 155; green, 155; blue, 155 }  ,draw opacity=1 ]   (79.35,223.2) -- (128.33,121.33) ;
	%Shape: Axis 2D [id:dp5274226497252805] 
	\draw  (50,223.2) -- (343.5,223.2)(79.35,54) -- (79.35,242) (336.5,218.2) -- (343.5,223.2) -- (336.5,228.2) (74.35,61) -- (79.35,54) -- (84.35,61)  ;
	%Straight Lines [id:da9759819044033384] 
	\draw  [dash pattern={on 0.84pt off 2.51pt}]  (79.5,122) -- (320,122) ;
	% Plotting does not support converting to Tikz
	%Curve Lines [id:da034935606148249976] 
	\draw [line width=1.5]    (79.35,223.2) .. controls (127.67,119.33) and (198.33,132) .. (317,124.67) ;

	% Text Node
	\draw (106,59) node    {$v( t)$};
	% Text Node
	\draw (358,225) node    {$t$};
	% Text Node
	\draw (65,120) node    {$\frac{g}{k}$};


	\end{tikzpicture}
\end{figure}
\FloatBarrier

\subparagraph{Piano inclinato con attrito} Si riprenda il problema del corpo sul piano inclinato supponendo che esso sia scabro. Si immagini di poterne variare l'inclinazione a proprio piacimento. L'obbiettivo è di cercare l'angolo $\vartheta$ massimo al di sopra del quale il corpo si muove, mostrando che esso non dipende dalla massa del corpo.
\[
	\norma{\vec{P}_x} \le \norma{\vec{F}_d} = \norma{\vec{P}_y}\mu_s 	\implies mg\sin\vartheta\le mg\cos\vartheta \mu_s
\]
\begin{equation}
	\label{eqn:attrito}
	\tan\vartheta \le \mu_s
\end{equation}
\begin{figure}[htpb]
	\centering
	

	\tikzset{every picture/.style={line width=0.75pt}} %set default line width to 0.75pt        

	\begin{tikzpicture}[x=0.75pt,y=0.75pt,yscale=-1,xscale=1]
	%uncomment if require: \path (0,300); %set diagram left start at 0, and has height of 300

	%Shape: Rectangle [id:dp9393907316583523] 
	\draw  [draw opacity=0][fill={rgb, 255:red, 155; green, 155; blue, 155 }  ,fill opacity=1 ] (279.55,99.37) -- (359.86,136.48) -- (338.66,182.37) -- (258.35,145.26) -- cycle ;
	%Shape: Right Triangle [id:dp4374809228121763] 
	\draw   (150.5,95.72) -- (491.06,252.09) -- (150.5,252.09) -- cycle ;
	%Straight Lines [id:da23541904718549822] 
	\draw    (309.11,140.87) -- (309.11,223.05) ;
	\draw [shift={(309.11,226.05)}, rotate = 270] [fill={rgb, 255:red, 0; green, 0; blue, 0 }  ][line width=0.08]  [draw opacity=0] (10.72,-5.15) -- (0,0) -- (10.72,5.15) -- (7.12,0) -- cycle    ;
	%Straight Lines [id:da8160650307927853] 
	\draw    (309.11,140.87) -- (338.82,154.6) ;
	\draw [shift={(341.54,155.86)}, rotate = 204.8] [fill={rgb, 255:red, 0; green, 0; blue, 0 }  ][line width=0.08]  [draw opacity=0] (10.72,-5.15) -- (0,0) -- (10.72,5.15) -- (7.12,0) -- cycle    ;
	%Straight Lines [id:da4458617710302386] 
	\draw    (309.11,140.87) -- (277.93,208.34) ;
	\draw [shift={(276.67,211.07)}, rotate = 294.8] [fill={rgb, 255:red, 0; green, 0; blue, 0 }  ][line width=0.08]  [draw opacity=0] (10.72,-5.15) -- (0,0) -- (10.72,5.15) -- (7.12,0) -- cycle    ;
	%Straight Lines [id:da0008803172745237564] 
	\draw    (340.28,73.4) -- (309.11,140.87) ;
	\draw [shift={(341.54,70.68)}, rotate = 114.8] [fill={rgb, 255:red, 0; green, 0; blue, 0 }  ][line width=0.08]  [draw opacity=0] (10.72,-5.15) -- (0,0) -- (10.72,5.15) -- (7.12,0) -- cycle    ;
	%Shape: Arc [id:dp3140052717514066] 
	\draw  [draw opacity=0] (308.94,173.75) .. controls (304.05,173.73) and (299.41,172.63) .. (295.24,170.69) -- (309.11,140.87) -- cycle ; \draw   (308.94,173.75) .. controls (304.05,173.73) and (299.41,172.63) .. (295.24,170.69) ;
	%Shape: Arc [id:dp02760366689875271] 
	\draw  [draw opacity=0] (453.15,251.89) .. controls (453.18,246.22) and (454.45,240.85) .. (456.71,236.02) -- (491.06,252.09) -- cycle ; \draw   (453.15,251.89) .. controls (453.18,246.22) and (454.45,240.85) .. (456.71,236.02) ;
	%Straight Lines [id:da20547538831189271] 
	\draw    (279.4,127.14) -- (309.11,140.87) ;
	\draw [shift={(276.67,125.89)}, rotate = 24.8] [fill={rgb, 255:red, 0; green, 0; blue, 0 }  ][line width=0.08]  [draw opacity=0] (10.72,-5.15) -- (0,0) -- (10.72,5.15) -- (7.12,0) -- cycle    ;
	%Shape: Rectangle [id:dp5397173605226044] 
	\draw  [color={rgb, 255:red, 0; green, 0; blue, 0 }  ,draw opacity=1 ][dash pattern={on 0.84pt off 2.51pt}] (309.11,140.87) -- (341.54,155.86) -- (309.11,226.05) -- (276.67,211.07) -- cycle ;

	% Text Node
	\draw (356.92,75.24) node    {$\vec{R}_{n}$};
	% Text Node
	\draw (363.87,160.86) node    {$\vec{P}_{x}$};
	% Text Node
	\draw (265.35,215.41) node    {$\vec{P}_{y}$};
	% Text Node
	\draw (435.86,238.09) node    {$\vartheta $};
	% Text Node
	\draw (256.27,113.06) node    {$\vec{F}_{d}$};
	% Text Node
	\draw (342.37,229.86) node    {$\vec{P} =m\vec{g}$};
	% Text Node
	\draw (299.36,184.59) node    {$\vartheta $};
	% Text Node
	\draw (301.36,119.09) node    {$m$};


	\end{tikzpicture}
\end{figure}
Se invece il corpo all'istante iniziale sta scendendo lungo il piano con velocità $v$, se non si verifica la ~\eqref{eqn:attrito}, vi sarà moto uniformemente accelerato, altrimenti si fermerà. Infine se $\tan\vartheta=\mu_s$, il moto sarà uniforme. L'unica differenza rispetto alla partenza da fermo è che si può avere moto.







































\section{Momento della forza e momento angolare}

Si introduce ora un concetto che sarà utile quando si affronterà lo studio della gravitazione universale (cap. 6). Nello studiare il moto di un punto materiale su una traiettoria si è osservata come grandezza cinematica la velocità vettoriale del punto, che può cambiare modulo, direzione e verso a causa dell'effetto delle forze. Ad esso si è attribuita anche una grandezza detta quantità di moto, che potrà variare (immaginando il caso in cui la massa inerziale è costante) in modulo, direzione e verso a causa dell'effetto della risultante delle forze. In inglese la quantità di moto è anche detta \textit{linear momentum}, momento lineare. Per capire meglio il significato di questo nome, si introduce il concetto di \textbf{momento angolare}, grandezza che risulterà particolarmente utile nei casi in cui il punto materiale si muove su di una traiettoria che gira attorno a un punto (potrebbe trattarsi di una traiettoria ellittica o parabolica).
\begin{figure}[htpb]
	\centering
	

	\tikzset{every picture/.style={line width=0.75pt}} %set default line width to 0.75pt        

	\begin{tikzpicture}[x=0.75pt,y=0.75pt,yscale=-1,xscale=1]
	%uncomment if require: \path (0,300); %set diagram left start at 0, and has height of 300

	% Plotting does not support converting to Tikz
	%Shape: Axis 2D [id:dp7726038513129663] 
	\draw  (128,195) -- (353.5,195)(241.5,27) -- (241.5,212) (346.5,190) -- (353.5,195) -- (346.5,200) (236.5,34) -- (241.5,27) -- (246.5,34)  ;
	%Shape: Circle [id:dp5040305763016306] 
	\draw  [fill={rgb, 255:red, 0; green, 0; blue, 0 }  ,fill opacity=1 ] (239.6,140.8) .. controls (239.6,139.58) and (240.58,138.6) .. (241.8,138.6) .. controls (243.02,138.6) and (244,139.58) .. (244,140.8) .. controls (244,142.02) and (243.02,143) .. (241.8,143) .. controls (240.58,143) and (239.6,142.02) .. (239.6,140.8) -- cycle ;
	%Straight Lines [id:da6870337422019563] 
	\draw [line width=1.5]    (241.8,140.8) -- (267.88,170.97) ;
	\draw [shift={(270.5,174)}, rotate = 229.16] [fill={rgb, 255:red, 0; green, 0; blue, 0 }  ][line width=0.08]  [draw opacity=0] (13.4,-6.43) -- (0,0) -- (13.4,6.44) -- (8.9,0) -- cycle    ;
	%Straight Lines [id:da9749674177874588] 
	\draw [line width=1.5]    (270.5,174) -- (313.46,121.76) ;
	\draw [shift={(316,118.67)}, rotate = 489.43] [fill={rgb, 255:red, 0; green, 0; blue, 0 }  ][line width=0.08]  [draw opacity=0] (13.4,-6.43) -- (0,0) -- (13.4,6.44) -- (8.9,0) -- cycle    ;
	%Curve Lines [id:da6365665117119264] 
	\draw [line width=1.5]    (168.5,73.25) .. controls (220.5,235.25) and (260,235.25) .. (313,73.75) ;

	% Text Node
	\draw (281.83,178.33) node    {$P$};
	% Text Node
	\draw (329.4,118.77) node    {$\vec{v}$};
	% Text Node
	\draw (263.67,134.97) node    {$\vec{r}^{( o)}$};


	\end{tikzpicture}
\end{figure}
\FloatBarrier
\vspace*{-2cm}
Il punto attorno al quale il punto materiale si muove prende il nome di \textbf{polo}. Il momento angolare è una grandezza vettoriale legata alla velocità del punto, in genere indicata con la lettera $\vec{L}$.
$\vec{L}^{(o)}$ si definisce come il prodotto vettoriale fra il vettore $\vec{r}_0$ e la quantità di moto del punto ed è sempre espresso rispetto al polo attorno a cui esso ruota.
\[
	\boxed{\vec{L}^{(o)}=\vec{r}\,^{(o)}\times m\vec{v}}
\]
È un vettore ortogonale al piano del moto, individuato da $\vec{r}\,^{(o)}$ e $\vec{v}$. Si chiama momento angolare perché è evidente che questa quantità posseduta dal punto materiale è non nulla tutte le volte che esso ha una velocità che presenta una componente ortogonale al vettore $\vec{r}\,^{(o)}$, è questa componente infatti che gli permette di ruotare attorno al polo. La componente parallela invece rappresenta la tendenza del punto ad avvicinarsi al polo $O$. Si osservi che il momento angolare si contrappone all'effetto che ha la quantità di moto, perché quest'ultima è sempre parallela alla traiettoria.

Così come la quantità di moto varia nel tempo sotto l'azione delle forze che agiscono sul punto $P$, analogamente ci sarà una quantità dinamica che fa variare nel tempo il momento angolare e che quindi entra in gioco quando una forza ha l'effetto di far ruotare il corpo. Se il punto materiale per qualche motivo è vincolato al polo $O$, tramite ad esempio un'asta rigida, l'effetto della forza $\vec{F}$ agente sul punto $P$ sarà che esso, non potendo traslare verso l'alto liberamente, ruota attorno al polo $O$. Per individuare questo effetto dinamico di rotazione, si definisce \textbf{momento della forza} rispetto al polo $O$ il prodotto vettoriale fra $\vec{r}\,^{(o)}$ e la forza $\vec{F}$. È quindi un vettore diretto ortogonalmente al piano formato dal vettore $\vec{r}\,^{(o)}$ e dal vettore $\vec{F}$. In seguito si studia come calcolare l'intensità del momento della forza in due modalità differenti.

\paragraph{Primo metodo} Si disegna una retta che ha la stessa direzione di $\vec{F}$ e si valuta la distanza assiale (angolo retto) fra il punto $P$ e questo asse, che ha la stessa direzione di $\vec{F}$. Questo tratto tratteggiato lo si chiama \textbf{braccio della forza}. Si ottiene che l'intensità del momento della forza è pari al prodotto fra l'intensità della forza e il suo braccio.
\begin{figure}[htpb]
	\centering
	

	\tikzset{every picture/.style={line width=0.75pt}} %set default line width to 0.75pt        

	\begin{tikzpicture}[x=0.75pt,y=0.75pt,yscale=-1,xscale=1]
	%uncomment if require: \path (0,300); %set diagram left start at 0, and has height of 300

	%Straight Lines [id:da7921800429795998] 
	\draw    (167.5,242) -- (313.5,121) ;
	%Straight Lines [id:da5767628400815077] 
	\draw  [dash pattern={on 0.84pt off 2.51pt}]  (167.5,242) -- (167.5,103.67) ;
	%Straight Lines [id:da09563530908775442] 
	\draw    (274.83,148.5) -- (274.83,100.33) ;
	\draw [shift={(274.83,151.5)}, rotate = 270] [fill={rgb, 255:red, 0; green, 0; blue, 0 }  ][line width=0.08]  [draw opacity=0] (10.72,-5.15) -- (0,0) -- (10.72,5.15) -- (7.12,0) -- cycle    ;
	%Straight Lines [id:da48703129030454484] 
	\draw  [dash pattern={on 0.84pt off 2.51pt}]  (168,153.33) -- (274.83,153.33) ;
	%Shape: Circle [id:dp7102470019836413] 
	\draw  [fill={rgb, 255:red, 0; green, 0; blue, 0 }  ,fill opacity=1 ] (165.67,242) .. controls (165.67,240.99) and (166.49,240.17) .. (167.5,240.17) .. controls (168.51,240.17) and (169.33,240.99) .. (169.33,242) .. controls (169.33,243.01) and (168.51,243.83) .. (167.5,243.83) .. controls (166.49,243.83) and (165.67,243.01) .. (165.67,242) -- cycle ;
	%Shape: Circle [id:dp5463186831665952] 
	\draw  [fill={rgb, 255:red, 0; green, 0; blue, 0 }  ,fill opacity=1 ] (273,153.33) .. controls (273,152.32) and (273.82,151.5) .. (274.83,151.5) .. controls (275.85,151.5) and (276.67,152.32) .. (276.67,153.33) .. controls (276.67,154.35) and (275.85,155.17) .. (274.83,155.17) .. controls (273.82,155.17) and (273,154.35) .. (273,153.33) -- cycle ;
	%Shape: Arc [id:dp33721727159311543] 
	\draw  [draw opacity=0] (274.83,126) .. controls (274.83,126) and (274.83,126) .. (274.83,126) .. controls (283.24,126) and (290.76,129.8) .. (295.78,135.77) -- (274.83,153.33) -- cycle ; \draw   (274.83,126) .. controls (274.83,126) and (274.83,126) .. (274.83,126) .. controls (283.24,126) and (290.76,129.8) .. (295.78,135.77) ;
	%Straight Lines [id:da12096800312738543] 
	\draw    (168,83.33) -- (274.83,83.33) ;
	\draw [shift={(274.83,83.33)}, rotate = 180] [color={rgb, 255:red, 0; green, 0; blue, 0 }  ][line width=0.75]    (0,5.59) -- (0,-5.59)   ;
	\draw [shift={(168,83.33)}, rotate = 180] [color={rgb, 255:red, 0; green, 0; blue, 0 }  ][line width=0.75]    (0,5.59) -- (0,-5.59)   ;

	% Text Node
	\draw (231.33,212) node    {$\vec{r}^{( o)}$};
	% Text Node
	\draw (166,255.33) node    {$O$};
	% Text Node
	\draw (283.33,162.33) node    {$P$};
	% Text Node
	\draw (262,109.67) node    {$\vec{F}$};
	% Text Node
	\draw (292.33,115.33) node    {$\vartheta $};
	% Text Node
	\draw (221.5,68) node   [align=left] {distanza assiale};


	\end{tikzpicture}
\end{figure}
\FloatBarrier
A pari intensità della forza il momento è più grande se la distanza dal cardine è più lunga e c'è se la forza ha una componente ortogonale al raggio che identifica la posizione del punto $P$ rispetto al polo $O$.
\[
	\boxed{\norma{\vec{M}^{(o)}}= \norma{\vec{r}\,^{(o)}} \norma{\vec{F}} \sin\vartheta}
\]
In caso contrario, la forza ha solo l'effetto di trazione oppure di compressione, non ha momento e quindi non permette all'oggetto di ruotare intorno al polo.

Se si studia il moto del punto $P$ rispetto a un polo si potrò calcolare il momento di ognuna delle forze agenti su di esso. La somma di tutti questi momenti da luogo alla \emph{risultante} dei momenti che stanno agendo sul punto materiale.
\begin{figure}[htpb]
	\centering
	

	\tikzset{every picture/.style={line width=0.75pt}} %set default line width to 0.75pt        

	\begin{tikzpicture}[x=0.75pt,y=0.75pt,yscale=-0.8,xscale=0.8]
	%uncomment if require: \path (0,300); %set diagram left start at 0, and has height of 300

	%Curve Lines [id:da7301824126083205] 
	\draw    (69.5,155) .. controls (95.5,80) and (159.5,196) .. (200,120) .. controls (240.5,44) and (298.5,123) .. (323.5,98) ;
	%Shape: Circle [id:dp27186947828666264] 
	\draw  [fill={rgb, 255:red, 0; green, 0; blue, 0 }  ,fill opacity=1 ] (135.67,72) .. controls (135.67,70.99) and (136.49,70.17) .. (137.5,70.17) .. controls (138.51,70.17) and (139.33,70.99) .. (139.33,72) .. controls (139.33,73.01) and (138.51,73.83) .. (137.5,73.83) .. controls (136.49,73.83) and (135.67,73.01) .. (135.67,72) -- cycle ;
	%Straight Lines [id:da47132651835927253] 
	\draw    (195.32,118.13) -- (137.5,72) ;
	\draw [shift={(197.67,120)}, rotate = 218.58] [fill={rgb, 255:red, 0; green, 0; blue, 0 }  ][line width=0.08]  [draw opacity=0] (10.72,-5.15) -- (0,0) -- (10.72,5.15) -- (7.12,0) -- cycle    ;
	%Shape: Circle [id:dp27888495348889486] 
	\draw  [fill={rgb, 255:red, 0; green, 0; blue, 0 }  ,fill opacity=1 ] (197.67,120) .. controls (197.67,118.99) and (198.49,118.17) .. (199.5,118.17) .. controls (200.51,118.17) and (201.33,118.99) .. (201.33,120) .. controls (201.33,121.01) and (200.51,121.83) .. (199.5,121.83) .. controls (198.49,121.83) and (197.67,121.01) .. (197.67,120) -- cycle ;
	%Straight Lines [id:da3814471042063463] 
	\draw    (199.91,115.19) -- (211.25,32) ;
	\draw [shift={(199.5,118.17)}, rotate = 277.77] [fill={rgb, 255:red, 0; green, 0; blue, 0 }  ][line width=0.08]  [draw opacity=0] (10.72,-5.15) -- (0,0) -- (10.72,5.15) -- (7.12,0) -- cycle    ;
	%Straight Lines [id:da063971460239854] 
	\draw    (204.33,119.96) -- (284.75,119) ;
	\draw [shift={(201.33,120)}, rotate = 359.31] [fill={rgb, 255:red, 0; green, 0; blue, 0 }  ][line width=0.08]  [draw opacity=0] (10.72,-5.15) -- (0,0) -- (10.72,5.15) -- (7.12,0) -- cycle    ;

	% Text Node
	\draw (202.33,133.5) node    {$P$};
	% Text Node
	\draw (122.83,72.5) node    {$O$};
	% Text Node
	\draw (223.83,54) node    {$\vec{F}_{1}$};
	% Text Node
	\draw (268.33,132.5) node    {$\vec{F}_{2}$};
	% Text Node
	\draw (156.33,107) node    {$\vec{r}^{( o)}$};


	\end{tikzpicture}
\end{figure}
\FloatBarrier
\[
	\vec{M}_{\text{TOT}}=\vec{M}^{(o)}_1+\vec{M}^{(o)}_2= \vec{r}\,^{(o)} \times\vec{F}_1+\vec{r}\,^{(o)} \times \vec{F}_2= \vec{r}\,^{(o)} \times (\vec{F}_1+\vec{F}_2)
\]
È utile osservare che poiché il punto materiale è sempre lo stesso, quando si fa la somma si può applicare la proprietà distributiva. Ciò significa che è la stessa cosa sommare i singoli momenti o calcolare la risultante delle forze per poi calcolarne il momento.

\paragraph{Secondo metodo} Si comincia ricordando che le forze hanno l'effetto di far variare la quantità di moto:
\[
	\vec{F}=\frac{d(m\vec{v})}{dt}=\frac{d\vec{p}}{dt}
\]
Data questa espressione si prende il momento angolare e lo si deriva nel tempo.
\[
	\frac{d\vec{L}^{(o)}}{dt}=\frac{d(\vec{r}\,^{(o)}\times m\vec{v})}{dt}=\frac{d(\vec{r}\,^{(o)})}{dt}\times m\vec{v}+\underbrace{\vec{r}\,^{(o)}\times m\frac{d\vec{v}}{dt}}_{\vec{M}^{(o)}}
\]
Si comincia a vedere già meglio il legame fra l'effetto dinamico e la sua causa. $\frac{d\vec{r}}{dt}$ è la velocità del punto materiale osservato da un osservatore solidale al polo $O$. Ciò significa che tale termine non è altro che la velocità del punto $P$ meno la velocità di $O$.
\begin{gather*}
	\vec{v}_{\text{rel}}=\vec{v}-\vec{v}_0 \implies (\vec{v}-\vec{v}_0)\times m\vec{v}=\underbrace{\vec{v}\times m\vec{v}}_{=0}-\vec{v}_0 \times m\vec{v}=-\vec{v}_0 \times m\vec{v} \\
	\boxed{\frac{d\vec{L}^{(o)}}{dt} =-\vec{v}_0 \times m\vec{v}+ \vec{M}^{(o)}}
\end{gather*}
L'effetto dinamico che hanno questi momenti delle forze è quello di far variare nel tempo il momento angolare del punto. Questo risultato prende il nome di \textbf{teorema del momento angolare} del punto materiale ed è una diretta conseguenza della legge di Newton. Infatti, l'utilizzo del momento angolare e del momento della forza e delle loro proprietà, anche se rilevanti in alcuni casi specifici, non fornisce nessuna informazione che non sia già ricavabile direttamente da essa. Il termine aggiuntivo $-\vec{v}_0 \times m\vec{v}$ si annulla se il polo $O$ è fermo oppure se si muove con velocità parallela a quella del punto. In questo caso il teorema diventa:
\[
	\boxed{\vec{M}^{(o)}= \frac{d\vec{L}^{(o)}}{dt}}
\]
Se sul punto materiale agiscono delle forze che non generano momento, si trova come conseguenza il \textbf{principio di conservazione del momento angolare}:
\[
	\boxed{\vec{M}^{(o)}=0 \implies \vec{L}^{(o)}= \text{costante}}
\]
Si chiama in tal modo perché si può dimostrare che è un principio fondamentale che deriva dalla simmetria dello spazio nel tempo.







































\section{Dinamica relativa}

Nel capitolo 2 si è determinato il legame sia in termini di velocità che di accelerazione fra ciò che rilevano l'osservatore assoluto e quello relativo. Vediamo come queste leggi si traducono sul moto di un corpo nel momento in cui su di esso agiscono delle forze. L'obbiettivo del paragrafo è quello di capire come l'osservatore dal sistema di riferimento mobile posso ancora usare i principi della dinamica newtoniana.
Il problema si presenta perché essi valgono solo se il sistema di riferimento è inerziale. In tal caso si ha:
\[
	\vec{F}=m\vec{a}
\]
dove le forze che compaiono a primo membro sono quelle reali, che derviano dalle interazioni fondamentali, e la risultante è proporzionale all'accelerazione misurata in quel sistema di riferimento. Si consideri ora un altro sistema di riferimento che si muove di moto traslatorio rettilineo uniforme rispetto ad un certo sistema inerziale. Si ha:
\[
	\vec{v}_{o'}=\text{costante} \quad \vec{a}_{o'}=0 \quad \vec{\omega}=0
\]
Dato che:
\[
	\vec{a}=\vec{a'}+\vec{a}_{o'}+\frac{d\vec{\omega}}{dt} \times \vec{r'}+\vec{\omega}\times (\vec{\omega} \times \vec{r'})+2\vec{\omega} \times \vec{v'} \quad \text{si ha} \quad \vec{a}=\vec{a'}
\]
Le accelerazioni di un punto misurate nei due sistemi di riferimento sono eguali. Se $a=0$ allora anche $a'=0$ e quindi anche il secondo sistema è inerziale. Questo porta a confermare il fatto che, definito un sistema di riferimento inerziale, tutti gli altri sistemi in moto rettilineo uniforme rispetto a questo sono anche essi inerziali. Per tali sistemi la legge di Newton si scrive allo stesso modo, ossia con gli stessi valori di $\vec{F}$ e $\vec{a}$. Conseguenza importante è che non è possibile stabilire tramite misure effettuate in questi diversi sistemi di riferimento, se uno di essi è in quiete o in moto. Si ricordi che tale situazione fisica viene descritta anche con il termine di relatività galileiana.

Tuttavia, se il moto del secondo sistema è accelerato rispetto al sistema inerziale, sia perché $a_{o'}$ o $\omega$ sono non nulle o per entrambe le ragioni, si osserva che la legge di Newton non è più valida, la forza vera che agisce sul punto considerato non è proporzionale alla sua accelerazione. Infatti, se $\vec{F}=m\vec{a}$ nel sistema inerziale, nel sistema mobile in moto accelerato non può sussistere la relazione $\vec{F}=m\vec{a'}$ perché $a$ e $a'$ sono diverse.
Quindi $F=ma'$ non è vera tutte le volte che esiste un'accelerazione di trascinamento e/o una di Coriolis. D'altra parte si può scrivere:
\[
	\vec{F}=m\vec{a}=m\vec{a}_{\text{rel}}+m\vec{a}_{\text{trasc}}+m\vec{a}_c
\]
Isolando l'accelerazione relativa:
\[
	m\vec{a}_{\text{rel}}=\vec{F} \underbrace{-m\vec{a}_{\text{trasc}}-m\vec{a}_c}_{\vec{F}_{\text{app}}}
\]
Il terimine $\vec{F}_{\text{app}}$ dimensionalmente è una forza e per questo motivo viene anche chiamato \textbf{forza apparente}. Quindi in un sistema di riferimento non inerziale il prodotto della massa del punto materiale per l'accelerazione misurata in quel sistema è eguale alla forza vera agente sul punto sommata a queste forze apparenti. Non si tratta di interazioni reali che effettivamente agiscono sul punto e che si possono ricondurre alle quattro interazioni fondamentali, ma costituiscono un termine correttivo che appare solo nel sistema non inerziale che permette di ricondursi a $F=ma'$ e vengono pertanto chiamate anche \textit{forze di inerzia}: grazie ad esse la seconda legge di Newton è rispettata.
\[
	\vec{F}_{\text{app}}=-m\vec{a}_{\text{trasc}}-m\vec{a}_c
\]
Si vede come allora sia possibile introdurre un secondo principio della dinamica modificato per i sistemi di riferimento non inerziali grazie a questo nuovo termine.
\[
	\vec{F}_{\text{app}}=(-m\vec{a}_{o'})+(-m\vec{\omega}\times \vec{\omega} \times \vec{r})+(-mr'\vec{\alpha})+(-2m\vec{\omega} \times \vec{v}_{\text{rel}})
\]
In particolare, la forza:
\[
	\vec{F}_{\text{cf}}=-m\vec{\omega}\times \vec{\omega} \times \vec{r}
\]
è la cosiddetta \textbf{forza apparente centrifuga}. Il suo effetto può essere facilmente percepito quando ci si trova su un'automobile che percorre una rotatoria e si ha l'impressione di essere spinti verso l'esterno. Invece:
\[
	\vec{F}_c= -2m\vec{\omega} \times \vec{v}_{\text{rel}}
\]
prende il nome di \textbf{forza di Coriolis}.
\[
	\vec{F}+\vec{F}_{\text{app}}=m\vec{a}_{\text{rel}}
\]
Si noti in particolare come in un sistema accelerato $F=0$ non comporti $a=0$ e quindi l'osservazione di un moto rettilineo uniforme. Questo risultato giustifica il nome di sistema non inerziale per un sistema accelerato: in esso la legge di inerzia non vale.


\subsubsection{Moto di trascinamento rettilineo accelerato}

\paragraph{Primo esempio} Si supponga di avere un vagone di un treno che inizialmente si muove di moto uniforme rettilineo.
\begin{figure}[htpb]
	\centering
	

	\tikzset{every picture/.style={line width=0.75pt}} %set default line width to 0.75pt        

	\begin{tikzpicture}[x=0.75pt,y=0.75pt,yscale=-1,xscale=1]
	%uncomment if require: \path (0,493); %set diagram left start at 0, and has height of 493

	%Straight Lines [id:da8845303315941491] 
	\draw    (50,190) -- (430,190) ;
	%Rounded Rect [id:dp5517065009785334] 
	\draw   (210,114) .. controls (210,106.27) and (216.27,100) .. (224,100) -- (366,100) .. controls (373.73,100) and (380,106.27) .. (380,114) -- (380,156) .. controls (380,163.73) and (373.73,170) .. (366,170) -- (224,170) .. controls (216.27,170) and (210,163.73) .. (210,156) -- cycle ;
	%Shape: Circle [id:dp28946861808735336] 
	\draw   (240,180) .. controls (240,174.48) and (244.48,170) .. (250,170) .. controls (255.52,170) and (260,174.48) .. (260,180) .. controls (260,185.52) and (255.52,190) .. (250,190) .. controls (244.48,190) and (240,185.52) .. (240,180) -- cycle ;
	%Shape: Circle [id:dp13801409799404407] 
	\draw   (270,180) .. controls (270,174.48) and (274.48,170) .. (280,170) .. controls (285.52,170) and (290,174.48) .. (290,180) .. controls (290,185.52) and (285.52,190) .. (280,190) .. controls (274.48,190) and (270,185.52) .. (270,180) -- cycle ;
	%Shape: Circle [id:dp29024094658021093] 
	\draw   (300,180) .. controls (300,174.48) and (304.48,170) .. (310,170) .. controls (315.52,170) and (320,174.48) .. (320,180) .. controls (320,185.52) and (315.52,190) .. (310,190) .. controls (304.48,190) and (300,185.52) .. (300,180) -- cycle ;
	%Shape: Circle [id:dp2732816373100859] 
	\draw   (330,180) .. controls (330,174.48) and (334.48,170) .. (340,170) .. controls (345.52,170) and (350,174.48) .. (350,180) .. controls (350,185.52) and (345.52,190) .. (340,190) .. controls (334.48,190) and (330,185.52) .. (330,180) -- cycle ;
	%Shape: Circle [id:dp9803871770324311] 
	\draw   (290,129) .. controls (290,123.48) and (294.48,119) .. (300,119) .. controls (305.52,119) and (310,123.48) .. (310,129) .. controls (310,134.52) and (305.52,139) .. (300,139) .. controls (294.48,139) and (290,134.52) .. (290,129) -- cycle ;
	%Straight Lines [id:da5918357181478389] 
	\draw    (300,159) -- (300,139) ;
	%Straight Lines [id:da8644803390899314] 
	\draw    (310,169) -- (300,159) ;
	%Straight Lines [id:da9988000045948477] 
	\draw    (310,149) -- (300,139) ;
	%Straight Lines [id:da5227285209954715] 
	\draw    (290,149) -- (300,139) ;
	%Straight Lines [id:da72039255595026] 
	\draw    (290,169) -- (300,159) ;

	%Shape: Circle [id:dp7254243014258992] 
	\draw   (140,149) .. controls (140,143.48) and (144.48,139) .. (150,139) .. controls (155.52,139) and (160,143.48) .. (160,149) .. controls (160,154.52) and (155.52,159) .. (150,159) .. controls (144.48,159) and (140,154.52) .. (140,149) -- cycle ;
	%Straight Lines [id:da3738111513126945] 
	\draw    (150,179) -- (150,159) ;
	%Straight Lines [id:da9266705565337741] 
	\draw    (160,189) -- (150,179) ;
	%Straight Lines [id:da549874811572632] 
	\draw    (160,169) -- (150,159) ;
	%Straight Lines [id:da4835894229807174] 
	\draw    (140,169) -- (150,159) ;
	%Straight Lines [id:da9048075753546523] 
	\draw    (140,189) -- (150,179) ;

	%Shape: Axis 2D [id:dp683490375288136] 
	\draw  (90,180) -- (120,180)(90,150) -- (90,180) -- cycle (113,175) -- (120,180) -- (113,185) (85,157) -- (90,150) -- (95,157)  ;
	%Shape: Axis 2D [id:dp01355233214226792] 
	\draw  (235,160) -- (265,160)(235,130) -- (235,160) -- cycle (258,155) -- (265,160) -- (258,165) (230,137) -- (235,130) -- (240,137)  ;
	%Straight Lines [id:da6449047403936963] 
	\draw    (370,140) -- (417,139.06) ;
	\draw [shift={(420,139)}, rotate = 538.85] [fill={rgb, 255:red, 0; green, 0; blue, 0 }  ][line width=0.08]  [draw opacity=0] (10.72,-5.15) -- (0,0) -- (10.72,5.15) -- (7.12,0) -- cycle    ;
	%Shape: Circle [id:dp6949534200545615] 
	\draw  [fill={rgb, 255:red, 0; green, 0; blue, 0 }  ,fill opacity=1 ] (340,125) .. controls (340,122.24) and (342.24,120) .. (345,120) .. controls (347.76,120) and (350,122.24) .. (350,125) .. controls (350,127.76) and (347.76,130) .. (345,130) .. controls (342.24,130) and (340,127.76) .. (340,125) -- cycle ;
	%Straight Lines [id:da7649546918616734] 
	\draw    (345,133) -- (345,167) ;
	\draw [shift={(345,170)}, rotate = 270] [fill={rgb, 255:red, 0; green, 0; blue, 0 }  ][line width=0.08]  [draw opacity=0] (10.72,-5.15) -- (0,0) -- (10.72,5.15) -- (7.12,0) -- cycle    ;
	\draw [shift={(345,130)}, rotate = 90] [fill={rgb, 255:red, 0; green, 0; blue, 0 }  ][line width=0.08]  [draw opacity=0] (10.72,-5.15) -- (0,0) -- (10.72,5.15) -- (7.12,0) -- cycle    ;
	%Rounded Rect [id:dp6299281142169286] 
	\draw   (210,274) .. controls (210,266.27) and (216.27,260) .. (224,260) -- (366,260) .. controls (373.73,260) and (380,266.27) .. (380,274) -- (380,316) .. controls (380,323.73) and (373.73,330) .. (366,330) -- (224,330) .. controls (216.27,330) and (210,323.73) .. (210,316) -- cycle ;
	%Straight Lines [id:da4399120364271609] 
	\draw    (318.98,281.02) -- (185.82,281.02) ;
	\draw [shift={(182.82,281.02)}, rotate = 360] [fill={rgb, 255:red, 0; green, 0; blue, 0 }  ][line width=0.08]  [draw opacity=0] (10.72,-5.15) -- (0,0) -- (10.72,5.15) -- (7.12,0) -- cycle    ;
	%Straight Lines [id:da11021083557463185] 
	\draw    (318.98,281.02) -- (318.98,355.75) ;
	\draw [shift={(318.98,358.75)}, rotate = 270] [fill={rgb, 255:red, 0; green, 0; blue, 0 }  ][line width=0.08]  [draw opacity=0] (10.72,-5.15) -- (0,0) -- (10.72,5.15) -- (7.12,0) -- cycle    ;
	%Straight Lines [id:da5241271764093178] 
	\draw  [dash pattern={on 0.84pt off 2.51pt}]  (318.98,358.75) -- (182.82,358.75) ;
	%Straight Lines [id:da06747514242420771] 
	\draw  [dash pattern={on 0.84pt off 2.51pt}]  (182.82,281.02) -- (182.82,358.75) ;
	%Straight Lines [id:da37342175067776906] 
	\draw    (318.98,281.02) -- (185.42,357.26) ;
	\draw [shift={(182.82,358.75)}, rotate = 330.28] [fill={rgb, 255:red, 0; green, 0; blue, 0 }  ][line width=0.08]  [draw opacity=0] (10.72,-5.15) -- (0,0) -- (10.72,5.15) -- (7.12,0) -- cycle    ;
	%Shape: Circle [id:dp3288407095021495] 
	\draw  [fill={rgb, 255:red, 0; green, 0; blue, 0 }  ,fill opacity=1 ] (313.98,281.02) .. controls (313.98,278.26) and (316.21,276.02) .. (318.98,276.02) .. controls (321.74,276.02) and (323.98,278.26) .. (323.98,281.02) .. controls (323.98,283.79) and (321.74,286.02) .. (318.98,286.02) .. controls (316.21,286.02) and (313.98,283.79) .. (313.98,281.02) -- cycle ;

	% Text Node
	\draw (129,177) node    {$x$};
	% Text Node
	\draw (80,147) node    {$y$};
	% Text Node
	\draw (80,180) node    {$O$};
	% Text Node
	\draw (274,157) node    {$x'$};
	% Text Node
	\draw (225,127) node    {$y'$};
	% Text Node
	\draw (225,160) node    {$O'$};
	% Text Node
	\draw (434.5,142.5) node    {$\vec{a}_{O'}$};
	% Text Node
	\draw (150.5,209.5) node   [align=left] {oss. fisso};
	% Text Node
	\draw (300.5,209.5) node   [align=left] {oss. mobile};
	% Text Node
	\draw (357,147) node    {$h$};
	% Text Node
	\draw (154,277.5) node    {$-m\vec{a}_{O'}$};
	% Text Node
	\draw (332,373.5) node    {$\vec{p} =m\vec{g}$};
	% Text Node
	\draw (173,363) node    {$\vec{F}_{tot}$};


	\end{tikzpicture}
\end{figure}
\FloatBarrier
All'istante $t_0$ esso viene accelerato in avanti con una certa accelerazione $\vec{a}_{O'}$. Si considerino successivamente un osservatore fisso fuori dal treno e uno mobile solidale ad esso e un oggetto di massa $m$ che si trova sollevato a una certa quota $h$ all'interno di un vagone. Nell'istante in cui il treno accelera esso viene lasciato libero di cadere. Si è interessati a capire quale sia il moto di caduta dell'oggetto visto da $O$ e da $O'$.

Negli istanti successivi a $t_0$ il corpo è soltanto sottoposto alla forza di gravità. Tuttavia il treno mentre esso cade si sposta più avanti così che l'oggetto tocca il pavimento del vagone più indietro rispetto a $O'$, di uno spostamento $d$. L'osservatore $O'$ rileverà quindi un moto di caduta parabolico. Dovrebbero esserci delle forze che giustificano questo moto, ma l'unica forza esistente è la forza peso. Bisogna quindi considerare l'esistenza di una forza apparente diretta verso sinistra parallela all'asse $x$. La forza apparente e la forza peso si sommano vettorialmente.

L'osservatore $O$ invece osserverà un normale moto di caduta verticale uniformemente accelerato.

\paragraph{Secondo esempio} Si immagini la stessa situazione vista nell'esempio precedente, ma al posto di avere un oggetto che viene lasciato cadere si ha un corpo appoggiato su un tavolo. Si trascuri l'effetto dell'attrito. Dal punto di vista dell'osservatore esterno, l'oggetto rimane fermo, $O'$ invece percepisce l'oggetto venire verso di sé. Per spiegare questo il cervello introduce una forza apparente.
\begin{figure}[htpb]
	\centering
	

	\tikzset{every picture/.style={line width=0.75pt}} %set default line width to 0.75pt        

	\begin{tikzpicture}[x=0.75pt,y=0.75pt,yscale=-1,xscale=1]
	%uncomment if require: \path (0,383); %set diagram left start at 0, and has height of 383

	%Straight Lines [id:da7481618489533346] 
	\draw    (50,150) -- (430,150) ;
	%Rounded Rect [id:dp5667741041831242] 
	\draw   (210,74) .. controls (210,66.27) and (216.27,60) .. (224,60) -- (366,60) .. controls (373.73,60) and (380,66.27) .. (380,74) -- (380,116) .. controls (380,123.73) and (373.73,130) .. (366,130) -- (224,130) .. controls (216.27,130) and (210,123.73) .. (210,116) -- cycle ;
	%Shape: Circle [id:dp2747437249820368] 
	\draw   (240,140) .. controls (240,134.48) and (244.48,130) .. (250,130) .. controls (255.52,130) and (260,134.48) .. (260,140) .. controls (260,145.52) and (255.52,150) .. (250,150) .. controls (244.48,150) and (240,145.52) .. (240,140) -- cycle ;
	%Shape: Circle [id:dp4412374085386952] 
	\draw   (270,140) .. controls (270,134.48) and (274.48,130) .. (280,130) .. controls (285.52,130) and (290,134.48) .. (290,140) .. controls (290,145.52) and (285.52,150) .. (280,150) .. controls (274.48,150) and (270,145.52) .. (270,140) -- cycle ;
	%Shape: Circle [id:dp34192828888252835] 
	\draw   (300,140) .. controls (300,134.48) and (304.48,130) .. (310,130) .. controls (315.52,130) and (320,134.48) .. (320,140) .. controls (320,145.52) and (315.52,150) .. (310,150) .. controls (304.48,150) and (300,145.52) .. (300,140) -- cycle ;
	%Shape: Circle [id:dp9465454617396569] 
	\draw   (330,140) .. controls (330,134.48) and (334.48,130) .. (340,130) .. controls (345.52,130) and (350,134.48) .. (350,140) .. controls (350,145.52) and (345.52,150) .. (340,150) .. controls (334.48,150) and (330,145.52) .. (330,140) -- cycle ;
	%Shape: Circle [id:dp5345063228583655] 
	\draw   (290,89) .. controls (290,83.48) and (294.48,79) .. (300,79) .. controls (305.52,79) and (310,83.48) .. (310,89) .. controls (310,94.52) and (305.52,99) .. (300,99) .. controls (294.48,99) and (290,94.52) .. (290,89) -- cycle ;
	%Straight Lines [id:da9418121610599619] 
	\draw    (300,119) -- (300,99) ;
	%Straight Lines [id:da4873352074315478] 
	\draw    (310,129) -- (300,119) ;
	%Straight Lines [id:da2265861890506471] 
	\draw    (310,109) -- (300,99) ;
	%Straight Lines [id:da4267121758287853] 
	\draw    (290,109) -- (300,99) ;
	%Straight Lines [id:da3908971786394406] 
	\draw    (290,129) -- (300,119) ;

	%Shape: Circle [id:dp1205389278798743] 
	\draw   (140,109) .. controls (140,103.48) and (144.48,99) .. (150,99) .. controls (155.52,99) and (160,103.48) .. (160,109) .. controls (160,114.52) and (155.52,119) .. (150,119) .. controls (144.48,119) and (140,114.52) .. (140,109) -- cycle ;
	%Straight Lines [id:da1777074619744483] 
	\draw    (150,139) -- (150,119) ;
	%Straight Lines [id:da8202026608698005] 
	\draw    (160,149) -- (150,139) ;
	%Straight Lines [id:da3057970641206704] 
	\draw    (160,129) -- (150,119) ;
	%Straight Lines [id:da09311157856084074] 
	\draw    (140,129) -- (150,119) ;
	%Straight Lines [id:da009624285579875158] 
	\draw    (140,149) -- (150,139) ;

	%Shape: Axis 2D [id:dp8662876823902428] 
	\draw  (90,140) -- (120,140)(90,110) -- (90,140) -- cycle (113,135) -- (120,140) -- (113,145) (85,117) -- (90,110) -- (95,117)  ;
	%Shape: Axis 2D [id:dp5231104317813651] 
	\draw  (235,120) -- (265,120)(235,90) -- (235,120) -- cycle (258,115) -- (265,120) -- (258,125) (230,97) -- (235,90) -- (240,97)  ;
	%Straight Lines [id:da7998457648446491] 
	\draw    (370,100) -- (417,99.06) ;
	\draw [shift={(420,99)}, rotate = 538.85] [fill={rgb, 255:red, 0; green, 0; blue, 0 }  ][line width=0.08]  [draw opacity=0] (10.72,-5.15) -- (0,0) -- (10.72,5.15) -- (7.12,0) -- cycle    ;
	%Straight Lines [id:da08530829866443712] 
	\draw    (305,241.67) -- (221.84,241.67) ;
	\draw [shift={(218.84,241.67)}, rotate = 360] [fill={rgb, 255:red, 0; green, 0; blue, 0 }  ][line width=0.08]  [draw opacity=0] (10.72,-5.15) -- (0,0) -- (10.72,5.15) -- (7.12,0) -- cycle    ;
	%Shape: Rectangle [id:dp12171107723467856] 
	\draw   (340,100) -- (350,100) -- (350,110) -- (340,110) -- cycle ;
	%Straight Lines [id:da017961366152073666] 
	\draw [line width=2.25]    (330,110) -- (360,110) ;
	%Straight Lines [id:da6976251861981426] 
	\draw [line width=2.25]    (330,130) -- (330,110) ;
	%Straight Lines [id:da5278964740683598] 
	\draw [line width=2.25]    (360,130) -- (360,110) ;
	%Shape: Rectangle [id:dp971974753281907] 
	\draw   (283.33,220) -- (326.67,220) -- (326.67,263.33) -- (283.33,263.33) -- cycle ;
	%Straight Lines [id:da6316657209710219] 
	\draw [line width=2.25]    (240,263.33) -- (370,263.33) ;
	%Straight Lines [id:da3197068016741924] 
	\draw [line width=2.25]    (240,290) -- (240,263.33) ;
	%Straight Lines [id:da15778515861977183] 
	\draw [line width=2.25]    (370,290) -- (370,263.33) ;
	%Straight Lines [id:da7707290971716076] 
	\draw    (388.16,241.67) -- (305,241.67) ;
	\draw [shift={(391.16,241.67)}, rotate = 180] [fill={rgb, 255:red, 0; green, 0; blue, 0 }  ][line width=0.08]  [draw opacity=0] (10.72,-5.15) -- (0,0) -- (10.72,5.15) -- (7.12,0) -- cycle    ;
	%Shape: Circle [id:dp8261532738782928] 
	\draw  [fill={rgb, 255:red, 0; green, 0; blue, 0 }  ,fill opacity=1 ] (302.88,241.67) .. controls (302.88,240.49) and (303.83,239.54) .. (305,239.54) .. controls (306.17,239.54) and (307.13,240.49) .. (307.13,241.67) .. controls (307.13,242.84) and (306.17,243.79) .. (305,243.79) .. controls (303.83,243.79) and (302.88,242.84) .. (302.88,241.67) -- cycle ;

	% Text Node
	\draw (129,137) node    {$x$};
	% Text Node
	\draw (80,107) node    {$y$};
	% Text Node
	\draw (80,140) node    {$O$};
	% Text Node
	\draw (274,117) node    {$x'$};
	% Text Node
	\draw (225,87) node    {$y'$};
	% Text Node
	\draw (225,120) node    {$O'$};
	% Text Node
	\draw (434.5,102.5) node    {$\vec{a}_{O'}$};
	% Text Node
	\draw (150.5,169.5) node   [align=left] {oss. fisso};
	% Text Node
	\draw (300.5,169.5) node   [align=left] {oss. mobile};
	% Text Node
	\draw (344.5,88) node    {$m$};
	% Text Node
	\draw (195,236.5) node    {$-m\vec{a}_{O}$};
	% Text Node
	\draw (412,235.5) node    {$\vec{R}_{t}$};


	\end{tikzpicture}
\end{figure}
\FloatBarrier
Si immagini ora che fra l'oggetto appoggiato e il tavolo ci sia un attrito statico che mantiene l'oggetto fermo. La forza di attrito bilancia la forza apparente e quindi l'osservatore al di fuori vede l'oggetto accelerare in avanti con accelerazione $\vec{a}_{O'}$. La forza di attrito è quella che permette al corpo di andare in avanti. La forza reale $R_t$ ha una sua reazione che si dovrà applicare sul tavolo (per la terza legge della dinamica), mentre la forza apparente non ha alcuna reazione su quest'ultimo perché non è una vera forza.

\subsubsection{Moto di trascinamento rotatorio uniforme}

\paragraph{Primo esempio} Si immagini di avere una giostra che sta ruotando intorno al proprio asse di rotazione con velocità angolare costante. Un bambino $A$ è seduto al centro della giostra e un altro $B$ è in piedi davanti a lui al lato opposto. $A$ vuole lanciare un pallone al suo amico.
\begin{figure}[htpb]
	\centering
	

	\tikzset{every picture/.style={line width=0.75pt}} %set default line width to 0.75pt        

	\begin{tikzpicture}[x=0.75pt,y=0.75pt,yscale=-1,xscale=1]
	%uncomment if require: \path (0,417); %set diagram left start at 0, and has height of 417

	%Shape: Can [id:dp1996235754689646] 
	\draw   (304.74,186.82) -- (304.74,220.57) .. controls (304.74,229.89) and (264.1,237.44) .. (213.96,237.44) .. controls (163.83,237.44) and (123.19,229.89) .. (123.19,220.57) -- (123.19,186.82) .. controls (123.19,177.5) and (163.83,169.94) .. (213.96,169.94) .. controls (264.1,169.94) and (304.74,177.5) .. (304.74,186.82) .. controls (304.74,196.14) and (264.1,203.69) .. (213.96,203.69) .. controls (163.83,203.69) and (123.19,196.14) .. (123.19,186.82) ;
	%Straight Lines [id:da1965995670183247] 
	\draw    (213.96,135.92) -- (213.96,186.82) ;
	\draw [shift={(213.96,132.92)}, rotate = 90] [fill={rgb, 255:red, 0; green, 0; blue, 0 }  ][line width=0.08]  [draw opacity=0] (10.72,-5.15) -- (0,0) -- (10.72,5.15) -- (7.12,0) -- cycle    ;
	%Straight Lines [id:da6025617807584005] 
	\draw    (320.82,171.46) -- (213.96,186.82) ;
	\draw [shift={(323.79,171.03)}, rotate = 171.82] [fill={rgb, 255:red, 0; green, 0; blue, 0 }  ][line width=0.08]  [draw opacity=0] (10.72,-5.15) -- (0,0) -- (10.72,5.15) -- (7.12,0) -- cycle    ;
	%Straight Lines [id:da686534953691958] 
	\draw    (237.27,160.36) -- (285.59,176.29) ;
	\draw [shift={(234.42,159.42)}, rotate = 18.25] [fill={rgb, 255:red, 0; green, 0; blue, 0 }  ][line width=0.08]  [draw opacity=0] (10.72,-5.15) -- (0,0) -- (10.72,5.15) -- (7.12,0) -- cycle    ;
	%Shape: Ellipse [id:dp14957173649018563] 
	\draw  [fill={rgb, 255:red, 0; green, 0; blue, 0 }  ,fill opacity=1 ] (282.55,176.29) .. controls (282.55,174.61) and (283.91,173.25) .. (285.59,173.25) .. controls (287.27,173.25) and (288.64,174.61) .. (288.64,176.29) .. controls (288.64,177.97) and (287.27,179.33) .. (285.59,179.33) .. controls (283.91,179.33) and (282.55,177.97) .. (282.55,176.29) -- cycle ;
	%Shape: Can [id:dp8485531513937126] 
	\draw   (549.5,186.56) -- (549.5,220.31) .. controls (549.5,229.63) and (508.86,237.19) .. (458.72,237.19) .. controls (408.59,237.19) and (367.95,229.63) .. (367.95,220.31) -- (367.95,186.56) .. controls (367.95,177.24) and (408.59,169.68) .. (458.72,169.68) .. controls (508.86,169.68) and (549.5,177.24) .. (549.5,186.56) .. controls (549.5,195.88) and (508.86,203.44) .. (458.72,203.44) .. controls (408.59,203.44) and (367.95,195.88) .. (367.95,186.56) ;
	%Straight Lines [id:da34569188456151245] 
	\draw  [dash pattern={on 0.84pt off 2.51pt}]  (536.03,177.85) -- (458.72,186.56) ;
	%Shape: Ellipse [id:dp5567109055251096] 
	\draw  [fill={rgb, 255:red, 0; green, 0; blue, 0 }  ,fill opacity=1 ] (457.57,186.56) .. controls (457.57,185.92) and (458.08,185.4) .. (458.72,185.4) .. controls (459.36,185.4) and (459.88,185.92) .. (459.88,186.56) .. controls (459.88,187.2) and (459.36,187.72) .. (458.72,187.72) .. controls (458.08,187.72) and (457.57,187.2) .. (457.57,186.56) -- cycle ;
	%Curve Lines [id:da8168874457095212] 
	\draw    (458.72,186.56) .. controls (505.54,182.2) and (520.78,186.83) .. (532.22,196.36) ;
	%Shape: Ellipse [id:dp4186108813920151] 
	\draw  [fill={rgb, 255:red, 0; green, 0; blue, 0 }  ,fill opacity=1 ] (521.26,190.37) .. controls (521.26,189.73) and (521.78,189.21) .. (522.42,189.21) .. controls (523.06,189.21) and (523.57,189.73) .. (523.57,190.37) .. controls (523.57,191.01) and (523.06,191.53) .. (522.42,191.53) .. controls (521.78,191.53) and (521.26,191.01) .. (521.26,190.37) -- cycle ;

	% Text Node
	\draw (329.51,171.3) node    {$\vec{r}$};
	% Text Node
	\draw (197.77,133.64) node    {$\vec{\omega }$};
	% Text Node
	\draw (254.43,145.16) node    {$\vec{\omega } \times \vec{r}$};
	% Text Node
	\draw (527.82,212.7) node    {$\vec{F}$};


	\end{tikzpicture}
\end{figure}
\FloatBarrier
La velocità lineare non è uguale per i punti sulla giostra: il bambino al centro gira su se stesso, ha velocità lineare pari a $0$, mentre $B$ ha una certa velocità lineare non nulla. Come conseguenza di ciò quando il pallone è arrivato al punto $x$, il bambino $B$ intanto si è spostato. La traiettoria che percepisce l'osservatore assoluto è dritta che cade verso il basso. L'osservatore che ruota osserva invece una traiettoria che curva come in figura.

\paragraph{Secondo esempio} Si consideri ora il sistema inerziale $O$ costituito da una coppia di assi cartesiani $x, y$ posti su un piano orizzontale, e il sistema non inerziale $O'$ costituito da un'altra coppia di assi $x', y'$ con la stessa origine e nello stesso piano, ruotanti con velocità angolare costante $\omega$. Si può ad esempio assumere gli assi $x', y'$ solidali ad un disco posto nel piano $x, y$ che ruota rispetto ad un asse passante per il suo centro e ortogonale al piano $x, y$. Se si pone un punto materiale sul disco, con attrito nullo, il punto $P$ rimane fermo mentre il disco gira sotto di esso. Se $P$ lasciasse una traccia, si osserverebbe una circonferenza di raggio $R$, con centro nell'origine comune dei due sistemi. Per l'osservatore $O$ il punto è in quiete, mentre per quello rotante $O'$ descrive un moto circolare uniforme. $O'$ deve allora ipotizzare che agiscano delle forze (centrifuga e di Coriolis) le quali, combinandosi, comunicano al punto l'accelerazione $a'$. Rimane il problema dell'origine di queste forze.

Si ipotizzi ore di legare con un filo il punto all'asse di rotazione e di dargli una velocità di modulo $\omega r$ in modo tale che ruoti con la stessa velocità del punto del disco su cui si trova. La situazione è opposta a quella precedente: per $O$ il punto descrive un moto circolare uniforme sotto l'azione della tensione del filo, mentre $O'$ vede il punto fermo. Esso è costretto a supporre che sul punto agisca una forza diretta verso l'esterno, che chiama forza centrifuga, bilanciata dalla tensione del filo. Per verificare l'ipotesi $O'$ traccia un segno radiale sul disco e recide il legame tra il punto e l'origine degli assi, immaginando di vederlo allontanarsi radialmente sotto l'azione della forza centrifuga, in quanto è stata annullata la forza esercitata dal filo. In effetti $O'$ osserva un moto del punto materiale, però lungo una traiettoria curvilinea, e deve quindi ammettere, come già fatto, che sui punti in moto nel suo sistema di rifermo, agisca un'altra forza che non si manifesta quando sono in quiete, la forza di Coriolis. Nel sistema inerziale invece il punto materiale all'istante in cui viene lasciato libero inizia a muoversi di moto rettilineo uniforme con direzione tangente alla circonferenza nella posizione in cui avviene il distacco dal vincolo.  In un sistema rotante è corretto attribuire alla forza centrifuga la tendenza allo spostamento radiale verso l'esterno e alla forza di Coriolis l'incurvamento della traiettoria osservata. La cosa importante è avere ben chiara l'origine di tali forze apparenti e utilizzarle correttamente dove appropriato e non estendere la loro esistenza ai sistemi inerziali.

\subsubsection{Il moto rispetto alla Terra}

\paragraph{Forza di Coriolis} Si trova discorso analogo a quello fatto in precedenza considerando gli effetti della rotazione della Terra intorno al proprio asse. Si ha di nuovo un effetto dovuto al fatto che un punto $P$ sopra la Terra si muove con una certa $\omega$ che è la stessa per tutti i punti del pianeta ma con una velocità lineare, diretta come il parallelo, che è massima all'equatore e va a via via diminuendo lungo i poli. Questo fatto fa si che, ad esempio, un osservatore posto su un aereo che si muove da sud a nord percepisca una deviazione verso destra. Esso infatti finisce per trovarsi su punti che via via si stanno muovendo con velocità lineare inferiore: questo è di nuovo un effetto della forza di Coriolis. Essa è tangente al parallelo e diretta verso est e il suo effetto a pari velocità relativa dell'aereo, è tanto più intenso quanto più ci si sposta ai poli, perché diventa sempre più importante la componente della velocità relativa diretta ortogonalmente a $\vec{\omega}$.

\paragraph{Forza centrifuga} Si immagini ora di avere un filo a piombo sospeso lungo la verticale. A causa dell'effetto della rotazione terrestre non si vede il lampadario diretto lungo la direzione dell'accelerazione di gravità. Sull'oggetto agisce la forza peso compensata dalla tensione del filo. Un osservatore assoluto vede il filo a piombo che ha la direzione che congiunge il punto materiale con il centro della Terra. L'osservatore mobile dovrà introdurre delle forze apparenti. L'unica in questo caso è la forza centrifuga, che sarà:
\[
	\vec{F}_{\text{cf}}=-m \vec{\omega} \times (\vec{\omega} \times \vec{r'} ) \implies \norma{\vec{F}_{\text{cf}}}=m\omega^2 R_t \cos\lambda
\]
Sulla Terra si percepisce un'accelerazione complessiva in modo tale che il filo a piombo non punterà veramente verso il centro della Terra ma è leggermente inclinato verso sud nell'emisfero boreale, verso Nord nell'emisfero australe (verso l'equatore).
\begin{figure}[htpb]
	\centering
	

	\tikzset{every picture/.style={line width=0.75pt}} %set default line width to 0.75pt        

	\begin{tikzpicture}[x=0.75pt,y=0.75pt,yscale=-1,xscale=1]
	%uncomment if require: \path (0,300); %set diagram left start at 0, and has height of 300

	%Shape: Circle [id:dp7489150544907581] 
	\draw   (156,146.75) .. controls (156,92.76) and (199.76,49) .. (253.75,49) .. controls (307.74,49) and (351.5,92.76) .. (351.5,146.75) .. controls (351.5,200.74) and (307.74,244.5) .. (253.75,244.5) .. controls (199.76,244.5) and (156,200.74) .. (156,146.75) -- cycle ;
	%Straight Lines [id:da6254894909113031] 
	\draw [line width=1.5]    (186,77) -- (132.5,77) ;
	\draw [shift={(128.5,77)}, rotate = 360] [fill={rgb, 255:red, 0; green, 0; blue, 0 }  ][line width=0.08]  [draw opacity=0] (13.4,-6.43) -- (0,0) -- (13.4,6.44) -- (8.9,0) -- cycle    ;
	%Straight Lines [id:da12818472298876005] 
	\draw  [dash pattern={on 0.84pt off 2.51pt}]  (156.5,146.75) -- (253.75,146.75) ;
	%Straight Lines [id:da8541133725441592] 
	\draw    (253.75,146.75) -- (186,77) ;
	%Shape: Boxed Line [id:dp3732214130705598] 
	\draw    (225.41,117.57) -- (186,77) ;
	\draw [shift={(227.5,119.73)}, rotate = 225.83] [fill={rgb, 255:red, 0; green, 0; blue, 0 }  ][line width=0.08]  [draw opacity=0] (10.72,-5.15) -- (0,0) -- (10.72,5.15) -- (7.12,0) -- cycle    ;
	%Straight Lines [id:da3181484584982588] 
	\draw  [dash pattern={on 0.84pt off 2.51pt}]  (170.5,119.73) -- (227.5,119.73) ;
	%Shape: Boxed Line [id:dp8534461017405279] 
	\draw  [dash pattern={on 0.84pt off 2.51pt}]  (170.5,120.24) -- (128.5,77) ;
	%Straight Lines [id:da7939485633198569] 
	\draw [line width=1.5]    (186,77) -- (171.85,116.47) ;
	\draw [shift={(170.5,120.24)}, rotate = 289.72] [fill={rgb, 255:red, 0; green, 0; blue, 0 }  ][line width=0.08]  [draw opacity=0] (13.4,-6.43) -- (0,0) -- (13.4,6.44) -- (8.9,0) -- cycle    ;
	%Shape: Arc [id:dp7184725777533214] 
	\draw  [draw opacity=0] (234.91,147.27) .. controls (234.9,147.1) and (234.9,146.92) .. (234.9,146.75) .. controls (234.9,141.27) and (237.24,136.34) .. (240.97,132.89) -- (253.75,146.75) -- cycle ; \draw   (234.91,147.27) .. controls (234.9,147.1) and (234.9,146.92) .. (234.9,146.75) .. controls (234.9,141.27) and (237.24,136.34) .. (240.97,132.89) ;

	% Text Node
	\draw (222.8,135) node    {$\lambda $};
	% Text Node
	\draw (240.8,105) node    {$R$};
	% Text Node
	\draw (212.8,82.6) node    {$\vec{P}$};
	% Text Node
	\draw (111.2,72.2) node    {$\vec{F}_{\text{cf}}$};


	\end{tikzpicture}
\end{figure}
\FloatBarrier
Si consideri ora invece un corpo che viene lasciato cadere in prossimità della superficie terrestre da un'altezza $h$. Se per esempio la sua velocità iniziale fosse nulla, l'azione della forza centrifuga comporterebbe uno spostamento verso l'equatore lungo un meridiano: invece la forza di Coriolis, tangente a un parallelo e rivolta come nella figura soprastante, provoca uno spostamento verso oriente in entrambi gli emisferi. L'effetto complessivo è una combinazione dei due.























































































































\chapter{Lavoro ed energia}

\section{Definizione di lavoro di una forza}

La scomposizione delle forze in componenti tangenti e normali fornisce due informazioni ben differenti e diventa interessante quando si studia la dinamica del punto materiale in termini energetici, andando a definire quella grandezza che prende il nome di lavoro.
Si supponga di avere un punto materiale che sta eseguendo uno spostamento in termini infinitesimi pari a $d\vec{r}$ su una traiettoria generica $\Gamma$ soggetto ad una forza $\vec{F}$. In questo contesto si definisce il \textbf{lavoro elementare} della forza $\vec{F}$ compiuto durante lo spostamento vettoriale $d\vec{r}$ la quantità scalare:
\[
	\boxed{d\mathcal{L}=\vec{F}\cdot d\vec{r}}
\]
\begin{figure}[htpb]
	\centering
	

	\tikzset{every picture/.style={line width=0.75pt}} %set default line width to 0.75pt        

	\begin{tikzpicture}[x=0.75pt,y=0.75pt,yscale=-1,xscale=1]
	%uncomment if require: \path (0,300); %set diagram left start at 0, and has height of 300

	%Curve Lines [id:da4209846702190032] 
	\draw    (107.14,162.09) .. controls (306.54,63.28) and (184.82,253.34) .. (303.5,195.97) .. controls (422.18,138.6) and (463.86,189.98) .. (498.36,170.06) ;
	%Straight Lines [id:da396487998523426] 
	\draw [line width=1.5]    (210,134.4) -- (286.28,78.37) ;
	\draw [shift={(289.5,76)}, rotate = 503.7] [fill={rgb, 255:red, 0; green, 0; blue, 0 }  ][line width=0.08]  [draw opacity=0] (13.4,-6.43) -- (0,0) -- (13.4,6.44) -- (8.9,0) -- cycle    ;
	%Straight Lines [id:da013966627498320783] 
	\draw [line width=1.5]    (210,134.4) -- (242.69,149.11) ;
	\draw [shift={(246.33,150.75)}, rotate = 204.23] [fill={rgb, 255:red, 0; green, 0; blue, 0 }  ][line width=0.08]  [draw opacity=0] (13.4,-6.43) -- (0,0) -- (13.4,6.44) -- (8.9,0) -- cycle    ;
	%Shape: Arc [id:dp12128273572561565] 
	\draw  [draw opacity=0] (226.39,122.26) .. controls (228.91,125.65) and (230.4,129.85) .. (230.4,134.4) .. controls (230.4,137.04) and (229.9,139.56) .. (228.99,141.88) -- (210,134.4) -- cycle ; \draw   (226.39,122.26) .. controls (228.91,125.65) and (230.4,129.85) .. (230.4,134.4) .. controls (230.4,137.04) and (229.9,139.56) .. (228.99,141.88) ;
	%Shape: Boxed Line [id:dp22162093486060597] 
	\draw [line width=0.75]  [dash pattern={on 0.84pt off 2.51pt}]  (289.5,76) -- (254.06,154.75) ;
	%Straight Lines [id:da0719027115068891] 
	\draw [line width=0.75]  [dash pattern={on 0.84pt off 2.51pt}]  (255.22,154.75) -- (210,134.4) ;

	% Text Node
	\draw (295.67,88.4) node    {$\vec{F}$};
	% Text Node
	\draw (211.57,155.9) node    {$d\vec{r}$};
	% Text Node
	\draw (244.87,126) node    {$\vartheta $};
	% Text Node
	\draw (470.67,157.6) node    {$\Gamma $};


	\end{tikzpicture}
\end{figure}
\FloatBarrier
Visto che si parla di uno spostamento infinitesimo, la quantità di lavoro che si ottiene è infinitesima e pertanto la si indica come $d\mathcal{L}$.
Ci si accorge che, andando a risolvere il prodotto scalare, esso vale:
\[
	\norma{\vec{F}}\norma{d\vec{r}}\cos\vartheta
\]
Si possono presentare tre casi:
\begin{itemize}
	\item $\vec{F}$ forma con $d\vec{r}$ un angolo minore di $\frac{\pi}{2}$, per cui l'accelerazione tangente è concorde con la velocità e la fa aumentare: $d\mathcal{L}$ risulta positivo e viene chiamato lavoro \textbf{motore}. Vuol dire che la forza $\vec{F}$ sta fungendo da motore, permette al corpo di proseguire nella sua direzione e aumentare la velocità.
	\item $\vec{F}$ forma un angolo maggiore di $\frac{\pi}{2}$ con $d\vec{r}$, il punto viene frenato e $d\mathcal{L}$ risulta negativo (lavoro \textbf{resistente}). La forza resistente ha l'effetto di rallentare il moto.
	\item Quando la forza è ortogonale allo spostamento si ha che il lavoro è nullo, infatti essa né accelera né rallenta il moto. Capiamo che le forze normali alla traiettoria, le cosiddette \textbf{forze centripete} non compiono mai lavoro.
\end{itemize}
Lungo un arco finito di traiettoria può presentarsi sempre la stessa situazione, così che la velocità finale è maggiore di quella iniziale nel primo caso, minore nel secondo, uguale nel terzo. Tuttavia le varie situazioni possono anche alternarsi e il risultato dipende dalla situazione predominante.
Si noti che $\norma{\vec{F}} \cos\vartheta$ non è altro che la proiezione di $\vec{F}$ in direzione tangente al moto a conferma del fatto che una forza compie lavoro solo se presenta tale componente.

Si è definito il lavoro elementare considerando un tratto molto piccolo di traiettoria. Se la si divide in tanti segmenti infinitesimi, il lavoro totale, complessivo, fatto dalla forza $\vec{F}$ quando il punto materiale si muove sulla traiettoria $\Gamma$ dalla posizione $A$ alla posizione $B$, è la somma di tutti i lavori elementari calcolati su ogni segmento infinitesimo. Essendo una somma continua, si trova un integrale di linea:
\[
	\boxed{\mathcal{L}_{\Gamma, A\to B}=\int_{\Gamma, A\to B} d\mathcal{L}=\int_{\Gamma, A\to B} F_t\,ds}
\]
Il lavoro è dato dalla somma di infinti contributi infinitesimi $F\,ds$.
Facendo l'analisi dimensionale:
\[
	[L]=[F][L]=[M][L]^2[T]^{-2} \to m\,N
\]
Tale unità di misura prende il nome di \emph{Joule}. Una forza di un Newton che agisce in direzione parallela allo spostamento compie il lavoro di un Joule su un metro di spostamento.







































\section{Teorema delle forze vive}

Valutare il lavoro fatto da una forza significa valutare l'effetto della forza sulla velocità di un corpo. Questo concetto è esplicitato dal teorema delle forze vive, o dell'energia cinetica. Si consideri una generica porzione della traiettoria di lunghezza $ds$. Immaginiamo che il punto materiale quando si trova in $P$ sia soggetto a $n$ forze.  Applicando il secondo principio della dinamica:
\[
	\vec{F}_1+\vec{F}_2+\dots +\vec{F}_n=m\vec{a}
\]
\begin{figure}[htpb]
	\centering
	

	\tikzset{every picture/.style={line width=0.75pt}} %set default line width to 0.75pt        

	\begin{tikzpicture}[x=0.75pt,y=0.75pt,yscale=-1,xscale=1]
	%uncomment if require: \path (0,300); %set diagram left start at 0, and has height of 300

	%Curve Lines [id:da769583073586938] 
	\draw    (120.5,65) .. controls (226.5,45) and (163.5,222) .. (309.5,202) ;
	%Straight Lines [id:da5627734005611151] 
	\draw [line width=1.5]    (201,134.4) -- (238.87,83.22) ;
	\draw [shift={(241.25,80)}, rotate = 486.5] [fill={rgb, 255:red, 0; green, 0; blue, 0 }  ][line width=0.08]  [draw opacity=0] (13.4,-6.43) -- (0,0) -- (13.4,6.44) -- (8.9,0) -- cycle    ;
	%Shape: Circle [id:dp28223035330479807] 
	\draw  [fill={rgb, 255:red, 0; green, 0; blue, 0 }  ,fill opacity=1 ] (197.75,134.4) .. controls (197.75,132.61) and (199.21,131.15) .. (201,131.15) .. controls (202.79,131.15) and (204.25,132.61) .. (204.25,134.4) .. controls (204.25,136.19) and (202.79,137.65) .. (201,137.65) .. controls (199.21,137.65) and (197.75,136.19) .. (197.75,134.4) -- cycle ;
	%Straight Lines [id:da1868879857416137] 
	\draw [line width=1.5]    (201,134.4) -- (260.18,164.2) ;
	\draw [shift={(263.75,166)}, rotate = 206.73] [fill={rgb, 255:red, 0; green, 0; blue, 0 }  ][line width=0.08]  [draw opacity=0] (13.4,-6.43) -- (0,0) -- (13.4,6.44) -- (8.9,0) -- cycle    ;
	%Straight Lines [id:da1480558067390092] 
	\draw [line width=1.5]    (201,134.4) -- (140.58,152.83) ;
	\draw [shift={(136.75,154)}, rotate = 343.03999999999996] [fill={rgb, 255:red, 0; green, 0; blue, 0 }  ][line width=0.08]  [draw opacity=0] (13.4,-6.43) -- (0,0) -- (13.4,6.44) -- (8.9,0) -- cycle    ;
	%Straight Lines [id:da5486592963361427] 
	\draw [line width=1.5]    (201,134.4) -- (213.81,163.93) ;
	\draw [shift={(215.4,167.6)}, rotate = 246.55] [fill={rgb, 255:red, 0; green, 0; blue, 0 }  ][line width=0.08]  [draw opacity=0] (13.4,-6.43) -- (0,0) -- (13.4,6.44) -- (8.9,0) -- cycle    ;

	% Text Node
	\draw (258.17,82.4) node    {$\vec{F}_{1}$};
	% Text Node
	\draw (274.17,151.4) node    {$\vec{F}_{2}$};
	% Text Node
	\draw (126.17,132.4) node    {$\vec{F}_{n}$};
	% Text Node
	\draw (193.77,168.2) node    {$d\vec{s}$};
	% Text Node
	\draw (186.17,124.6) node    {$P$};


	\end{tikzpicture}
\end{figure}
\FloatBarrier
Se si va a calcolare il lavoro di tutte le forze mentre il punto si starà muovendo sulla traiettoria da $A$ a $B$, andando a sfruttare il secondo principio della dinamica scomposto in componenti tangenti e normali:
\begin{equation*}
	\begin{aligned}
		\mathcal{L} &= \int_{\Gamma, A\to B} F_t\,ds=\int_{\Gamma, A\to B} ma_t\,ds=\int_{\Gamma, A\to B} m\frac{dv}{dt}\,ds= \\
		&= \int_{\Gamma, A\to B} mv\,dv=\frac{1}{2}[ mv^2 ]^B_A=\frac{1}{2}mv^2_B-\frac{1}{2}mv^2_A
	\end{aligned}
\end{equation*}
Questa relazione afferma che quando su un punto materiale agiscono delle forze, il lavoro compiuto da tutte le forze ha l'effetto di variare la velocità del punto materiale in modo tale che $\mathcal{L} \frac{1}{2}m(v^2_B-v^2_A)$
Questa espressione rappresenta un'energia a cui si da il nome di energia cinetica. Si definisce l'\textbf{energia cinetica} posseduta da un punto materiale come la quantità energetica:
\[
	\boxed{E_k=\frac{1}{2}mv^2}
\]
Allora il teorema delle forze vive afferma che il lavoro di tutte le forze compiute su un punto materiale provoca la variazione dell'energia cinetica di esso quando si sposta dalla posizione iniziale in $A$ e quella finale in $B$. 
\begin{figure}[htpb]
	\centering
	

	\tikzset{every picture/.style={line width=0.75pt}} %set default line width to 0.75pt        

	\begin{tikzpicture}[x=0.75pt,y=0.75pt,yscale=-1,xscale=1]
	%uncomment if require: \path (0,300); %set diagram left start at 0, and has height of 300

	%Curve Lines [id:da22223273061886006] 
	\draw    (140.5,85) .. controls (255.33,43) and (248,167) .. (388.67,135) ;
	%Shape: Circle [id:dp7960730507131906] 
	\draw  [fill={rgb, 255:red, 0; green, 0; blue, 0 }  ,fill opacity=1 ] (210.42,81.4) .. controls (210.42,79.61) and (211.87,78.15) .. (213.67,78.15) .. controls (215.46,78.15) and (216.92,79.61) .. (216.92,81.4) .. controls (216.92,83.19) and (215.46,84.65) .. (213.67,84.65) .. controls (211.87,84.65) and (210.42,83.19) .. (210.42,81.4) -- cycle ;
	%Shape: Circle [id:dp1184180970622013] 
	\draw  [fill={rgb, 255:red, 0; green, 0; blue, 0 }  ,fill opacity=1 ] (321.75,138.73) .. controls (321.75,136.94) and (323.21,135.48) .. (325,135.48) .. controls (326.79,135.48) and (328.25,136.94) .. (328.25,138.73) .. controls (328.25,140.53) and (326.79,141.98) .. (325,141.98) .. controls (323.21,141.98) and (321.75,140.53) .. (321.75,138.73) -- cycle ;

	% Text Node
	\draw (228.17,69.73) node    {$A$};
	% Text Node
	\draw (327.5,120.4) node    {$B$};


	\end{tikzpicture}
\end{figure}
\FloatBarrier
Si chiama anche teorema delle forze vive perché comunica che le uniche forze che hanno l'effetto di far variare il valore della velocità sono quelle tangenti al moto.
In realtà la definizione di $E_k$ non è univoca ma è definita a meno di una costante aggiuntiva, che si sceglie uguale a zero in modo tale che un corpo a riposo abbia energia cinetica nulla. In questo modo tale concetto assume anche un significato fisico. La nozione di lavoro infatti, e quindi anche quella di variazione di energia cinetica, è necessariamente legata a quella di spostamento: se non c'è spostamento non può esserci lavoro, qualunque sia la forza applicata. Il lavoro fatto da una forza dipende in generale dalla traiettoria su cui si sta muovendo il corpo. Non si perde la dipendenza dalla traiettoria nella definizione di energia cinetica, infatti, a seconda di essa, il valore con cui il punto arriva in $B$ sarà in generale diversa.

Le due seguenti relazioni:
\[
	\boxed{F_t=ma_t \iff \mathcal{L}_\Gamma=\Delta E_k}
\]
portano ovviamente allo stesso risultato e il loro campo di validità è lo stesso (i sistemi di riferimento inerziali). Si dice che il teorema dell'energia cinetica è invariante rispetto a tali sistemi.

Si può riscrivere il teorema dell'energia cinetica (e anche di conservazione dell'energia meccanica di conseguenza) per i sistemi non inerziali. Si dovrà tenere conto del lavoro compiuto anche da tutte le forze apparenti:
\[
	\mathcal{L}_{\text{F}_{\text{reali}}}+\mathcal{L}_{\text{F}_{\text{app}}}=\Delta E_{ \text{k, rel}}
\]
Nel calcolo del lavoro bisogna valutare quello delle forze mentre il punto materiali si muove sulla traiettoria percepita dall'osservatore relativo. Se le forze apparenti non fossero costanti mentre il punto si muove si tratterebbe di risolvere un'equazione differenziale, problema non che non verrà affrontato.







































\section{Forze conservative e non}

\subsection{Esempi di lavoro}

Si espongono in seguito alcuni esempi di calcolo di lavoro compiuto da alcune forze.

\subsubsection{Lavoro della forza peso}

Un punto materiale si sta muovendo da una posizione $A$ a una posizione $B$ sulla traiettoria. Si definisce un asse $z$ rivolto verso l'alto e si chiamano $z_a$ la quota iniziale a cui si trova il punto e $z_b$ quella finale. In un punto generico si va a mettere in evidenza la direzione della forza peso e la direzione dello spostamento. Il lavoro della forza peso lungo la traiettoria $\Gamma$ che va da $A$ a $B$ non è altro che:
\[
	\mathcal{L}_{\Gamma, A \to B}=\int_{\Gamma, A \to B} m\vec{g} \cdot \vec{dr}
\]
In questo caso lo spostamento continua a cambiare la propria direzione mentre la forza peso la mantiene costante. Quindi conviene scomporre $d\vec{r}$ e considerare la sua proiezione su $z$.
\[
	d\vec{r}=dx\vec{u}_x+dy\vec{u}_y+dz\vec{u}_z \quad m\vec{g}=-mg\vec{u}_z
\]
Sopravvive solo la terza componente perché le altre due sono ortogonali alla forza peso.
\[
	\boxed{\mathcal{L}_{\Gamma, A \to B}=\int_{\Gamma, A \to B} -mg\,dz=mg(z_a-z_b)}
\]
\begin{figure}[htpb]
	\centering
	

	\tikzset{every picture/.style={line width=0.75pt}} %set default line width to 0.75pt        

	\begin{tikzpicture}[x=0.75pt,y=0.75pt,yscale=-1,xscale=1]
	%uncomment if require: \path (0,300); %set diagram left start at 0, and has height of 300

	%Straight Lines [id:da4567550117967014] 
	\draw    (179.5,226) -- (179.5,67) ;
	\draw [shift={(179.5,64)}, rotate = 450] [fill={rgb, 255:red, 0; green, 0; blue, 0 }  ][line width=0.08]  [draw opacity=0] (10.72,-5.15) -- (0,0) -- (10.72,5.15) -- (7.12,0) -- cycle    ;
	%Straight Lines [id:da8799949196323655] 
	\draw  [dash pattern={on 0.84pt off 2.51pt}]  (179.33,208) -- (206.5,208) ;
	%Curve Lines [id:da992092583362937] 
	\draw    (206.5,208) .. controls (240.5,138) and (247.5,216) .. (287.5,186) .. controls (327.5,156) and (309.5,118) .. (352.5,138) .. controls (395.5,158) and (401.5,139) .. (411.5,98) ;
	%Shape: Circle [id:dp1915008511555767] 
	\draw  [fill={rgb, 255:red, 0; green, 0; blue, 0 }  ,fill opacity=1 ] (203.67,208) .. controls (203.67,206.44) and (204.94,205.17) .. (206.5,205.17) .. controls (208.06,205.17) and (209.33,206.44) .. (209.33,208) .. controls (209.33,209.56) and (208.06,210.83) .. (206.5,210.83) .. controls (204.94,210.83) and (203.67,209.56) .. (203.67,208) -- cycle ;
	%Shape: Circle [id:dp5508839626190245] 
	\draw  [fill={rgb, 255:red, 0; green, 0; blue, 0 }  ,fill opacity=1 ] (408.67,98) .. controls (408.67,96.44) and (409.94,95.17) .. (411.5,95.17) .. controls (413.06,95.17) and (414.33,96.44) .. (414.33,98) .. controls (414.33,99.56) and (413.06,100.83) .. (411.5,100.83) .. controls (409.94,100.83) and (408.67,99.56) .. (408.67,98) -- cycle ;
	%Straight Lines [id:da7581801323720281] 
	\draw  [dash pattern={on 0.84pt off 2.51pt}]  (179.33,98) -- (411.5,98) ;
	%Straight Lines [id:da7547784460352234] 
	\draw    (352.5,138) -- (352.5,201.33) ;
	\draw [shift={(352.5,204.33)}, rotate = 270] [fill={rgb, 255:red, 0; green, 0; blue, 0 }  ][line width=0.08]  [draw opacity=0] (10.72,-5.15) -- (0,0) -- (10.72,5.15) -- (7.12,0) -- cycle    ;
	%Straight Lines [id:da3231983593688519] 
	\draw    (352.5,138) -- (390.51,155.02) ;
	\draw [shift={(393.25,156.25)}, rotate = 204.13] [fill={rgb, 255:red, 0; green, 0; blue, 0 }  ][line width=0.08]  [draw opacity=0] (10.72,-5.15) -- (0,0) -- (10.72,5.15) -- (7.12,0) -- cycle    ;
	%Shape: Circle [id:dp43575236520295135] 
	\draw  [fill={rgb, 255:red, 0; green, 0; blue, 0 }  ,fill opacity=1 ] (349.67,138) .. controls (349.67,136.44) and (350.94,135.17) .. (352.5,135.17) .. controls (354.06,135.17) and (355.33,136.44) .. (355.33,138) .. controls (355.33,139.56) and (354.06,140.83) .. (352.5,140.83) .. controls (350.94,140.83) and (349.67,139.56) .. (349.67,138) -- cycle ;

	% Text Node
	\draw (163.67,61.33) node    {$z$};
	% Text Node
	\draw (160.67,95.33) node    {$z_{B}$};
	% Text Node
	\draw (159.67,204.33) node    {$z_{A}$};
	% Text Node
	\draw (216.67,212.67) node    {$A$};
	% Text Node
	\draw (429.33,89.67) node    {$B$};
	% Text Node
	\draw (371.67,198) node    {$m\vec{g}$};
	% Text Node
	\draw (404.5,161) node    {$d\vec{r}$};


	\end{tikzpicture}
\end{figure}
\FloatBarrier
Si noti che il lavoro non dipende dalla traiettoria ma solo dalla quota finale e da quella iniziale. Questo significa che se il punto per andare da $A$ a $B$ ci va seguendo diverse traiettorie, il lavoro fatto dalla forza peso è lo stesso. Il segno meno significa che quando il punto materiale viene sollevato di quota il lavoro è negativo. Viceversa se è il punto materiale che scivola giù, il lavoro diventa positivo. Infatti, se il punto $B$ si trova in una posizione più bassa di $A$, $AB$ è lo spostamento naturale di un punto $P$ sottoposto alla sola forza peso. Se invece il punto $B$ è più in alto rispetto ad $A$ significa che il punto deve avere una sufficiente velocità iniziale così che la diminuzione di energia cinetica eguagli il lavoro. Altrimenti è necessario applicare al punto un'altra forza il cui lavoro motore superi in modulo il lavoro resistente della forza peso.

\subsubsection{Lavoro di una forza elastica}

Si immagini di avere un punto materiale di massa $M$ vincolato a una molla di costante elastica $k$. La molla viene allungata, per cui il punto materiale oscilla seguendo una legge di moto armonico (che verrà in seguito affrontato). Si è interessati a capire qual è il lavoro fatto dalla forza elastica quando la molla passa per le posizioni generiche $A$ e $B$. Quando allungata, essa genera una forza di richiamo che tende a riportare il punto materiale nella posizione $O$.
\[
	\boxed{\mathcal{L}_{\Gamma, A \to B}=\int_{\Gamma, A \to B} \vec{F}\cdot d\vec{r}= \int_{x_A}^{x_B} -kx\,dx=\frac{1}{2}k(x_A^2-x_B^2)}
\]
\begin{figure}[htpb]
	\centering
	

	\tikzset{every picture/.style={line width=0.75pt}} %set default line width to 0.75pt        

	\begin{tikzpicture}[x=0.75pt,y=0.75pt,yscale=-1,xscale=1]
	%uncomment if require: \path (0,300); %set diagram left start at 0, and has height of 300

	%Shape: Rectangle [id:dp4234502411788541] 
	\draw  [draw opacity=0][fill={rgb, 255:red, 155; green, 155; blue, 155 }  ,fill opacity=1 ] (204,129) -- (281.5,129) -- (281.5,204) -- (204,204) -- cycle ;
	%Shape: Axis 2D [id:dp8158083942748016] 
	\draw  (77,203.9) -- (364.5,203.9)(105.75,68) -- (105.75,219) (357.5,198.9) -- (364.5,203.9) -- (357.5,208.9) (100.75,75) -- (105.75,68) -- (110.75,75)  ;
	%Shape: Spring [id:dp9074821559743382] 
	\draw   (106,165) .. controls (106.88,155) and (110.13,145) .. (116.63,145) .. controls (129.63,145) and (129.63,185) .. (123.63,185) .. controls (117.63,185) and (117.63,145) .. (130.63,145) .. controls (143.63,145) and (143.63,185) .. (137.63,185) .. controls (131.63,185) and (131.63,145) .. (144.63,145) .. controls (157.63,145) and (157.63,185) .. (151.63,185) .. controls (145.63,185) and (145.63,145) .. (158.63,145) .. controls (171.63,145) and (171.63,185) .. (165.63,185) .. controls (159.63,185) and (159.63,145) .. (172.63,145) .. controls (185.63,145) and (185.63,185) .. (179.63,185) .. controls (173.63,185) and (173.63,145) .. (186.63,145) .. controls (199.63,145) and (199.63,185) .. (193.63,185) .. controls (187.63,185) and (187.63,145) .. (200.63,145) .. controls (202.08,145) and (203.36,145.5) .. (204.5,146.38) ;
	%Straight Lines [id:da4992112063762819] 
	\draw    (242.75,166.5) -- (273.75,166.5) ;
	\draw [shift={(276.75,166.5)}, rotate = 180] [fill={rgb, 255:red, 0; green, 0; blue, 0 }  ][line width=0.08]  [draw opacity=0] (10.72,-5.15) -- (0,0) -- (10.72,5.15) -- (7.12,0) -- cycle    ;
	%Straight Lines [id:da1169902927225368] 
	\draw    (211.75,166.5) -- (242.75,166.5) ;
	\draw [shift={(208.75,166.5)}, rotate = 0] [fill={rgb, 255:red, 0; green, 0; blue, 0 }  ][line width=0.08]  [draw opacity=0] (10.72,-5.15) -- (0,0) -- (10.72,5.15) -- (7.12,0) -- cycle    ;
	%Shape: Circle [id:dp39681623393842536] 
	\draw  [fill={rgb, 255:red, 0; green, 0; blue, 0 }  ,fill opacity=1 ] (239.92,166.5) .. controls (239.92,164.94) and (241.19,163.67) .. (242.75,163.67) .. controls (244.31,163.67) and (245.58,164.94) .. (245.58,166.5) .. controls (245.58,168.06) and (244.31,169.33) .. (242.75,169.33) .. controls (241.19,169.33) and (239.92,168.06) .. (239.92,166.5) -- cycle ;
	%Straight Lines [id:da5686270262356254] 
	\draw    (222,200.33) -- (222,208) ;
	%Straight Lines [id:da28261884806148196] 
	\draw    (262,200.33) -- (262,208) ;

	% Text Node
	\draw (374.67,211.33) node    {$x$};
	% Text Node
	\draw (243.67,113.33) node    {$M$};
	% Text Node
	\draw (261.83,148.33) node    {$d\vec{r}$};
	% Text Node
	\draw (223.67,216.33) node    {$x_{A}$};
	% Text Node
	\draw (263.67,216.33) node    {$x_{B}$};
	% Text Node
	\draw (97.67,213.33) node    {$0$};
	% Text Node
	\draw (227.17,147.33) node    {$\vec{F}$};


	\end{tikzpicture}
\end{figure}
\FloatBarrier
Anche per la forza elastica è possibile non considerare più la dipendenza dalla traiettoria. Il lavoro fatto dalla molla è lo stesso quando la massa va da $A$ a $B$ oppure quando partendo da $A$ fa un oscillazione completa
e poi torna su $B$. Le reazioni vincolari non vanno considerare perché sono ortogonali allo spostamento.

\subsubsection{Lavoro della forza di attrito radente}

La forza di attrito radente statico non compie lavoro perché, come già ribadito, quando non si ha spostamento non è mai possibile avere lavoro. Si consideri allora la forza di attrito radente dinamico. Sia un corpo appoggiato a un piano scabro con coefficiente di attrito $\mu_d$ che si sta muovendo in avanti. La forza di attrito dinamico sarà tangente al piano di appoggio e opposta allo spostamento.
\[
	\mathcal{L}_{\Gamma, A \to B}=\int_A^B \vec{F}_{\text{attr}} \cdot d\vec{r}
\]
Scomponendo le forze in componenti intrinseche alla traiettoria:
\[
	\vec{F}_{\text{attr}}=-\mu_d\norma{\vec{R}_n} \vec{u}_t \quad d\vec{r}=ds\vec{u}_t
\]
Supponendo che il punto si muova sullo stesso piano, $R_n$ è costante.
\[
	\boxed{\mathcal{L}_{\Gamma, A \to B}=\int_A^B -\mu_d\norma{\vec{R}_n} \,ds=-\mu_d\norma{\vec{R}_n}L_\Gamma}
\]
\begin{figure}[htpb]
	\centering

	% Pattern Info
	 
	\tikzset{
	pattern size/.store in=\mcSize, 
	pattern size = 5pt,
	pattern thickness/.store in=\mcThickness, 
	pattern thickness = 0.3pt,
	pattern radius/.store in=\mcRadius, 
	pattern radius = 1pt}
	\makeatletter
	\pgfutil@ifundefined{pgf@pattern@name@_c6kb6rdxs}{
	\pgfdeclarepatternformonly[\mcThickness,\mcSize]{_c6kb6rdxs}
	{\pgfqpoint{0pt}{-\mcThickness}}
	{\pgfpoint{\mcSize}{\mcSize}}
	{\pgfpoint{\mcSize}{\mcSize}}
	{
	\pgfsetcolor{\tikz@pattern@color}
	\pgfsetlinewidth{\mcThickness}
	\pgfpathmoveto{\pgfqpoint{0pt}{\mcSize}}
	\pgfpathlineto{\pgfpoint{\mcSize+\mcThickness}{-\mcThickness}}
	\pgfusepath{stroke}
	}}
	\makeatother
	\tikzset{every picture/.style={line width=0.75pt}} %set default line width to 0.75pt        

	\begin{tikzpicture}[x=0.75pt,y=0.75pt,yscale=-1,xscale=1]
	%uncomment if require: \path (0,300); %set diagram left start at 0, and has height of 300

	%Shape: Rectangle [id:dp5747589760441774] 
	\draw  [draw opacity=0][pattern=_c6kb6rdxs,pattern size=6.5249999999999995pt,pattern thickness=0.75pt,pattern radius=0pt, pattern color={rgb, 255:red, 155; green, 155; blue, 155}] (115.5,204) -- (396.5,204) -- (396.5,231) -- (115.5,231) -- cycle ;
	%Shape: Rectangle [id:dp5186133906953796] 
	\draw  [draw opacity=0][fill={rgb, 255:red, 155; green, 155; blue, 155 }  ,fill opacity=1 ] (164,129) -- (241.5,129) -- (241.5,204) -- (164,204) -- cycle ;
	%Straight Lines [id:da980881591168157] 
	\draw    (202.75,166.5) -- (295,166.5) ;
	\draw [shift={(298,166.5)}, rotate = 180] [fill={rgb, 255:red, 0; green, 0; blue, 0 }  ][line width=0.08]  [draw opacity=0] (10.72,-5.15) -- (0,0) -- (10.72,5.15) -- (7.12,0) -- cycle    ;
	%Shape: Circle [id:dp803196961729584] 
	\draw  [fill={rgb, 255:red, 0; green, 0; blue, 0 }  ,fill opacity=1 ] (199.92,166.5) .. controls (199.92,164.94) and (201.19,163.67) .. (202.75,163.67) .. controls (204.31,163.67) and (205.58,164.94) .. (205.58,166.5) .. controls (205.58,168.06) and (204.31,169.33) .. (202.75,169.33) .. controls (201.19,169.33) and (199.92,168.06) .. (199.92,166.5) -- cycle ;
	%Straight Lines [id:da10740669680439652] 
	\draw    (115.5,204) -- (410.5,204) ;
	\draw [shift={(413.5,204)}, rotate = 180] [fill={rgb, 255:red, 0; green, 0; blue, 0 }  ][line width=0.08]  [draw opacity=0] (10.72,-5.15) -- (0,0) -- (10.72,5.15) -- (7.12,0) -- cycle    ;
	%Straight Lines [id:da23089935827826258] 
	\draw    (125.83,166.5) -- (202.75,166.5) ;
	\draw [shift={(122.83,166.5)}, rotate = 0] [fill={rgb, 255:red, 0; green, 0; blue, 0 }  ][line width=0.08]  [draw opacity=0] (10.72,-5.15) -- (0,0) -- (10.72,5.15) -- (7.12,0) -- cycle    ;
	%Straight Lines [id:da39793383823784767] 
	\draw    (354.75,166.5) -- (411.67,166.5) ;
	\draw [shift={(414.67,166.5)}, rotate = 180] [fill={rgb, 255:red, 0; green, 0; blue, 0 }  ][line width=0.08]  [draw opacity=0] (10.72,-5.15) -- (0,0) -- (10.72,5.15) -- (7.12,0) -- cycle    ;
	%Straight Lines [id:da8122406232454551] 
	\draw    (202.75,166.5) -- (202.75,192.75) ;
	\draw [shift={(202.75,195.75)}, rotate = 270] [fill={rgb, 255:red, 0; green, 0; blue, 0 }  ][line width=0.08]  [draw opacity=0] (10.72,-5.15) -- (0,0) -- (10.72,5.15) -- (7.12,0) -- cycle    ;
	%Straight Lines [id:da5706791620810392] 
	\draw    (202.75,117.25) -- (202.75,166.5) ;
	\draw [shift={(202.75,114.25)}, rotate = 90] [fill={rgb, 255:red, 0; green, 0; blue, 0 }  ][line width=0.08]  [draw opacity=0] (10.72,-5.15) -- (0,0) -- (10.72,5.15) -- (7.12,0) -- cycle    ;
	%Straight Lines [id:da8196494490501676] 
	\draw    (202.75,164.5) -- (259.67,164.5) ;
	\draw [shift={(262.67,164.5)}, rotate = 180] [fill={rgb, 255:red, 0; green, 0; blue, 0 }  ][line width=0.08]  [draw opacity=0] (10.72,-5.15) -- (0,0) -- (10.72,5.15) -- (7.12,0) -- cycle    ;

	% Text Node
	\draw (425.67,202.33) node    {$x$};
	% Text Node
	\draw (257.33,148.33) node    {$d\vec{r}$};
	% Text Node
	\draw (129.17,141.33) node    {$\vec{F}_{att}$};
	% Text Node
	\draw (335.17,190.5) node    {$\mu _{d}$};
	% Text Node
	\draw (381.33,151.83) node    {$\vec{u}_{t}$};
	% Text Node
	\draw (222.33,181.83) node    {$\vec{R}_{n}$};
	% Text Node
	\draw (320.83,152.83) node    {$\vec{F}_{\text{motore}}$};
	% Text Node
	\draw (263,117) node  [font=\footnotesize] [align=left] {reazione vincolare};


	\end{tikzpicture}
\end{figure}
\FloatBarrier
Sommando ogni infinitesima parte $ds$ della traiettoria, si ottiene la sua lunghezza. Quindi si nota che il lavoro compiuto dalla forza di attrito invece dipende da $\Gamma$, inoltre è sempre negativo, è lavoro resistente. Perché possa verificarsi il moto deve agire o un'altra forza che produca un lavoro motore, oppure, in assenza di questa, il punto deve possedere una certa velocità iniziale, ovvero una certa energia cinetica, che diminuisce lungo il percorso.

\subsection{Definizione}

È possibile ora andare a classificare le forze che compiono lavoro in due categorie, a seconda che il lavoro dipenda dalla traiettoria o meno.
Si dice che una forza è \textbf{conservativa} se il lavoro fatto da essa non dipende dalla traiettoria ma solo dalla posizione finale e iniziale. Tutte le altre forze tali per cui il loro lavoro dipende dal percorso seguito prendono il nome di forze non conservative.

Si considerino le sole forze conservative. L'obbiettivo è quello di scrivere in forma analitica la definizione appena data. Se per il calcolo del lavoro infatti si può utilizzare qualsiasi percorso che colleghi $A$ e $B$, esso può essere espresso come differenza dei valori che una funzione delle coordinate assume in $A$ e $B$. Si può sempre trovare una funzione della posizione $f(P)$ in cui si trova il punto materiale tale per cui il lavoro della forza che si sta considerando è dato dalla differenza fra $f(A)$ e $f(B)$.
\[
	\exists \quad f(P)\,: \, \mathcal{L}_{A \to B}= f(A)-f(B)
\]
Ciò comporta che se si inverte il segno di percorrenza, cioè se si va da $B$ ad $A$, cambia solo il segno del lavoro. Inoltre, per un percorso chiuso $ABA$ lungo la traiettoria si avrà un lavoro nullo.
\[
	\boxed{\text{Forza conservativa} \iff \oint \vec{F} \cdot d\vec{s} = 0}
\]
Questa proprietà si può assumere come definizione di forza conservativa.
Dimensionalmente la funzione $f$ sarà un'energia, che prende il nome di 	\textbf{energia potenziale}, posseduta dal punto materiale e associata a una certa forza conservativa. Si distingue dall'energia cinetica perché quest'ultima è associata al lavoro compiuto da \emph{tutte} le forze agenti, non solo da un'unica forza conservativa. Dunque per tutte le forze conservative vale la relazione:
\[
	\mathcal{L}=E_{\text{pot}}(A)-E_{\text{pot}}(B)=-\Delta E_{\text{pot}}
\]
Non esiste una formula generale per l'energia potenziale, ma l'espressione esplicita dipende dalla particolare forza cui essa si riferisce.

\paragraph{Significato fisico} Si vuole spostare un punto materiale da una posizione $B$ generica a una posizione di riferimento $A$, ad esempio quella in cui il corpo tende ad andare sotto l'azione della forza. L'energia potenziale in questo punto $U(B)$ equivale al lavoro che deve compiere la forza conservativa per portare l'oggetto fino ad $A$. Essa si chiama quindi così perché è l'energia posseduta da un corpo che \emph{potenzialmente} si potrà trasformare in lavoro per tornare alla posizione di riferimento. Quando durante il moto l'energia potenziale diminuisce, il lavoro compiuto dalla forza è positivo e può essere utilizzato. Come già detto, per un percorso chiuso se si spende lavoro per fare aumentare l'energia potenziale nel passaggio da $A$ a $B$, quando si torna da $B$ ad $A$ si ricava esattamente quanto speso, quindi una forza conservativa non fornisce lavoro lungo un ciclo.

\paragraph{Energia potenziale della forza peso} Si va ora a cercare questa funzione per la forza peso:
\[
	\mathcal{L}_{\text{F,peso}_{A\to B}}=mg(z_a-z_b)=mgz_a-mgz_b=E_{\text{pot}}(A)-E_{ \text{pot} } (B)=-\Delta E_{ \text{pot} }
\]
Risulta che l'energia potenziale della forza peso è l'espressione $mgz$. La definizione di energia potenziale è data come variazione, infatti l'espressione $mgz$ la si potrebbe scrivere come $mgz+$cost. Tutte le definizioni di energia potenziale sono date a meno di una costante arbitraria, che è possibile scegliere a proprio piacimento. La scelta più conveniente per la forza peso, definita un asse $z$, è quella di considerare $0$ l'energia potenziale a quota $0$, dove l'oggetto tende a tornare se lasciato libero.

\paragraph{Energia potenziale della forza elastica}
\[
	\mathcal{L}_{F,el A\to B}=\frac{1}{2}k(x_a^2-x_b^2)=E(A)-E(B)
\]
La possibile espressione dell'energia potenziale della forza elastica quando la molla è allungata di un generico tratto $x$ è:
\[
	\frac{1}{2}kx^2+\text{cost}
\]
Anche in questo caso la costante viene scelta zero quando la molla è a riposo. A differenza dell'energia potenziale della forza peso, essa è sempre positiva. Dal punto di vista fisico ciò accade perché una molla spontaneamente tenderà sempre a riportare il corpo nella posizione di riposo.

Per forze non conservative non vale la proprietà di invarianza del lavoro rispetto al percorso e non è quindi possibile esprimere questo tramite la differenza dei valori di una funzione delle coordinate: per esse non si può introdurre l'energia potenziale, ma continua comunque a valere il teorema dell'energia cinetica. Una classe particolare di forze non conservative sono le forze di attrito, dette anche forze \textbf{dissipative}.







































\section{Teorema dell'energia meccanica}

Si riprenda il teorema dell'energia cinetica:
\[
	\mathcal{L}_\Gamma, A\to B= \Delta E_k
\]
Si riscrive il termine come il lavoro fatto dalle forze conservative e non mentre il punto materiale si sposta sulla traiettoria $\Gamma$ da $A$ a $B$.
\[
	\mathcal{L}_{\text{F.N.C.} \Gamma, A \to B}+\mathcal{L}_{\text{F.C.} \Gamma, A \to B}=\Delta E_k
\]
Il lavoro delle forze conservative si può definire come l'opposto della variazione di energia potenziale:
\[
	\mathcal{L}_{FC}=-\Delta E_{\text{p}}
\]
Si ha allora che:
\[
	\boxed{\mathcal{L}_{\text{FNC}}=\Delta E_k + \Delta E_p}
\]
A questo punto si va a definire un terzo contributo energetico detto \textbf{energia meccanica}, dato dalla somma dell'energia cinetica posseduta dal punto materiale più la somma di tutti i contributi dell'energia potenziale.
\[
	\boxed{\Delta E_{\text{mecc}}=\Delta E_k + \Delta E_p}
\]
Si trova così:
\[
	\mathcal{L}_{\text{FNC}}=\Delta E_\text{mecc}
\]
Questa espressione è nota come \textbf{teorema dell'energia meccanica}.

In particolare, l'energia meccanica di un punto materiale che si muove sotto l'azione di forze conservative resta costante durante il moto, si conserva.
\[
	\Delta E_{\text{mecc}}=0
\]
Questo risultato è noto come \textbf{principio di conservazione dell'energia meccanica} e vale se sul corpo agiscono solo forze conservative. Esse si chiamano così proprio perché permettono all'energia meccanica di conservarsi. Il teorema dell'energia meccanica non è altro che un modo diverso per esprimere il teorema dell'energia cinetica.
\[
	\boxed{\mathcal{L}_{\text{FNC}, \Gamma, A \to B}= \Delta E_{\text{mecc}} \iff \mathcal{L}_{\Gamma, A \to B}= \Delta E_k}
\]
D'altra parte la seconda espressione è identica a usare il secondo principio della dinamica sulla direzione tangente. In realtà si può dimostrare che il principio di conservazione dell'energia meccanica ha validità molto più ampia. Esso infatti è uno dei tre principi fondamentali di conservazione su cui si basa tutta la fisica, non solo quella classica ma quella relativistica e quantistica. In ambiti più ampi questi tre principi di conservazione discendono dalle proprietà di simmetria dello spazio e di omogeneità del tempo.

Il fatto che l'energia meccanica si conservi significa che durante il moto avviene una trasformazione da una forma di energia all'altra, per tramite di lavoro compiuto e assorbito, ma il contenuto energetico totale non cambia.
In presenza di forze non conservative l'energia meccanica non resta costante e la sua variazione è eguale al lavoro delle forze non conservative. In qualunque processo meccanico si osserva sperimentalmente che è sempre presente una forza di attrito che si oppone al moto. A questa situazione si può però porre rimedio con l'intervento di altre forze non conservative e l'energia meccanica del punto può essere mantenuta costante o anche aumentare. Se però si considera un sistema più vasto, costituito dal punto e dal meccanismo che ha fornito il lavoro non conservativo, si trova sempre che l'energia meccanica complessiva non si conserva, ma diminuisce a causa di effetti dissipativi. Nei fenomeni macroscopici questa appare una legge naturale.

\paragraph{Esempio} Si supponga di avere un corpo che si trova a una certa quota $h$ di un piano scabro inclinato di un certo angolo $\vartheta$, esso è inizialmente fermo e viene fatto scivolare su un piano d'appoggio (si muove). Si vuole ricavare la velocità finale.
\begin{figure}[htb]
	\centering
	

	% Pattern Info
	 
	\tikzset{
	pattern size/.store in=\mcSize, 
	pattern size = 5pt,
	pattern thickness/.store in=\mcThickness, 
	pattern thickness = 0.3pt,
	pattern radius/.store in=\mcRadius, 
	pattern radius = 1pt}
	\makeatletter
	\pgfutil@ifundefined{pgf@pattern@name@_wp7nxmbtm}{
	\pgfdeclarepatternformonly[\mcThickness,\mcSize]{_wp7nxmbtm}
	{\pgfqpoint{-\mcThickness}{-\mcThickness}}
	{\pgfpoint{\mcSize}{\mcSize}}
	{\pgfpoint{\mcSize}{\mcSize}}
	{
	\pgfsetcolor{\tikz@pattern@color}
	\pgfsetlinewidth{\mcThickness}
	\pgfpathmoveto{\pgfpointorigin}
	\pgfpathlineto{\pgfpoint{0}{\mcSize}}
	\pgfusepath{stroke}
	}}
	\makeatother
	\tikzset{every picture/.style={line width=0.75pt}} %set default line width to 0.75pt        

	\begin{tikzpicture}[x=0.75pt,y=0.75pt,yscale=-1,xscale=1]
	%uncomment if require: \path (0,300); %set diagram left start at 0, and has height of 300

	%Shape: Rectangle [id:dp6778794918506883] 
	\draw  [draw opacity=0][pattern=_wp7nxmbtm,pattern size=4.425000000000001pt,pattern thickness=0.75pt,pattern radius=0pt, pattern color={rgb, 255:red, 155; green, 155; blue, 155}] (195.74,107.49) -- (429.25,215.39) -- (423.01,228.9) -- (189.5,121) -- cycle ;
	%Shape: Rectangle [id:dp9140602617424805] 
	\draw  [draw opacity=0][fill={rgb, 255:red, 155; green, 155; blue, 155 }  ,fill opacity=1 ] (299.55,99.37) -- (379.86,136.48) -- (358.66,182.37) -- (278.35,145.26) -- cycle ;
	%Shape: Right Triangle [id:dp6608914674089241] 
	\draw   (170.5,95.72) -- (511.06,252.09) -- (170.5,252.09) -- cycle ;
	%Straight Lines [id:da25182783637159534] 
	\draw    (329.11,140.87) -- (329.11,224.24) ;
	\draw [shift={(329.11,227.24)}, rotate = 270] [fill={rgb, 255:red, 0; green, 0; blue, 0 }  ][line width=0.08]  [draw opacity=0] (10.72,-5.15) -- (0,0) -- (10.72,5.15) -- (7.12,0) -- cycle    ;
	%Straight Lines [id:da6672905585046607] 
	\draw    (329.11,140.87) -- (295.81,208.2) ;
	\draw [shift={(294.48,210.89)}, rotate = 296.32] [fill={rgb, 255:red, 0; green, 0; blue, 0 }  ][line width=0.08]  [draw opacity=0] (10.72,-5.15) -- (0,0) -- (10.72,5.15) -- (7.12,0) -- cycle    ;
	%Straight Lines [id:da45934902095695485] 
	\draw    (358.87,80.12) -- (328.6,141.88) ;
	\draw [shift={(360.2,77.43)}, rotate = 116.11] [fill={rgb, 255:red, 0; green, 0; blue, 0 }  ][line width=0.08]  [draw opacity=0] (10.72,-5.15) -- (0,0) -- (10.72,5.15) -- (7.12,0) -- cycle    ;
	%Shape: Arc [id:dp6015484662404116] 
	\draw  [draw opacity=0] (329.26,185.25) .. controls (329.04,185.25) and (328.82,185.25) .. (328.6,185.25) .. controls (321.74,185.25) and (315.25,183.66) .. (309.48,180.82) -- (328.6,141.88) -- cycle ; \draw   (329.26,185.25) .. controls (329.04,185.25) and (328.82,185.25) .. (328.6,185.25) .. controls (321.74,185.25) and (315.25,183.66) .. (309.48,180.82) ;
	%Shape: Arc [id:dp8671287529080107] 
	\draw  [draw opacity=0] (473.15,251.89) .. controls (473.18,246.22) and (474.45,240.85) .. (476.71,236.02) -- (511.06,252.09) -- cycle ; \draw   (473.15,251.89) .. controls (473.18,246.22) and (474.45,240.85) .. (476.71,236.02) ;
	%Straight Lines [id:da3117130690188432] 
	\draw    (150.5,95.72) -- (150.5,252.09) ;
	\draw [shift={(150.5,252.09)}, rotate = 270] [color={rgb, 255:red, 0; green, 0; blue, 0 }  ][line width=0.75]    (0,5.59) -- (0,-5.59)   ;
	\draw [shift={(150.5,95.72)}, rotate = 270] [color={rgb, 255:red, 0; green, 0; blue, 0 }  ][line width=0.75]    (0,5.59) -- (0,-5.59)   ;
	%Straight Lines [id:da6085455179233916] 
	\draw    (511.06,272.09) -- (170.5,272.09) ;
	\draw [shift={(170.5,272.09)}, rotate = 360] [color={rgb, 255:red, 0; green, 0; blue, 0 }  ][line width=0.75]    (0,5.59) -- (0,-5.59)   ;
	\draw [shift={(511.06,272.09)}, rotate = 360] [color={rgb, 255:red, 0; green, 0; blue, 0 }  ][line width=0.75]    (0,5.59) -- (0,-5.59)   ;
	%Straight Lines [id:da35968133807125735] 
	\draw    (328.6,141.88) -- (360.52,156.96) ;
	\draw [shift={(363.23,158.24)}, rotate = 205.29] [fill={rgb, 255:red, 0; green, 0; blue, 0 }  ][line width=0.08]  [draw opacity=0] (10.72,-5.15) -- (0,0) -- (10.72,5.15) -- (7.12,0) -- cycle    ;
	%Straight Lines [id:da00019852261181596553] 
	\draw    (296.69,126.81) -- (328.6,141.88) ;
	\draw [shift={(293.97,125.53)}, rotate = 25.29] [fill={rgb, 255:red, 0; green, 0; blue, 0 }  ][line width=0.08]  [draw opacity=0] (10.72,-5.15) -- (0,0) -- (10.72,5.15) -- (7.12,0) -- cycle    ;
	%Straight Lines [id:da5817894701007842] 
	\draw  [dash pattern={on 0.84pt off 2.51pt}]  (294.48,210.89) -- (329.11,227.24) ;
	%Straight Lines [id:da34165750572159537] 
	\draw  [dash pattern={on 0.84pt off 2.51pt}]  (363.73,157.23) -- (329.11,227.24) ;

	% Text Node
	\draw (380.92,82.74) node    {$\vec{R}_{n}$};
	% Text Node
	\draw (387.87,162.36) node    {$\vec{P}_{x}$};
	% Text Node
	\draw (282.35,212.91) node    {$\vec{P}_{y}$};
	% Text Node
	\draw (458.86,239.09) node    {$\vartheta $};
	% Text Node
	\draw (273.27,112.56) node    {$\vec{R}_{t}$};
	% Text Node
	\draw (139.36,173.59) node    {$h$};
	% Text Node
	\draw (524.36,271.59) node    {$L$};
	% Text Node
	\draw (339.35,228.41) node    {$\vec{P}$};
	% Text Node
	\draw (316.36,194.59) node    {$\vartheta $};


	\end{tikzpicture}
\end{figure}
\FloatBarrier
\begin{align*}
	\mathcal{L}_{\text{F.N.C.}_{\Gamma, A \to B}} &= \vec{F}_{\text{attr}} \cdot d\vec{r}= \norma{\vec{F}_{\text{attr}}}L \tag*{$L=\frac{h}{\sin\vartheta}$} \\
	&= mg\cos\vartheta \mu_d \frac{h}{\sin\vartheta}=\cot\vartheta \,mgh\, \mu_d
\end{align*}
La forza peso è l'unica che da un contributo potenziale:
\begin{gather*}
	\mathcal{L}_{\text{F.C.}_{\Gamma, A \to B}}=mg\sin\vartheta \frac{h}{\sin\vartheta} = mgh \\
\end{gather*}
Allora
\begin{align*}
	\mathcal{L}_{\text{F.N.C.}_{\Gamma, A \to B}} &= E_{\text{mecc, fin}} - E_{\text{mecc, in}} \\
	\cot\vartheta \,mgh\, \mu_d &= \frac{1}{2}mv^2 +0 - (0 + mgh) \\
	\cot\vartheta \,gh\, \mu_d &= \frac{1}{2}v^2 -gh \\
	gh + \cot\vartheta \,gh\, \mu_d &= \frac{1}{2}v^2 \\
	2gh (1 + \cot\vartheta \, \mu_d) &= v^2 \\
	v_{\text{fin}} &= \sqrt{2gh\left(1+\frac{\mu_d}{\tan\vartheta}\right)}
\end{align*}
























































































































\chapter{Il moto armonico}

\section{Cinematica del moto armonico}

Un moto armonico semplice è un moto la cui legge oraria è una funzione sinusoidale o cosinusoidale nel tempo. Si tratta di un andamento periodico, perché ad intervalli di tempo eguale il punto ripassa nella stessa posizione con la stessa velocità.
\begin{equation}
	\label{eqn:armonico}
	x(t)=A\cos(\omega_0 t+\Phi)
\end{equation}
\begin{figure}[htpb]
	\centering
	

	\tikzset{every picture/.style={line width=0.75pt}} %set default line width to 0.75pt        

	\begin{tikzpicture}[x=0.75pt,y=0.75pt,yscale=-1,xscale=1]
	%uncomment if require: \path (0,300); %set diagram left start at 0, and has height of 300

	% Plotting does not support converting to Tikz
	%Shape: Axis 2D [id:dp732307782075978] 
	\draw  (110,155) -- (501.5,155)(145.5,49) -- (145.5,257) (494.5,150) -- (501.5,155) -- (494.5,160) (140.5,56) -- (145.5,49) -- (150.5,56)  ;
	%Straight Lines [id:da7766397595927959] 
	\draw [line width=0.75]  [dash pattern={on 0.84pt off 2.51pt}]  (147,89) -- (482.5,89) ;
	%Straight Lines [id:da6094182586898851] 
	\draw [line width=0.75]  [dash pattern={on 0.84pt off 2.51pt}]  (147,219) -- (482.5,219) ;
	%Curve Lines [id:da2449173035027563] 
	\draw [line width=1.5]    (147,89) .. controls (159,90.75) and (169,117.25) .. (181.5,154.25) .. controls (209,240.75) and (221,241.25) .. (249.5,155.25) .. controls (278,69.25) and (289,68.75) .. (317,154.75) .. controls (346,239.25) and (355,240.75) .. (385,154.75) .. controls (415,68.75) and (423,69.75) .. (453.5,154.75) .. controls (467,193.75) and (472.5,218.25) .. (485.5,219.75) ;

	% Text Node
	\draw (124,88) node    {$+A$};
	% Text Node
	\draw (124,218) node    {$-A$};
	% Text Node
	\draw (517,154) node    {$t$};
	% Text Node
	\draw (129,49) node    {$x$};


	\end{tikzpicture}
\end{figure}
Il massimo spostamento dall'origine è $A$, che per questo prende il nome di ampiezza dell'oscillazione. Nell'argomento del coseno compare, oltre che a una funzione del tempo, un angolo che rappresenta la condizione iniziale del moto all'istante $t=0$ e per questo lo si chiama \textbf{fase iniziale}. Non è detto infatti che per $t=0$ la posizione del punto debba per forza coincidere con l'origine, ma potrebbe corrispondere ad un altro punto diverso da $0$. $\omega_0$ è il coefficiente di proporzionalità della variabile tempo e prende il nome di \textbf{pulsazione del moto}: è un parametro che rileva quanto sono rapide le oscillazioni.

Temporalmente il minimo intervallo di tempo tale per cui oltre ad esso il moto si ripete identico a se stesso (è infatti periodico),  è noto come \textbf{periodo}:
\[
	T \quad | \quad x(t)=x(t+T) \quad \forall t
\]
Possiamo calcolarlo come segue:
\begin{align*}
	x(t) &= x(t+T) \\
	A\cos(\omega_0 t+\Phi) &= A\cos(\omega_0 t +\omega_0 T+\Phi) \\
	\omega_0 t+\Phi &= \omega_0 t+\Phi+\omega_0 T \pm 2\pi \\
	T &= \frac{2\pi}{\omega}
\end{align*}
La pulsazione è dunque legata al periodo da questa relazione:
\[
	\boxed{\omega_0=\frac{2\pi}{T}}
\]
Si noti come il moto si ripeta velocemente quando la pulsazione è grande mentre è lento per bassi valori di essa. Infatti al crescere della pulsazione si accorcia il periodo e viceversa.

È comodo inoltre definire la \textbf{frequenza del moto}, ossia il numero di cicli, di periodi in un secondo. Più alta è la pulsazione, maggiore è la frequenza.
\[
	\boxed{f=\frac{1}{T}} \implies \boxed{\omega=2\pi f}
\]
Se si deriva la ~\eqref{eqn:armonico} una prima volta:
\[
	v(t)=\frac{dx}{dt}=-A\omega_0\sin(\omega_0 t+\Phi)
\]
\begin{figure}[htpb]
	\centering
	

	\tikzset{every picture/.style={line width=0.75pt}} %set default line width to 0.75pt        

	\begin{tikzpicture}[x=0.75pt,y=0.75pt,yscale=-1,xscale=1]
	%uncomment if require: \path (0,300); %set diagram left start at 0, and has height of 300

	% Plotting does not support converting to Tikz
	%Shape: Axis 2D [id:dp905263594609613] 
	\draw  (130,164.6) -- (521.5,164.6)(165.5,58.6) -- (165.5,266.6) (514.5,159.6) -- (521.5,164.6) -- (514.5,169.6) (160.5,65.6) -- (165.5,58.6) -- (170.5,65.6)  ;
	%Straight Lines [id:da3678422717394072] 
	\draw [line width=0.75]  [dash pattern={on 0.84pt off 2.51pt}]  (167,98.6) -- (502.5,98.6) ;
	%Straight Lines [id:da1458773151059063] 
	\draw [line width=0.75]  [dash pattern={on 0.84pt off 2.51pt}]  (167,228.6) -- (502.5,228.6) ;
	%Curve Lines [id:da5552577452439422] 
	\draw [line width=1.5]    (165.5,164.6) .. controls (191.5,250.25) and (202.5,250.75) .. (232.33,162.33) .. controls (261.33,77.33) and (271.33,76.67) .. (300.33,162.67) .. controls (329.33,248) and (338.67,248.67) .. (368.33,162.67) .. controls (398,78) and (407.33,77.33) .. (436.67,162.33) .. controls (466.67,248.67) and (476.67,248.67) .. (504.67,162.33) ;

	% Text Node
	\draw (137,97.6) node    {$+\omega _{0} A$};
	% Text Node
	\draw (137,227.6) node    {$-\omega _{0} A$};
	% Text Node
	\draw (537,163.6) node    {$t$};
	% Text Node
	\draw (149,58.6) node    {$v$};


	\end{tikzpicture}
\end{figure}
La velocità nel tempo varia con un seno dello stesso angolo e quindi oscillerà nel tempo con un periodo che è lo stesso ma con una funzione sfasata. L'oscillazione è compresa fra $A\omega_0$ e $-A\omega_0$. Analogamente si può calcolare l'accelerazione scalare come derivata della velocità nel tempo, ottenendo:
\[
	a(t)=\frac{dv}{dt}=-A\omega^2_0 t\cos(\omega_0 t+\Phi)
\]
L'accelerazione si annulla nel centro di oscillazione dove la velocità è massima e assume il valore massimo in modulo agli estremi, dove si inverte la velocità: inoltre essa è sempre proporzionale ed opposta allo spostamento dal centro di oscillazione. $a(t)$ e $x(t)$ sono una l'opposto dell'altra a parte una costante di proporzionalità. In particolare $a$ varia fra $\pm \omega^2_0 A$.
A parte il valore dell'ampiezza, le tre funzioni $x(t)$, $v(t)$ e $a(t)$ mostrano lo stesso andamento temporale: la forma e il periodo sono eguali, vi è solo uno spostamento di una rispetto all'altra lungo l'asse dei tempi. Quest'ultima caratteristica viene indicata dicendo che le funzioni sono sfasate fra loro.
In particolare la velocità è sfasata di $\frac{\pi}{2}$ rispetto allo spostamento (è in \emph{quadratura di fase}) mentre l'accelerazione è sfasata di $\pi$ sempre rispetto ad esso (è in \emph{opposizione di fase}). In pratica l'accelerazione è una oscillazione che avviene sempre come un seno ma il meno rappresenta tale sfasatura. Questa osservazione è ancora più interessante dal punto di vista analitico. Se infatti si ragiona su ciò, si nota che l'accelerazione è in relazione con $x(t)$ come segue:
\[
	a(t)=-A\omega^2_0 t\cos(\omega_0 t+\Phi)=-\omega^2_0\,x(t)
\]
Si ottiene quanto appena osservato: l'accelerazione è opposta a $x(t)$ a meno di una costante di proporzionalità.
Si può scrivere la relazione come equazione differenziale (relazione che lega cioè una funzione alle sue derivate):
\[
	\frac{d^2x}{dt^2}+\omega^2_0\,x(t)=0 \quad \forall t
\]
\begin{figure}[ht]
	\centering

	\tikzset{every picture/.style={line width=0.75pt}} %set default line width to 0.75pt        

	\begin{tikzpicture}[x=0.75pt,y=0.75pt,yscale=-1,xscale=1]
	%uncomment if require: \path (0,511); %set diagram left start at 0, and has height of 511

	%Shape: Axis 2D [id:dp07361889277532274] 
	\draw  (150,161.67) -- (541.5,161.67)(185.5,105) -- (185.5,216.2) (534.5,156.67) -- (541.5,161.67) -- (534.5,166.67) (180.5,112) -- (185.5,105) -- (190.5,112)  ;
	%Straight Lines [id:da48985214580087777] 
	\draw [line width=0.75]  [dash pattern={on 0.84pt off 2.51pt}]  (258.5,472) -- (258.5,103.6) ;
	% Plotting does not support converting to Tikz
	%Shape: Axis 2D [id:dp058897931948480364] 
	\draw  (150,291.67) -- (541.5,291.67)(185.5,235) -- (185.5,346.2) (534.5,286.67) -- (541.5,291.67) -- (534.5,296.67) (180.5,242) -- (185.5,235) -- (190.5,242)  ;
	%Shape: Axis 2D [id:dp6700287498310264] 
	\draw  (150,421.67) -- (541.5,421.67)(185.5,365) -- (185.5,476.2) (534.5,416.67) -- (541.5,421.67) -- (534.5,426.67) (180.5,372) -- (185.5,365) -- (190.5,372)  ;
	%Straight Lines [id:da5710725199192004] 
	\draw [line width=0.75]  [dash pattern={on 0.84pt off 2.51pt}]  (321.5,472) -- (321.5,103.6) ;
	%Curve Lines [id:da7575964396202544] 
	\draw [line width=1.5]    (185.5,161.67) .. controls (215.5,124.75) and (229.5,126.25) .. (259.5,161) .. controls (289.5,195.75) and (292,193.25) .. (321.5,161) .. controls (351,128.75) and (359.5,120.25) .. (394.5,161) .. controls (429.5,201.75) and (439,182.75) .. (457.5,162) ;
	%Curve Lines [id:da8424652798712617] 
	\draw [line width=1.5]    (185.5,419.97) .. controls (215.5,456.14) and (229.5,454.67) .. (259.5,420.62) .. controls (289.5,386.58) and (292,389.03) .. (321.5,420.62) .. controls (351,452.22) and (359.5,460.55) .. (394.5,420.62) .. controls (429.5,380.7) and (439,399.32) .. (457.5,419.64) ;
	%Curve Lines [id:da8424652798712617] 
	\draw [line width=1.5]    (185.5,419.97) .. controls (215.5,456.14) and (229.5,454.67) .. (259.5,420.62) .. controls (289.5,386.58) and (292,389.03) .. (321.5,420.62) .. controls (351,452.22) and (359.5,460.55) .. (394.5,420.62) .. controls (429.5,380.7) and (439,399.32) .. (457.5,419.64) ;
	%Curve Lines [id:da8976654485931836] 
	\draw [line width=1.5]    (186.17,264.42) .. controls (191.67,263.92) and (200.67,259.42) .. (227.67,292.42) .. controls (254.67,325.42) and (273.17,322.42) .. (292.67,291.42) .. controls (312.17,260.42) and (329.17,251.42) .. (358.67,288.79) .. controls (388.17,326.16) and (396.67,328.71) .. (431.67,288.79) .. controls (441.17,277.92) and (450.67,264.42) .. (465.17,263.92) ;

	% Text Node
	\draw (557,161.6) node    {$t$};
	% Text Node
	\draw (173,99.6) node    {$x$};
	% Text Node
	\draw (557,291.6) node    {$t$};
	% Text Node
	\draw (173,229.6) node    {$v$};
	% Text Node
	\draw (557,421.6) node    {$t$};
	% Text Node
	\draw (173,359.6) node    {$a$};


	\end{tikzpicture}
	%\caption{Andamento di spazio, velocità e accelerazione in un moto armonico}
    %\label{fig:equazioniMotoArmoicoDerivate}
\end{figure}
Essa è detta \textbf{equazione differenziale caratteristica del moto armonico}. Un moto armonico è dunque un moto che soddisfa tale relazione. Le funzioni seno e coseno e le loro combinazioni lineari, sono tutte e sole le funzioni che soddisfano a questa condizione nel campo reale.

Quest'ultimo fatto porta a osservare esplicitamente che le proprietà generali del moto armonico semplice restano eguali se invece della funzione ~\eqref{eqn:armonico} si fosse utilizzata la funzione coseno. Le due funzioni infatti differiscono solo per un termine di sfasamento pari a $\frac{\pi}{2}$.







































\section{Esempi di moti armonici semplici}

\subsection{Forza elastica}

Si consideri il caso di un corpo di massa $m$ appoggiato a un piano orizzontale perfettamente liscio, vincolato a una molla di costante elastica $k$. Inizialmente il sistema si trova in una condizione di riposo. Si definisce un asse $x$ in cui l'origine corrisponde alla posizione in cui si trova il corpo a riposo. La molla viene allungata di uno spostamento $x$, così che il corpo trasli in avanti. L'obbiettivo è di ricavare la legge oraria a cui esso sarà soggetto. Dal punto di vista dinamico, in direzione verticale agisce la forza peso perfettamente bilanciata dalla reazione normale. In direzione orizzontale vi è la forza di richiamo della molla.

Quando il corpo ritorna nella posizione di equilibrio, esso non si ferma perché ha acquisito velocità, ma continua a deformare la molla che lo richiama indietro. Il moto sarà un'oscillazione perpetua: un moto armonico.
\[
	x: \qquad -kx=ma
\]
Ricordando che $x$ non sarà costante nel tempo:
\[
	-kx=m\frac{d^2x}{dx^2} \implies \frac{d^2x}{dt^2}+\frac{k}{m}x=0
\]
Si trova che il sistema meccanico dato dal corpo attaccato alla molla reagisce alla perturbazione dell'equilibrio con una accelerazione di richiamo proporzionale allo spostamento subito: si tratta di un moto armonico. Dato che in esso il coefficiente di proporzionalità è l'opposto del quadrato della pulsazione, nel caso di una molla si trova:
\[
	\frac{d^2x}{dt^2}+\omega^2_0 x=0 \implies \omega_0=\sqrt{\frac{k}{m}}
\]
L'asse $x$ del primo grafico nella figura vista prima rappresenta la posizione nel tempo, è come se si potesse disegnare a lato la molla. Quando passa nella posizione di riposo il corpo è stato accelerato per aumentare la sua velocità che è diventata massima. Quindi il corpo va avanti fino a che la molla non si comprime.  Nei punti di massima compressione o massimo allungamento l'accelerazione è massima o minima e la velocità è nulla (il corpo si ferma) mentre nel punto di equilibrio non vi è accelerazione ma una velocità che è massima.

Abbiamo visto un esempio di moto armonico dato da una forza in direzione tangente al moto, in cui essa modifica il valore della velocità. Vediamo ora un esempio con una \textbf{forza centripeta}, ossia che fa variare la direzione della velocità.

\subsection{Il pendolo semplice}

Il pendolo semplice è costituito da un punto materiale appeso tramite una fune inestensibile e di massa trascurabile. La situazione di equilibrio statico è quella per cui il filo è sospeso lungo la verticale. La forza esercitata da esso è $\vec{T}=m\vec{g}$. Se si sposta il punto dalla verticale, ossia se viene data al filo una certa inclinazione che si indica con l'angolo $\vartheta$, esso inizia ad oscillare attorno alla posizione di equilibrio, lungo un arco di circonferenza di raggio $L$. Il corpo in assenza di attriti continua a oscillare e quello che si ottiene è anche in questo caso un moto armonico semplice.

\begin{figure}[htpb]
	\centering
	

	% Pattern Info
	 
	\tikzset{
	pattern size/.store in=\mcSize, 
	pattern size = 5pt,
	pattern thickness/.store in=\mcThickness, 
	pattern thickness = 0.3pt,
	pattern radius/.store in=\mcRadius, 
	pattern radius = 1pt}
	\makeatletter
	\pgfutil@ifundefined{pgf@pattern@name@_spdknvuu7}{
	\pgfdeclarepatternformonly[\mcThickness,\mcSize]{_spdknvuu7}
	{\pgfqpoint{0pt}{0pt}}
	{\pgfpoint{\mcSize+\mcThickness}{\mcSize+\mcThickness}}
	{\pgfpoint{\mcSize}{\mcSize}}
	{
	\pgfsetcolor{\tikz@pattern@color}
	\pgfsetlinewidth{\mcThickness}
	\pgfpathmoveto{\pgfqpoint{0pt}{0pt}}
	\pgfpathlineto{\pgfpoint{\mcSize+\mcThickness}{\mcSize+\mcThickness}}
	\pgfusepath{stroke}
	}}
	\makeatother
	\tikzset{every picture/.style={line width=0.75pt}} %set default line width to 0.75pt        

	\begin{tikzpicture}[x=0.75pt,y=0.75pt,yscale=-1,xscale=1]
	%uncomment if require: \path (0,351); %set diagram left start at 0, and has height of 351

	%Shape: Rectangle [id:dp19908194141643554] 
	\draw  [draw opacity=0][pattern=_spdknvuu7,pattern size=6pt,pattern thickness=0.75pt,pattern radius=0pt, pattern color={rgb, 255:red, 222; green, 222; blue, 222}] (80,41.5) -- (492.5,41.5) -- (492.5,69) -- (80,69) -- cycle ;
	%Straight Lines [id:da8854451216886683] 
	\draw    (80,69) -- (492.5,69) ;
	%Shape: Arc [id:dp8757101389668807] 
	\draw  [draw opacity=0][dash pattern={on 0.84pt off 2.51pt}] (383.26,168.27) .. controls (349.7,222.48) and (289.69,258.59) .. (221.25,258.59) .. controls (171.41,258.59) and (126.03,239.44) .. (92.09,208.09) -- (221.25,68.17) -- cycle ; \draw  [dash pattern={on 0.84pt off 2.51pt}] (383.26,168.27) .. controls (349.7,222.48) and (289.69,258.59) .. (221.25,258.59) .. controls (171.41,258.59) and (126.03,239.44) .. (92.09,208.09) ;
	%Shape: Circle [id:dp9647791556970287] 
	\draw  [fill={rgb, 255:red, 0; green, 0; blue, 0 }  ,fill opacity=1 ] (218,69) .. controls (218,67.21) and (219.46,65.75) .. (221.25,65.75) .. controls (223.04,65.75) and (224.5,67.21) .. (224.5,69) .. controls (224.5,70.79) and (223.04,72.25) .. (221.25,72.25) .. controls (219.46,72.25) and (218,70.79) .. (218,69) -- cycle ;
	%Straight Lines [id:da02623529304163541] 
	\draw  [dash pattern={on 0.84pt off 2.51pt}]  (221.25,68.17) -- (221.25,258.31) ;
	%Straight Lines [id:da43326551309148487] 
	\draw    (221.25,68.17) -- (328.9,223.64) ;
	%Shape: Ellipse [id:dp9763602525450497] 
	\draw  [draw opacity=0][fill={rgb, 255:red, 155; green, 155; blue, 155 }  ,fill opacity=1 ] (315.7,223.64) .. controls (315.7,216.35) and (321.61,210.44) .. (328.9,210.44) .. controls (336.19,210.44) and (342.1,216.35) .. (342.1,223.64) .. controls (342.1,230.93) and (336.19,236.83) .. (328.9,236.83) .. controls (321.61,236.83) and (315.7,230.93) .. (315.7,223.64) -- cycle ;
	%Straight Lines [id:da3604521273003396] 
	\draw [line width=1.5]    (328.9,223.64) -- (328.9,314.75) ;
	\draw [shift={(328.9,318.75)}, rotate = 270] [fill={rgb, 255:red, 0; green, 0; blue, 0 }  ][line width=0.08]  [draw opacity=0] (13.4,-6.43) -- (0,0) -- (13.4,6.44) -- (8.9,0) -- cycle    ;
	%Straight Lines [id:da16498016296525497] 
	\draw [line width=1.5]    (331.14,221.4) -- (283.26,151.81) ;
	\draw [shift={(280.99,148.51)}, rotate = 415.47] [fill={rgb, 255:red, 0; green, 0; blue, 0 }  ][line width=0.08]  [draw opacity=0] (13.4,-6.43) -- (0,0) -- (13.4,6.44) -- (8.9,0) -- cycle    ;
	%Straight Lines [id:da0071920943467709275] 
	\draw [line width=1.5]    (328.9,223.64) -- (249.7,108.19) ;
	\draw [shift={(247.44,104.89)}, rotate = 415.55] [fill={rgb, 255:red, 0; green, 0; blue, 0 }  ][line width=0.08]  [draw opacity=0] (13.4,-6.43) -- (0,0) -- (13.4,6.44) -- (8.9,0) -- cycle    ;
	%Shape: Arc [id:dp8965644261235546] 
	\draw  [draw opacity=0] (263.41,128.79) .. controls (251.46,137.12) and (236.92,142) .. (221.25,142) .. controls (220.96,142) and (220.67,142) .. (220.37,141.99) -- (221.25,68.17) -- cycle ; \draw   (263.41,128.79) .. controls (251.46,137.12) and (236.92,142) .. (221.25,142) .. controls (220.96,142) and (220.67,142) .. (220.37,141.99) ;
	%Straight Lines [id:da40890744138766677] 
	\draw    (210.07,70.41) -- (210.07,256.07) ;
	\draw [shift={(210.07,256.07)}, rotate = 270] [color={rgb, 255:red, 0; green, 0; blue, 0 }  ][line width=0.75]    (0,5.59) -- (0,-5.59)   ;
	\draw [shift={(210.07,70.41)}, rotate = 270] [color={rgb, 255:red, 0; green, 0; blue, 0 }  ][line width=0.75]    (0,5.59) -- (0,-5.59)   ;
	%Straight Lines [id:da01871836867451715] 
	\draw [line width=1.5]    (328.9,223.64) -- (287.11,250.95) ;
	\draw [shift={(283.76,253.14)}, rotate = 326.83000000000004] [fill={rgb, 255:red, 0; green, 0; blue, 0 }  ][line width=0.08]  [draw opacity=0] (13.4,-6.43) -- (0,0) -- (13.4,6.44) -- (8.9,0) -- cycle    ;
	%Straight Lines [id:da07884360167382432] 
	\draw    (328.9,223.64) -- (397.85,323.22) ;
	%Straight Lines [id:da4270548204578508] 
	\draw [line width=1.5]    (383.64,140.9) -- (355.87,100.55) ;
	\draw [shift={(353.61,97.25)}, rotate = 415.47] [fill={rgb, 255:red, 0; green, 0; blue, 0 }  ][line width=0.08]  [draw opacity=0] (13.4,-6.43) -- (0,0) -- (13.4,6.44) -- (8.9,0) -- cycle    ;
	%Straight Lines [id:da3070934302985917] 
	\draw [line width=1.5]    (383.64,140.9) -- (423.99,113.14) ;
	\draw [shift={(427.29,110.87)}, rotate = 505.47] [fill={rgb, 255:red, 0; green, 0; blue, 0 }  ][line width=0.08]  [draw opacity=0] (13.4,-6.43) -- (0,0) -- (13.4,6.44) -- (8.9,0) -- cycle    ;
	%Straight Lines [id:da8303876221439561] 
	\draw [line width=0.75]  [dash pattern={on 0.84pt off 2.51pt}]  (374.04,289.25) -- (328.9,318.75) ;
	%Straight Lines [id:da8777416855485225] 
	\draw [line width=1.5]    (371.77,285.95) -- (328.9,223.64) ;
	\draw [shift={(374.04,289.25)}, rotate = 235.47] [fill={rgb, 255:red, 0; green, 0; blue, 0 }  ][line width=0.08]  [draw opacity=0] (13.4,-6.43) -- (0,0) -- (13.4,6.44) -- (8.9,0) -- cycle    ;
	%Straight Lines [id:da1372972802325747] 
	\draw [line width=0.75]  [dash pattern={on 0.84pt off 2.51pt}]  (328.9,318.75) -- (283.76,253.14) ;
	%Shape: Arc [id:dp43221342740275226] 
	\draw  [draw opacity=0] (346.1,248.36) .. controls (341.22,251.76) and (335.29,253.75) .. (328.9,253.75) .. controls (328.78,253.75) and (328.66,253.75) .. (328.54,253.75) -- (328.9,223.64) -- cycle ; \draw   (346.1,248.36) .. controls (341.22,251.76) and (335.29,253.75) .. (328.9,253.75) .. controls (328.78,253.75) and (328.66,253.75) .. (328.54,253.75) ;

	% Text Node
	\draw (272.79,102.1) node    {$\vec{T}$};
	% Text Node
	\draw (307.84,148.33) node    {$\vec{R}_{n}$};
	% Text Node
	\draw (276.78,265.3) node    {$\vec{R}_{t}$};
	% Text Node
	\draw (328.79,328.76) node    {$m\vec{g}$};
	% Text Node
	\draw (352.27,223.56) node    {$P$};
	% Text Node
	\draw (248.44,148.97) node    {$\vartheta $};
	% Text Node
	\draw (332.95,197.12) node    {$m$};
	% Text Node
	\draw (201.21,159.14) node    {$L$};
	% Text Node
	\draw (442.29,108.1) node    {$\vec{u}_{t}$};
	% Text Node
	\draw (340.79,96.6) node    {$\vec{u}_{n}$};
	% Text Node
	\draw (339.44,261.47) node    {$\vartheta $};


	\end{tikzpicture}
\end{figure}

Per calcolare la legge oraria, conviene descrivere la posizione del punto in termini di posizione angolare. Il nostro obbiettivo è ricavare $\vartheta (t)$, con verso positivo quando va in verso antiorario. Andiamo a utilizzare la seconda legge della dinamica. Le forze agenti su punto $P$ sono il peso $m\vec{g}$ e la tensione del filo $T$, per cui il moto è regolato da $m\vec{g}+\vec{T}=m\vec{a}$. Conviene scomporre la seconda legge della dinamica in direzione tangente e normale alla traiettoria. Si prende la direzione tangente concorde al verso positivo dell'angolo, mentre la direzione normale sarà diretta verso l'interno della concavità.
\[
	\begin{cases} \vec{u}_t: \quad -mg\sin\vartheta=ma_t \\ \vec{u}_n: \quad T-mg\cos\vartheta=ma_n \end{cases}
\]
Il segno negativo della componente lungo la traiettoria è dovuto solo fatto che la forza ha segno opposto rispetto a quello della coordinata $s$ sulla traiettoria. Fisicamente $R_t=-mg\sin\vartheta$ è una forza di richiamo che tende a riportare il punto sulla verticale, anche se non è di direzione costante come nel caso delle forze elastiche.
\[
	-g\sin\vartheta=\frac{dv}{dt}=\frac{d\omega L}{dt}=\frac{d\omega}{dt}\,L=L\frac{d^2\vartheta}{dt^2}
\]
L'accelerazione normale sarà $\frac{v^2}{L}$. Si usa la legge al di sopra perché informa su come varia nel tempo la velocità scalare, la scomposizione tangente è sempre quella che informa sulla legge oraria.
\[
	-g\frac{\sin\vartheta}{L}=\frac{d^2\vartheta}{dt^2}
\]
Si è trovata una relazione che lega la posizione angolare alla derivata seconda. La soluzione non è esattamente quella del moto armonico ma, per valori molto piccoli dell'angolo, si può attuare l'approssimazione:
\[
	\sin\vartheta=\vartheta+\frac{\vartheta^3}{3!}+\dots \implies \sin\vartheta \simeq \vartheta \implies -\frac{g}{L}\vartheta(t)=\frac{d^2\vartheta}{dt^2}
\]
Per $\vartheta \le 0.122 \,\text{rad}, \,\sin\vartheta$ può essere approssimato a $\vartheta$ commettendo un errore relativo che è sempre minore di $10^{-3}$. Si ha allora:
\begin{equation}
	\frac{d^2\vartheta}{dt^2}+\frac{g}{L}\vartheta(t)=0
\end{equation}
Si trova così l'equazione differenziale per piccole oscillazioni. In conclusione il moto del pendolo è oscillatorio armonico quando l'ampiezza delle oscillazioni è piccola. La legge oraria del moto è:
\[
	\vartheta(t)=\vartheta_0\cos(\omega t+\Phi)
\]
Si può affermare che il pendolo si muove di moto armonico con pulsazione caratteristica pari a:
\[
	\omega=\sqrt{\frac{g}{L}}
\]
Si nota che il periodo del pendolo non dipende dalla massa ma dall'accelerazione di gravità e dalla lunghezza del filo. Inoltre esso non dipende nemmeno dall'ampiezza (isocronismo delle piccole oscillazioni).
Se si vuole ottenere la legge oraria dello spostamento lungo l'arco di circonferenza si sfrutta la relazione:
\[
	s(t)=L\,\vartheta(t)=L\vartheta_0\cos(\omega t+\Phi)
\]
Lo studio del moto del pendolo è interessante perché mostra un moto armonico che non avviene su una traiettoria rettilinea, permettendo di osservare che la legge oraria si ottiene concentrandosi sulle componenti tangenti delle forza, proprio perché sono quelle che modificano il valore della velocità. Tali forze tangenti si chiamano anche forze vive. L'altra scomposizione della seconda legge della dinamica in direzione normale alla traiettoria dà una informazione sulle forze che permettono al punto materiale di riuscire a mantenere una traiettoria curvilinea. La tensione di una fune è un esempio tipico di una forza che permette a un corpo di curvare. Essa non si mantiene costante nel tempo ma cambia valore al variare della posizione. Perché il filo non si rompa la tensione deve essere minore di $T_{MAX}$. La velocità in un moto armonico è minima agli estremi, dove il moto si inverte, ed è massima sulla verticale, in questo punto anche la forza peso è massima. Quindi il punto critico in termini di tensione massima è quest'ultimo.
Il moto del pendolo semplice può avvenire con velocità iniziale nulla purché l'angolo sia diverso da zero. Quando l'ampiezza delle oscillazioni non è piccola il moto è ancora periodico, ma non armonico, e il periodo dipende dall'ampiezza.







































\section{Studio energetico di un moto armonico}

Dal punto di vista energetico, un corpo che si muove sottoposto alla forza elastica, conservativa, avrà un energia meccanica costante perché forza peso e reazione normale non compiono lavoro.

Quando la molla raggiunge il punto di massima compressione l'energia cinetica si annulla. Essa diventa poi massima quando il corpo ripassa per la posizione di riposo e via dicendo.
L'ampiezza vale: $A^2\omega_0^2$.
\[
	E_k=\frac{1}{2}mv^2(t)=\frac{1}{2}mA^2\omega_0^2\sin^2(\omega_0 t+\Phi)
\]
C'è un continuo scambio di contributo energetico fra energia cinetica e energia potenziale. Dimostriamolo analiticamente:
\begin{gather*}
	E_{\text{pot}}=\frac{1}{2}kx^2(t)=\frac{1}{2}kA^2\cos^2(\omega_0 t+\Phi) \\
	E_{\text{mecc}}=\frac{1}{2}kA^2\cos^2(\omega_0 t+\Phi)+\frac{1}{2}mA^2\omega_0^2\sin^2(\omega_0 t+\Phi)
\end{gather*}
attuando la sostituzione $\omega_0=\sqrt{\frac{k}{m}}$ si ha:
\[
	E_{\text{mecc}}=\frac{1}{2}kA^2\cos^2(\omega_0 t+\Phi)+\frac{1}{2}kA^2\\sin^2(\omega_0 t+\Phi)=\frac{1}{2}kA^2
\]
Sebbene i contributi energetici oscillino nel tempo si nota che l'energia meccanica è sempre costante.
\begin{figure}[htpb]
	\centering
	

	\tikzset{every picture/.style={line width=0.75pt}} %set default line width to 0.75pt        

	\begin{tikzpicture}[x=0.75pt,y=0.75pt,yscale=-0.9,xscale=0.9]
	%uncomment if require: \path (0,822); %set diagram left start at 0, and has height of 822

	% Plotting does not support converting to Tikz
	%Shape: Axis 2D [id:dp9554882578986952] 
	\draw  (79,181.67) -- (470.5,181.67)(114.5,125) -- (114.5,236.2) (463.5,176.67) -- (470.5,181.67) -- (463.5,186.67) (109.5,132) -- (114.5,125) -- (119.5,132)  ;
	%Straight Lines [id:da5272545840794085] 
	\draw [line width=0.75]  [dash pattern={on 0.84pt off 2.51pt}]  (180.5,632) -- (180.5,123.6) ;
	%Shape: Axis 2D [id:dp3531664678382227] 
	\draw  (79,311.67) -- (470.5,311.67)(114.5,255) -- (114.5,366.2) (463.5,306.67) -- (470.5,311.67) -- (463.5,316.67) (109.5,262) -- (114.5,255) -- (119.5,262)  ;
	% Plotting does not support converting to Tikz
	%Shape: Axis 2D [id:dp6720567158093034] 
	\draw  (79,441.67) -- (470.5,441.67)(114.5,385) -- (114.5,496.2) (463.5,436.67) -- (470.5,441.67) -- (463.5,446.67) (109.5,392) -- (114.5,385) -- (119.5,392)  ;
	%Straight Lines [id:da8589279622426254] 
	\draw [line width=0.75]  [dash pattern={on 0.84pt off 2.51pt}]  (246.5,632) -- (246.5,123.6) ;
	%Straight Lines [id:da9613052196013632] 
	\draw [line width=0.75]  [dash pattern={on 0.84pt off 2.51pt}]  (315.5,632) -- (315.5,123.6) ;
	% Plotting does not support converting to Tikz
	%Shape: Axis 2D [id:dp7437622625843014] 
	\draw  (79,573.28) -- (470.5,573.28)(114.5,511) -- (114.5,633.2) (463.5,568.28) -- (470.5,573.28) -- (463.5,578.28) (109.5,518) -- (114.5,511) -- (119.5,518)  ;
	%Shape: Axis 2D [id:dp029865456838393145] 
	\draw  (79,723.2) -- (470.5,723.2)(114.55,654) -- (114.55,774.2) (463.5,718.2) -- (470.5,723.2) -- (463.5,728.2) (109.55,661) -- (114.55,654) -- (119.55,661)  ;
	%Straight Lines [id:da6865054473899408] 
	\draw [line width=0.75]  [dash pattern={on 0.84pt off 2.51pt}]  (114.55,672.2) -- (456.55,672.2) ;
	%Curve Lines [id:da186605963203905] 
	\draw [line width=1.5]    (114.33,154.33) .. controls (125.25,154.88) and (136.25,167.38) .. (147,181) .. controls (173.75,214.13) and (186.75,214.38) .. (214,181.33) .. controls (241.25,148.29) and (254,147.63) .. (282.33,180.67) .. controls (310.67,213.71) and (320.25,215.13) .. (349.67,181.33) .. controls (379.08,147.54) and (389.25,147.88) .. (419.33,181) ;
	%Curve Lines [id:da29313157364483367] 
	\draw [line width=1.5]    (114.5,311.67) .. controls (141.25,277.28) and (154.75,275.54) .. (182,309.85) .. controls (209.25,344.15) and (222,344.84) .. (250.33,310.54) .. controls (278.67,276.23) and (288.25,274.76) .. (317.67,309.85) .. controls (347.08,344.93) and (357.25,344.58) .. (387.33,310.19) .. controls (401.57,291.57) and (411.57,284.14) .. (421.86,284.43) ;
	%Curve Lines [id:da2916033657930399] 
	\draw [line width=1.5]    (114.5,441.67) .. controls (126.67,441.17) and (136,390.5) .. (148.33,390.17) .. controls (160.67,389.83) and (167.67,440.5) .. (181,441.17) .. controls (194.33,441.83) and (200,390.5) .. (214,390.83) .. controls (228,391.17) and (233,441.5) .. (246.67,441.17) .. controls (260.33,440.83) and (265.67,390.17) .. (280,390.5) .. controls (294.33,390.83) and (302.33,441.17) .. (315,441.17) .. controls (327.67,441.17) and (335,390.17) .. (348,390.17) .. controls (361,390.17) and (365.67,441.17) .. (379,440.83) .. controls (392.33,440.5) and (398,390.83) .. (412,390.5) ;
	%Curve Lines [id:da7476925369183249] 
	\draw [line width=1.5]    (114.33,522.17) .. controls (126.67,521.83) and (133.67,572.5) .. (147,573.17) .. controls (160.33,573.83) and (166,522.5) .. (180,522.83) .. controls (194,523.17) and (199,573.5) .. (212.67,573.17) .. controls (226.33,572.83) and (231.67,522.17) .. (246,522.5) .. controls (260.33,522.83) and (268.33,573.17) .. (281,573.17) .. controls (293.67,573.17) and (301,522.17) .. (314,522.17) .. controls (327,522.17) and (331.67,573.17) .. (345,572.83) .. controls (358.33,572.5) and (364,522.83) .. (378,522.5) .. controls (392,522.17) and (398.67,573.5) .. (412,573.5) ;
	%Curve Lines [id:da22569658489583788] 
	\draw [line width=1.5]    (114.33,672.17) .. controls (126.67,671.83) and (133.67,722.5) .. (147,723.17) .. controls (160.33,723.83) and (166,672.5) .. (180,672.83) .. controls (194,673.17) and (199,723.5) .. (212.67,723.17) .. controls (226.33,722.83) and (231.67,672.17) .. (246,672.5) .. controls (260.33,672.83) and (268.33,723.17) .. (281,723.17) .. controls (293.67,723.17) and (301,672.17) .. (314,672.17) .. controls (327,672.17) and (331.67,723.17) .. (345,722.83) .. controls (358.33,722.5) and (364,672.83) .. (378,672.5) .. controls (392,672.17) and (398.67,723.5) .. (412,723.5) ;
	%Curve Lines [id:da7782549887315116] 
	\draw [line width=1.5]    (114.5,723.67) .. controls (126.67,723.17) and (136,672.5) .. (148.33,672.17) .. controls (160.67,671.83) and (167.67,722.5) .. (181,723.17) .. controls (194.33,723.83) and (200,672.5) .. (214,672.83) .. controls (228,673.17) and (233,723.5) .. (246.67,723.17) .. controls (260.33,722.83) and (265.67,672.17) .. (280,672.5) .. controls (294.33,672.83) and (302.33,723.17) .. (315,723.17) .. controls (327.67,723.17) and (335,672.17) .. (348,672.17) .. controls (361,672.17) and (365.67,723.17) .. (379,722.83) .. controls (392.33,722.5) and (398,672.83) .. (412,672.5) ;

	% Text Node
	\draw (102,119.6) node    {$x$};
	% Text Node
	\draw (102,249.6) node    {$v$};
	% Text Node
	\draw (93,379.6) node    {$E_{k}$};
	% Text Node
	\draw (93,516.6) node    {$E_{p}$};
	% Text Node
	\draw (70,660.6) node    {$E_{p} +E_{k}$};
	% Text Node
	\draw (490,668.6) node    {$\frac{1}{2} kA^{2}$};


	\end{tikzpicture}
\end{figure}























































































































\chapter{Gravitazione}

\section{Le leggi di Keplero}

Dal punto di vista storico la gravitazione universale è frutto di osservazioni sperimentali di diversi fisici. Agli inizi del 1500 era stata avanzata da Copernico l'ipotesi eliocentrica: il Sole, e non la Terra, è il corpo celeste attorno al quale si svolge il moto dei pianeti. Successivamente, le posizioni assunte da questi ultimi nel tempo erano state oggetto di numerose e accurate misure da parte di Brahe alla fine del '500. Su tali misure si basò Keplero per formulare, tra il 1600 e il 1620, le sue tre leggi.

\textbf{Prima legge}

\noindent\fbox{%
	\parbox{\textwidth}{%
		\emph{I pianeti percorrono orbite ellittiche intorno al Sole che occupa uno dei fuochi dell'ellisse.}
	}%
}

In tale ambito, si definisce l'eccentricità di un'ellisse come il rapporto fra asse maggiore e asse minore. Se è uguale a uno è una circonferenza, se tende a infinito l'ellisse degenera in una retta (e il punto andrebbe avanti e indietro).

\textbf{Seconda legge} 

\noindent\fbox{%
	\parbox{\textwidth}{%
		\emph{La velocità areale con cui il raggio vettore che unisce il Sole ad un pianeta descrive l'orbita è costante.}
	}%
}

Questa legge è formulata anche dicendo che il punto spazza aree uguali in tempi uguali. Si immagini che mentre il pianeta ruota il raggio vettore lasci una traccia, in un tempo $\Delta t$ avrà spaziato un area.
\begin{figure}[htpb]
	\centering
	

	% Pattern Info
	 
	\tikzset{
	pattern size/.store in=\mcSize, 
	pattern size = 5pt,
	pattern thickness/.store in=\mcThickness, 
	pattern thickness = 0.3pt,
	pattern radius/.store in=\mcRadius, 
	pattern radius = 1pt}
	\makeatletter
	\pgfutil@ifundefined{pgf@pattern@name@_wlc8yueuj}{
	\pgfdeclarepatternformonly[\mcThickness,\mcSize]{_wlc8yueuj}
	{\pgfqpoint{0pt}{-\mcThickness}}
	{\pgfpoint{\mcSize}{\mcSize}}
	{\pgfpoint{\mcSize}{\mcSize}}
	{
	\pgfsetcolor{\tikz@pattern@color}
	\pgfsetlinewidth{\mcThickness}
	\pgfpathmoveto{\pgfqpoint{0pt}{\mcSize}}
	\pgfpathlineto{\pgfpoint{\mcSize+\mcThickness}{-\mcThickness}}
	\pgfusepath{stroke}
	}}
	\makeatother

	% Pattern Info
	 
	\tikzset{
	pattern size/.store in=\mcSize, 
	pattern size = 5pt,
	pattern thickness/.store in=\mcThickness, 
	pattern thickness = 0.3pt,
	pattern radius/.store in=\mcRadius, 
	pattern radius = 1pt}
	\makeatletter
	\pgfutil@ifundefined{pgf@pattern@name@_vq8azi3lx}{
	\pgfdeclarepatternformonly[\mcThickness,\mcSize]{_vq8azi3lx}
	{\pgfqpoint{0pt}{-\mcThickness}}
	{\pgfpoint{\mcSize}{\mcSize}}
	{\pgfpoint{\mcSize}{\mcSize}}
	{
	\pgfsetcolor{\tikz@pattern@color}
	\pgfsetlinewidth{\mcThickness}
	\pgfpathmoveto{\pgfqpoint{0pt}{\mcSize}}
	\pgfpathlineto{\pgfpoint{\mcSize+\mcThickness}{-\mcThickness}}
	\pgfusepath{stroke}
	}}
	\makeatother
	\tikzset{every picture/.style={line width=0.75pt}} %set default line width to 0.75pt        

	\begin{tikzpicture}[x=0.75pt,y=0.75pt,yscale=-1,xscale=1]
	%uncomment if require: \path (0,300); %set diagram left start at 0, and has height of 300

	%Shape: Polygon Curved [id:ds7888048739385256] 
	\draw  [draw opacity=0][pattern=_wlc8yueuj,pattern size=3.75pt,pattern thickness=0.75pt,pattern radius=0pt, pattern color={rgb, 255:red, 222; green, 222; blue, 222}] (366.5,194) .. controls (354.5,182.75) and (344.5,173.75) .. (316.75,148) .. controls (340.5,124.75) and (349.5,116.75) .. (366.5,100) .. controls (385,140.75) and (381.5,162.25) .. (366.5,194) -- cycle ;
	%Shape: Polygon Curved [id:ds3660792116229794] 
	\draw  [draw opacity=0][pattern=_vq8azi3lx,pattern size=3.75pt,pattern thickness=0.75pt,pattern radius=0pt, pattern color={rgb, 255:red, 222; green, 222; blue, 222}] (122.5,121) .. controls (164,126.75) and (267.5,140.75) .. (316.75,148) .. controls (258,148.25) and (172,149.75) .. (118.5,151) .. controls (117.5,141.25) and (120,129.75) .. (122.5,121) -- cycle ;
	%Shape: Ellipse [id:dp10272028811620681] 
	\draw   (118,148) .. controls (118,88.35) and (176.54,40) .. (248.75,40) .. controls (320.96,40) and (379.5,88.35) .. (379.5,148) .. controls (379.5,207.65) and (320.96,256) .. (248.75,256) .. controls (176.54,256) and (118,207.65) .. (118,148) -- cycle ;
	%Straight Lines [id:da08228735118508945] 
	\draw    (366.5,100) -- (316.75,148) ;
	%Straight Lines [id:da8949547722445947] 
	\draw    (366.5,194) -- (316.75,148) ;
	%Straight Lines [id:da6396212065868376] 
	\draw    (316.75,148) -- (122.5,121) ;
	%Straight Lines [id:da132233877114581] 
	\draw    (316.75,148) -- (118.5,151) ;
	%Straight Lines [id:da14243283627350656] 
	\draw    (179.5,229.47) -- (316.75,148) ;
	\draw [shift={(176.92,231)}, rotate = 329.31] [fill={rgb, 255:red, 0; green, 0; blue, 0 }  ][line width=0.08]  [draw opacity=0] (10.72,-5.15) -- (0,0) -- (10.72,5.15) -- (7.12,0) -- cycle    ;
	%Shape: Circle [id:dp19722738347889135] 
	\draw  [draw opacity=0][fill={rgb, 255:red, 184; green, 184; blue, 184 }  ,fill opacity=1 ] (301,148) .. controls (301,139.3) and (308.05,132.25) .. (316.75,132.25) .. controls (325.45,132.25) and (332.5,139.3) .. (332.5,148) .. controls (332.5,156.7) and (325.45,163.75) .. (316.75,163.75) .. controls (308.05,163.75) and (301,156.7) .. (301,148) -- cycle ;
	%Shape: Circle [id:dp22507537647047027] 
	\draw  [draw opacity=0][fill={rgb, 255:red, 128; green, 128; blue, 128 }  ,fill opacity=1 ] (164.42,235) .. controls (164.42,231.09) and (167.59,227.92) .. (171.5,227.92) .. controls (175.41,227.92) and (178.58,231.09) .. (178.58,235) .. controls (178.58,238.91) and (175.41,242.08) .. (171.5,242.08) .. controls (167.59,242.08) and (164.42,238.91) .. (164.42,235) -- cycle ;

	% Text Node
	\draw (136,137) node    {$A_{1}$};
	% Text Node
	\draw (354,145) node    {$A_{2}$};
	% Text Node
	\draw (438.5,140.5) node    {$A_{1} =A_{2}$};
	% Text Node
	\draw (269.33,214.67) node   [align=left] {raggio vettore};
	% Text Node
	\draw (142.67,251.33) node   [align=left] {pianeta};
	% Text Node
	\draw (311.33,120) node   [align=left] {Sole};


	\end{tikzpicture}
\end{figure}
\FloatBarrier
Quando il pianeta è lontano dal Sole esso si muoverà più lentamente, viceversa la velocità sarà massima quando si troverà molto vicino ad esso. In un'ellisse i punti in cui l'asse maggiore interseca tale curva sono rispettivamente il punto di minima e massima distanza dal fuoco. Essi prendono il nome di \textbf{perielio} e \textbf{afelio}.

\textbf{Terza legge} 

\noindent\fbox{%
	\parbox{\textwidth}{%
		\emph{Detto $T$ il periodo impiegato da un pianeta per compiere un giro intorno al Sole (periodo di rivoluzione), si osserva che il quadrato del periodo di rivoluzione di ogni pianeta è proporzionale al cubo del semiasse maggiore.} \[T^2=kr^3\]
	}%
}

Questa proporzionalità è la stessa per tutti i pianeti del sistema solare. La costante non dipende dalla massa del pianeta che sta ruotando. Tale legge mette in evidenza un legame fra tutti i pianeti del sistema solare: quelli più lontani dal Sole orbiteranno con un tempo ovviamente più lungo per mantenere la proporzionalità.







































\section{Forza di gravitazione universale}

Fu poi Newton a dare una interpretazione metodologica, basata su un concetto fisico, di queste tre osservazioni sperimentali attuate da Keplero. Se esse infatti danno una descrizione cinematica del moto dei pianeti, la spiegazione dinamica venne trovata proprio da Newton nel 1666. Affinché le tre leggi siano vere, il corpo non può che essere soggetto a un'unica forza detta di gravitazione universale. Essa è tale per cui, detta $r$ la mutua distanza fra i centri dei pianeti,  è inversamente proporzionale al quadrato di $r$ e direttamente proporzionale al prodotto fra le masse, tramite una costante di proporzionalità. Immaginando che la massa $M$ sia il Sole e la $m$ sia il pianeta, si ha:
\[
	\norma{\vec{F}_\gamma}=\frac{mM}{r^2}\gamma \qquad \text{con} \quad\gamma=6.67 \cdot 10^{-11} m^3 kg^{-1} s^{-2}
\]
Fu poi Cavendish a capire che doveva trattarsi di una forza attrattiva fra due corpi.
\begin{figure}[htpb]
	\centering
	

	\tikzset{every picture/.style={line width=0.75pt}} %set default line width to 0.75pt        

	\begin{tikzpicture}[x=0.75pt,y=0.75pt,yscale=-1,xscale=1]
	%uncomment if require: \path (0,300); %set diagram left start at 0, and has height of 300

	%Shape: Ellipse [id:dp9941739679943589] 
	\draw   (399.5,168) .. controls (399.5,227.65) and (340.96,276) .. (268.75,276) .. controls (196.54,276) and (138,227.65) .. (138,168) .. controls (138,108.35) and (196.54,60) .. (268.75,60) .. controls (340.96,60) and (399.5,108.35) .. (399.5,168) -- cycle ;
	%Shape: Circle [id:dp0635364371860474] 
	\draw  [draw opacity=0][fill={rgb, 255:red, 128; green, 128; blue, 128 }  ,fill opacity=1 ] (353.08,81) .. controls (353.08,84.91) and (349.91,88.08) .. (346,88.08) .. controls (342.09,88.08) and (338.92,84.91) .. (338.92,81) .. controls (338.92,77.09) and (342.09,73.92) .. (346,73.92) .. controls (349.91,73.92) and (353.08,77.09) .. (353.08,81) -- cycle ;
	%Straight Lines [id:da5431299853077853] 
	\draw    (426.5,168) -- (138,168) ;
	\draw [shift={(429.5,168)}, rotate = 180] [fill={rgb, 255:red, 0; green, 0; blue, 0 }  ][line width=0.08]  [draw opacity=0] (10.72,-5.15) -- (0,0) -- (10.72,5.15) -- (7.12,0) -- cycle    ;
	%Straight Lines [id:da6109887920330987] 
	\draw    (200.75,43) -- (200.75,168) ;
	\draw [shift={(200.75,40)}, rotate = 90] [fill={rgb, 255:red, 0; green, 0; blue, 0 }  ][line width=0.08]  [draw opacity=0] (10.72,-5.15) -- (0,0) -- (10.72,5.15) -- (7.12,0) -- cycle    ;
	%Straight Lines [id:da08744928129546925] 
	\draw    (337.42,86.37) -- (200.75,168) ;
	\draw [shift={(340,84.83)}, rotate = 149.15] [fill={rgb, 255:red, 0; green, 0; blue, 0 }  ][line width=0.08]  [draw opacity=0] (10.72,-5.15) -- (0,0) -- (10.72,5.15) -- (7.12,0) -- cycle    ;
	%Shape: Arc [id:dp006379504475305886] 
	\draw  [draw opacity=0] (235.99,146.94) .. controls (239.68,153.1) and (241.8,160.3) .. (241.8,168) -- (200.75,168) -- cycle ; \draw   (235.99,146.94) .. controls (239.68,153.1) and (241.8,160.3) .. (241.8,168) ;
	%Shape: Circle [id:dp6764092735681855] 
	\draw  [draw opacity=0][fill={rgb, 255:red, 184; green, 184; blue, 184 }  ,fill opacity=1 ] (216.5,168) .. controls (216.5,176.7) and (209.45,183.75) .. (200.75,183.75) .. controls (192.05,183.75) and (185,176.7) .. (185,168) .. controls (185,159.3) and (192.05,152.25) .. (200.75,152.25) .. controls (209.45,152.25) and (216.5,159.3) .. (216.5,168) -- cycle ;
	%Shape: Boxed Line [id:dp8971684018104789] 
	\draw    (226.82,152.43) -- (200.75,168) ;
	\draw [shift={(229.4,150.89)}, rotate = 149.15] [fill={rgb, 255:red, 0; green, 0; blue, 0 }  ][line width=0.08]  [draw opacity=0] (10.72,-5.15) -- (0,0) -- (10.72,5.15) -- (7.12,0) -- cycle    ;
	%Shape: Boxed Line [id:dp9239139265790635] 
	\draw    (185.18,141.93) -- (200.75,168) ;
	\draw [shift={(183.64,139.35)}, rotate = 59.15] [fill={rgb, 255:red, 0; green, 0; blue, 0 }  ][line width=0.08]  [draw opacity=0] (10.72,-5.15) -- (0,0) -- (10.72,5.15) -- (7.12,0) -- cycle    ;

	% Text Node
	\draw (252,153.2) node    {$\vartheta $};
	% Text Node
	\draw (438.8,168.8) node    {$x$};
	% Text Node
	\draw (188,42.8) node    {$y$};
	% Text Node
	\draw (173.2,145.2) node    {$\vec{u}_{\vartheta }$};
	% Text Node
	\draw (220.8,135.2) node    {$\vec{u}_{r}$};
	% Text Node
	\draw (195.27,173.4) node  [font=\small]  {$O$};
	% Text Node
	\draw (218.4,187.6) node    {$M$};
	% Text Node
	\draw (363.2,69.6) node    {$m$};
	% Text Node
	\draw (278.8,105.2) node    {$\vec{r}$};


	\end{tikzpicture}
\end{figure}
\FloatBarrier
Invece che un sistema di riferimento cartesiano, in tale ambito risulta più pratico l'utilizzo di un sistema di riferimento in coordinate polari, dove il centro viene messo in corrispondenza del corpo attrattore. Si va poi a identificare il raggio vettore $\vec{r}$.  La posizione $P$ è identificata da una coppia di valori: la distanza radiale (lunghezza del raggio vettore) e l'angolo $\vartheta$, formato da $\vec{r}$ e da una direzione di riferimento fissa (ad esempio l'asse orizzontale). Ecco le relazioni che intercorrono con il sistema cartesiano:
\begin{gather*}
	\begin{cases} x=r\cos\vartheta \\ y=r\sin\vartheta \end{cases} \iff P(r,\vartheta)=P(x,y) \\
	r=\sqrt{x^2+y^2} \qquad \tan\vartheta=\frac{y}{x}
\end{gather*}
Si definiscono i versori $\vec{u}_r$ e $\vec{u}_\vartheta$, tangente alla traiettoria e con verso concorde a quello di percorrenza. A questo punto diventa molto semplice esprimere la forza gravitazionale in forma vettoriale.
\[
	\boxed{\vec{F}=-\frac{mM}{r^2}\gamma\, \vec{u}_r}
\]
Si noti che la forza di gravitazione universale è una \textbf{forza centrale}. In meccanica classica, una forza centrale è una forza diretta lungo la congiungente del punto di applicazione e un punto fisso, detto centro della forza, e tale che in ogni momento il modulo sia funzione esclusivamente del raggio-vettore tra il punto di applicazione della forza e il centro. Conseguenza di ciò è che la forza di interazione gravitazionale permette la conservazione del momento angolare ed è conservativa. Questi fatti verranno affrontati in seguito.

La velocità del punto materiale ha in generale una componente che è parallela al raggio, \emph{velocità radiale} e una ortogonale ad esso, \emph{velocità trasversa}. Durante il moto la velocità radiale dice quanto rapidamente il pianeta si avvicina o si allontana dal Sole. L'altra informa su quanto rapidamente esso sta ruotando intorno a questo.
Si ricava la loro espressione:
\begin{gather*}
	\vec{v}=\frac{d\vec{r}}{dt}=\frac{d(r\vec{u}_r)}{dt}=\frac{dr}{dt}\,\vec{u}_r+r\,\frac{d\vec{u}_r}{dt}=\frac{dr}{dt}\,\vec{u}_r+r\frac{d\vartheta}{dt}\,\vec{u}_\vartheta=\vec{v}_r+r\underbrace{(\vec{\omega}\times\vec{u}_r)}_{\text{vettore $\parallel$ a } \vec{u}_\vartheta} \\
	\vec{v}=\vec{v}_r+r\frac{d\vartheta}{dt}\, \vec{u}_\vartheta \implies \vec{v}=\vec{v}_r+\vec{v}_\vartheta
\end{gather*}
\begin{figure}[htpb]
	\centering
	

	\tikzset{every picture/.style={line width=0.75pt}} %set default line width to 0.75pt        

	\begin{tikzpicture}[x=0.75pt,y=0.75pt,yscale=-1,xscale=1]
	%uncomment if require: \path (0,300); %set diagram left start at 0, and has height of 300

	%Shape: Ellipse [id:dp41169445171757313] 
	\draw   (394.5,169) .. controls (394.5,228.65) and (335.96,277) .. (263.75,277) .. controls (191.54,277) and (133,228.65) .. (133,169) .. controls (133,109.35) and (191.54,61) .. (263.75,61) .. controls (335.96,61) and (394.5,109.35) .. (394.5,169) -- cycle ;
	%Shape: Circle [id:dp19752370514212458] 
	\draw  [draw opacity=0][fill={rgb, 255:red, 128; green, 128; blue, 128 }  ,fill opacity=1 ] (348.08,82) .. controls (348.08,85.91) and (344.91,89.08) .. (341,89.08) .. controls (337.09,89.08) and (333.92,85.91) .. (333.92,82) .. controls (333.92,78.09) and (337.09,74.92) .. (341,74.92) .. controls (344.91,74.92) and (348.08,78.09) .. (348.08,82) -- cycle ;
	%Shape: Circle [id:dp9056973707211953] 
	\draw  [draw opacity=0][fill={rgb, 255:red, 184; green, 184; blue, 184 }  ,fill opacity=1 ] (211.5,169) .. controls (211.5,177.7) and (204.45,184.75) .. (195.75,184.75) .. controls (187.05,184.75) and (180,177.7) .. (180,169) .. controls (180,160.3) and (187.05,153.25) .. (195.75,153.25) .. controls (204.45,153.25) and (211.5,160.3) .. (211.5,169) -- cycle ;
	%Shape: Circle [id:dp03866659809345174] 
	\draw  [fill={rgb, 255:red, 0; green, 0; blue, 0 }  ,fill opacity=1 ] (131.17,169) .. controls (131.17,167.99) and (131.99,167.17) .. (133,167.17) .. controls (134.01,167.17) and (134.83,167.99) .. (134.83,169) .. controls (134.83,170.01) and (134.01,170.83) .. (133,170.83) .. controls (131.99,170.83) and (131.17,170.01) .. (131.17,169) -- cycle ;
	%Shape: Circle [id:dp785149968468944] 
	\draw  [fill={rgb, 255:red, 0; green, 0; blue, 0 }  ,fill opacity=1 ] (392.67,169) .. controls (392.67,167.99) and (393.49,167.17) .. (394.5,167.17) .. controls (395.51,167.17) and (396.33,167.99) .. (396.33,169) .. controls (396.33,170.01) and (395.51,170.83) .. (394.5,170.83) .. controls (393.49,170.83) and (392.67,170.01) .. (392.67,169) -- cycle ;
	%Shape: Circle [id:dp3881830809031668] 
	\draw  [draw opacity=0][fill={rgb, 255:red, 128; green, 128; blue, 128 }  ,fill opacity=1 ] (222.75,268.67) .. controls (222.75,272.58) and (219.58,275.75) .. (215.67,275.75) .. controls (211.75,275.75) and (208.58,272.58) .. (208.58,268.67) .. controls (208.58,264.75) and (211.75,261.58) .. (215.67,261.58) .. controls (219.58,261.58) and (222.75,264.75) .. (222.75,268.67) -- cycle ;

	% Text Node
	\draw (360.47,74.47) node    {$A$};
	% Text Node
	\draw (198.5,276.13) node    {$B$};
	% Text Node
	\draw (98.5,167.5) node   [align=left] {perielio};
	% Text Node
	\draw (421.5,167.5) node   [align=left] {afelio};


	\end{tikzpicture}
\end{figure}
\FloatBarrier
Si noti che in corrispondenza dell'afelio e del perielio la componente radiale non c'è, la velocità ha soltanto componente trasversale. Ciò accade perché la velocità radiale informa su come varia la distanza dal Sole. Essa diminuisce, raggiunge un punto di minimo, poi aumenta fino a raggiungere il massimo e via dicendo. $r(\vartheta)$ varia come in figura. Per i valori che corrispondono all'afelio e al perielio si hanno punti a tangente orizzontale. In corrispondenza di questi la derivata non può che essere nulla.
\begin{figure}[htpb]
	\centering
	

	\tikzset{every picture/.style={line width=0.75pt}} %set default line width to 0.75pt        

	\begin{tikzpicture}[x=0.75pt,y=0.75pt,yscale=-1,xscale=1]
	%uncomment if require: \path (0,300); %set diagram left start at 0, and has height of 300

	% Plotting does not support converting to Tikz
	%Shape: Axis 2D [id:dp6077987803013458] 
	\draw  (81,241) -- (440.5,241)(104.5,71) -- (104.5,259) (433.5,236) -- (440.5,241) -- (433.5,246) (99.5,78) -- (104.5,71) -- (109.5,78)  ;
	%Straight Lines [id:da7365662394897443] 
	\draw  [dash pattern={on 0.84pt off 2.51pt}]  (105,130) -- (413.5,130) ;
	%Straight Lines [id:da7879993821891] 
	\draw  [dash pattern={on 0.84pt off 2.51pt}]  (105,182) -- (413.5,182) ;
	%Curve Lines [id:da7192072802610738] 
	\draw [line width=1.5]    (104.5,155.75) .. controls (113.5,146.75) and (124.5,130.25) .. (139.5,129.75) .. controls (160,129.75) and (184.5,181.75) .. (207.5,181.75) .. controls (230.5,181.75) and (252.5,130.25) .. (274.5,130.25) .. controls (296.5,130.25) and (323.5,182.25) .. (344.5,181.75) .. controls (365.5,181.25) and (386,129.75) .. (413.5,130) ;

	% Text Node
	\draw (207.5,194.5) node   [align=left] {perielio};
	% Text Node
	\draw (275.5,116.5) node   [align=left] {afelio};
	% Text Node
	\draw (93,72) node    {$r$};
	% Text Node
	\draw (459,246) node    {$\vartheta ,t$};


	\end{tikzpicture}
\end{figure}
\FloatBarrier
Si noti che la forza gravitazionale non è una forza centripeta. Per esserlo dovrebbe essere diretta perpendicolarmente alla traiettoria. Ma essa ha sia una componente tangente che una trasversale. Ci sono solo due punti in cui la forza gravitazionale si comporta come forza centripeta in un orbita ellittica, e questi sono proprio l'afelio e il perielio. Si osserva però che, in generale:
\[
	\gamma\,\frac{mM}{r_\text{afelio}^2}=\frac{mv_\text{afelio}^2}{r_\text{afelio}}
\]
Si noti che, a differenza delle forze finora considerate, che si manifestano al contatto macroscopico tra i corpi, la forza gravitazionale si manifesta distanza, senza che le masse vengono a contatto. In effetti tutte le interazioni fondamentali conosciute sono forze a distanza, che differiscono però nel raggio di azione (oltre che per altre proprietà): la forza gravitazionale e la forza tra cariche elettriche hanno la stessa dipendenza dalla distanza e si dice che il loro raggio di azione è infinito, invece la forza forte e quella debole decrescono molto più rapidamente con la distanza e sono efficaci solo a livello subatomico.







































\section{Verifica delle leggi di Keplero}

La forza gravitazionale rispetto al polo non genera momento perché non ha una componente ortogonale al raggio vettore, ma è sempre parallela ad esso: come diretta conseguenza, un pianeta ha un momento angolare costante nel tempo. Esso non collassa perché è in movimento, la sua velocità iniziale gli permette di ruotare. L'obbiettivo è studiare quali sono gli effetti cinematici che ciò comporta. Verrebbe da pensare che il punto si muove di moto circolare uniforme, ma questo è solo un caso particolare della situazione che si presenta.
Si consideri un oggetto che si muove su una traiettoria con momento angolare costante, in direzione, verso e modulo.

\paragraph{1} La conseguenza del fatto che la direzione di $\vec{L}$ si mantiene costante è che il moto non può essere tridimensionale, ma avviene su un piano, formato da $\vec{r}$ e $\vec{v}$. Si ha così una verifica della prima legge di Keplero. Il verso di $\vec{L}$ costante vuol dire invece che il punto non può mai invertire il moto.

\paragraph{2} Si calcoli ora il modulo del momento angolare:
\[
	\norma{\vec{L}}=\norma{\vec{r}\times m\vec{v}}=\norma{\vec{r}\times m(\vec{v}_r+\vec{v}_\vartheta)}=\norma{\vec{r}\times m \vec{v}_\vartheta}
\]
$\vec{r}$ e $\vec{v}_\vartheta$ sono ortogonali quindi il prodotto vettoriale fra di essi non è altro che il prodotto fra i loro moduli.
\[
	\norma{\vec{L}}=\norma{\vec{r}} \cdot \norma{\vec{v}_\vartheta} m=r^2\omega m=m\,r^2\,\frac{d\vartheta}{dt}=\text{costante}
\]
Si immagini che il punto si stia muovendo di un piccolo tratto infinitesimo:  la sua posizione angolare varia di $d\vartheta$. Si definisce \textbf{velocità areale} o areoale, $\frac{dA}{dt}$, la quantità scalare pari alla variazione nel tempo dell'area spaziata dal raggio vettore. Si procede approssimando l'area tratteggiata all'area di un triangolo.
\begin{figure}[htpb]
	\centering
	

	% Pattern Info
	 
	\tikzset{
	pattern size/.store in=\mcSize, 
	pattern size = 5pt,
	pattern thickness/.store in=\mcThickness, 
	pattern thickness = 0.3pt,
	pattern radius/.store in=\mcRadius, 
	pattern radius = 1pt}
	\makeatletter
	\pgfutil@ifundefined{pgf@pattern@name@_ny7vciohh}{
	\pgfdeclarepatternformonly[\mcThickness,\mcSize]{_ny7vciohh}
	{\pgfqpoint{0pt}{-\mcThickness}}
	{\pgfpoint{\mcSize}{\mcSize}}
	{\pgfpoint{\mcSize}{\mcSize}}
	{
	\pgfsetcolor{\tikz@pattern@color}
	\pgfsetlinewidth{\mcThickness}
	\pgfpathmoveto{\pgfqpoint{0pt}{\mcSize}}
	\pgfpathlineto{\pgfpoint{\mcSize+\mcThickness}{-\mcThickness}}
	\pgfusepath{stroke}
	}}
	\makeatother
	\tikzset{every picture/.style={line width=0.75pt}} %set default line width to 0.75pt        

	\begin{tikzpicture}[x=0.75pt,y=0.75pt,yscale=-1,xscale=1]
	%uncomment if require: \path (0,300); %set diagram left start at 0, and has height of 300

	%Shape: Polygon Curved [id:ds16872019378222625] 
	\draw  [draw opacity=0][pattern=_ny7vciohh,pattern size=3pt,pattern thickness=0.75pt,pattern radius=0pt, pattern color={rgb, 255:red, 222; green, 222; blue, 222}] (249.2,65.4) .. controls (258.6,62.4) and (275,61.2) .. (283.75,61) .. controls (270.2,82.4) and (243,126.8) .. (215.75,169) .. controls (227,132.8) and (237.4,101.6) .. (249.2,65.4) -- cycle ;
	%Shape: Ellipse [id:dp9861585294009343] 
	\draw   (414.5,169) .. controls (414.5,228.65) and (355.96,277) .. (283.75,277) .. controls (211.54,277) and (153,228.65) .. (153,169) .. controls (153,109.35) and (211.54,61) .. (283.75,61) .. controls (355.96,61) and (414.5,109.35) .. (414.5,169) -- cycle ;
	%Straight Lines [id:da6105323859497809] 
	\draw    (215.75,169) -- (283.75,61) ;
	%Straight Lines [id:da870717036200692] 
	\draw    (215.75,169) -- (249.2,65.4) ;
	%Straight Lines [id:da7058693436437289] 
	\draw    (215.75,169) -- (264.72,64.91) ;
	\draw [shift={(266,62.2)}, rotate = 475.2] [fill={rgb, 255:red, 0; green, 0; blue, 0 }  ][line width=0.08]  [draw opacity=0] (10.72,-5.15) -- (0,0) -- (10.72,5.15) -- (7.12,0) -- cycle    ;
	%Shape: Circle [id:dp7452873351674358] 
	\draw  [draw opacity=0][fill={rgb, 255:red, 184; green, 184; blue, 184 }  ,fill opacity=1 ] (231.5,169) .. controls (231.5,177.7) and (224.45,184.75) .. (215.75,184.75) .. controls (207.05,184.75) and (200,177.7) .. (200,169) .. controls (200,160.3) and (207.05,153.25) .. (215.75,153.25) .. controls (224.45,153.25) and (231.5,160.3) .. (231.5,169) -- cycle ;
	%Straight Lines [id:da45722653487185494] 
	\draw    (250.18,57.02) -- (278.77,53.38) ;
	\draw [shift={(281.75,53)}, rotate = 532.74] [fill={rgb, 255:red, 0; green, 0; blue, 0 }  ][line width=0.08]  [draw opacity=0] (10.72,-5.15) -- (0,0) -- (10.72,5.15) -- (7.12,0) -- cycle    ;
	\draw [shift={(247.2,57.4)}, rotate = 352.74] [fill={rgb, 255:red, 0; green, 0; blue, 0 }  ][line width=0.08]  [draw opacity=0] (10.72,-5.15) -- (0,0) -- (10.72,5.15) -- (7.12,0) -- cycle    ;

	% Text Node
	\draw (263.77,37.77) node    {$rd\vartheta $};
	% Text Node
	\draw (260.27,116.47) node    {$\vec{r}$};


	\end{tikzpicture}
\end{figure}
\FloatBarrier
L'altezza è proprio la distanza radiale, la base $r\,d\vartheta$ è l'arco di circonferenza di raggio $r$ e ampiezza angolare $d\vartheta$.
\[
	v_\text{areale}=\frac{dA}{dt}=\frac{\frac{rd\vartheta\,r}{2}}{dt}=\frac{d\vartheta}{dt}\cdot \frac{1}{2} r^2=\frac{\omega r^2}{2}=\frac{v_\vartheta r}{2}=\frac{mv_\vartheta r}{2m}
\]
Quindi:
\[
	v_\text{areale}=\frac{\norma{\vec{L}}}{2m}
\]
Se il momento angolare è costante si ha che la velocità areale è costante. Trovare un moto in cui i momenti delle forze sono nulli e quindi il momento angolare è costante, vuole dire avere una traiettoria piana con velocità areale costante. Viene così verificata la seconda legge di Keplero.

\paragraph{Osservazione} È stato detto che il moto circolare uniforme è un caso particolare che si ha nel momento in cui il momento angolare è costante. Un corpo soggetto a forza gravitazionale che si muove lungo una traiettoria circolare mantiene costante la distanza con il pianeta attrattore. Il fatto che $\vec{r}$ sia costante implica l'assenza della componente radiale della velocità; in tutti i punti c'è solo la velocità trasversale. Infatti la velocità radiale dice quanto varia la distanza dal centro nel tempo, ma questa non cambia. È come se tutti i punti fossero l'afelio e il perielio.
\[
	v_\text{areale}=\frac{v_\vartheta r}{2}=\frac{\omega r^2}{2}
\]
Da questa espressione si vede che $r$ costante e velocità areale costante implicano una velocità angolare costante e quindi un moto uniforme.
\begin{figure}[htpb]
	\centering
	

	\tikzset{every picture/.style={line width=0.75pt}} %set default line width to 0.75pt        

	\begin{tikzpicture}[x=0.75pt,y=0.75pt,yscale=-1,xscale=1]
	%uncomment if require: \path (0,300); %set diagram left start at 0, and has height of 300

	%Straight Lines [id:da0745484543757402] 
	\draw    (444.5,162.25) -- (444.5,121.25) ;
	\draw [shift={(444.5,118.25)}, rotate = 450] [fill={rgb, 255:red, 0; green, 0; blue, 0 }  ][line width=0.08]  [draw opacity=0] (10.72,-5.15) -- (0,0) -- (10.72,5.15) -- (7.12,0) -- cycle    ;
	%Shape: Circle [id:dp6339082486573857] 
	\draw  [draw opacity=0][fill={rgb, 255:red, 184; green, 184; blue, 184 }  ,fill opacity=1 ] (367,162.25) .. controls (367,170.95) and (359.95,178) .. (351.25,178) .. controls (342.55,178) and (335.5,170.95) .. (335.5,162.25) .. controls (335.5,153.55) and (342.55,146.5) .. (351.25,146.5) .. controls (359.95,146.5) and (367,153.55) .. (367,162.25) -- cycle ;
	%Shape: Circle [id:dp6563003174772981] 
	\draw   (258,162.25) .. controls (258,110.75) and (299.75,69) .. (351.25,69) .. controls (402.75,69) and (444.5,110.75) .. (444.5,162.25) .. controls (444.5,213.75) and (402.75,255.5) .. (351.25,255.5) .. controls (299.75,255.5) and (258,213.75) .. (258,162.25) -- cycle ;
	%Straight Lines [id:da1874688148167316] 
	\draw    (258,203.25) -- (258,162.25) ;
	\draw [shift={(258,206.25)}, rotate = 270] [fill={rgb, 255:red, 0; green, 0; blue, 0 }  ][line width=0.08]  [draw opacity=0] (10.72,-5.15) -- (0,0) -- (10.72,5.15) -- (7.12,0) -- cycle    ;
	%Straight Lines [id:da038252941880675184] 
	\draw    (351.25,69) -- (310.25,69) ;
	\draw [shift={(307.25,69)}, rotate = 360] [fill={rgb, 255:red, 0; green, 0; blue, 0 }  ][line width=0.08]  [draw opacity=0] (10.72,-5.15) -- (0,0) -- (10.72,5.15) -- (7.12,0) -- cycle    ;
	%Straight Lines [id:da6941678012243162] 
	\draw    (392.25,255.5) -- (351.25,255.5) ;
	\draw [shift={(395.25,255.5)}, rotate = 180] [fill={rgb, 255:red, 0; green, 0; blue, 0 }  ][line width=0.08]  [draw opacity=0] (10.72,-5.15) -- (0,0) -- (10.72,5.15) -- (7.12,0) -- cycle    ;
	%Straight Lines [id:da6984646792309464] 
	\draw    (354.25,163.25) -- (397.07,84.63) ;
	\draw [shift={(398.5,82)}, rotate = 478.57] [fill={rgb, 255:red, 0; green, 0; blue, 0 }  ][line width=0.08]  [draw opacity=0] (10.72,-5.15) -- (0,0) -- (10.72,5.15) -- (7.12,0) -- cycle    ;
	%Straight Lines [id:da33278676717158584] 
	\draw    (348.68,156.62) -- (391.5,78) ;
	\draw [shift={(347.25,159.25)}, rotate = 298.57] [fill={rgb, 255:red, 0; green, 0; blue, 0 }  ][line width=0.08]  [draw opacity=0] (10.72,-5.15) -- (0,0) -- (10.72,5.15) -- (7.12,0) -- cycle    ;

	% Text Node
	\draw (387.27,130.47) node    {$\vec{r}$};
	% Text Node
	\draw (356.47,105.27) node    {$\vec{F}_{\gamma }$};


	\end{tikzpicture}
\end{figure}
\FloatBarrier
In questo particolare caso si ha:
\[
	\norma{\vec{F}_\gamma}=\gamma \frac{mM}{r^2}=m\omega^2r
\]

\paragraph{3} La terza legge di Keplero si dimostra qui nel caso di orbite circolari. Le orbite dei pianeti, pur essendo certamente ellittiche, sono molto prossime a circonferenze. Se la velocità areale è costante, il moto di un pianeta è circolare uniforme. Il periodo di rivoluzione, ossia il tempo per compiere un giro completo, è dato da:
\[
	T=\frac{2\pi}{\omega}
\]
La forza che agisce sul pianeta, permettendogli di percorrere una traiettoria circolare con velocità costante, deve essere esclusivamente centripeta, e si scrive:
\begin{gather*}
	F_\gamma=m\omega^2r \implies \omega^2=\frac{F_\gamma}{mr} \\
	T^2=\frac{4\pi^2}{\omega^2}=\frac{4\pi^2mr}{F_\gamma}=\frac{4\pi^2mr^3}{\gamma mM} \implies T^2=\underbrace{\frac{4\pi^2}{\gamma M}}_{=k} r^3 \implies T^2=kr^3
\end{gather*}
Questa relazione vale per tutti i pianeti del sistema solare perché non dipende dalla loro massa, ma solo da quella del pianeta attrattore, il Sole.







































\section{Energia potenziale gravitazionale}

L'obbiettivo della sezione è quello di verificare se la forza gravitazionale è conservativa o meno.
\[
	\mathcal{L}=\int_{\Gamma, A\to B}\vec{F}_\gamma \cdot d\vec{r}
\]
\begin{figure}[htpb]
	\centering
	

	\tikzset{every picture/.style={line width=0.75pt}} %set default line width to 0.75pt        

	\begin{tikzpicture}[x=0.75pt,y=0.75pt,yscale=-1,xscale=1]
	%uncomment if require: \path (0,300); %set diagram left start at 0, and has height of 300

	%Shape: Ellipse [id:dp26398419772349535] 
	\draw   (434.5,189) .. controls (434.5,248.65) and (375.96,297) .. (303.75,297) .. controls (231.54,297) and (173,248.65) .. (173,189) .. controls (173,129.35) and (231.54,81) .. (303.75,81) .. controls (375.96,81) and (434.5,129.35) .. (434.5,189) -- cycle ;
	%Straight Lines [id:da6748300763011188] 
	\draw    (187.68,108.49) -- (270.86,83.99) ;
	\draw [shift={(184.81,109.34)}, rotate = 343.59] [fill={rgb, 255:red, 0; green, 0; blue, 0 }  ][line width=0.08]  [draw opacity=0] (10.72,-5.15) -- (0,0) -- (10.72,5.15) -- (7.12,0) -- cycle    ;
	%Shape: Rectangle [id:dp11471698125766983] 
	\draw  [dash pattern={on 0.84pt off 2.51pt}] (207.49,56.69) -- (270.86,83.99) -- (248.17,136.65) -- (184.81,109.34) -- cycle ;
	%Straight Lines [id:da37709233960590205] 
	\draw    (249.36,133.89) -- (270.86,83.99) ;
	\draw [shift={(248.17,136.65)}, rotate = 293.31] [fill={rgb, 255:red, 0; green, 0; blue, 0 }  ][line width=0.08]  [draw opacity=0] (10.72,-5.15) -- (0,0) -- (10.72,5.15) -- (7.12,0) -- cycle    ;
	%Straight Lines [id:da8201337662239865] 
	\draw    (210.25,57.88) -- (270.86,83.99) ;
	\draw [shift={(207.49,56.69)}, rotate = 23.31] [fill={rgb, 255:red, 0; green, 0; blue, 0 }  ][line width=0.08]  [draw opacity=0] (10.72,-5.15) -- (0,0) -- (10.72,5.15) -- (7.12,0) -- cycle    ;

	%Shape: Circle [id:dp4076016936839617] 
	\draw  [draw opacity=0][fill={rgb, 255:red, 184; green, 184; blue, 184 }  ,fill opacity=1 ] (251.5,189) .. controls (251.5,197.7) and (244.45,204.75) .. (235.75,204.75) .. controls (227.05,204.75) and (220,197.7) .. (220,189) .. controls (220,180.3) and (227.05,173.25) .. (235.75,173.25) .. controls (244.45,173.25) and (251.5,180.3) .. (251.5,189) -- cycle ;
	%Shape: Circle [id:dp7855035876590197] 
	\draw  [fill={rgb, 255:red, 0; green, 0; blue, 0 }  ,fill opacity=1 ] (175.67,161.5) .. controls (175.67,160.49) and (176.49,159.67) .. (177.5,159.67) .. controls (178.51,159.67) and (179.33,160.49) .. (179.33,161.5) .. controls (179.33,162.51) and (178.51,163.33) .. (177.5,163.33) .. controls (176.49,163.33) and (175.67,162.51) .. (175.67,161.5) -- cycle ;
	%Shape: Circle [id:dp9700834372146263] 
	\draw  [fill={rgb, 255:red, 0; green, 0; blue, 0 }  ,fill opacity=1 ] (374.67,99) .. controls (374.67,97.99) and (375.49,97.17) .. (376.5,97.17) .. controls (377.51,97.17) and (378.33,97.99) .. (378.33,99) .. controls (378.33,100.01) and (377.51,100.83) .. (376.5,100.83) .. controls (375.49,100.83) and (374.67,100.01) .. (374.67,99) -- cycle ;

	% Text Node
	\draw (384.77,87.77) node    {$A$};
	% Text Node
	\draw (167.6,112.63) node    {$d\vec{r}$};
	% Text Node
	\draw (166.77,155.77) node    {$B$};
	% Text Node
	\draw (243.6,55.13) node    {$d\vec{r}_{\vartheta }$};
	% Text Node
	\draw (276.77,114.8) node    {$d\vec{r}_{r}$};
	% Text Node
	\draw (332.77,65.77) node    {$\Gamma $};

	\end{tikzpicture}
\end{figure}
\FloatBarrier
Dove $d\vec{r}$ è il vettore spostamento, che non va confuso con il raggio. Lo si scompone in: $d\vec{r}=d\vec{r}_r+d\vec{r}_\vartheta$
A questo punto si ha:
\begin{align*}
	\mathcal{L} &= \int_{\Gamma, A \to B} \vec{F}_\gamma \cdot (d\vec{r}_r+d\vec{r}_\vartheta)=\int_{\Gamma, A \to B} \vec{F}_\gamma \cdot d\vec{r}_r= \int_{\Gamma, A \to B} -\gamma \frac{mM}{r^2}\,\vec{u}_r \cdot d\vec{r}_r= \\
	&= \int_{\Gamma, A \to B} - \gamma \frac{mM}{r^2} \,dr_r=-\gamma mM\,\biggl[-\frac{1}{r}\biggr]_{r_A}^{r_B}=\frac{\gamma mM}{r_B}-\frac{\gamma mM}{r_A}
\end{align*}
Contano soltanto la distanza radiale iniziale e quella finale, quindi la forza gravitazionale è conservativa. Si può definire la funzione energia potenziale imponendo che il lavoro sia pari all'opposto della sua variazione.
\[
	\boxed{E_{p,\text{gravitazionale}}=-\gamma \frac{mM}{r}+\text{cost}}
\]
Il pianeta orbitante tende ad essere portato verso il Sole, il segno negativo deriva dunque dal fatto che la forza gravitazionale è attrattiva. Se si sostituisce $r=0$, si ottiene energia potenziale infinita. È invece ragionevole che a distanza infinita, quindi quando non c'è interazione, l'energia potenziale sia nulla e quindi la costante si assume pari a $0$.
\begin{figure}[htpb]
	\centering
	

	\tikzset{every picture/.style={line width=0.75pt}} %set default line width to 0.75pt        

	\begin{tikzpicture}[x=0.75pt,y=0.75pt,yscale=-1,xscale=1]
	%uncomment if require: \path (0,325); %set diagram left start at 0, and has height of 325

	% Plotting does not support converting to Tikz
	%Shape: Axis 2D [id:dp4497706014371563] 
	\draw  (97,104) -- (435.5,104)(123.31,74) -- (123.31,286) (428.5,99) -- (435.5,104) -- (428.5,109) (118.31,81) -- (123.31,74) -- (128.31,81)  ;
	%Curve Lines [id:da8498389368782535] 
	\draw [line width=1.5]    (137,266) .. controls (141,100) and (230,125) .. (384,114) ;

	% Text Node
	\draw (105,70) node    {$E_{p}$};
	% Text Node
	\draw (452,102) node    {$r$};


	\end{tikzpicture}
\end{figure}
\FloatBarrier
Nell'applicare la definizione di lavoro se ne è andata la componente trasversale ed è rimasta quella radiale. Quella appena ricavata prendere il nome di \textbf{energia potenziale gravitazionale}.

Siccome la forza è conservativa, l'energia meccanica, somma dell'energia cinetica e dell'energia gravitazionale, resta costante e quindi se $E_k$ aumenta $E_p$ deve diminuire. Essendo nulla per distanza infinita, $E_p$ deve essere negativa per distante finite.
\[
	E_\text{mecc}=E_k+E_p=\frac{1}{2}mv^2-\gamma\frac{mM}{r}
\]
Riscrivendo l'energia cinetica nelle sue componenti radiale e trasversale:
\[
	E_\text{mecc}=\frac{1}{2}mv_r^2+\frac{1}{2}mv_\vartheta^2-\gamma \frac{mM}{r}=\frac{1}{2}mv_r^2+\underbrace{\underbrace{\frac{1}{2}\frac{\norma{\vec{L}}^2}{mr^2}}_B-\gamma\frac{mM}{r}}_A
\]
Per come è stato riscritto il termine $B$, esso è diventato una funzione che dipende da $L$, costante, e da un termine funzione della posizione. È allora un'energia posizionale, ossia potenziale. $A$ prende il nome di \textbf{energia potenziale gravitazionale efficace}. Si è attribuito così un significato fisico energetico diverso al secondo membro. Questa energia porta con sé una parte dell'energia cinetica, ma, in presenza di forze centrali, si può esprimere come funzione delle coordinate in quanto dipende solo dalla distanza di $m$ da $M$.







































\section{Studio delle orbite}

In questa nuova visione si attribuisce al punto materiale un'unica energia cinetica che informa di quanto velocemente esso si allontana o si avvicina al Sole. È come se non si stesse più osservando il moto come osservatore assoluto posto in $O$, ma dal punto di vista di un osservatore relativo $O'$ che gira con la stessa velocità angolare del pianeta. $O'$ non lo vede muoversi, ma solo avvicinarsi e allontanarsi. Ponendosi come osservatore relativo, deve apparire una forza apparente centrifuga, opposta al raggio, che spiega lo spostamento del pianeta che $O'$ rileva. Essa nel sistema di riferimento apparente compie un lavoro pari al termine $B$. Si è così trasformato il problema dello studio del moto in un problema a una sola variabile, misurata rispetto alla coordinata radiale. L'energia potenziale efficace ha la forma rappresentata di seguito.
\begin{figure}[htpb]
	\centering
	


	\tikzset{every picture/.style={line width=0.75pt}} %set default line width to 0.75pt        

	\begin{tikzpicture}[x=0.75pt,y=0.75pt,yscale=-1,xscale=1]
	%uncomment if require: \path (0,346); %set diagram left start at 0, and has height of 346

	%Shape: Axis 2D [id:dp40528865472744124] 
	\draw  (117,185) -- (455.5,185)(142.5,78) -- (142.5,290) (448.5,180) -- (455.5,185) -- (448.5,190) (137.5,85) -- (142.5,78) -- (147.5,85)  ;
	%Curve Lines [id:da10823584761686678] 
	\draw [line width=1.5]    (142.5,97) .. controls (167,202.5) and (183,275.5) .. (233,276.5) .. controls (283,277.5) and (334,193.5) .. (399.5,194) ;

	% Text Node
	\draw (103,79) node    {$E_{p,\text{efficace}}$};
	% Text Node
	\draw (475,184) node    {$r$};


	\end{tikzpicture}
\end{figure}
\FloatBarrier
Per $r$ grande predomina il termine di energia potenziale gravitazionale. Quindi per $r$ che va a infinito il grafico approssima quello di $E_p$. Per $r$ che va a zero prevale l'energia potenziale efficace (centrifuga). In mezzo i due contributi pesano entrambi e danno luogo a un minimo relativo. L'energia cinetica radiale non può avere segno qualunque ma è sempre positiva o al massimo nulla. Questo vuole dire che la differenza fra l'energia totale e quella efficace gravitazionale deve essere sempre positiva.
\begin{gather*}
	E_\text{mecc}=E_\text{pot, eff}+E_\text{k, rad} \\
	E_\text{k,rad}=E_\text{mecc}-E_\text{pot, eff} >0 \implies E_\text{mecc}>E_\text{pot, eff}
\end{gather*}
Questa informazione è importante perché afferma che per il pianeta sono permesse solo quelle posizioni che fanno si che il grafico della sua energia meccanica sia sempre sopra al grafico dell'energia potenziale gravitazionale efficace. Se l'energia meccanica è al di sotto della curva, il pianeta non riesce a ruotare attorno al pianeta attrattore e collassa su di esso. Quelli sotto la curva sono valori di energia non permessi. L'energia totale può assumere valori positivi, negativi, o nulli.

Quando essa è positiva o nulla, $r$ ha un valore minimo. La distanza fra $M$ e $m$ non può scendere al di sotto di un certo valore che dipende da $E_\text{mecc}$ e da $L$. A parità di $E_\text{mecc}$, maggiore è $L$ maggiore è la distanza minima. Si osservi che quando $r$ è minimo, uno dei termini che compongono l'energia cinetica si annulla, per cui l'energia meccanica coincide con quella potenziale efficace.

\paragraph{1} Per $E_\text{mecc}=0$ le distanze consentite sono tutte quelle in cui l'energia meccanica sta sopra l'energia potenziale efficace. $r$ tenderà a infinito, nel punto di minimo la velocità ha soltanto componente trasversale e il punto si allontanerà all'infinito, facendo tendere l'energia potenziale efficace a $0$. La traiettoria è quella di una \textbf{parabola}. Anche l'energia cinetica radiale va a $0$. Quindi il punto arriva all'infinito con velocità nulla.

\paragraph{2} Per $E_\text{mecc}$ positiva si ha di nuovo una traiettoria aperta. Il pianeta arriva a distanza infinita avendo ancora energia cinetica radiale e quindi una certa velocità. Si può dimostrare che la traiettoria è un'\textbf{iperbole}.

\paragraph{3} Quando l'energia totale è negativa, la distanza fra $m$ e $M$ ha un valore minimo e massimo. Si può dimostrare che la traiettoria è appunto una \textbf{traiettoria ellittica} dove la distanza minima è il perielio e la massima è l'afelio. Nei punti $A$ e $B$ l'energia cinetica radiale è zero. In particolare, nel caso in cui l'energia meccanica ha valore pari a quello minimo dell'energia potenziale efficace, è permesso solo un valore di $r$ e quindi la configurazione è quella di un'\textbf{orbita circolare}, tutte le altre hanno un livello energetico superiore.
\begin{figure}[ht]
	\centering
	


	\tikzset{every picture/.style={line width=0.75pt}} %set default line width to 0.75pt        

	\begin{tikzpicture}[x=0.75pt,y=0.75pt,yscale=-1,xscale=1]
	%uncomment if require: \path (0,306); %set diagram left start at 0, and has height of 306

	%Shape: Axis 2D [id:dp06633050074921831] 
	\draw  (137,185) -- (475.5,185)(162.5,78) -- (162.5,290) (468.5,180) -- (475.5,185) -- (468.5,190) (157.5,85) -- (162.5,78) -- (167.5,85)  ;
	%Straight Lines [id:da8964693514751385] 
	\draw  [dash pattern={on 0.84pt off 2.51pt}]  (171.67,141.67) -- (171.67,185.67) ;
	%Straight Lines [id:da567489127467361] 
	\draw  [dash pattern={on 0.84pt off 2.51pt}]  (162.67,141.67) -- (394,141.67) ;
	%Curve Lines [id:da05339900489855598] 
	\draw [line width=1.5]    (162.5,100) .. controls (187,205.5) and (203,278.5) .. (253,279.5) .. controls (303,280.5) and (354,196.5) .. (419.5,197) ;

	% Text Node
	\draw (143,82) node    {$E_{p}$};
	% Text Node
	\draw (495,184) node    {$r$};
	% Text Node
	\draw (189.67,124.67) node    {$r_{min}$};
	% Text Node
	\draw (99,140) node    {$E_{\text{meccanica}} \geqslant 0$};


	\end{tikzpicture}
\end{figure}
\FloatBarrier
\begin{figure}[htpb]
	\centering
	

	\tikzset{every picture/.style={line width=0.75pt}} %set default line width to 0.75pt        

	\begin{tikzpicture}[x=0.75pt,y=0.75pt,yscale=-1,xscale=1]
	%uncomment if require: \path (0,310); %set diagram left start at 0, and has height of 310

	%Shape: Axis 2D [id:dp6630535401419646] 
	\draw  (137,145) -- (475.5,145)(162.5,38) -- (162.5,250) (468.5,140) -- (475.5,145) -- (468.5,150) (157.5,45) -- (162.5,38) -- (167.5,45)  ;
	%Straight Lines [id:da10628897100542933] 
	\draw  [dash pattern={on 0.84pt off 2.51pt}]  (202.17,145.17) -- (202.17,191.17) ;
	%Straight Lines [id:da6958420279748636] 
	\draw  [dash pattern={on 0.84pt off 2.51pt}]  (162.67,191.67) -- (394,191.67) ;
	%Straight Lines [id:da07604953384880853] 
	\draw  [dash pattern={on 0.84pt off 2.51pt}]  (335.17,145.17) -- (335.17,191.17) ;
	%Curve Lines [id:da9957461492216604] 
	\draw [line width=1.5]    (162.5,57) .. controls (187,162.5) and (203,235.5) .. (253,236.5) .. controls (303,237.5) and (354,153.5) .. (419.5,154) ;

	% Text Node
	\draw (145,44) node    {$E_{p}$};
	% Text Node
	\draw (495,144) node    {$r$};
	% Text Node
	\draw (206.67,130.17) node    {$r_{min}$};
	% Text Node
	\draw (101,191) node    {$E_{\text{meccanica}} < 0$};
	% Text Node
	\draw (333.67,130.17) node    {$r_{max}$};


	\end{tikzpicture}
\end{figure}
\FloatBarrier
Quindi studiando il segno e il valore dell'energia totale si possono ottenere informazioni molto utili sulla forma della traiettoria seguita dal corpo. Si immagini ora di voler far partire un razzo dalla Terra. L'energia potenziale gravitazionale è fissata perché è nota la posizione del razzo. Perché esso possa decollare, gli si conferisce una certo valore della velocità, e quindi un'energia cinetica iniziale.
Dato che l'energia totale è costante, la somma di energia cinetica e potenziale dà un valore che a questo punto è fissato e che durante il moto si manterrà tale e quale. A seconda quindi della velocità che si conferisce al razzo, l'orbita percorsa sarà diversa e ci si riconduce a uno dei tre casi descritti in precedenza.

Oltre che scegliendo il modulo della velocità iniziale, si può far variare l'orbita seguita anche imponendo la direzione della velocità. Questo infatti si traduce nello stabilire un certo modulo del momento angolare. L'energia potenziale efficace dipende da quest'ultimo e quindi, nonostante mantenga la stessa forma, a seconda del valore di $L$ il suo andamento può allargarsi o restringersi.
\begin{figure}[htpb]
	\centering
	

	\tikzset{every picture/.style={line width=0.75pt}} %set default line width to 0.75pt        

	\begin{tikzpicture}[x=0.75pt,y=0.75pt,yscale=-1,xscale=1]
	%uncomment if require: \path (0,300); %set diagram left start at 0, and has height of 300

	%Shape: Axis 2D [id:dp008375906793796739] 
	\draw  (101,171) -- (439.5,171)(126.5,64) -- (126.5,276) (432.5,166) -- (439.5,171) -- (432.5,176) (121.5,71) -- (126.5,64) -- (131.5,71)  ;
	%Straight Lines [id:da17443468183766164] 
	\draw  [dash pattern={on 0.84pt off 2.51pt}]  (164.17,171.17) -- (164.17,217.17) ;
	%Straight Lines [id:da42800203214304755] 
	\draw  [dash pattern={on 0.84pt off 2.51pt}]  (126.67,217.67) -- (358,217.67) ;
	%Straight Lines [id:da9495313997069219] 
	\draw  [dash pattern={on 0.84pt off 2.51pt}]  (299.17,171.17) -- (299.17,217.17) ;
	%Straight Lines [id:da7012721870034195] 
	\draw  [dash pattern={on 0.84pt off 2.51pt}]  (126.67,117.67) -- (358,117.67) ;
	%Straight Lines [id:da7410405305183971] 
	\draw  [dash pattern={on 0.84pt off 2.51pt}]  (217.17,171.5) -- (217.17,263.17) ;
	%Straight Lines [id:da36685552188825965] 
	\draw  [dash pattern={on 0.84pt off 2.51pt}]  (126.67,263.67) -- (217,263.67) ;
	%Curve Lines [id:da2586176186680875] 
	\draw [line width=1.5]    (126.5,83) .. controls (151,188.5) and (167,261.5) .. (217,262.5) .. controls (267,263.5) and (318,179.5) .. (383.5,180) ;

	% Text Node
	\draw (111,54) node    {$E$};
	% Text Node
	\draw (459,170) node    {$r$};
	% Text Node
	\draw (168.67,156.17) node    {$r_{min}$};
	% Text Node
	\draw (296.67,156.17) node    {$r_{max}$};
	% Text Node
	\draw (219.67,155.67) node    {$r$};
	% Text Node
	\draw (90.67,257.67) node    {$E_{min}$};
	% Text Node
	\draw (86.17,215.67) node    {$E< 0$};
	% Text Node
	\draw (86.17,159.17) node    {$E=0$};
	% Text Node
	\draw (86.17,116.17) node    {$E >0$};


	\end{tikzpicture}
\end{figure}
\FloatBarrier
È interessante osservare che iperbole, parabola ed ellisse fanno parte di una grande famiglia di curve che prende il nome di \emph{coniche}. In generale infatti si può dimostrare che, partendo dal secondo principio della dinamica,  un satellite o un pianeta in moto in un campo di forze gravitazionali si muove su una traiettoria che è quella di una conica. Esiste una equazione univoca che permette di definire l'equazione di qualsiasi conica, variando semplicemente un parametro e che può essere data sia in componenti cartesiane che polari.
\begin{figure}[h!]
	\centering
	

	\tikzset{every picture/.style={line width=0.75pt}} %set default line width to 0.75pt        

	\begin{tikzpicture}[x=0.75pt,y=0.75pt,yscale=-0.8,xscale=0.8]
	%uncomment if require: \path (0,300); %set diagram left start at 0, and has height of 300

	%Shape: Circle [id:dp41009238915658974] 
	\draw  [fill={rgb, 255:red, 0; green, 0; blue, 0 }  ,fill opacity=1 ] (333.67,150) .. controls (333.67,148.34) and (335.01,147) .. (336.67,147) .. controls (338.32,147) and (339.67,148.34) .. (339.67,150) .. controls (339.67,151.66) and (338.32,153) .. (336.67,153) .. controls (335.01,153) and (333.67,151.66) .. (333.67,150) -- cycle ;
	%Shape: Circle [id:dp8101755239068957] 
	\draw   (256.33,150) .. controls (256.33,105.63) and (292.3,69.67) .. (336.67,69.67) .. controls (381.03,69.67) and (417,105.63) .. (417,150) .. controls (417,194.37) and (381.03,230.33) .. (336.67,230.33) .. controls (292.3,230.33) and (256.33,194.37) .. (256.33,150) -- cycle ;
	%Shape: Ellipse [id:dp42250607336258317] 
	\draw   (176,150) .. controls (176,98.91) and (224.28,57.5) .. (283.83,57.5) .. controls (343.39,57.5) and (391.67,98.91) .. (391.67,150) .. controls (391.67,201.09) and (343.39,242.5) .. (283.83,242.5) .. controls (224.28,242.5) and (176,201.09) .. (176,150) -- cycle ;
	%Curve Lines [id:da01919817252796241] 
	\draw    (245.33,4.33) .. controls (419.67,102.67) and (426.33,193.33) .. (245,295.33) ;
	%Curve Lines [id:da9838056002806577] 
	\draw    (302.33,4.67) .. controls (306.57,12.22) and (310.58,19.38) .. (314.37,26.19) .. controls (382.78,149.07) and (378.75,156.29) .. (302,294) ;

	% Text Node
	\draw (462,213.67) node    {$\text{circonferenza } e=0$};
	% Text Node
	\draw (157.67,218) node    {$\text{ellisse } e< 1$};
	% Text Node
	\draw (220.67,29.42) node    {$\text{parabola } e=1$};
	% Text Node
	\draw (375.17,29.42) node    {$\text{iperbole } e >1$};
	% Text Node
	\draw (488.67,73.42) node    {$e=\text{eccentricità}$};


	\end{tikzpicture}
\end{figure}
\FloatBarrier







































\section{Problema dei due corpi}

Si è studiato il moto di un corpo materiale soggetto a forza gravitazionale nell'ipotesi che il corpo attrattore sia fermo. Ovviamente questa è un'approssimazione, perché se il corpo di massa $m$ risente della forza attrattiva generata dal Sole, vuole dire che viceversa, per il principio di azione reazione, il pianeta di massa $m$ genererà una forza uguale e contraria sul Sole. L'approssimazione è corretta perché la massa del Sole è così grande che l'effetto di questa forza generata su di lui non da luogo a nessun effetto dinamico apprezzabile. Tuttavia, il problema dello studio degli effetti della forza gravitazionale quando si considerano due corpi, è molto interessante, perché lo si usa per vedere come interagiscono fra di loro ad esempio due pianeti o due corpi di massa paragonabile. Questo problema è noto come problema dei due corpi. Siano $m_1$ ed $m_2$ i due pianeti, $\vec{F}_1$ la forza generata da $m_2$ su $m_1$ e $\vec{F}_2$ la forza generata da $m_1$ su $m_2$. Si consideri un osservatore inerziale che si mette in un punto dello spazio fermo, questo non può più coincidere con una delle due masse se si suppone che si muovano. Si va a definire il vettore posizione $\vec{r}_1$ che identifica la posizione di $m_1$ rispetto all'osservatore assoluto e analogamente il vettore $\vec{r}_2$ che identifica la posizione della massa $m_2$.
\begin{figure}[htpb]
	\centering
	

	\tikzset{every picture/.style={line width=0.75pt}} %set default line width to 0.75pt        

	\begin{tikzpicture}[x=0.75pt,y=0.75pt,yscale=-1,xscale=1]
	%uncomment if require: \path (0,300); %set diagram left start at 0, and has height of 300

	%Straight Lines [id:da6863459773640985] 
	\draw    (114.5,238) -- (271.44,107.42) ;
	\draw [shift={(273.75,105.5)}, rotate = 500.24] [fill={rgb, 255:red, 0; green, 0; blue, 0 }  ][line width=0.08]  [draw opacity=0] (10.72,-5.15) -- (0,0) -- (10.72,5.15) -- (7.12,0) -- cycle    ;
	%Straight Lines [id:da5313627472873643] 
	\draw    (114.5,238) -- (134.02,111.87) ;
	\draw [shift={(134.48,108.9)}, rotate = 458.8] [fill={rgb, 255:red, 0; green, 0; blue, 0 }  ][line width=0.08]  [draw opacity=0] (10.72,-5.15) -- (0,0) -- (10.72,5.15) -- (7.12,0) -- cycle    ;
	%Straight Lines [id:da4027192773772521] 
	\draw    (134.48,102.4) -- (268.5,99.61) ;
	\draw [shift={(271.5,99.55)}, rotate = 538.81] [fill={rgb, 255:red, 0; green, 0; blue, 0 }  ][line width=0.08]  [draw opacity=0] (10.72,-5.15) -- (0,0) -- (10.72,5.15) -- (7.12,0) -- cycle    ;
	%Shape: Circle [id:dp5223940100275606] 
	\draw  [fill={rgb, 255:red, 0; green, 0; blue, 0 }  ,fill opacity=1 ] (112.75,238) .. controls (112.75,237.03) and (113.53,236.25) .. (114.5,236.25) .. controls (115.47,236.25) and (116.25,237.03) .. (116.25,238) .. controls (116.25,238.97) and (115.47,239.75) .. (114.5,239.75) .. controls (113.53,239.75) and (112.75,238.97) .. (112.75,238) -- cycle ;
	%Shape: Circle [id:dp597764102802874] 
	\draw  [fill={rgb, 255:red, 0; green, 0; blue, 0 }  ,fill opacity=1 ] (127.98,102.4) .. controls (127.98,98.81) and (130.89,95.9) .. (134.48,95.9) .. controls (138.07,95.9) and (140.98,98.81) .. (140.98,102.4) .. controls (140.98,105.99) and (138.07,108.9) .. (134.48,108.9) .. controls (130.89,108.9) and (127.98,105.99) .. (127.98,102.4) -- cycle ;
	%Shape: Circle [id:dp7278212874749974] 
	\draw  [fill={rgb, 255:red, 0; green, 0; blue, 0 }  ,fill opacity=1 ] (272,100.05) .. controls (272,96.46) and (274.91,93.55) .. (278.5,93.55) .. controls (282.09,93.55) and (285,96.46) .. (285,100.05) .. controls (285,103.64) and (282.09,106.55) .. (278.5,106.55) .. controls (274.91,106.55) and (272,103.64) .. (272,100.05) -- cycle ;
	%Straight Lines [id:da2625538502562448] 
	\draw [line width=1.5]    (134.48,105.9) -- (182.25,104.61) ;
	\draw [shift={(186.25,104.5)}, rotate = 538.45] [fill={rgb, 255:red, 0; green, 0; blue, 0 }  ][line width=0.08]  [draw opacity=0] (13.4,-6.43) -- (0,0) -- (13.4,6.44) -- (8.9,0) -- cycle    ;
	%Straight Lines [id:da010873890958111199] 
	\draw [line width=1.5]    (221.48,103.79) -- (269.25,102.5) ;
	\draw [shift={(217.48,103.9)}, rotate = 358.45] [fill={rgb, 255:red, 0; green, 0; blue, 0 }  ][line width=0.08]  [draw opacity=0] (13.4,-6.43) -- (0,0) -- (13.4,6.44) -- (8.9,0) -- cycle    ;

	% Text Node
	\draw (107.5,248.5) node    {$O$};
	% Text Node
	\draw (107,165.5) node    {$\vec{r}_{1}$};
	% Text Node
	\draw (199,191) node    {$\vec{r}_{2}$};
	% Text Node
	\draw (114.5,93.5) node    {$m_{1}$};
	% Text Node
	\draw (297,87.5) node    {$m_{2}$};
	% Text Node
	\draw (203,86.5) node    {$\vec{r}$};
	% Text Node
	\draw (248,80) node    {$\vec{F}_{2}$};
	% Text Node
	\draw (158,83) node    {$\vec{F}_{1}$};


	\end{tikzpicture}
\end{figure}
\FloatBarrier
Si nota che il vettore $\vec{r}$ che dà la distanza fra i due pianeti, è tale per cui, per costruzione geometrica:
\[
	\vec{r}_1+\vec{r}=\vec{r}_2 \implies \vec{r}=\vec{r}_2-\vec{r}_1
\]
Per studiare un problema di questo tipo si applica il secondo principio della dinamica:
\begin{align*}
	\vec{F}_1 &= m_1\vec{a}_1=m_1\frac{d^2\vec{r}_1}{dt^2} \\
	\vec{F}_2 &= m_1\vec{a}_2=m_2\frac{d^2\vec{r}_2}{dt^2}
\end{align*}
Si dividono le due relazioni membro a membro:
\[
	\frac{\vec{F}_1}{m_1}=\frac{d^2\vec{r}_1}{dt^2} \quad \frac{\vec{F}_2}{m_2}=\frac{d^2\vec{r}_2}{dt^2}
\]
Si sa per il principio di azione reazione che queste due forze $\vec{F}_1$ e $\vec{F}_2$ non sono indipendenti fra di loro ma sono una l'opposta dell'altra. Si chiami allora $\vec{F}$ la forza $\vec{F}_2$, $\vec{F}_1$ sarà semplicemente $-\vec{F}$.
\[
	\vec{F}_2=\vec{F} \quad \vec{F}_1=-\vec{F}
\]
Sottraendo membro a membro:
\[
	\frac{\vec{F}}{m_2}+\frac{\vec{F}}{m_1}=\frac{d^2(\vec{r}_2-\vec{r_1})}{dt^2}
\]
La derivazione è un operazione lineare, quindi fare la differenza delle derivate è uguale a fare la derivata della differenza. Quindi si avrà:
\[
	\vec{F} \biggl(\frac{1}{m_1}+\frac{1}{m_2} \biggr)=\frac{d^2\vec{r}}{dt^2}
\]
A questo punto si può definire una quantità detta \textbf{massa ridotta}, $\mu$, che ha dimensione di una massa ed è tale per cui:
\[
	\frac{1}{\mu}=\biggl(\frac{1}{m_1}+\frac{1}{m_2} \biggr) \implies \vec{F}\frac{1}{\mu}=\frac{d^2\vec{r}}{dt^2} \implies \vec{F}=\mu \frac{d^2\vec{r}}{dt^2}
\]
Si ottiene finalmente una relazione affermante che la forza che genera uno dei due corpi, è uguale alla massa ridotta per la derivata seconda di $\vec{r}$ rispetto al tempo. Quindi, se si vuole studiare il moto di due corpi soggetti entrambi alla mutua interazione gravitazionale, invece che concentrarsi sul moto assoluto di $m_1$ e $m_2$, si può osservare il moto relativo dei due corpi uno rispetto all'altro, semplicemente sostituendo alla massa di uno dei due corpi, la massa ridotta. Si è infatti riscritto il moto in funzione della variabile $\vec{r}$.  Si semplifica il problema dei due corpi nel problema di un corpo solo. Nel caso del moto della Terra intorno al Sole, dove quindi uno dei due corpi è molto molto maggiore in massa dell'altro, si ha:
\[
	M \gg m \implies \frac{1}{\mu}=\frac{1}{M}+\frac{1}{m}  \approx \frac{1}{M} \implies \mu \approx M
\]
Si è immaginato il Sole come fermo, perché fondamentalmente la sua massa è molto maggiore e la massa ridotta diventa semplicemente la massa del pianeta. Se le masse dei due pianeti invece sono di valore simile, allora si deve semplicemente sostituire alla massa di un pianeta, la massa ridotta.

Affrontare il problema dei tre corpi è un problema molto complesso che non ha soluzione analitica. In generale, studiare il moto relativo di un certo numero di corpi soggetti a mutua interazione è molto complicato.























































































































\chapter{Dinamica dei sistemi di particelle}

\section{Sistemi di punti materiali}

Ci si sposta ora dalla dinamica del punto materiale alla dinamica dei sistemi di punti materiali. Essi sono sistemi in cui figurano numerosi oggetti ancora approssimati a corpi puntiformi che interagiscono tra di loro e contemporaneamente interagiscono con l'ambiente circostante.
Un approccio potrebbe essere quello di considerare ogni punto materiale e scrivere la seconda equazione della dinamica. Tuttavia si otterrebbe un sistema troppo complesso di questo tipo:
\[
	\begin{cases} \vec{F}_1^\text{tot}=m\vec{a}_1 \\ \vec{F}_2^\text{tot}=m\vec{a}_2 \\ \vdots \\ \vec{F}_n^\text{tot}=m\vec{a}_n  \end{cases} \text{ $n$ equazioni vettoriali in $3n$ incognite}
\]
In più non tutte le forze sono facilmente ricavabili a priori. I punti materiali ovviamente potranno scambiarsi delle forze, quindi ad esempio la massa $m_1$ interagendo con la massa $m_2$ sarà soggetta a una forza $F_1$. Inoltre, focalizzando l'attenzione sulla generica massa $i$-esima, questa non sarà solo soggetta all'interazione con gli altri corpi, ma anche a quella delle forze esterne, ad esempio la forza peso di attrazione al centro della Terra.

Si definiscono \textbf{forze esterne} le forze generate su un certo punto materiale sotto l'azione di parti esterne al sistema. Al contrario, le \textbf{forze interne} saranno quelle generate su una massa da un'altra parte del sistema.
La distinzione tra forze interne ed esterne dipende da come viene definito il sistema di punti. Se per esempio si ingloba nel sistema una parte del resto dell'universo, alcune forze, precedentemente considerate esterne, diventano interne.

Considerando la seconda legge della dinamica scritta per il corpo di massa $i$-esima, si può dire che la somma di tutte le forze che agiscono su di essa, danno luogo alla sua accelerazione vettoriale. Scomponendo tale somma nel contributo delle forze esterne e in quello delle forze interne:
\[
	\vec{F}_i^\text{ext}+\vec{F}_i^\text{int}=m\vec{a}_i
\]
Si è così definito dal punto di vista dinamico come cambia la terminologia quando si parla di dinamica dei sistemi.

È possibile anche parlare delle quantità di moto complessiva del sistema di punti. Si definisce \textbf{quantità di moto totale} di esso la somma vettoriale delle quantità di moto di ogni punto materiale.
\[
	\vec{p}_\text{tot}=\sum_{i=1}^n \vec{p}_i
\]
Si scrive la seconda legge della dinamica per tutti i punti fino all'$n$-esimo. Si troverà un sistema di $n$ equazioni vettoriali, cioè $3n$ equazioni scalari e quindi $3n$ incognite.
\[
	\begin{cases}
		\vec{F}_1^\text{int} + \vec{F}_1^\text{ext}=m\vec{a}_1=m\frac{d\vec{p}_1}{dt} \\
		\vec{F}_2^\text{int} + \vec{F}_2^\text{ext}=m\vec{a}_2=m\frac{d\vec{p}_2}{dt} \\
		\vdots \\
		\vec{F}_n^\text{int} + \vec{F}_n^\text{ext}=m\vec{a}_n=m\frac{d\vec{p}_n}{dt}
	\end{cases}
\]

\paragraph{Esempio} Si consideri un portachiavi in cui ogni punto è una chiave. Quando lo si lancia, le varie chiavi sono soggette sicuramente all'azione della forza peso, ma esse in realtà non si muovono perché venendo a contatto con le chiavi vicine vengono spinte. Quindi l'azione delle forze interne va a modificare la traiettoria di ogni singolo punto. È impossibile riuscire a ricavare la loro azione per via dell'impossibilità di ricavare su di esse un'informazione quantitativa.

Quindi invece che risolvere il sistema in questa maniera, si va a generare un'unica equazione sommando membro a membro queste equazioni.
\[
	\sum_{i=1}^n \vec{F}_i^\text{ext}+\sum_{i=1}^n \vec{F}_i^\text{int}=\sum_{i=1}^nm_i\vec{a}_i=\frac{d\sum_{i=1}^n\vec{p}_i}{dt}=\frac{d\vec{p}_\text{tot}}{dt}
\]
La sommatoria delle forze interne si annulla perché per il principio di azione reazione per ogni forza interna ce ne sarà un'altra uguale e contraria. Si ottiene allora un'equazione in cui la sommatoria di tutte le forze esterne provocano la variazione della quantità di moto complessiva del sistema, eliminando le forze interne. La sommatoria delle forze esterne viene anche scritta come risultante delle forze esterne $\vec{R}_\text{tot}$:
\[
	\boxed{\vec{R}_\text{tot}=\frac{d\vec{p}_\text{tot}}{dt}}
\]
Questa equazione è un'equazione fondamentale che prende il nome di \textbf{prima equazione cardinale della dinamica dei sistemi}. Se si guarda un sistema di punti materiali ognuno dei quali si muove in maniera complicata, si può studiarne il moto complessivo in termini di quantità di moto, per poter così considerare soltanto l'effetto delle forze esterne. È una relazione molto comoda e sintetica da utilizzare.
Si dice che un sistema è \textbf{isolato} se la risultante delle forze esterne è uguale a zero. In questo caso la prima equazione diventa una legge di conservazione detta \textbf{principio di conservazione della quantità di moto}:
\[
	\boxed{\vec{R}^\text{est}=0 \implies \vec{p}_\text{tot}=\text{costante}}
\]

\paragraph{Esempio sulla legge di conservazione} Si immagini di avere un frammento (inizialmente fermo) sospeso in aria. Esso, a causa di una forza interna, esplode, rompendosi in due frammenti: il sistema è isolato.
\begin{figure}[htpb]
	\centering
	

	\tikzset{every picture/.style={line width=0.75pt}} %set default line width to 0.75pt        

	\begin{tikzpicture}[x=0.75pt,y=0.75pt,yscale=-1,xscale=1]
	%uncomment if require: \path (0,300); %set diagram left start at 0, and has height of 300

	%Shape: Polygon Curved [id:ds41536764033360196] 
	\draw   (134,95) .. controls (112.5,57) and (244,75) .. (224,95) .. controls (204,115) and (204,125) .. (224,155) .. controls (244,185) and (154,185) .. (134,155) .. controls (114,125) and (155.5,133) .. (134,95) -- cycle ;
	%Straight Lines [id:da7352415804326031] 
	\draw    (210,128) -- (265.58,141.3) ;
	\draw [shift={(268.5,142)}, rotate = 193.46] [fill={rgb, 255:red, 0; green, 0; blue, 0 }  ][line width=0.08]  [draw opacity=0] (10.72,-5.15) -- (0,0) -- (10.72,5.15) -- (7.12,0) -- cycle    ;
	%Straight Lines [id:da14490670515067294] 
	\draw    (84.42,97.7) -- (140,111) ;
	\draw [shift={(81.5,97)}, rotate = 13.46] [fill={rgb, 255:red, 0; green, 0; blue, 0 }  ][line width=0.08]  [draw opacity=0] (10.72,-5.15) -- (0,0) -- (10.72,5.15) -- (7.12,0) -- cycle    ;

	% Text Node
	\draw (100,123) node    {$\vec{v}_{1}$};
	% Text Node
	\draw (251,118) node    {$\vec{v}_{2}$};


	\end{tikzpicture}
\end{figure}
\FloatBarrier
Se uno di essi si muove in una direzione, l'altro frammento partirà in direzione uguale e contraria. Questo perché il sistema era inizialmente fermo e quindi, la quantità di moto che era zero inizialmente, deve continuare a essere tale. Il sistema rimane complessivamente fermo anche se in realtà i frammenti partono in direzione opposta. Essi non hanno necessariamente la stessa velocità, perché quello che si deve osservare è la quantità di moto totale, cioè la somma:
\[
	m_1\vec{v}_1+m_2\vec{v}_2=0
\]
È evidente che il frammento di massa maggiore partirà con una velocità minore in modulo.







































\section{Centro di massa}

Si immagini di avere $n$ punti materiali, rilevati rispetto a un osservatore assoluto. È possibile definire la posizione di un punto fittizio detto \textbf{centro di massa}, il cui vettore posizione è dato dalla posizione media di tutti i punti materiali pesati per la loro massa. Attuare una media ponderata di questo tipo significa dare più importanza alle masse più elevate. Il centro di massa di un sistema tende a trovarsi dove è concentrata maggiormente la sua massa e non coincide necessariamente con un suo punto. La definizione matematica è:
\[
	\boxed{\vec{r}_\text{CM} =\sum_{i=1}^n \frac{m_i\vec{r}_i}{M_\text{tot}}}
\]
\begin{figure}[htpb]
	\centering
	

	\tikzset{every picture/.style={line width=0.75pt}} %set default line width to 0.75pt        

	\begin{tikzpicture}[x=0.75pt,y=0.75pt,yscale=-1,xscale=1]
	%uncomment if require: \path (0,300); %set diagram left start at 0, and has height of 300

	%Straight Lines [id:da9212332807493775] 
	\draw [color={rgb, 255:red, 155; green, 155; blue, 155 }  ,draw opacity=1 ]   (105.25,118.25) -- (69.5,242) ;
	%Straight Lines [id:da44266553778346474] 
	\draw [color={rgb, 255:red, 155; green, 155; blue, 155 }  ,draw opacity=1 ]   (155.25,98.25) -- (69.5,242) ;
	%Straight Lines [id:da29671944764078395] 
	\draw [color={rgb, 255:red, 155; green, 155; blue, 155 }  ,draw opacity=1 ]   (215.25,148.25) -- (69.5,242) ;
	%Straight Lines [id:da9922801504038579] 
	\draw [color={rgb, 255:red, 155; green, 155; blue, 155 }  ,draw opacity=1 ]   (175.25,198.25) -- (69.5,242) ;
	%Straight Lines [id:da04217664681854427] 
	\draw [color={rgb, 255:red, 155; green, 155; blue, 155 }  ,draw opacity=1 ]   (125.25,188.25) -- (69.5,242) ;
	%Straight Lines [id:da7222564679512768] 
	\draw [color={rgb, 255:red, 155; green, 155; blue, 155 }  ,draw opacity=1 ]   (148.4,139.8) -- (69.5,242) ;
	%Shape: Axis 2D [id:dp29791761046641163] 
	\draw  (50,242) -- (295.5,242)(69.5,52) -- (69.5,257) (288.5,237) -- (295.5,242) -- (288.5,247) (64.5,59) -- (69.5,52) -- (74.5,59)  ;
	%Shape: Circle [id:dp7005511906572408] 
	\draw  [fill={rgb, 255:red, 0; green, 0; blue, 0 }  ,fill opacity=1 ] (100,118.25) .. controls (100,115.35) and (102.35,113) .. (105.25,113) .. controls (108.15,113) and (110.5,115.35) .. (110.5,118.25) .. controls (110.5,121.15) and (108.15,123.5) .. (105.25,123.5) .. controls (102.35,123.5) and (100,121.15) .. (100,118.25) -- cycle ;
	%Shape: Circle [id:dp36061172179055134] 
	\draw  [fill={rgb, 255:red, 0; green, 0; blue, 0 }  ,fill opacity=1 ] (150,98.25) .. controls (150,95.35) and (152.35,93) .. (155.25,93) .. controls (158.15,93) and (160.5,95.35) .. (160.5,98.25) .. controls (160.5,101.15) and (158.15,103.5) .. (155.25,103.5) .. controls (152.35,103.5) and (150,101.15) .. (150,98.25) -- cycle ;
	%Shape: Circle [id:dp6157944650911422] 
	\draw  [fill={rgb, 255:red, 0; green, 0; blue, 0 }  ,fill opacity=1 ] (170,198.25) .. controls (170,195.35) and (172.35,193) .. (175.25,193) .. controls (178.15,193) and (180.5,195.35) .. (180.5,198.25) .. controls (180.5,201.15) and (178.15,203.5) .. (175.25,203.5) .. controls (172.35,203.5) and (170,201.15) .. (170,198.25) -- cycle ;
	%Shape: Circle [id:dp9455255125788649] 
	\draw  [fill={rgb, 255:red, 0; green, 0; blue, 0 }  ,fill opacity=1 ] (210,148.25) .. controls (210,145.35) and (212.35,143) .. (215.25,143) .. controls (218.15,143) and (220.5,145.35) .. (220.5,148.25) .. controls (220.5,151.15) and (218.15,153.5) .. (215.25,153.5) .. controls (212.35,153.5) and (210,151.15) .. (210,148.25) -- cycle ;
	%Shape: Circle [id:dp7342008389787866] 
	\draw  [fill={rgb, 255:red, 0; green, 0; blue, 0 }  ,fill opacity=1 ] (120,188.25) .. controls (120,185.35) and (122.35,183) .. (125.25,183) .. controls (128.15,183) and (130.5,185.35) .. (130.5,188.25) .. controls (130.5,191.15) and (128.15,193.5) .. (125.25,193.5) .. controls (122.35,193.5) and (120,191.15) .. (120,188.25) -- cycle ;
	%Shape: Circle [id:dp8542177920959095] 
	\draw   (146.5,136.25) .. controls (146.5,133.35) and (148.85,131) .. (151.75,131) .. controls (154.65,131) and (157,133.35) .. (157,136.25) .. controls (157,139.15) and (154.65,141.5) .. (151.75,141.5) .. controls (148.85,141.5) and (146.5,139.15) .. (146.5,136.25) -- cycle ;

	% Text Node
	\draw (104,96) node    {$m_{1}$};
	% Text Node
	\draw (155,78) node    {$m_{2}$};
	% Text Node
	\draw (239,146) node    {$m_{3}$};
	% Text Node
	\draw (195,208) node    {$m_{4}$};
	% Text Node
	\draw (143,172.5) node    {$m_{n}$};
	% Text Node
	\draw (179.5,130) node    {$CM$};


	\end{tikzpicture}
\end{figure}
\FloatBarrier
Definita la posizione del centro di massa, si può anche definire la sua velocità, derivando la relazione:
\[
	\vec{v}_\text{CM}=\frac{d\vec{r}_\text{CM}}{dt}=\frac{d\sum_{i=1}^nm_i\vec{r}_i}{dt} \frac{1}{M_\text{tot}}=\sum_{i=1}^n\frac{dm_i\vec{r}_i}{dt} \frac{1}{M_\text{tot}}=\frac{\sum_{i=1}^nm_i\vec{v}_i}{M_\text{tot}}
\]
La velocità del centro di massa non è altro che la media delle velocità dei punti materiali pesata per le corrispondenti masse. Da questo risultato si ottiene un'informazione molto importante sul centro di massa. Infatti:
\[
	\boxed{M_\text{tot}\vec{v}_\text{CM}=\sum_{i=1}^nm_i \vec{v}_i=\vec{p}_\text{tot}}
\]
Questo risultato prende il nome di \textbf{primo teorema del centro di massa}. In termini di quantità di moto, si può considerare il sistema nel suo complesso come un unico punto materiale in cui è concentrata tutta la massa e che coincide con la posizione del centro di massa. Si va ad applicare questo risultato alla prima equazione cardinale della dinamica. Nell'ipotesi che il sistema non vari la sua massa:
\[
	\boxed{\vec{R}_\text{tot}=\frac{d\vec{p}_\text{tot}}{dt}=\frac{dM_\text{tot}\vec{v}_\text{CM}}{dt}=M_\text{tot}\frac{d\vec{v}_\text{CM}}{dt}=M_\text{tot}\vec{a}_\text{CM}}
\]
L'accelerazione del centro di massa è la media pesata delle varie accelerazioni. Si vede quindi che Il suo moto è causato dalle forze esterne.  Questo risultato prende il nome di \textbf{secondo teorema del centro di massa} e, dal punto di vista delle informazioni che fornisce, è del tutto equivalente alla prima equazione cardinale della dinamica. Il suo vantaggio è che fa vedere il problema dello studio del sistema di punti materiali in maniera molto semplice.
In aggiunta esso può diventare un principio di conservazione quando sul sistema non agiscono forze esterne. In questo caso il centro di massa si muove di moto rettilineo uniforme o, se era inizialmente fermo, rimane in quiete.
Si è trasformato il complesso problema dello studio del moto di un sistema di punti materiali allo studio del moto di un singolo punto materiale che è il centro di massa.

\paragraph{Esempio} Se il sistema è isolato, le equazioni ottenute diventano:
\[
	\vec{R}_\text{ext}=0 \quad \vec{a}_\text{CM}=0 \quad \vec{v}_\text{CM}=\text{cost}
\]
Il centro di massa si muove di moto rettilineo uniforme, di cui la quiete è un caso particolare. Questa relazione è interessante anche quando il sistema non è isolato nel suo complesso ma in una sola direzione. Per capire ciò, si immagini di avere un cannone di massa $M$, con dentro una massa $m$. Inizialmente il sistema è fermo, ad un certo istante il cannone viene innescato e la palla parte con una certa velocità nota $\vec{v}_0$.
\begin{figure}[htpb]
	\centering
	

	\tikzset{every picture/.style={line width=0.75pt}} %set default line width to 0.75pt        

	\begin{tikzpicture}[x=0.75pt,y=0.75pt,yscale=-1,xscale=1]
	%uncomment if require: \path (0,300); %set diagram left start at 0, and has height of 300

	%Straight Lines [id:da8804303858640294] 
	\draw    (61,210) -- (330,210) ;
	%Shape: Circle [id:dp6243569737365466] 
	\draw   (120,200) .. controls (120,194.48) and (124.48,190) .. (130,190) .. controls (135.52,190) and (140,194.48) .. (140,200) .. controls (140,205.52) and (135.52,210) .. (130,210) .. controls (124.48,210) and (120,205.52) .. (120,200) -- cycle ;
	%Shape: Circle [id:dp5679046904606062] 
	\draw   (180,200) .. controls (180,194.48) and (184.48,190) .. (190,190) .. controls (195.52,190) and (200,194.48) .. (200,200) .. controls (200,205.52) and (195.52,210) .. (190,210) .. controls (184.48,210) and (180,205.52) .. (180,200) -- cycle ;
	%Shape: Rectangle [id:dp6403059717351121] 
	\draw   (110,170) -- (210,170) -- (210,190) -- (110,190) -- cycle ;
	%Straight Lines [id:da5020169261481033] 
	\draw    (130,170) -- (130,120) -- (240,70) -- (260,90) -- (190,150) -- (190,170) ;
	%Straight Lines [id:da3957252727046501] 
	\draw    (155,109) -- (190,150) ;
	%Shape: Circle [id:dp32993320697849726] 
	\draw  [fill={rgb, 255:red, 0; green, 0; blue, 0 }  ,fill opacity=1 ] (225,90) .. controls (225,84.48) and (229.48,80) .. (235,80) .. controls (240.52,80) and (245,84.48) .. (245,90) .. controls (245,95.52) and (240.52,100) .. (235,100) .. controls (229.48,100) and (225,95.52) .. (225,90) -- cycle ;
	%Shape: Spring [id:dp6487389658408163] 
	\draw   (176.89,133.92) .. controls (175,130.53) and (174.77,125.82) .. (179.08,122.4) .. controls (187.7,115.56) and (197.22,127.56) .. (192.52,131.29) .. controls (187.82,135.02) and (178.3,123.02) .. (186.91,116.19) .. controls (195.53,109.35) and (205.05,121.35) .. (200.35,125.08) .. controls (195.65,128.81) and (186.13,116.81) .. (194.75,109.97) .. controls (203.36,103.13) and (212.88,115.13) .. (208.18,118.86) .. controls (203.48,122.59) and (193.96,110.59) .. (202.58,103.75) .. controls (211.2,96.92) and (220.72,108.92) .. (216.02,112.65) .. controls (211.32,116.38) and (201.8,104.38) .. (210.41,97.54) .. controls (219.03,90.7) and (228.55,102.7) .. (223.85,106.43) .. controls (219.15,110.16) and (209.63,98.16) .. (218.25,91.32) .. controls (226.86,84.49) and (236.39,96.49) .. (231.69,100.22) .. controls (226.99,103.95) and (217.46,91.95) .. (226.08,85.11) .. controls (227.27,84.16) and (228.48,83.58) .. (229.67,83.28) ;
	%Shape: Boxed Line [id:dp10601723496206072] 
	\draw    (235,90) -- (282.59,49.37) ;
	\draw [shift={(284.88,47.42)}, rotate = 499.51] [fill={rgb, 255:red, 0; green, 0; blue, 0 }  ][line width=0.08]  [draw opacity=0] (10.72,-5.15) -- (0,0) -- (10.72,5.15) -- (7.12,0) -- cycle    ;
	%Straight Lines [id:da13288739314820308] 
	\draw    (235,90) -- (313.75,90) ;
	%Straight Lines [id:da24008536549031012] 
	\draw    (73.75,164) -- (73.75,112) ;
	\draw [shift={(73.75,167)}, rotate = 270] [fill={rgb, 255:red, 0; green, 0; blue, 0 }  ][line width=0.08]  [draw opacity=0] (10.72,-5.15) -- (0,0) -- (10.72,5.15) -- (7.12,0) -- cycle    ;
	%Shape: Arc [id:dp08315862136385044] 
	\draw  [draw opacity=0] (257.48,70.13) .. controls (262.06,75.31) and (264.88,82.08) .. (265,89.5) -- (235,90) -- cycle ; \draw   (257.48,70.13) .. controls (262.06,75.31) and (264.88,82.08) .. (265,89.5) ;

	% Text Node
	\draw (85.5,128) node    {$\vec{g}$};
	% Text Node
	\draw (299,46) node    {$\vec{v}_{0}$};
	% Text Node
	\draw (221,62) node    {$m$};
	% Text Node
	\draw (275,78.25) node    {$\vartheta $};
	% Text Node
	\draw (148,152) node    {$M$};


	\end{tikzpicture}
\end{figure}
\FloatBarrier
Si sa che il piano su cui è appoggiato il cannone è perfettamente liscio. Quando si spara la palla il cannone rincula indietro. Il sistema infatti non è isolato ma vi è la forza peso, per cui agisce una forza esterna in direzione verticale. In direzione parallela al piano però non c'è alcuna forza esterna. Questo vuole dire che allora il sistema è isolato in una certa direzione, in questo caso la $x$. Allora si conserva la componente lungo essa della velocità del centro di massa. La sua velocità inizialmente era zero. Quando il cannone spara, per il principio di azione e reazione, il cannone viene spinto indietro. Se la velocità iniziale è zero, lo sarà anche quando la palla di cannone è partita.
\[
	\frac{mv_0 \cos\vartheta+ Mv*}{m+M} =0
\]
Si ottiene:
\[
	v*=-\frac{m}{M}v_0\cos\vartheta
\]
Il cannone va indietro con una velocità molto piccola a causa della sua massa elevata.
\[
	\norma{\vec{v}_c}=\frac{mv_0\cos\vartheta}{M}
\]
In direzione $y$, le forze, entrambe esterne, sono la reazione normale e la forza peso. La responsabile dell'impulso è la reazione normale. Essa inizialmente serve solo a bilanciare il peso della palla di cannone, ma nel momento in cui la palla viene sparata, riceve una variazione molto intesa del suo valore, che quindi aumenta istantaneamente.

La variazione di quantità di moto in direzione $y$ sarà:
\[
	\Delta p_{\text{tot},y}=mv_0\sin\vartheta-0
\]
Questo effetto lo da l'impulso della reazione normale nell'intervallo di tempo $\Delta t$ dello sparo. Si può calcolare il valore medio della reazione normale:
\[
	\Delta p_{\text{tot},y}=mv_0\sin\vartheta=\int_0^{\Delta t} R_n \,dt=\bar{R}_n\Delta t
\]
Si ha ora un corpo rigido, somma di tanti punti materiali. Questo può ruotare in molti modi intorno al suo centro di massa senza che esso si sposti. In tal caso il teorema sul centro di massa non dice nulla. Bisogna aggiungere ancora qualche informazione per capire come descrivere complessivamente il moto di un sistema di questo tipo.
Per capire come un corpo ruota attorno a un centro di massa, si deve richiamare il teorema del momento angolare, il quale afferma che se un punto materiale è soggetto a una risultante delle forze che genera un momento rispetto ad un certo polo $O$, l'azione dinamica a cui dà luogo il moto delle forze, è quello di far variare nel tempo il momento angolare di esso.

\begin{figure}[htpb]
	\centering
	

	\tikzset{every picture/.style={line width=0.75pt}} %set default line width to 0.75pt        

	\begin{tikzpicture}[x=0.75pt,y=0.75pt,yscale=-1,xscale=1]
	%uncomment if require: \path (0,300); %set diagram left start at 0, and has height of 300

	%Straight Lines [id:da9001806911610699] 
	\draw  [dash pattern={on 0.84pt off 2.51pt}]  (98.5,185) -- (316.5,91) ;
	%Straight Lines [id:da13465404967438555] 
	\draw [line width=0.75]    (101.25,183.81) -- (177.35,151) ;
	\draw [shift={(98.5,185)}, rotate = 336.67] [fill={rgb, 255:red, 0; green, 0; blue, 0 }  ][line width=0.08]  [draw opacity=0] (10.72,-5.15) -- (0,0) -- (10.72,5.15) -- (7.12,0) -- cycle    ;
	%Straight Lines [id:da13465404967438555] 
	\draw [line width=1.5]    (237.65,125) -- (312.83,92.58) ;
	\draw [shift={(316.5,91)}, rotate = 516.6700000000001] [fill={rgb, 255:red, 0; green, 0; blue, 0 }  ][line width=0.08]  [draw opacity=0] (13.4,-6.43) -- (0,0) -- (13.4,6.44) -- (8.9,0) -- cycle    ;
	%Straight Lines [id:da3937309759048766] 
	\draw [line width=0.75]    (237.65,125) -- (313.75,92.19) ;
	\draw [shift={(316.5,91)}, rotate = 516.6700000000001] [fill={rgb, 255:red, 0; green, 0; blue, 0 }  ][line width=0.08]  [draw opacity=0] (10.72,-5.15) -- (0,0) -- (10.72,5.15) -- (7.12,0) -- cycle    ;
	%Shape: Circle [id:dp12362703310493028] 
	\draw  [fill={rgb, 255:red, 0; green, 0; blue, 0 }  ,fill opacity=1 ] (173.95,151) .. controls (173.95,149.12) and (175.47,147.6) .. (177.35,147.6) .. controls (179.23,147.6) and (180.75,149.12) .. (180.75,151) .. controls (180.75,152.88) and (179.23,154.4) .. (177.35,154.4) .. controls (175.47,154.4) and (173.95,152.88) .. (173.95,151) -- cycle ;
	%Shape: Circle [id:dp3472803664189661] 
	\draw  [fill={rgb, 255:red, 0; green, 0; blue, 0 }  ,fill opacity=1 ] (234.25,125) .. controls (234.25,123.12) and (235.77,121.6) .. (237.65,121.6) .. controls (239.53,121.6) and (241.05,123.12) .. (241.05,125) .. controls (241.05,126.88) and (239.53,128.4) .. (237.65,128.4) .. controls (235.77,128.4) and (234.25,126.88) .. (234.25,125) -- cycle ;
	%Shape: Circle [id:dp32662397763234785] 
	\draw  [fill={rgb, 255:red, 0; green, 0; blue, 0 }  ,fill opacity=1 ] (267.25,178) .. controls (267.25,176.12) and (268.77,174.6) .. (270.65,174.6) .. controls (272.53,174.6) and (274.05,176.12) .. (274.05,178) .. controls (274.05,179.88) and (272.53,181.4) .. (270.65,181.4) .. controls (268.77,181.4) and (267.25,179.88) .. (267.25,178) -- cycle ;
	%Straight Lines [id:da43912600871808216] 
	\draw [line width=0.75]    (177.35,204.67) -- (177.35,151) ;
	\draw [shift={(177.35,207.67)}, rotate = 270] [fill={rgb, 255:red, 0; green, 0; blue, 0 }  ][line width=0.08]  [draw opacity=0] (10.72,-5.15) -- (0,0) -- (10.72,5.15) -- (7.12,0) -- cycle    ;

	% Text Node
	\draw (129.33,146) node    {$\vec{F}_{int}$};
	% Text Node
	\draw (197.33,178) node    {$\vec{F}_{est}$};
	% Text Node
	\draw (174,132) node    {$m_{2}$};
	% Text Node
	\draw (231.33,107.33) node    {$m_{1}$};
	% Text Node
	\draw (287.33,176.67) node    {$O$};


	\end{tikzpicture}
\end{figure}
\FloatBarrier
Si applica tale teorema a un sistema di punti materiali. Si mette in evidenza per ognuna delle masse un certo polo $O$ e si calcola per ogni forza il momento generato. Le forze si distinguono sempre in interne ed esterne.
\[
	\vec{M}_1^{\text{int}, (o)}+\vec{M}_1^{\text{ext}, (o)}=\frac{d\vec{L_1}}{dt}+\vec{v}_0\times m_1\vec{v}_1
\]
Si va a riscrivere la relazione tutte le masse fino alla massa $n$-esima.
\[
	\vec{M}_n^{\text{int}, (o)}+\vec{M}_n^{\text{ext}, (o)}=\frac{d\vec{L_n}}{dt}+\vec{v}_0\times m_n\vec{v}_n
\]
Si va a sommare membro a membro, ottenendo:
\[
	\sum_{i=1}^n \vec{M}_i^{\text{int}, (o)}+\sum_{i=1}^n \vec{M}_i^{\text{ext}, (o)}=\sum_{i=1}^n \frac{d\vec{L_i}}{dt}+\sum_{i=1}^n \vec{v}_0\times m_i\vec{v}_i
\]
C'è un termine comune a tutti i prodotti che non dipende dal pedice $i$, $\vec{v_0}$, che quindi si può portare fuori dalla sommatoria:
\begin{equation}
	\label{ciao}
	\sum_{i=1}^n(\vec{v}_0\times m_i\vec{v}_i)=\vec{v}_0 \times \sum_{i=1}^nm_i\vec{v}_i=\vec{v}_0\times (M_\text{tot}\vec{v}_\text{CM})
\end{equation}
Ricordando che $\frac{\sum_{i=1}^nd\vec{L}_i^{(o)}}{dt}=\frac{d\sum_{i=1}^n \vec{L}_i^{(o)}}{dt}$, si va a definire una nuova grandezza che è la somma di tutti i momenti angolari: $\vec{L}_\text{tot}^{(o)}$
Il termine $\sum_{i=1}^n \vec{M}^\text{est}$ lo si chiama \textbf{risultante dei momenti delle forze esterne} ed è un termine che in generale non si elide. Per quanto riguarda il momento delle forze interne, quando si attua il prodotto vettoriale per i vari momenti bisogna sempre considerare il braccio della forza, non $\vec{r}$. Si vede che le coppie di forze hanno diversa distanza da $O$ ma hanno lo stesso braccio. La forza $\vec{F}_1$ genera sulla massa $m_1$ un momento diretto come in figura.
\begin{figure}[htpb]
	\centering
	

	\tikzset{every picture/.style={line width=0.75pt}} %set default line width to 0.75pt        

	\begin{tikzpicture}[x=0.75pt,y=0.75pt,yscale=-1,xscale=1]
	%uncomment if require: \path (0,300); %set diagram left start at 0, and has height of 300

	%Straight Lines [id:da7593915250022287] 
	\draw  [dash pattern={on 0.84pt off 2.51pt}]  (118.5,205) -- (336.5,111) ;
	%Straight Lines [id:da330620271995024] 
	\draw [line width=0.75]    (121.25,203.81) -- (197.35,171) ;
	\draw [shift={(118.5,205)}, rotate = 336.67] [fill={rgb, 255:red, 0; green, 0; blue, 0 }  ][line width=0.08]  [draw opacity=0] (10.72,-5.15) -- (0,0) -- (10.72,5.15) -- (7.12,0) -- cycle    ;
	%Straight Lines [id:da6426180835226845] 
	\draw [line width=0.75]    (257.65,145) -- (333.75,112.19) ;
	\draw [shift={(336.5,111)}, rotate = 516.6700000000001] [fill={rgb, 255:red, 0; green, 0; blue, 0 }  ][line width=0.08]  [draw opacity=0] (10.72,-5.15) -- (0,0) -- (10.72,5.15) -- (7.12,0) -- cycle    ;
	%Shape: Circle [id:dp1763812532155713] 
	\draw  [fill={rgb, 255:red, 0; green, 0; blue, 0 }  ,fill opacity=1 ] (193.95,171) .. controls (193.95,169.12) and (195.47,167.6) .. (197.35,167.6) .. controls (199.23,167.6) and (200.75,169.12) .. (200.75,171) .. controls (200.75,172.88) and (199.23,174.4) .. (197.35,174.4) .. controls (195.47,174.4) and (193.95,172.88) .. (193.95,171) -- cycle ;
	%Shape: Circle [id:dp8887216402681932] 
	\draw  [fill={rgb, 255:red, 0; green, 0; blue, 0 }  ,fill opacity=1 ] (254.25,145) .. controls (254.25,143.12) and (255.77,141.6) .. (257.65,141.6) .. controls (259.53,141.6) and (261.05,143.12) .. (261.05,145) .. controls (261.05,146.88) and (259.53,148.4) .. (257.65,148.4) .. controls (255.77,148.4) and (254.25,146.88) .. (254.25,145) -- cycle ;
	%Shape: Circle [id:dp7130126320472034] 
	\draw  [fill={rgb, 255:red, 0; green, 0; blue, 0 }  ,fill opacity=1 ] (194.58,242.67) .. controls (194.58,240.79) and (196.11,239.27) .. (197.98,239.27) .. controls (199.86,239.27) and (201.38,240.79) .. (201.38,242.67) .. controls (201.38,244.54) and (199.86,246.07) .. (197.98,246.07) .. controls (196.11,246.07) and (194.58,244.54) .. (194.58,242.67) -- cycle ;
	%Straight Lines [id:da4643521475870709] 
	\draw [line width=0.75]    (197.98,242.67) -- (197.38,177.4) ;
	\draw [shift={(197.35,174.4)}, rotate = 449.47] [fill={rgb, 255:red, 0; green, 0; blue, 0 }  ][line width=0.08]  [draw opacity=0] (10.72,-5.15) -- (0,0) -- (10.72,5.15) -- (7.12,0) -- cycle    ;
	%Straight Lines [id:da7200382141356945] 
	\draw  [dash pattern={on 0.84pt off 2.51pt}]  (140.5,267.33) -- (358.5,173.33) ;
	%Straight Lines [id:da5773842074758675] 
	\draw [line width=0.75]    (197.98,242.67) -- (256.04,150.93) ;
	\draw [shift={(257.65,148.4)}, rotate = 482.33] [fill={rgb, 255:red, 0; green, 0; blue, 0 }  ][line width=0.08]  [draw opacity=0] (10.72,-5.15) -- (0,0) -- (10.72,5.15) -- (7.12,0) -- cycle    ;
	%Shape: Boxed Line [id:dp6958853312388102] 
	\draw  [dash pattern={on 0.84pt off 2.51pt}]  (257.65,145) -- (284.1,206.33) ;
	%Shape: Boxed Line [id:dp6136755725042817] 
	\draw  [dash pattern={on 0.84pt off 2.51pt}]  (197.35,171) -- (223.8,232.33) ;

	% Text Node
	\draw (139.33,166) node    {$\vec{F}_{1,int}$};
	% Text Node
	\draw (182,210.27) node    {$\vec{r}_{1}$};
	% Text Node
	\draw (254,126) node    {$m_{2}$};
	% Text Node
	\draw (190.67,152.67) node    {$m_{1}$};
	% Text Node
	\draw (198,264) node    {$O$};
	% Text Node
	\draw (253.6,178.4) node    {$\vec{r}_{2}$};
	% Text Node
	\draw (282.4,166.4) node    {$b_{2}$};
	% Text Node
	\draw (218,183.33) node    {$b_{1}$};


	\end{tikzpicture}
\end{figure}
\FloatBarrier
L'altra genera un momento in verso orario. Le due forze danno luogo a momenti uguali in modulo e opposti in verso. Per ogni coppia di forze, i momenti delle forze interne sono sempre uguali a zero.

Il termine ~\eqref{ciao} è zero se come polo si sceglie il centro di massa, oppure se c'è un polo che si muove com velocità parallela al centro di massa o se esso è fisso nel sistema di riferimento inerziale. Questa equazione
\[
	\boxed{\vec{M}_\text{tot}^\text{ext,(o)}=\frac{d\vec{L}_\text{tot}^{(o)}}{dt}}
\]
è nota come \textbf{seconda equazione cardinale della dinamica}. Le forze esterne sono applicate a punti diversi, quindi calcolandone il momento si hanno bracci vettori diversi e non si può dire che il momento della risultante è uguale alla risultante dei momenti.

Per capire l'affermazione:
\[
	\vec{R}_\text{ext}=0 \notimplies \vec{M}_\text{tot}^\text{ext,(o)}=0
\]
si pensi alla ruota di una bicicletta.
\begin{figure}[htpb]
	\centering
	

	\tikzset{every picture/.style={line width=0.75pt}} %set default line width to 0.75pt        

	\begin{tikzpicture}[x=0.75pt,y=0.75pt,yscale=-1,xscale=1]
	%uncomment if require: \path (0,300); %set diagram left start at 0, and has height of 300

	%Shape: Circle [id:dp4404111454676727] 
	\draw   (134,158.75) .. controls (134,120.23) and (165.23,89) .. (203.75,89) .. controls (242.27,89) and (273.5,120.23) .. (273.5,158.75) .. controls (273.5,197.27) and (242.27,228.5) .. (203.75,228.5) .. controls (165.23,228.5) and (134,197.27) .. (134,158.75) -- cycle ;
	%Straight Lines [id:da9995352945073703] 
	\draw    (134,158.75) -- (273.5,158.75) ;
	%Straight Lines [id:da3033281097064129] 
	\draw    (203.75,89) -- (203.75,228.5) ;
	%Straight Lines [id:da2400256208242031] 
	\draw    (154.43,109.43) -- (253.07,208.07) ;
	%Straight Lines [id:da6171032237559488] 
	\draw    (253.07,109.43) -- (154.43,208.07) ;
	%Straight Lines [id:da9767183926970966] 
	\draw    (203.75,228.5) -- (270.25,228.5) ;
	\draw [shift={(273.25,228.5)}, rotate = 180] [fill={rgb, 255:red, 0; green, 0; blue, 0 }  ][line width=0.08]  [draw opacity=0] (10.72,-5.15) -- (0,0) -- (10.72,5.15) -- (7.12,0) -- cycle    ;
	%Straight Lines [id:da2493252744323593] 
	\draw    (137.25,89) -- (203.75,89) ;
	\draw [shift={(134.25,89)}, rotate = 0] [fill={rgb, 255:red, 0; green, 0; blue, 0 }  ][line width=0.08]  [draw opacity=0] (10.72,-5.15) -- (0,0) -- (10.72,5.15) -- (7.12,0) -- cycle    ;
	%Curve Lines [id:da3859918903211981] 
	\draw    (282,184.67) .. controls (290.82,169.55) and (288.33,151.45) .. (284.09,136) ;
	\draw [shift={(283.33,133.33)}, rotate = 433.74] [fill={rgb, 255:red, 0; green, 0; blue, 0 }  ][line width=0.08]  [draw opacity=0] (10.72,-5.15) -- (0,0) -- (10.72,5.15) -- (7.12,0) -- cycle    ;
	%Shape: Boxed Bezier Curve [id:dp7410834647628459] 
	\draw    (127.94,132.67) .. controls (119.12,147.79) and (121.61,165.88) .. (125.85,181.33) ;
	\draw [shift={(126.61,184)}, rotate = 253.74] [fill={rgb, 255:red, 0; green, 0; blue, 0 }  ][line width=0.08]  [draw opacity=0] (10.72,-5.15) -- (0,0) -- (10.72,5.15) -- (7.12,0) -- cycle    ;
	%Shape: Circle [id:dp20185600318945807] 
	\draw  [fill={rgb, 255:red, 0; green, 0; blue, 0 }  ,fill opacity=1 ] (316,108.75) .. controls (316,107.23) and (317.23,106) .. (318.75,106) .. controls (320.27,106) and (321.5,107.23) .. (321.5,108.75) .. controls (321.5,110.27) and (320.27,111.5) .. (318.75,111.5) .. controls (317.23,111.5) and (316,110.27) .. (316,108.75) -- cycle ;

	% Text Node
	\draw (246,146) node    {$r$};
	% Text Node
	\draw (289,227) node    {$\vec{F}_{1}$};
	% Text Node
	\draw (118.33,84.33) node    {$\vec{F}_{2}$};
	% Text Node
	\draw (337.33,106.33) node    {$\vec{u}_{z}$};


	\end{tikzpicture}
\end{figure}
\FloatBarrier
Il sistema è costituito da tutte le masse che vanno ad appoggiarsi su di essa.  Si applicano dall'esterno due forze uguali e contrarie in modo che la bicicletta si metta a ruotare. Questo è un esempio in cui la risultante delle forze esterne è zero, ma i momenti delle forze si sommano e si ha:
\[
	\begin{cases} \vec{M}_1=Fr\vec{u}_z \\ \vec{M}_2=Fr\vec{u}_z \end{cases} \implies \vec{M}_\text{tot}=2Fr\vec{u}_z
\]
Il momento totale è diverso da zero perché le forze non giacciono sulla stessa retta di applicazione e hanno quindi la caratteristica di non far traslare il sistema ma di farlo ruotare.

Si capisce quindi come la prima equazione cardinale della dinamica permetta di studiare il moto del centro di massa, mentre la seconda parla del moto rotativo di tutti gli altri punti rispetto ad esso. Le due equazioni sono indipendenti.

Se sul sistema agiscono delle forze che non danno luogo a un momento rispetto al polo $O$, si ha anche un questo caso un principio di conservazione.

\paragraph{Esempio} Si consideri la bacchetta di una majorette che ruota con una certa velocità angolare $\vec{\omega}_0$.
\begin{figure}[htpb]
	\centering
	

	\tikzset{every picture/.style={line width=0.75pt}} %set default line width to 0.75pt        

	\begin{tikzpicture}[x=0.75pt,y=0.75pt,yscale=-1,xscale=1]
	%uncomment if require: \path (0,300); %set diagram left start at 0, and has height of 300

	%Shape: Rectangle [id:dp3822716726396491] 
	\draw   (134.5,130) -- (317,130) -- (317,138) -- (134.5,138) -- cycle ;
	%Shape: Circle [id:dp8453822070451111] 
	\draw  [fill={rgb, 255:red, 0; green, 0; blue, 0 }  ,fill opacity=1 ] (223.79,134) .. controls (223.79,132.92) and (224.67,132.04) .. (225.75,132.04) .. controls (226.83,132.04) and (227.71,132.92) .. (227.71,134) .. controls (227.71,135.08) and (226.83,135.96) .. (225.75,135.96) .. controls (224.67,135.96) and (223.79,135.08) .. (223.79,134) -- cycle ;

	% Text Node
	\draw    (120, 133) circle [x radius= 14.53, y radius= 14.53]   ;
	\draw (120,133) node    {$m$};
	% Text Node
	\draw    (331, 133) circle [x radius= 14.53, y radius= 14.53]   ;
	\draw (331,133) node    {$m$};
	% Text Node
	\draw (226.29,117.43) node    {$O$};


	\end{tikzpicture}
\end{figure}
\FloatBarrier
Poi la bacchetta, che era inizialmente lunga $2d$, per un meccanismo interno che fa scattare una molla, si allunga del doppio. La domanda che ci si pone (trascurando tutti gli attriti) è quale sia la velocità angolare di rotazione dopo lo scatto del meccanismo interno che ha fatto allungare l'asta. Nel sistema, le forze esterne sono pari a $0$, perché forza peso e reazione normale si bilanciano. Si può affermare che in questo caso i momenti delle forze esterne sono uguali a zero e quindi il momento angolare totale del sistema sarà costante. Si deve conservare il momento angolare, non la velocità angolare.
\begin{gather*}
	\vec{R}_\text{ext}=\frac{d\vec{p}_\text{tot}}{dt} =M_\text{tot}\vec{a}_\text{CM} \\
	\vec{M}_\text{tot}^\text{ext, (o)}=\frac{d\vec{L}_\text{tot}}{dt} \quad \vec{M}_\text{tot}^\text{ext}=0 \implies \vec{L}_\text{tot}^{(o)}=\text{cost}
\end{gather*}
Spostando più massa lontano dal centro di rotazione,  gli oggetti hanno velocità lineare maggiore a parità di $\omega$. Ma $v$ non può variare quindi deve diminuire $\omega$, per permette al momento angolare di conservarsi.
\begin{gather*}
	\norma{v}=\omega_0 D \quad \vec{L}_\text{tot, in}=\omega_0 D^2m\vec{u}_z+\omega_0 D^2m\vec{u}_z=2\omega_0 D^2m\vec{u}_z \\
	\vec{L}_\text{tot, fin}=m4D^2\omega_\text{fin}\vec{u}_z+m4D^2\omega_\text{fin}\vec{u}_z=8mD^2\omega_\text{fin}\vec{u}_z \\
	2\omega_0 D^2m\vec{u}_z= 8mD^2\omega_\text{fin}\vec{u}_z \implies \omega_\text{fin}=\frac{\omega_0}{4}
\end{gather*}







































\section{Teorema dell'energia cinetica}

Si vuole trovare una relazione complessiva che dà l'effetto di tutte le forze sulla variazione di energia cinetica totale del sistema. Si va allora a definire un'altra grandezza complessiva, l'\textbf{energia cinetica totale}:
\[
	E_\text{k, tot}=\sum_{i=1}^nE_{k, i}=\Delta E_\text{k, i}
\]
\begin{equation}
	\label{A}
	\mathcal{L}=\Delta E_k
\end{equation}
Scriviamo la relazione ~\eqref{A} per la generica massa $i-$esima distinguendo forze esterne e interne:
\[
	\mathcal{L}_i^\text{est}+\mathcal{L}_i^\text{int}=\Delta E_{k, i}
\]
Sommando membro a membro si ottiene:
\[
	\boxed{\sum_{i=1}^n \mathcal{L}_i^\text{int}+\sum_{i=1}^n\mathcal{L}_i^\text{est}=\sum_{i=1}^n \Delta E_{k, i}= \Delta E_\text{k, tot}  \quad \text{Teorema dell'energia cinetica}}
\]
La somma dei lavori di tutte le forze esterne daranno luogo a un lavoro che in generale non è nullo e si può andare a calcolare. È evidente che se le traiettorie su cui si muovono le singole masse sono diverse, il termine non va a zero. Questo è il \textbf{teorema dell'energia cinetica} per sistemi di punti materiali, il quale afferma che l'energia cinetica può variare il suo lavoro sia a causa delle forze esterne che a causa delle forze interne.

Si può definire l'\textbf{energia meccanica totale} come somma dell'energia meccanica di ogni punto del sistema, essa varia secondo la relazione:
\[
	\Delta E_\text{mecc}=\mathcal{L}^\text{n.c.}_\text{int}+\mathcal{L}^\text{n.c.}_\text{est}
\]

\paragraph{Esempio} Si immagini di avere un cuneo di massa $M$ su cui è possibile appoggiare un oggetto che può scivolarci sopra. Il piano d'appoggio è perfettamente liscio.
\begin{figure}[htpb]
	\centering
	

	\tikzset{every picture/.style={line width=0.75pt}} %set default line width to 0.75pt        

	\begin{tikzpicture}[x=0.75pt,y=0.75pt,yscale=-1,xscale=1]
	%uncomment if require: \path (0,300); %set diagram left start at 0, and has height of 300

	%Curve Lines [id:da08474185160861603] 
	\draw    (99,86) .. controls (156.5,256) and (234.5,255) .. (289,86) ;
	%Straight Lines [id:da3600402848302091] 
	\draw    (99,86) -- (99,255) ;
	%Straight Lines [id:da9890701156002613] 
	\draw    (289,86) -- (289,255) ;
	%Straight Lines [id:da14088227593390834] 
	\draw    (99,255) -- (289,255) ;
	%Straight Lines [id:da20165149083647327] 
	\draw    (196,237) -- (196,190) ;
	\draw [shift={(196,187)}, rotate = 450] [fill={rgb, 255:red, 0; green, 0; blue, 0 }  ][line width=0.08]  [draw opacity=0] (10.72,-5.15) -- (0,0) -- (10.72,5.15) -- (7.12,0) -- cycle    ;
	%Straight Lines [id:da3591981245258391] 
	\draw    (173.67,229) -- (173.67,275) ;
	\draw [shift={(173.67,278)}, rotate = 270] [fill={rgb, 255:red, 0; green, 0; blue, 0 }  ][line width=0.08]  [draw opacity=0] (10.72,-5.15) -- (0,0) -- (10.72,5.15) -- (7.12,0) -- cycle    ;
	%Straight Lines [id:da5542442706390196] 
	\draw    (126.5,241.67) -- (66,241.67) ;
	\draw [shift={(63,241.67)}, rotate = 360] [fill={rgb, 255:red, 0; green, 0; blue, 0 }  ][line width=0.08]  [draw opacity=0] (10.72,-5.15) -- (0,0) -- (10.72,5.15) -- (7.12,0) -- cycle    ;
	%Shape: Rectangle [id:dp24134016790920998] 
	\draw  [fill={rgb, 255:red, 222; green, 222; blue, 222 }  ,fill opacity=1 ] (134.72,112.43) -- (153.55,150.52) -- (131.78,161.29) -- (112.95,123.19) -- cycle ;
	%Straight Lines [id:da1394008418051096] 
	\draw    (132.25,137.86) -- (173.6,115.91) ;
	\draw [shift={(176.25,114.5)}, rotate = 512.04] [fill={rgb, 255:red, 0; green, 0; blue, 0 }  ][line width=0.08]  [draw opacity=0] (10.72,-5.15) -- (0,0) -- (10.72,5.15) -- (7.12,0) -- cycle    ;
	%Straight Lines [id:da12135194002047722] 
	\draw    (90.9,159.81) -- (132.25,137.86) ;
	\draw [shift={(88.25,161.21)}, rotate = 332.04] [fill={rgb, 255:red, 0; green, 0; blue, 0 }  ][line width=0.08]  [draw opacity=0] (10.72,-5.15) -- (0,0) -- (10.72,5.15) -- (7.12,0) -- cycle    ;
	%Straight Lines [id:da11104915980985175] 
	\draw    (132.25,184.86) -- (132.25,137.86) ;
	\draw [shift={(132.25,187.86)}, rotate = 270] [fill={rgb, 255:red, 0; green, 0; blue, 0 }  ][line width=0.08]  [draw opacity=0] (10.72,-5.15) -- (0,0) -- (10.72,5.15) -- (7.12,0) -- cycle    ;
	%Shape: Circle [id:dp285260685691167] 
	\draw  [fill={rgb, 255:red, 0; green, 0; blue, 0 }  ,fill opacity=1 ] (130.25,137.86) .. controls (130.25,136.75) and (131.15,135.86) .. (132.25,135.86) .. controls (133.35,135.86) and (134.25,136.75) .. (134.25,137.86) .. controls (134.25,138.96) and (133.35,139.86) .. (132.25,139.86) .. controls (131.15,139.86) and (130.25,138.96) .. (130.25,137.86) -- cycle ;
	%Shape: Boxed Line [id:dp8945452096934903] 
	\draw    (132.25,137.86) -- (154.2,179.21) ;
	\draw [shift={(155.61,181.86)}, rotate = 242.04] [fill={rgb, 255:red, 0; green, 0; blue, 0 }  ][line width=0.08]  [draw opacity=0] (10.72,-5.15) -- (0,0) -- (10.72,5.15) -- (7.12,0) -- cycle    ;
	%Straight Lines [id:da9166808228451746] 
	\draw    (110.3,96.51) -- (132.25,137.86) ;
	\draw [shift={(108.89,93.86)}, rotate = 62.04] [fill={rgb, 255:red, 0; green, 0; blue, 0 }  ][line width=0.08]  [draw opacity=0] (10.72,-5.15) -- (0,0) -- (10.72,5.15) -- (7.12,0) -- cycle    ;

	% Text Node
	\draw (213,230.67) node    {$\vec{R}_{n}$};
	% Text Node
	\draw (191.5,106) node    {$\vec{R}_{n}$};
	% Text Node
	\draw (68.5,157.5) node    {$-\vec{R}_{n}$};
	% Text Node
	\draw (129.5,201) node    {$m\vec{g}$};
	% Text Node
	\draw (130.17,89.33) node    {$\vec{R}_{t}$};
	% Text Node
	\draw (170.83,163.33) node    {$-\vec{R}_{t}$};


	\end{tikzpicture}
\end{figure}
\FloatBarrier
Se l'oggetto parte da fermo, quello che accade è che tenderà a scivolare a causa dell'effetto della forza peso ed è naturale pensare che se non c'è attrito, esso, partendo da faremo, trasformerà la sua energia potenziale in energia cinetica che poi, ripartendo verso l'alto, si ritrasformerà in energia potenziale. L'oggetto oscillerà sul cuneo. A causa del fatto che questo si sta muovendo complessivamente verso destra, ci si aspetta che il cuneo dovrà partire verso sinistra per la conservazione della quantità di moto. Dal punto di vista delle forze esterne, agiscono il peso e la reazione normale al piano d'appoggio, tutte in direzione verticale.

Le forze totali esterne sono $0$ in direzione $x$ e questo comporta che la quantità di moto totale del sistema valutata in tale direzione dovrà essere costante. Il sistema inizialmente aveva quantità di moto iniziale zero, quindi essa si deve continuare a mantenere tale: l'ascissa del centro di massa non deve variare nel tempo. Quando la massa si sposta verso destra il cuneo, per compensare, si sposta a sinistra. Se per assurdo la massa si spostasse e arrivasse dall'altra parte senza che il cuneo si muovesse, il centro di massa si sposterebbe in basso. Da un punto di vista fisico, ciò che fa spostare il cuneo è l'effetto di una forza interna. La reazione normale sull'oggetto è generata dal cuneo; allora la massa $m$ genererà su di esso una forza uguale e contraria.

Si immagini di complicare la situazione avendo la forza d'attrito, interna, che rallenta il movimento. Il suo effetto è nullo nel valutare la prima equazione cardinale della dinamica. Ma dal punto di vista energetico queste forze interne hanno un effetto. L'attrito va a dissipare l'energia potenziale, così che l'oggetto arriverà al lato opposto del cuneo a quote via via inferiori. Quest'ultimo subirà una forza $-\vec{R}_n$ e $-\vec{R}_t$. L'oggetto scivolerà verso il basso. Ma è come se si muovesse su un sistema di riferimento relativo che trasla, perché il cuneo si muove. $\vec{R}_t$, $-\vec{R}_t$ si elidono. Vale Lo stesso per $\vec{R}_n$ e $-\vec{R}_n$. Il calcolo del lavoro è il lavoro della forza d'attrito valutata su tutto lo spostamento del cuneo. Si noti infatti che il cuneo e la massa seguono traiettorie diverse quindi, benché $\vec{R}_t$ e $-\vec{R}_t$ siano uguali ed opposte, il loro lavoro è valutato su una traiettoria differente.







































\section{Urti tra due punti materiali}

\[
	\vec{R}_\text{est}=\frac{d\vec{p}_\text{tot}}{dt} \qquad \Delta E_k=\mathcal{L}_\text{int}+\mathcal{L}_\text{est} \qquad \vec{M}_\text{tot}=\frac{d\vec{L}_\text{tot}}{dt}
\]
Queste sono le tre relazioni che possono aiutare nello studio del moto di un sistema di punti. Un esempio tipico di applicazione di ciò è lo studio del fenomeno di urto fra punti.

Quando due punti materiali vengono a contatto e interagiscono per un intervallo di tempo trascurabile rispetto a quello di osservazione del sistema, si parla di \textbf{urto} tra due punti. È evidente che quando due corpi si scontrano, entrano in gioco forze interne molto elevate che passano dall'essere nulle prima del fenomeno di urto, all'essere di intensità molto elevate durante tale fenomeno. Studiandone l'andamento nell'intervallo di tempo, si può vedere come esse abbiano un carattere fortemente impulsivo. Le forze esterne invece si mantengono per lo più costanti durante il fenomeno di urto e se variano variano molto poco il proprio valore. Inoltre esse sono in generale molto meno intense rispetto a quelle interne.
\begin{figure}[htpb]
	\centering
	

	\tikzset{every picture/.style={line width=0.75pt}} %set default line width to 0.75pt        

	\begin{tikzpicture}[x=0.75pt,y=0.75pt,yscale=-1,xscale=1]
	%uncomment if require: \path (0,300); %set diagram left start at 0, and has height of 300

	%Shape: Axis 2D [id:dp9573824779450613] 
	\draw  (50.67,220) -- (318.17,220)(66.17,53) -- (66.17,235) (311.17,215) -- (318.17,220) -- (311.17,225) (61.17,60) -- (66.17,53) -- (71.17,60)  ;
	%Curve Lines [id:da6395267149757182] 
	\draw [line width=1.5]    (110.5,220) .. controls (140.25,190.5) and (140.67,86.33) .. (184.67,86.33) .. controls (228.67,86.33) and (230.75,190) .. (259.5,220) ;
	%Straight Lines [id:da8778937063159087] 
	\draw    (110.5,230) -- (259.5,230) ;
	\draw [shift={(259.5,230)}, rotate = 180] [color={rgb, 255:red, 0; green, 0; blue, 0 }  ][line width=0.75]    (0,5.59) -- (0,-5.59)   ;
	\draw [shift={(110.5,230)}, rotate = 180] [color={rgb, 255:red, 0; green, 0; blue, 0 }  ][line width=0.75]    (0,5.59) -- (0,-5.59)   ;
	%Straight Lines [id:da4441851887710484] 
	\draw [line width=1.5]    (66.17,180) -- (285.17,180) ;

	% Text Node
	\draw (48,54.67) node    {$|\vec{F} |$};
	% Text Node
	\draw (331.33,218) node    {$t$};
	% Text Node
	\draw (185.33,242) node    {$\tau \ \text{urto}$};
	% Text Node
	\draw (304,178) node    {$F_{\text{est}}$};


	\end{tikzpicture}
\end{figure}
\FloatBarrier
Si va a riscrivere la prima equazione cardinale della dinamica in termini di teorema dell'impulso. I corpi che si urtano diventano il sistema di corpi.
\[
	\int_0^\tau \vec{R}_\text{est}\,dt=\Delta \vec{p}_\text{tot}
\]
Se l'intervallo di tempo è molto molto piccolo, tale per cui può essere approssimato a zero, il primo termine va a 0: ciò comporta che durante questi fenomeni la quantità di moto del sistema si mantiene costante. Allora, il moto del centro di massa non viene alterato dall'urto. Analogamente si può attuare la stessa osservazione per la seconda equazione cardinale della dinamica: i momenti delle forze esterne durante l'urto hanno un effetto praticamente nullo e quindi il momento angolare totale rispetto a un qualunque polo $O$ del sistema è costante durante il fenomeno.
\[
	\int_0^\tau \vec{M}_\text{est}\,dt=\Delta \vec{L}^{(o)} \implies \vec{L}_\text{tot}=\text{cost}
\]
Quando i fenomeni di urto interessano un sistema che ruota è utile questa relazione. In una situazione di urto, i punti si movono verso il centro di massa ed entrano in contatto in corrispondenza di esso.

Infine è possibile attuare delle considerazione energetiche. Ci si chiede se e come varia l'energia totale del sistema durante i fenomeni di urto. In questo caso bisogna tenere conto del fatto che le forze interne, che hanno intensità molto elevata, entrano in gioco nel far variare l'energia totale del sistema. Se ci sono forze non conservative l'energia totale varia. Le forze esterne si possono di nuovo trascurare, ciò che però è interessante notare è che l'effetto dissipativo delle forze interne fa variare l'energia cinetica del sistema. La variazione di energia potenziale è all'incirca zero. È infatti un'energia posizionale che dà un valore a seconda della posizione dei corpi, ma il fenomeno è di così breve durata da giustificare l'assunzione che durante l'interazione i punti non si muovono in modo apprezzabile.
\[
	\mathcal{L}_\text{int}^\text{NC}+\mathcal{L}_\text{est}^\text{NC}=\Delta E_\text{mecc}=\Delta E_\text{k, tot}+\underbrace{\Delta E_\text{p, tot}}_{=0}
\]
Quindi se ci sono forze interne dissipative queste vanno a variare l'energia cinetica totale del sistema.

\subsection{Urti elastici e anaelastici}

Dal punto di vista energetico, si possono distinguere gli urti in due grandi tipologie: urti \textbf{ideali} o \textbf{elastici}, in cui le forze interne sono tutte conservative, e quindi non si ha effetto di dissipazione per causa loro, e urti \textbf{anaelastici}.

\paragraph{Urti elastici} Si definisce come urto elastico un urto durante il quale si conserva anche l'energia cinetica del sistema. Questo comporta che le forze interne, che si manifestano durante l'urto, siano conservative. I due corpi reali che si urtano subiscono, durante l'urto, delle deformazioni elastiche, riprendendo la configurazione iniziale subito dopo di esso.
Il termine:
\[
	\mathcal{L}_\text{int}^\text{NC}=\Delta E_\text{k, tot}
\]
Va a zero e l'energia totale del sistema è costante.
Nello studio di un sistema elastico si possono utilizzare quindi le equazioni:
\[
	P_\text{in}=P_\text{fin} \quad E_\text{k, in}=E_\text{k, fin}
\]
Si pensi all'urto tra palle da biliardo, le forze che si scambiano durante l'urto non variano l'energia del sistema e si osserva trasferimento di energia cinetica da un corpo all'altro.

\paragraph{urti anaelastici} Si tratta di urti che avvengono in presenza di forze interne non conservative con conseguente variazione di energia cinetica durante il fenomeno. Questa seconda classe rappresenta la vera categoria degli urti, la prima è principalmente un'idealizzazione. Se le forze interne sono dissipative l'energia cinetica diminuirà, anche se ci sono casi in cui le forze interne possono far aumentare l'energia del sistema.

\paragraph{Urto completamente anaelastico} Tra gli urti anaelastici, si trova una sottoclasse di fenomeni di urto che prende il nome di \textbf{urti completamente anaelastici}: sono quei fenomeni di urto in cui i corpi che si sono urtati dopo l'evento rimangono insieme, formando un unico corpo puntiforme di massa $m_1+m_2$ che si muove con una certa velocità finale. Per far unire i corpi insieme devono aver agito forze interne che hanno dissipato l'energia cinetica. I due corpi, che si assimilano a punti materiali, durante l'urto si deformano in modo permanente e restano compenetrati. Il lavoro compiuto, a spese dell'energia cinetica iniziale, per fare avvenire la deformazione, non viene più recuperato, cosa che equivale ad affermare che le forze interne sviluppate nell'atto non sono conservative. Inoltre, dopo l'urto completamente anaelastico, non c'è più moto rispetto al centro di massa.

\subsection{Urti centrali e non centrali}

Gli urti si classificano anche in centrali e non centrali.

\paragraph{Urti centrali} Gli urti \textbf{centrali} sono quei fenomeni di urto in cui i corpi interessati viaggiano in una certa direzione e dopo il fenomeno continuano a viaggiare nella stessa direzione (anche se il verso può essere opposto). In pratica gli urti centrali sono semplici da studiare perché dal punto di vista della quantità di moto il problema vettoriale è monodimensionale. Si osserva il tutto nell'unica direzione in cui avviene l'urto.

\paragraph{Urti non centrali} Gli urti \textbf{non centrali} sono invece quelli che avvengono non nella stessa direzione. I corpi possono avere dopo di essi una direzione qualunque.

L'obbiettivo dello studio del fenomeno di urto è ricavare la velocità dei due corpi dopo di esso. In \textit{urti centrali elastici}, in cui le forze esterne non sono impulsive, si conserva la quantità di moto totale del sistema. Essendo elastico, l'energia cinetica totale del sistema è costante.
\begin{gather*}
	p_\text{tot}=\text{cost} \quad m_1 v_1+m_2 v_2=m_1 v_{1,\text{fin}}+m_2 v_{2,\text{fin}} \\
	E_\text{k, tot}=\text{cost} \quad \frac{1}{2}m_1 v_1^2+\frac{1}{2}m_2 v_2^2=\frac{1}{2}m_1 v_{1,\text{fin}}^2+\frac{1}{2}m_2 v_{2,\text{fin}}^2
\end{gather*}
Si hanno due incognite, il problema è risolvibile.
\begin{gather*}
	v_{1,\text{fin}}=\frac{m_1-m_2}{m_1+m_2} v_1+\underbrace{\frac{2m_2 v_2}{m_1+m_2}}_A \\
	v_{2,\text{fin}}=\frac{2m_1 v_1}{m_1+m_2}+\underbrace{\frac{m_2-m_1 v_2}{m_1+m_2}}_B
\end{gather*}
\begin{figure}[htpb]
	\centering
	


	\tikzset{every picture/.style={line width=0.75pt}} %set default line width to 0.75pt        

	\begin{tikzpicture}[x=0.75pt,y=0.75pt,yscale=-1,xscale=1]
	%uncomment if require: \path (0,392); %set diagram left start at 0, and has height of 392

	%Straight Lines [id:da23785241410349744] 
	\draw    (118.75,106.75) -- (163.54,99.48) ;
	\draw [shift={(166.5,99)}, rotate = 530.78] [fill={rgb, 255:red, 0; green, 0; blue, 0 }  ][line width=0.08]  [draw opacity=0] (10.72,-5.15) -- (0,0) -- (10.72,5.15) -- (7.12,0) -- cycle    ;
	%Straight Lines [id:da8050824923621605] 
	\draw    (231.75,174.75) -- (226.86,133.98) ;
	\draw [shift={(226.5,131)}, rotate = 443.16] [fill={rgb, 255:red, 0; green, 0; blue, 0 }  ][line width=0.08]  [draw opacity=0] (10.72,-5.15) -- (0,0) -- (10.72,5.15) -- (7.12,0) -- cycle    ;
	%Shape: Circle [id:dp9455071582439742] 
	\draw  [draw opacity=0][fill={rgb, 255:red, 128; green, 128; blue, 128 }  ,fill opacity=1 ] (102,106.75) .. controls (102,97.5) and (109.5,90) .. (118.75,90) .. controls (128,90) and (135.5,97.5) .. (135.5,106.75) .. controls (135.5,116) and (128,123.5) .. (118.75,123.5) .. controls (109.5,123.5) and (102,116) .. (102,106.75) -- cycle ;
	%Shape: Circle [id:dp34714784890720884] 
	\draw  [draw opacity=0][fill={rgb, 255:red, 184; green, 184; blue, 184 }  ,fill opacity=1 ] (215,174.75) .. controls (215,165.5) and (222.5,158) .. (231.75,158) .. controls (241,158) and (248.5,165.5) .. (248.5,174.75) .. controls (248.5,184) and (241,191.5) .. (231.75,191.5) .. controls (222.5,191.5) and (215,184) .. (215,174.75) -- cycle ;
	%Shape: Circle [id:dp3095792636280448] 
	\draw  [draw opacity=0][fill={rgb, 255:red, 128; green, 128; blue, 128 }  ,fill opacity=1 ] (174.75,95) .. controls (174.75,85.75) and (182.25,78.25) .. (191.5,78.25) .. controls (200.75,78.25) and (208.25,85.75) .. (208.25,95) .. controls (208.25,104.25) and (200.75,111.75) .. (191.5,111.75) .. controls (182.25,111.75) and (174.75,104.25) .. (174.75,95) -- cycle ;
	%Straight Lines [id:da9403834889901768] 
	\draw    (191.5,78.25) -- (188.71,38.99) ;
	\draw [shift={(188.5,36)}, rotate = 445.94] [fill={rgb, 255:red, 0; green, 0; blue, 0 }  ][line width=0.08]  [draw opacity=0] (10.72,-5.15) -- (0,0) -- (10.72,5.15) -- (7.12,0) -- cycle    ;
	%Straight Lines [id:da8870317581239819] 
	\draw    (230.5,102.25) -- (272.13,69.84) ;
	\draw [shift={(274.5,68)}, rotate = 502.1] [fill={rgb, 255:red, 0; green, 0; blue, 0 }  ][line width=0.08]  [draw opacity=0] (10.72,-5.15) -- (0,0) -- (10.72,5.15) -- (7.12,0) -- cycle    ;
	%Shape: Circle [id:dp9617349145835501] 
	\draw  [draw opacity=0][fill={rgb, 255:red, 184; green, 184; blue, 184 }  ,fill opacity=1 ] (205.75,110) .. controls (205.75,100.75) and (213.25,93.25) .. (222.5,93.25) .. controls (231.75,93.25) and (239.25,100.75) .. (239.25,110) .. controls (239.25,119.25) and (231.75,126.75) .. (222.5,126.75) .. controls (213.25,126.75) and (205.75,119.25) .. (205.75,110) -- cycle ;
	%Straight Lines [id:da004649770034952372] 
	\draw    (15.5,279) -- (407.5,279) ;
	%Shape: Circle [id:dp49171906743296834] 
	\draw  [draw opacity=0][fill={rgb, 255:red, 128; green, 128; blue, 128 }  ,fill opacity=1 ] (42,262.25) .. controls (42,253) and (49.5,245.5) .. (58.75,245.5) .. controls (68,245.5) and (75.5,253) .. (75.5,262.25) .. controls (75.5,271.5) and (68,279) .. (58.75,279) .. controls (49.5,279) and (42,271.5) .. (42,262.25) -- cycle ;
	%Shape: Circle [id:dp5105287976415633] 
	\draw  [draw opacity=0][fill={rgb, 255:red, 184; green, 184; blue, 184 }  ,fill opacity=1 ] (118,262.25) .. controls (118,253) and (125.5,245.5) .. (134.75,245.5) .. controls (144,245.5) and (151.5,253) .. (151.5,262.25) .. controls (151.5,271.5) and (144,279) .. (134.75,279) .. controls (125.5,279) and (118,271.5) .. (118,262.25) -- cycle ;
	%Straight Lines [id:da2603449450705637] 
	\draw    (75.5,262.25) -- (102.5,262.25) ;
	\draw [shift={(105.5,262.25)}, rotate = 180] [fill={rgb, 255:red, 0; green, 0; blue, 0 }  ][line width=0.08]  [draw opacity=0] (10.72,-5.15) -- (0,0) -- (10.72,5.15) -- (7.12,0) -- cycle    ;
	%Straight Lines [id:da6285993598469684] 
	\draw    (151.5,262.25) -- (171.5,262.25) ;
	\draw [shift={(174.5,262.25)}, rotate = 180] [fill={rgb, 255:red, 0; green, 0; blue, 0 }  ][line width=0.08]  [draw opacity=0] (10.72,-5.15) -- (0,0) -- (10.72,5.15) -- (7.12,0) -- cycle    ;
	%Shape: Circle [id:dp8061709344353809] 
	\draw  [draw opacity=0][fill={rgb, 255:red, 128; green, 128; blue, 128 }  ,fill opacity=1 ] (252,262.25) .. controls (252,253) and (259.5,245.5) .. (268.75,245.5) .. controls (278,245.5) and (285.5,253) .. (285.5,262.25) .. controls (285.5,271.5) and (278,279) .. (268.75,279) .. controls (259.5,279) and (252,271.5) .. (252,262.25) -- cycle ;
	%Shape: Circle [id:dp4212709157924077] 
	\draw  [draw opacity=0][fill={rgb, 255:red, 184; green, 184; blue, 184 }  ,fill opacity=1 ] (328,262.25) .. controls (328,253) and (335.5,245.5) .. (344.75,245.5) .. controls (354,245.5) and (361.5,253) .. (361.5,262.25) .. controls (361.5,271.5) and (354,279) .. (344.75,279) .. controls (335.5,279) and (328,271.5) .. (328,262.25) -- cycle ;
	%Straight Lines [id:da44241450213871514] 
	\draw    (285.5,262.25) -- (306.5,262.25) ;
	\draw [shift={(309.5,262.25)}, rotate = 180] [fill={rgb, 255:red, 0; green, 0; blue, 0 }  ][line width=0.08]  [draw opacity=0] (10.72,-5.15) -- (0,0) -- (10.72,5.15) -- (7.12,0) -- cycle    ;
	%Straight Lines [id:da8915950053595205] 
	\draw    (361.5,262.25) -- (397.5,262.25) ;
	\draw [shift={(400.5,262.25)}, rotate = 180] [fill={rgb, 255:red, 0; green, 0; blue, 0 }  ][line width=0.08]  [draw opacity=0] (10.72,-5.15) -- (0,0) -- (10.72,5.15) -- (7.12,0) -- cycle    ;

	% Text Node
	\draw (152,121) node    {$\vec{v}_{1}$};
	% Text Node
	\draw (247,144) node    {$\vec{v}_{2}$};
	% Text Node
	\draw (273,98) node    {$\vec{v}_{2,\text{fin}}$};
	% Text Node
	\draw (213,53) node    {$\vec{v}_{1,\text{fin}}$};
	% Text Node
	\draw (89,234.5) node    {$\vec{v}_{1}$};
	% Text Node
	\draw (161,234.5) node    {$\vec{v}_{2}$};
	% Text Node
	\draw (301,235) node    {$\vec{v}_{1,\text{fin}}$};
	% Text Node
	\draw (379,235) node    {$\vec{v}_{2,\text{fin}}$};


	\end{tikzpicture}
\end{figure}
\FloatBarrier
Un caso interessante si ha quando una massa va a urtare un'altra massa inizialmente ferma. Il risultato che si ottiene è lo stesso senza i termini $A$ e $B$. La pallina nel secondo caso parte con velocità identica a quella della prima: si dice che i due corpi si scambiano la quantità di moto.
\[
	v_{1,\text{fin}} \begin{cases}  >0 & \text{se $m_1>m_2$} \\ <0 & \text{se $m_1<m_2$} \\ =0 &\text{se $m_1=m_2$} \end{cases}
\]
Nel caso di \textit{urti centrali anaelastici}, si può imporre la conservazione della quantità di moto. Essendo l'urto anaelastico, l'energia cinetica del sistema non è costante e varia nel tempo. Come nel caso precedente, si hanno due incognite ma una sola equazione scalare. Quindi non è sufficiente usare la sola conservazione della quantità di moto. Si deve avere un'informazione ad esempio su come è variata l'energia cinetica totale del sistema. Diversa è la situazione se l'urto è completamente anaelastico. Si impone infatti di nuovo la conservazione della quantità di moto, ma se l'urto è anaelastico completo i corpi si muovono insieme con la stessa velocità, quindi si ha una sola incognita.

Si rimuove ora l'ipotesi che le forze esterne non siano impulsive. In tal caso il loro effetto non è più trascurabile durante l'urto. Si immagini di avere un blocco di massa $M$ su un piano d'appoggio orizzontale.
\begin{figure}[htpb]
	\centering
	

	\tikzset{every picture/.style={line width=0.75pt}} %set default line width to 0.75pt        

	\begin{tikzpicture}[x=0.75pt,y=0.75pt,yscale=-1,xscale=1]
	%uncomment if require: \path (0,300); %set diagram left start at 0, and has height of 300

	%Shape: Rectangle [id:dp12759655748744736] 
	\draw  [draw opacity=0][fill={rgb, 255:red, 216; green, 216; blue, 216 }  ,fill opacity=1 ] (235.5,144) -- (313,144) -- (313,219) -- (235.5,219) -- cycle ;
	%Straight Lines [id:da99326183921629] 
	\draw    (39.5,219) -- (431.5,219) ;
	%Straight Lines [id:da4219217975595948] 
	\draw    (274.25,181.5) -- (274.25,92) ;
	\draw [shift={(274.25,89)}, rotate = 450] [fill={rgb, 255:red, 0; green, 0; blue, 0 }  ][line width=0.08]  [draw opacity=0] (10.72,-5.15) -- (0,0) -- (10.72,5.15) -- (7.12,0) -- cycle    ;
	%Straight Lines [id:da835878792476836] 
	\draw    (187.75,73.25) -- (246.23,124.03) ;
	\draw [shift={(248.5,126)}, rotate = 220.97] [fill={rgb, 255:red, 0; green, 0; blue, 0 }  ][line width=0.08]  [draw opacity=0] (10.72,-5.15) -- (0,0) -- (10.72,5.15) -- (7.12,0) -- cycle    ;
	%Straight Lines [id:da9881246029528177] 
	\draw  [dash pattern={on 0.84pt off 2.51pt}]  (187.75,73.25) -- (267.25,73.25) ;
	%Shape: Arc [id:dp002943181307110132] 
	\draw  [draw opacity=0] (217,73.06) .. controls (217,73.12) and (217,73.19) .. (217,73.25) .. controls (217,80.69) and (214.22,87.49) .. (209.64,92.65) -- (187.75,73.25) -- cycle ; \draw   (217,73.06) .. controls (217,73.12) and (217,73.19) .. (217,73.25) .. controls (217,80.69) and (214.22,87.49) .. (209.64,92.65) ;
	%Shape: Circle [id:dp4761852792534427] 
	\draw  [draw opacity=0][fill={rgb, 255:red, 128; green, 128; blue, 128 }  ,fill opacity=1 ] (171,73.25) .. controls (171,64) and (178.5,56.5) .. (187.75,56.5) .. controls (197,56.5) and (204.5,64) .. (204.5,73.25) .. controls (204.5,82.5) and (197,90) .. (187.75,90) .. controls (178.5,90) and (171,82.5) .. (171,73.25) -- cycle ;

	% Text Node
	\draw (292,117) node    {$\vec{R}_{n}$};
	% Text Node
	\draw (296,200) node    {$M$};
	% Text Node
	\draw (220.67,117.33) node    {$\vec{v}_{0}$};
	% Text Node
	\draw (225,85.67) node    {$\vartheta $};
	% Text Node
	\draw (156.33,72) node    {$m$};


	\end{tikzpicture}
\end{figure}
\FloatBarrier
Su di esso viene lanciato un oggetto di massa $m$ con velocità $v_o$, inclinata di un certo angolo rispetto alla direzione orizzontale. Le forze esterne che agiscono durante l'urto sono la forza peso della massa $m$ e quella della massa $M$ che non variano durante l'urto, ma c'è un altra forza impulsiva: la reazione normale. Essa varia perché quando la pallina va a conficcarsi nell'oggetto di massa $M$,  tende a scaricare una certa forza lungo la verticale, bilanciata dalla reazione normale, che aumenta istantaneamente. Quando agiscono forze esterne che hanno carattere impulsivo, non è più possibile trascurarle e quindi non è più vero che si conserva la quantità di moto totale durante l'urto. La reazione normale ha direzione solo verticale, quindi nella direzione non verticale non agiscono forze esterne impulsive. Il sistema varierà la sua quantità di moto lungo la direzione verticale, ma non lungo quella orizzontale. Si può applicare ancora il principio di conservazione della quantità di moto in una sola direzione. Si immagini che dopo l'urto la massa si conficchi nel corpo di massa $M$: si avrà un sistema di massa complessiva $m+M$ che viaggia in orizzontale con una certa velocità finale incognita. La quantità di moto scomposta in direzione $x$ iniziale si trasforma tutta in quantità di moto finale. La quantità di moto è variata perché la forza in direzione verticale ha assorbito la quantità di moto iniziale.
\begin{gather*}
	p_\text{x, tot}=\text{cost.} \qquad mv_0\cos\vartheta=(m+M)v_\text{fin} \qquad \Delta p_\text{y, tot} \ne 0 \\
	\implies 0+mv_0\sin\vartheta=\int_0^\tau R_n \, dt
\end{gather*}







































\section{I teoremi di K\"onig}

Si termina lo studio della dinamica dei sistemi di punti smaterializzati con l'introduzione di due teoremi molto importanti: essi prendono il nome di teoremi di K\"onig. Le grandezze caratteristiche di un sistema di punti materiali che finora sono state definite sono le seguenti:
\begin{itemize}
	\item $\vec{p}_\text{tot}$, quantità di moto complessiva. Si tratta della somma della quantità di moto di ogni singolo punto materiale e fornisce un'informazione su come sta traslando il sistema nel suo complesso.
	\item $\vec{L}_\text{tot}$, momento angolare complessivo rispetto al polo $O$. Esso non è altro che la somma dei momenti angolari posseduti da ogni punto materiale.
	\item $E_\text{k, tot}$, energia cinetica totale. È la somma scalare dell'energia cinetica posseduta da ogni punto materiale.
\end{itemize}
Accanto a queste grandezze è possibile definire quelle analoghe riferite al centro di massa. Se in esso si va a concentrare la massa totale del sistema, si definisce la sua quantità di moto come:
\[
	\vec{p}_\text{CM} = M_\text{tot} \, \vec{v}_\text{CM}
\]
Si può anche definire il \textbf{momento angolare del centro di massa} come:
\[
	\boxed{\vec{L}_\text{CM}^{(o)} = \vec{M}_\text{CM}^{(o)} \times M_{tot} \, \vec{v}_\text{CM}}
\]
Ci si potrebbe chiedere se è possibile descrivere il sistema di punti semplicemente in termini di centro di massa. Per il primo teorema del centro di massa, si sa che la quantità di moto totale del sistema è uguale a quella del solo CM. Si può dire la stessa cosa per l'energia cinetica e il momento angolare? I due teoremi di K\"onig rispondono proprio a questa domanda. Il momento angolare del centro di massa dà informazioni sul fatto che tale punto CM possiede un certo momento angolare, quindi che in qualche maniera sta ruotando attorno al polo $O$. Se il momento angolare del centro di massa è nullo tuttavia non si può affermare che i vari punti del sistema non abbiano alcun momento angolare rispetto a tale polo. Esso può ruotare mentre il centro di massa non ha momento angolare, ad esempio perché sta traslando rispetto al polo o perché il sistema sta ruotando attorno ad esso. La velocità del CM sarà nulla, ma il sistema possiederà sicuramente una certa energia cinetica perché le sue parti ruotano. È evidente che non ha senso immaginare di descrivere il sistema di punti in termini del solo centro di massa. I teoremi di K\"onig daranno una descrizione di queste due quantità in termini di centro di massa e di una quantità aggiuntiva.

\paragraph{Primo teorema di K\"onig} Si supponga di avere il sistema di punti materiali e un punto fisso $O$.
\begin{figure}[htpb]
	\centering
	

	\tikzset{every picture/.style={line width=0.75pt}} %set default line width to 0.75pt        

	\begin{tikzpicture}[x=0.75pt,y=0.75pt,yscale=-1,xscale=1]
	%uncomment if require: \path (0,300); %set diagram left start at 0, and has height of 300

	%Straight Lines [id:da3973464237777369] 
	\draw [color={rgb, 255:red, 155; green, 155; blue, 155 }  ,draw opacity=1 ]   (190,220) -- (288,220) ;
	\draw [shift={(290,220)}, rotate = 180] [color={rgb, 255:red, 155; green, 155; blue, 155 }  ,draw opacity=1 ][line width=0.75]    (10.93,-4.9) .. controls (6.95,-2.3) and (3.31,-0.67) .. (0,0) .. controls (3.31,0.67) and (6.95,2.3) .. (10.93,4.9)   ;
	%Straight Lines [id:da7302780746120818] 
	\draw [color={rgb, 255:red, 155; green, 155; blue, 155 }  ,draw opacity=1 ]   (190,220) -- (190,142) ;
	\draw [shift={(190,140)}, rotate = 450] [color={rgb, 255:red, 155; green, 155; blue, 155 }  ,draw opacity=1 ][line width=0.75]    (10.93,-4.9) .. controls (6.95,-2.3) and (3.31,-0.67) .. (0,0) .. controls (3.31,0.67) and (6.95,2.3) .. (10.93,4.9)   ;
	%Straight Lines [id:da7635809314408244] 
	\draw [color={rgb, 255:red, 155; green, 155; blue, 155 }  ,draw opacity=1 ]   (190,220) -- (141.41,268.59) ;
	\draw [shift={(140,270)}, rotate = 315] [color={rgb, 255:red, 155; green, 155; blue, 155 }  ,draw opacity=1 ][line width=0.75]    (10.93,-4.9) .. controls (6.95,-2.3) and (3.31,-0.67) .. (0,0) .. controls (3.31,0.67) and (6.95,2.3) .. (10.93,4.9)   ;
	%Straight Lines [id:da9778907913109491] 
	\draw [color={rgb, 255:red, 155; green, 155; blue, 155 }  ,draw opacity=1 ]   (316.18,160.24) -- (367.12,160.24) ;
	\draw [shift={(369.12,160.24)}, rotate = 180] [color={rgb, 255:red, 155; green, 155; blue, 155 }  ,draw opacity=1 ][line width=0.75]    (10.93,-4.9) .. controls (6.95,-2.3) and (3.31,-0.67) .. (0,0) .. controls (3.31,0.67) and (6.95,2.3) .. (10.93,4.9)   ;
	%Straight Lines [id:da5224303885204684] 
	\draw [color={rgb, 255:red, 155; green, 155; blue, 155 }  ,draw opacity=1 ]   (316.18,160.24) -- (316.18,119.88) ;
	\draw [shift={(316.18,117.88)}, rotate = 450] [color={rgb, 255:red, 155; green, 155; blue, 155 }  ,draw opacity=1 ][line width=0.75]    (10.93,-4.9) .. controls (6.95,-2.3) and (3.31,-0.67) .. (0,0) .. controls (3.31,0.67) and (6.95,2.3) .. (10.93,4.9)   ;
	%Straight Lines [id:da8855900305148521] 
	\draw [color={rgb, 255:red, 155; green, 155; blue, 155 }  ,draw opacity=1 ]   (316.18,160.24) -- (291.12,185.29) ;
	\draw [shift={(289.71,186.71)}, rotate = 315] [color={rgb, 255:red, 155; green, 155; blue, 155 }  ,draw opacity=1 ][line width=0.75]    (10.93,-4.9) .. controls (6.95,-2.3) and (3.31,-0.67) .. (0,0) .. controls (3.31,0.67) and (6.95,2.3) .. (10.93,4.9)   ;
	%Straight Lines [id:da05727546365654601] 
	\draw    (190,220) -- (313.47,161.52) ;
	\draw [shift={(316.18,160.24)}, rotate = 514.65] [fill={rgb, 255:red, 0; green, 0; blue, 0 }  ][line width=0.08]  [draw opacity=0] (10.72,-5.15) -- (0,0) -- (10.72,5.15) -- (7.12,0) -- cycle    ;
	%Straight Lines [id:da8494068632959506] 
	\draw    (190,220) -- (228.89,122.79) ;
	\draw [shift={(230,120)}, rotate = 471.8] [fill={rgb, 255:red, 0; green, 0; blue, 0 }  ][line width=0.08]  [draw opacity=0] (10.72,-5.15) -- (0,0) -- (10.72,5.15) -- (7.12,0) -- cycle    ;
	%Straight Lines [id:da8664667653130169] 
	\draw    (316.18,160.24) -- (232.72,121.27) ;
	\draw [shift={(230,120)}, rotate = 385.03] [fill={rgb, 255:red, 0; green, 0; blue, 0 }  ][line width=0.08]  [draw opacity=0] (10.72,-5.15) -- (0,0) -- (10.72,5.15) -- (7.12,0) -- cycle    ;
	%Shape: Circle [id:dp14778114923410857] 
	\draw  [fill={rgb, 255:red, 0; green, 0; blue, 0 }  ,fill opacity=1 ] (228.25,120) .. controls (228.25,119.03) and (229.03,118.25) .. (230,118.25) .. controls (230.97,118.25) and (231.75,119.03) .. (231.75,120) .. controls (231.75,120.97) and (230.97,121.75) .. (230,121.75) .. controls (229.03,121.75) and (228.25,120.97) .. (228.25,120) -- cycle ;
	%Shape: Circle [id:dp1926347954809542] 
	\draw  [fill={rgb, 255:red, 0; green, 0; blue, 0 }  ,fill opacity=1 ] (357.65,109.4) .. controls (357.65,108.43) and (358.43,107.65) .. (359.4,107.65) .. controls (360.37,107.65) and (361.15,108.43) .. (361.15,109.4) .. controls (361.15,110.37) and (360.37,111.15) .. (359.4,111.15) .. controls (358.43,111.15) and (357.65,110.37) .. (357.65,109.4) -- cycle ;
	%Shape: Circle [id:dp9267696113178119] 
	\draw  [fill={rgb, 255:red, 0; green, 0; blue, 0 }  ,fill opacity=1 ] (406.85,96.6) .. controls (406.85,95.63) and (407.63,94.85) .. (408.6,94.85) .. controls (409.57,94.85) and (410.35,95.63) .. (410.35,96.6) .. controls (410.35,97.57) and (409.57,98.35) .. (408.6,98.35) .. controls (407.63,98.35) and (406.85,97.57) .. (406.85,96.6) -- cycle ;
	%Shape: Circle [id:dp21813701641800876] 
	\draw  [fill={rgb, 255:red, 0; green, 0; blue, 0 }  ,fill opacity=1 ] (415.25,137.8) .. controls (415.25,136.83) and (416.03,136.05) .. (417,136.05) .. controls (417.97,136.05) and (418.75,136.83) .. (418.75,137.8) .. controls (418.75,138.77) and (417.97,139.55) .. (417,139.55) .. controls (416.03,139.55) and (415.25,138.77) .. (415.25,137.8) -- cycle ;
	%Shape: Circle [id:dp6494232798227508] 
	\draw  [fill={rgb, 255:red, 0; green, 0; blue, 0 }  ,fill opacity=1 ] (463.15,109.9) .. controls (463.15,108.93) and (463.93,108.15) .. (464.9,108.15) .. controls (465.87,108.15) and (466.65,108.93) .. (466.65,109.9) .. controls (466.65,110.87) and (465.87,111.65) .. (464.9,111.65) .. controls (463.93,111.65) and (463.15,110.87) .. (463.15,109.9) -- cycle ;

	% Text Node
	\draw (303,220) node    {$x$};
	% Text Node
	\draw (177,130) node    {$y$};
	% Text Node
	\draw (143,280) node    {$z$};
	% Text Node
	\draw (380,160.24) node    {$x'$};
	% Text Node
	\draw (307.29,112.59) node    {$y'$};
	% Text Node
	\draw (293.29,194) node    {$z'$};
	% Text Node
	\draw (197,230) node    {$O$};
	% Text Node
	\draw (323.5,171) node    {$O'$};
	% Text Node
	\draw (231,104.67) node    {$m_{i}$};
	% Text Node
	\draw (361.2,93) node    {$m_{2}$};
	% Text Node
	\draw (410.4,80.2) node    {$m_{3}$};
	% Text Node
	\draw (418.8,121.4) node    {$m_{n}$};
	% Text Node
	\draw (223.1,166.9) node    {$\vec{r}_{i}$};
	% Text Node
	\draw (272,125) node    {$\vec{r} '_{i}$};
	% Text Node
	\draw (260.2,165.6) node    {$\vec{r}_{CM}$};
	% Text Node
	\draw (466.7,93.5) node    {$m_{1}$};


	\end{tikzpicture}
\end{figure}
\FloatBarrier
Il generico punto di massa $m_i$ sarà descritto dal vettore posizione $\vec{r}_i$. Si potrà osservare il moto del sistema di punti guardando ciò che accade da un osservatore posto in $O$ oppure sul centro di massa. Quest'ultimo sistema di riferimento prende il nome di \emph{sistema di riferimento del centro di massa}. In genere gli assi vengono presi paralleli e equiversi rispetto a quelli del sistema di riferimento assoluto in $O$.

Si definisce la posizione della generica massa $m_i$ rispetto al sistema di riferimento del centro di massa con il vettore posizione $\vec{r}_i'$. Sussiste allora la seguente relazione:
\[
	\forall i: \vec{r}_i = \vec{r}_\text{CM} + \vec{r\,'}_i
\]
Questo discorso si può tradurre anche in termini di velocità. Infatti, per la legge di composizione delle velocità:
\begin{gather*}
	\vec{v}_{\text{ass} } = \vec{v}_{\text{rel} } + \vec{v}_{\text{traslazione}}\\
	\vec{v}_i = \vec{v'}_i + \vec{v}_\text{CM}
\end{gather*}
Si comincia andando a considerare la definizione di momento angolare totale del sistema di punti rispetto al polo $O$ fermo nello spazio.
\begin{figure}[htpb]
	\centering

	\tikzset{every picture/.style={line width=0.75pt}} %set default line width to 0.75pt        

	\begin{tikzpicture}[x=0.75pt,y=0.75pt,yscale=-1,xscale=1]
	%uncomment if require: \path (0,300); %set diagram left start at 0, and has height of 300

	%Straight Lines [id:da3973464237777369] 
	\draw [color={rgb, 255:red, 155; green, 155; blue, 155 }  ,draw opacity=1 ]   (190,220) -- (288,220) ;
	\draw [shift={(290,220)}, rotate = 180] [color={rgb, 255:red, 155; green, 155; blue, 155 }  ,draw opacity=1 ][line width=0.75]    (10.93,-4.9) .. controls (6.95,-2.3) and (3.31,-0.67) .. (0,0) .. controls (3.31,0.67) and (6.95,2.3) .. (10.93,4.9)   ;
	%Straight Lines [id:da7302780746120818] 
	\draw [color={rgb, 255:red, 155; green, 155; blue, 155 }  ,draw opacity=1 ]   (190,220) -- (190,142) ;
	\draw [shift={(190,140)}, rotate = 450] [color={rgb, 255:red, 155; green, 155; blue, 155 }  ,draw opacity=1 ][line width=0.75]    (10.93,-4.9) .. controls (6.95,-2.3) and (3.31,-0.67) .. (0,0) .. controls (3.31,0.67) and (6.95,2.3) .. (10.93,4.9)   ;
	%Straight Lines [id:da7635809314408244] 
	\draw [color={rgb, 255:red, 155; green, 155; blue, 155 }  ,draw opacity=1 ]   (190,220) -- (141.41,268.59) ;
	\draw [shift={(140,270)}, rotate = 315] [color={rgb, 255:red, 155; green, 155; blue, 155 }  ,draw opacity=1 ][line width=0.75]    (10.93,-4.9) .. controls (6.95,-2.3) and (3.31,-0.67) .. (0,0) .. controls (3.31,0.67) and (6.95,2.3) .. (10.93,4.9)   ;
	%Straight Lines [id:da9778907913109491] 
	\draw [color={rgb, 255:red, 155; green, 155; blue, 155 }  ,draw opacity=1 ]   (316.18,160.24) -- (367.12,160.24) ;
	\draw [shift={(369.12,160.24)}, rotate = 180] [color={rgb, 255:red, 155; green, 155; blue, 155 }  ,draw opacity=1 ][line width=0.75]    (10.93,-4.9) .. controls (6.95,-2.3) and (3.31,-0.67) .. (0,0) .. controls (3.31,0.67) and (6.95,2.3) .. (10.93,4.9)   ;
	%Straight Lines [id:da5224303885204684] 
	\draw [color={rgb, 255:red, 155; green, 155; blue, 155 }  ,draw opacity=1 ]   (316.18,160.24) -- (316.18,119.88) ;
	\draw [shift={(316.18,117.88)}, rotate = 450] [color={rgb, 255:red, 155; green, 155; blue, 155 }  ,draw opacity=1 ][line width=0.75]    (10.93,-4.9) .. controls (6.95,-2.3) and (3.31,-0.67) .. (0,0) .. controls (3.31,0.67) and (6.95,2.3) .. (10.93,4.9)   ;
	%Straight Lines [id:da8855900305148521] 
	\draw [color={rgb, 255:red, 155; green, 155; blue, 155 }  ,draw opacity=1 ]   (316.18,160.24) -- (291.12,185.29) ;
	\draw [shift={(289.71,186.71)}, rotate = 315] [color={rgb, 255:red, 155; green, 155; blue, 155 }  ,draw opacity=1 ][line width=0.75]    (10.93,-4.9) .. controls (6.95,-2.3) and (3.31,-0.67) .. (0,0) .. controls (3.31,0.67) and (6.95,2.3) .. (10.93,4.9)   ;
	%Straight Lines [id:da05727546365654601] 
	\draw    (190,220) -- (313.47,161.52) ;
	\draw [shift={(316.18,160.24)}, rotate = 514.65] [fill={rgb, 255:red, 0; green, 0; blue, 0 }  ][line width=0.08]  [draw opacity=0] (10.72,-5.15) -- (0,0) -- (10.72,5.15) -- (7.12,0) -- cycle    ;
	%Straight Lines [id:da8494068632959506] 
	\draw    (190,220) -- (228.89,122.79) ;
	\draw [shift={(230,120)}, rotate = 471.8] [fill={rgb, 255:red, 0; green, 0; blue, 0 }  ][line width=0.08]  [draw opacity=0] (10.72,-5.15) -- (0,0) -- (10.72,5.15) -- (7.12,0) -- cycle    ;
	%Straight Lines [id:da8664667653130169] 
	\draw    (316.18,160.24) -- (232.72,121.27) ;
	\draw [shift={(230,120)}, rotate = 385.03] [fill={rgb, 255:red, 0; green, 0; blue, 0 }  ][line width=0.08]  [draw opacity=0] (10.72,-5.15) -- (0,0) -- (10.72,5.15) -- (7.12,0) -- cycle    ;
	%Shape: Circle [id:dp14778114923410857] 
	\draw  [fill={rgb, 255:red, 0; green, 0; blue, 0 }  ,fill opacity=1 ] (228.25,120) .. controls (228.25,119.03) and (229.03,118.25) .. (230,118.25) .. controls (230.97,118.25) and (231.75,119.03) .. (231.75,120) .. controls (231.75,120.97) and (230.97,121.75) .. (230,121.75) .. controls (229.03,121.75) and (228.25,120.97) .. (228.25,120) -- cycle ;
	%Shape: Circle [id:dp1926347954809542] 
	\draw  [fill={rgb, 255:red, 0; green, 0; blue, 0 }  ,fill opacity=1 ] (357.65,109.4) .. controls (357.65,108.43) and (358.43,107.65) .. (359.4,107.65) .. controls (360.37,107.65) and (361.15,108.43) .. (361.15,109.4) .. controls (361.15,110.37) and (360.37,111.15) .. (359.4,111.15) .. controls (358.43,111.15) and (357.65,110.37) .. (357.65,109.4) -- cycle ;
	%Shape: Circle [id:dp9267696113178119] 
	\draw  [fill={rgb, 255:red, 0; green, 0; blue, 0 }  ,fill opacity=1 ] (406.85,96.6) .. controls (406.85,95.63) and (407.63,94.85) .. (408.6,94.85) .. controls (409.57,94.85) and (410.35,95.63) .. (410.35,96.6) .. controls (410.35,97.57) and (409.57,98.35) .. (408.6,98.35) .. controls (407.63,98.35) and (406.85,97.57) .. (406.85,96.6) -- cycle ;
	%Shape: Circle [id:dp21813701641800876] 
	\draw  [fill={rgb, 255:red, 0; green, 0; blue, 0 }  ,fill opacity=1 ] (415.25,137.8) .. controls (415.25,136.83) and (416.03,136.05) .. (417,136.05) .. controls (417.97,136.05) and (418.75,136.83) .. (418.75,137.8) .. controls (418.75,138.77) and (417.97,139.55) .. (417,139.55) .. controls (416.03,139.55) and (415.25,138.77) .. (415.25,137.8) -- cycle ;
	%Shape: Circle [id:dp6494232798227508] 
	\draw  [fill={rgb, 255:red, 0; green, 0; blue, 0 }  ,fill opacity=1 ] (463.15,109.9) .. controls (463.15,108.93) and (463.93,108.15) .. (464.9,108.15) .. controls (465.87,108.15) and (466.65,108.93) .. (466.65,109.9) .. controls (466.65,110.87) and (465.87,111.65) .. (464.9,111.65) .. controls (463.93,111.65) and (463.15,110.87) .. (463.15,109.9) -- cycle ;

	% Text Node
	\draw (303,220) node    {$x$};
	% Text Node
	\draw (177,130) node    {$y$};
	% Text Node
	\draw (143,280) node    {$z$};
	% Text Node
	\draw (380,160.24) node    {$x'$};
	% Text Node
	\draw (307.29,112.59) node    {$y'$};
	% Text Node
	\draw (293.29,194) node    {$z'$};
	% Text Node
	\draw (197,230) node    {$O$};
	% Text Node
	\draw (323.5,171) node    {$O'$};
	% Text Node
	\draw (231,104.67) node    {$m_{i}$};
	% Text Node
	\draw (361.2,93) node    {$m_{2}$};
	% Text Node
	\draw (410.4,80.2) node    {$m_{3}$};
	% Text Node
	\draw (418.8,121.4) node    {$m_{n}$};
	% Text Node
	\draw (223.1,166.9) node    {$\vec{r}_{i}$};
	% Text Node
	\draw (272,125) node    {$\vec{r} '_{i}$};
	% Text Node
	\draw (260.2,165.6) node    {$\vec{r}_{CM}$};
	% Text Node
	\draw (466.7,93.5) node    {$m_{1}$};


	\end{tikzpicture}
\end{figure}
\FloatBarrier
A partire da questa definizione si va a sostituire le relazioni trovate sopra:
\begin{equation*}
	\begin{aligned}
		\vec{L}_{tot}^{(o)} &= \sum_i \vec{r}_i\times m_i\vec{v}_i \\
		&= \sum_i (\vec{r'}_i + \vec{r}_\text{CM}) \times m_i (\vec{v'}_i + \vec{v}_\text{CM}) \\
		&= \sum_i m_i\cdot \vec{r'}_i \times \vec{v'}_i + \sum_i m_i \, \vec{r'}_i \times \vec{v}_\text{CM} \\
		&\quad + \sum_i m_i\, \vec{r}_\text{CM}\times \vec{v'}_i + \sum_i m_i \, \vec{r}_\text{CM}\times \vec{v}_\text{CM}
	\end{aligned}
\end{equation*}
\begin{enumerate}
	\item Il primo termine dentro la sommatoria lo si chiama \emph{momento angolare di ogni punto materiale rispetto al centro di massa}. È il momento angolare che si attribuisce ad ogni punto nel sistema di riferimento del centro di massa. La somma di tutti questi termini la si chiama $L_{\text{totale}}'$.
	\item Nel secondo termine si nota che dentro la sommatoria c'è un componente che non dipende dal pedice e che quindi si può portare fuori da essa ($\vec{v}_\text{CM}$). $m_i$ è uno scalare: $(\sum_i m_i\vec{r'}_i)\times \vec{v}_\text{CM}$. Il termine tra parentesi è pari a $\vec{r'}_\text{CM} \, M_\text{tot}$, in cui figura la posizione del centro di massa nel sistema di riferimento del centro di massa. Essendo il punto individuato da $\vec{r}_\text{CM}$ coincidente con l'origine di tale sistema di riferimento, si deduce che la sommatoria vale $0$.
	\item Riguardo al terzo termine, si può notare che il vettore posizione del centro di massa non dipende dal pedice $i$ e può essere quindi portato fuori dalla sommatoria. Rimane un termine che contiene la sommatoria:
	\[
		\sum_i (\vec{v'}_i\,m_i ) \quad \vec{v}_{cm} = \sum_i \frac{m_i\vec{v}_i  }{M_{tot} } \quad \vec{v'}_{cm} = \sum_i \frac{m_i\vec{v}_i}{M_{tot} } = 0
	\]
	Per motivi analoghi, si deduce che anche questo termine è nullo.
	\item Per quanto concerne l'ultimo termine, si può portare fuori dalla sommatoria tutto tranne la massa $i$-esima. Si ha semplicemente la somma di tante masse, che fa la massa totale del sistema. Il termine che si ottiene è il momento angolare del centro di massa rispetto al polo $O$.
\end{enumerate}
Il risultato che si ottiene è il primo teorema di K\"onig, il quale afferma che il momento angolare totale del sistema si può vedere come la somma dei due contributi rilevati.
Il primo informa sulla rotazione intorno al polo del centro di massa. Accanto a questo è sempre necessario tenere conto anche del moto di rotazione delle varie parti del sistema intorno al centro di massa, rappresentato dal secondo termine. Ecco quindi il doppio contributo che entra in gioco nel momento angolare.
\[
	\vec{L}_{tot}^{(o)} = \vec{r}_{cm}\times M_{tot}\,\vec{v}_{cm} + \sum_i (\vec{r'}\times m_i\,\vec{v'}_i)
\]
In tale dimostrazione era fondamentale capire che le quantità del secondo e terzo termine sono nulle.
\begin{gather*}
	\frac{\sum_i m_i\vec{r}_i   }{M_{tot} } = \vec{r'}_{cm} = 0 \qquad \frac{\sum_i m_i\vec{v}_i }{M_{tot} } = \vec{v'}_{cm} = 0 \\
	\boxed{\vec{L}_{tot}^{(o)}   = \vec{L}_{cm}^{(o)} + \vec{L'}_{tot}}
\end{gather*}

\paragraph{Secondo teorema di K\"onig} È possibile enunciare e dimostrare un secondo teorema relativo all'energia cinetica totale del sistema. Esso afferma che l'energia cinetica totale del sistema si può vedere come di nuovo suddivisa in due contributi: l'energia cinetica del solo centro di massa e l'energia cinetica dell'intero sistema rispetto al centro di massa. Si vede che l'energia cinetica del centro di massa e dell'intero sistema saranno:
\[
	E_{K,cm} = \frac{1}{2} M_{tot}\cdot v_{cm}^2 \qquad E_{K,tot} = \sum_i E_{K,i}
\]
Nel secondo termine, in particolare nell'$i$-esima componente, si trova la velocità relativa al centro di massa. La dimostrazione è del tutto analoga:
\[
	E_{K,tot} = \sum_i \frac{1}{2} m_i\,v_i^2
\]
Si va a riscrivere la velocità $i$-esima nella somma vettoriale della velocità del centro di massa e quella relativa al centro di massa:
\[
\vec{v}_i = \vec{v}_{cm} + \vec{v'}
\]
Si noti che vale la relazione $\vec{v}_i \cdot \vec{v}_i = v_i^2$, molto utile per introdurre in una notazione scalare una relazione vettoriale. Si può quindi affermare che:
\[ (\vec{v}_\text{CM}+\vec{v'})^2= (\vec{v}_\text{CM}+\vec{v'}) \cdot (\vec{v}_\text{CM}+\vec{v'}) \]
Dovranno nascere quattro contributi:
\begin{equation*}
	\begin{aligned}
		E_{K,tot} &= \sum_i \frac{1}{2} m_i (\vec{v}_{cm}+\vec{v'}  ) \cdot (\vec{v}_{cm}+\vec{v'}  ) \\
		&= \sum_i \frac{1}{2} m_i v_{cm}^2 + \sum_i \frac{1}{2} m_i\,\vec{v}_{cm}\cdot \vec{v'}_i + \sum_i \frac{1}{2} m_i \vec{v'} \cdot \vec{v}_{cm} + \sum_i \frac{1}{2} m_i v'^2_i
	\end{aligned}
\end{equation*}
Il secondo e il terzo termine sono del tutto uguali perché il prodotto scalare gode della proprietà commutativa. $\vec{v}_{cm}$ è un termine che non ha un pedice $i$ e che quindi può essere portato fuori dalla sommatoria. Allora si otterrà:
\[
	\vec{v}_{cm} \cdot \sum_i \vec{v'}_i m_i = \vec{v}_{cm}\cdot M_{tot}\cdot \vec{v'}_{cm} = 0
\]
$\vec{v'}_{cm}$ è zero nel sistema di riferimento del centro di massa.
Rimangono soltanto i primo e il quarto termine, che hanno un significato molto semplice da vedere. Il primo termine sarà:
\[
	\frac{1}{2} v_{cm}^2 \sum_i m_i = \frac{1}{2} v_{cm}^2 M_{tot}
\]
Questo termine è l'energia cinetica di un punto che si muove come il centro di massa e che quindi rappresenta l'\textbf{energia cinetica del centro di massa}.
L'ultimo termine rimasto è la somma di tante energie cinetiche dove compare per ogni massa la velocità che essa ha nel sistema di riferimento del centro di massa. Si tratta dell'energia cinetica totale del sistema nel sistema di riferimento del centro di massa.

Il secondo teorema di Koenig è quindi dato da.
\[
	\boxed{E_{K,tot} = E_{K,cm} + E'_K}
\]
Grazie a questo teorema, si deduce la necessità di tenere conto del fatto che le varie parti del sistema si muovono rispetto al centro di massa e hanno bisogno di una certa energia per mettersi in movimento.
Come per il primo si può affermare che l'energia cinetica non può essere descritta solo in termini del centro di massa.
Questi due teoremi hanno importanza notevole nello studio della dinamica del corpo rigido.























































































































\chapter{Elementi di dinamica del corpo rigido}

\section{Il corpo rigido}

\subsection{Definizione e osservazioni generali}
Un corpo rigido è un corpo che non può più essere approssimato a un punto materiale ma che ha una certa estensione. Il fatto che sia rigido comporta assenza di movimento delle parti interne del sistema. Esso si può immaginare come un particolare sistema di punti materiali semplificato. In questo caso si va a suddividere il corpo esteso in tante particelle molto piccole di massa infinitesima $dm$ che andranno a sostituire le masse $i$-esime.
\begin{figure}[htpb]
	\centering
	

	\tikzset{every picture/.style={line width=0.75pt}} %set default line width to 0.75pt        

	\begin{tikzpicture}[x=0.75pt,y=0.75pt,yscale=-1,xscale=1]
	%uncomment if require: \path (0,300); %set diagram left start at 0, and has height of 300

	%Shape: Polygon Curved [id:ds9077710605730196] 
	\draw   (221,83.31) .. controls (172.5,64) and (317.5,61) .. (349.66,76.41) .. controls (381.83,91.82) and (383.12,109.45) .. (334.5,124) .. controls (285.88,138.55) and (251.5,149) .. (204.01,145.45) .. controls (156.52,141.91) and (269.51,102.63) .. (221,83.31) -- cycle ;
	%Shape: Cube [id:dp5839980293890457] 
	\draw   (267,87.58) -- (272.88,81.7) -- (294.5,81.7) -- (294.5,95.42) -- (288.62,101.3) -- (267,101.3) -- cycle ; \draw   (294.5,81.7) -- (288.62,87.58) -- (267,87.58) ; \draw   (288.62,87.58) -- (288.62,101.3) ;

	% Text Node
	\draw (278,115) node    {$dm$};


	\end{tikzpicture}
\end{figure}
\FloatBarrier
Infatti, invece di utilizzare il concetto di sommatoria si ha quello di somma continua, che non è altro che l'integrale esteso all'intero volume del corpo, o esteso alla superficie o esteso alla linea, se il corpo è approssimabile a una superficie o a una linea.

Mentre quando si aveva il sistema di punti materiali, in generale una massa si poteva muovere rispetto all'altra, nel corpo rigido ciò non avviene. Si richiamano le equazioni cardinali della dinamica definite per un sistema di punti materiali:
\[
	\boxed{\vec{R}_{est} = M_{tot}\cdot \vec{a}_{cm} \qquad \vec{M}^{(o)}_{est} = \frac{d\vec{L}^{(o)}_{tot}}{dt} \qquad \mathcal{L}_{int} + \mathcal{L}_{est} = \Delta E_{K,tot}   }
\]
Queste relazioni sono applicabili al corpo rigido. Le forze interne in esso non compiono mai lavoro perché vista la rigidità, non mettono in movimento nessuna parte dell'oggetto. Quindi le uniche forze che vanno considerate sono quelle esterne. La seconda semplificazione è che se in un sistema di punti materiali si hanno molti gradi di libertà,  un corpo rigido è caratterizzato da $6$ di essi. Infatti, dato che questo corpo non si deforma, lo si definisce andando a individuare la posizione di un punto. A questo punto il corpo può ancora muoversi rispetto ad esso, per questo gli si conferisce un angolo di rotazione rispetto ai tre assi. Assegnare un angolo di rotazione e la posizione di un punto, significa individuare $3+3=6$ coordinate nello spazio. \emph{Se il sistema è definito da $6$ gradi di libertà, servono solo sei equazioni scalari per conoscere completamente il moto complessivo del corpo rigido}. Esse sono date dalla prima e la seconda equazione cardinale della dinamica. Questo primo risultato conferma l'osservazione che il moto di un corpo rigido è più complesso di quello di un punto materiale.

\paragraph{Osservazione} Indicati con $l$ i gradi di libertà, così come per un corpo rigido si ha $l=6$, si noti che:
\begin{itemize}
	\item un punto materiale avrà $l=3$;
	\item $n$ punti materiali indipendenti avranno $l=3n$;
	\item un punto vincolato a muoversi lungo una linea avrà $l=1$, sopra una superficie avrà invece $l=2$;
	\item due punti vincolati ad avere sempre la stessa distanza fra loro avranno $l=5$.
\end{itemize}
Si risolve quindi completamente il sistema andando a studiare l'effetto della risultante delle forze e dei momenti delle forze. Il teorema dell'energia cinetica può essere sostituito a una delle sei relazioni scalari per studiare eventualmente in maniera più semplice il moto del corpo dal punto di vista energetico.

\subsection{Il centro di massa nel corpo rigido}

A questo punto si può andare a capire cosa vuol dire passare da una notazione con la sommatoria a una con l'integrale. Si consideri un corpo rigido di forma qualunque. L'obbiettivo è quello di calcolare la posizione di $cm$ rispetto alla posizione di un certo osservatore $O$. Si è definito il centro di massa come:
\[
	\vec{r}_\text{CM} = \frac{\sum_i m_i r_i}{M}
\]
Ora si approssima il corpo e lo si vede come costituito da tanti volumetti infinitesimi.
\begin{figure}[htpb]
	\centering
	

	\tikzset{every picture/.style={line width=0.75pt}} %set default line width to 0.75pt        

	\begin{tikzpicture}[x=0.75pt,y=0.75pt,yscale=-1,xscale=1]
	%uncomment if require: \path (0,300); %set diagram left start at 0, and has height of 300

	%Shape: Can [id:dp03669144619399156] 
	\draw   (366.5,76.73) -- (366.5,200.27) .. controls (366.5,207.3) and (335.05,213) .. (296.25,213) .. controls (257.45,213) and (226,207.3) .. (226,200.27) -- (226,76.73) .. controls (226,69.7) and (257.45,64) .. (296.25,64) .. controls (335.05,64) and (366.5,69.7) .. (366.5,76.73) .. controls (366.5,83.76) and (335.05,89.46) .. (296.25,89.46) .. controls (257.45,89.46) and (226,83.76) .. (226,76.73) ;
	%Shape: Can [id:dp6672302852048766] 
	\draw  [fill={rgb, 255:red, 155; green, 155; blue, 155 }  ,fill opacity=1 ] (288.5,136.5) -- (288.5,157.5) .. controls (288.5,158.88) and (285.81,160) .. (282.5,160) .. controls (279.19,160) and (276.5,158.88) .. (276.5,157.5) -- (276.5,136.5) .. controls (276.5,135.12) and (279.19,134) .. (282.5,134) .. controls (285.81,134) and (288.5,135.12) .. (288.5,136.5) .. controls (288.5,137.88) and (285.81,139) .. (282.5,139) .. controls (279.19,139) and (276.5,137.88) .. (276.5,136.5) ;
	%Straight Lines [id:da1696202666868456] 
	\draw    (112,175) -- (276.54,149.46) ;
	\draw [shift={(279.5,149)}, rotate = 531.1800000000001] [fill={rgb, 255:red, 0; green, 0; blue, 0 }  ][line width=0.08]  [draw opacity=0] (10.72,-5.15) -- (0,0) -- (10.72,5.15) -- (7.12,0) -- cycle    ;
	%Shape: Circle [id:dp9601969907529446] 
	\draw  [fill={rgb, 255:red, 0; green, 0; blue, 0 }  ,fill opacity=1 ] (110.5,175) .. controls (110.5,174.17) and (111.17,173.5) .. (112,173.5) .. controls (112.83,173.5) and (113.5,174.17) .. (113.5,175) .. controls (113.5,175.83) and (112.83,176.5) .. (112,176.5) .. controls (111.17,176.5) and (110.5,175.83) .. (110.5,175) -- cycle ;

	% Text Node
	\draw (281.5,168) node    {$dm$};
	% Text Node
	\draw (306,144) node    {$dV$};
	% Text Node
	\draw (107,188) node    {$O$};
	% Text Node
	\draw (183.5,151.5) node    {$\vec{r}$};
	% Text Node
	\draw (269.5,126.5) node    {$P$};


	\end{tikzpicture}
\end{figure}
\FloatBarrier
\[
	\vec{r}_\text{CM} = \frac{\int_{\text{volume}} dm\,\vec{r} (P)}{M}
\]
$\vec{r} (P)$ non è costante ma è funzione della posizione. Andare a ricoprire in maniera continua il volume, significa fare una somma continua su tutto il volume, quello che si chiama \emph{integrale di volume}, o \emph{integrale triplo}.
\[
	\int_{\text{volume}} = \iiint
\]
In generale i corpi rigidi non sono copri omogenei, la massa non è distribuita in maniera uniforme sul volume del corpo. Per dare questa informazione si introduce il concetto di \textbf{densità volumetrica}, ovvero il rapporto tra la massa infinitesima del corpo e il volumetto infinitesimo $dV$.
\[
	\rho = \frac{dm}{dV}
\]
Nota questa grandezza, che dimensionalmente è una massa su una lunghezza al cubo e si misurerà in $kg/m^3$, si può andare a introdurla dentro alla definizione di centro di massa: invece di scrivere $dm$, la si sostituisce con la densità per il volumetto infinitesimo.
\[
	\vec{r}_\text{CM} = \frac{\int_{\text{volume}}\rho (P)\, \vec{r} (P)\, dr}{M}
\]
La variabile di integrazione è il volumetto infinitesimo, per ognuno di essi bisogna valutare quanto vale il prodotto fra la densità del corpo in quel punto e il vettore posizione che ne identifica la posizione. Questa diventa le definizione di posizione del centro di massa per un corpo tridimensionale. Se il corpo è omogeneo, $\rho$ è costante e sarà pari a $M/V$. Si ha in tal caso:
\[
	\vec{r}_\text{CM} = \rho \frac{\int_{\text{volume}}\vec{r} (P)dV }{M} = \frac{M}{V} \frac{\int_{\text{volume}}\vec{r} (P)dV }{M} = \frac{\int_{\text{volume}}\vec{r} (P)dV }{V}
\]
Questa definizione può essere estesa al caso in cui il corpo invece che essere tridimensionale, è approssimabile a una superficie.
\begin{figure}[htpb]
	\centering
	

	\tikzset{every picture/.style={line width=0.75pt}} %set default line width to 0.75pt        

	\begin{tikzpicture}[x=0.75pt,y=0.75pt,yscale=-1,xscale=1]
	%uncomment if require: \path (0,300); %set diagram left start at 0, and has height of 300

	%Shape: Parallelogram [id:dp9250279294336241] 
	\draw   (276.35,92) -- (349.5,92) -- (318.15,212) -- (245,212) -- cycle ;
	%Shape: Square [id:dp3084993720003819] 
	\draw  [fill={rgb, 255:red, 155; green, 155; blue, 155 }  ,fill opacity=1 ] (288.5,149) -- (297,149) -- (297,157.5) -- (288.5,157.5) -- cycle ;
	%Straight Lines [id:da7733792598985865] 
	\draw    (124,180) -- (288.54,154.46) ;
	\draw [shift={(291.5,154)}, rotate = 531.1800000000001] [fill={rgb, 255:red, 0; green, 0; blue, 0 }  ][line width=0.08]  [draw opacity=0] (10.72,-5.15) -- (0,0) -- (10.72,5.15) -- (7.12,0) -- cycle    ;
	%Shape: Circle [id:dp06205929212849104] 
	\draw  [fill={rgb, 255:red, 0; green, 0; blue, 0 }  ,fill opacity=1 ] (122.5,180) .. controls (122.5,179.17) and (123.17,178.5) .. (124,178.5) .. controls (124.83,178.5) and (125.5,179.17) .. (125.5,180) .. controls (125.5,180.83) and (124.83,181.5) .. (124,181.5) .. controls (123.17,181.5) and (122.5,180.83) .. (122.5,180) -- cycle ;

	% Text Node
	\draw (299,134.5) node    {$dS$};
	% Text Node
	\draw (288,169) node    {$dm$};
	% Text Node
	\draw (119,193) node    {$O$};
	% Text Node
	\draw (195.5,156.5) node    {$\vec{r}$};
	% Text Node
	\draw (313.5,154.5) node    {$P$};


	\end{tikzpicture}
\end{figure}
\FloatBarrier
Si divide il corpo in tanti elementi $m_i$ infinitesimi in cui la superficie è $dS$ e si introduce il concetto di \textbf{densità superficiale}, $\sigma$, che non è altro che il rapporto tra la massa infinitesima e l'elementino di superficie infinitesimo.
\[
	\sigma=\frac{dm}{dS}
\]
Riprendendo la definizione data del vettore posizione del centro di massa, la somma continua è un \emph{integrale di superficie} dove al posto di $dm$ si può sostituire $\sigma \,dS$. L'elemento di integrazione è $dS$.
\[
	\vec{r}_\text{CM} = \frac{\int_{\text{superficie}}\vec{r} (P) \sigma(P) dS}{M}
	\]
Se però il corpo è omogeneo:
\[
	\vec{r}_\text{CM} = \frac{M}{S} \frac{\int_{\text{superficie}}\vec{r} (P)dS}{M} = \frac{\int_{\text{superficie}}\vec{r} (P)dS}{S}
\]
Infine si può considerare il caso in cui il corpo rigido può essere approssimato ad una linea.
\begin{figure}[htpb]
	\centering
	

	\tikzset{every picture/.style={line width=0.75pt}} %set default line width to 0.75pt        

	\begin{tikzpicture}[x=0.75pt,y=0.75pt,yscale=-1,xscale=1]
	%uncomment if require: \path (0,300); %set diagram left start at 0, and has height of 300

	%Curve Lines [id:da02030950644760532] 
	\draw    (98.5,124) .. controls (232.5,94) and (266.5,171) .. (414.5,125) ;
	%Shape: Rectangle [id:dp3105990545838513] 
	\draw  [fill={rgb, 255:red, 155; green, 155; blue, 155 }  ,fill opacity=1 ] (261.6,131.6) -- (274.9,134.45) -- (274.4,136.8) -- (261.1,133.95) -- cycle ;
	%Straight Lines [id:da07326605801202035] 
	\draw    (98,161) -- (262.54,135.46) ;
	\draw [shift={(265.5,135)}, rotate = 531.1800000000001] [fill={rgb, 255:red, 0; green, 0; blue, 0 }  ][line width=0.08]  [draw opacity=0] (10.72,-5.15) -- (0,0) -- (10.72,5.15) -- (7.12,0) -- cycle    ;
	%Shape: Circle [id:dp4453267725448391] 
	\draw  [fill={rgb, 255:red, 0; green, 0; blue, 0 }  ,fill opacity=1 ] (96.5,161) .. controls (96.5,160.17) and (97.17,159.5) .. (98,159.5) .. controls (98.83,159.5) and (99.5,160.17) .. (99.5,161) .. controls (99.5,161.83) and (98.83,162.5) .. (98,162.5) .. controls (97.17,162.5) and (96.5,161.83) .. (96.5,161) -- cycle ;

	% Text Node
	\draw (257.6,144) node    {$dm$};
	% Text Node
	\draw (258.2,117.2) node    {$dl$};
	% Text Node
	\draw (93,174) node    {$O$};
	% Text Node
	\draw (169.5,137.5) node    {$\vec{r}$};
	% Text Node
	\draw (276.5,121.5) node    {$P$};


	\end{tikzpicture}
\end{figure}
\FloatBarrier
In tali casi, si va a suddividere la linea in tanti elementini di linea infinitesimi. Si definisce una \textbf{densità lineare} come il rapporto tra la massa infinitesima e la lunghezza $dl$.
\[
	\lambda (P)= \frac{dm}{dl}
\]
Ovviamente questa grandezza è una massa su una lunghezza.
\[
	\vec{r}_\text{CM} = \frac{\int_{\text{linea} }dm\,\vec{r} (P) }{M} =\vec{r}_\text{CM} = \frac{\int_{\text{linea} }\lambda(P)\,\vec{r} (P)\, dl}{M}
\]
Se il corpo è omogeneo, la densità lineare si può considerare come costante in ogni punto del corpo ed è data da il rapporto fra la massa e la sua lunghezza. Allora la definizione di vettore posizione del centro di massa sarà:
\[
	\vec{r}_\text{CM} = \frac{M}{L} \frac{\int_{\text{linea} }\vec{r} (P) dl}{M} = \frac{\int_{\text{linea} }\vec{r} (P) dl}{L}
\]
Se un corpo omogeneo è simmetrico rispetto ad un punto, un asse o un piano, il centro di massa rispettivamente coincide col centro di simmetria o è un punto dell'asse o del piano di simmetria. Se esistono più assi o piani di simmetria, il centro di massa sta sulla loro intersezione.

Si consideri un'asta rigida a forma di semi anello di raggio $R$, di densità uniforme.
\begin{figure}[htpb]
	\centering
	

	\tikzset{every picture/.style={line width=0.75pt}} %set default line width to 0.75pt        

	\begin{tikzpicture}[x=0.75pt,y=0.75pt,yscale=-0.9,xscale=0.9]
	%uncomment if require: \path (0,300); %set diagram left start at 0, and has height of 300

	%Shape: Axis 2D [id:dp20288318873505573] 
	\draw  (68,241.25) -- (479.5,241.25)(273.5,39.25) -- (273.5,241.25) (472.5,236.25) -- (479.5,241.25) -- (472.5,246.25) (268.5,46.25) -- (273.5,39.25) -- (278.5,46.25)  ;
	%Curve Lines [id:da27128963796582406] 
	\draw [color={rgb, 255:red, 0; green, 0; blue, 0 }  ,draw opacity=1 ][line width=1.5]    (133.5,241.25) .. controls (129.5,70.25) and (402.5,41.25) .. (413.5,241.25)(130.95,216.11) -- (138.87,217.24)(136.27,193.06) -- (143.88,195.53)(145.04,172.31) -- (152.11,176.05)(158.19,152.25) -- (164.45,157.23)(174.44,135.2) -- (179.73,141.2)(193.21,121.25) -- (197.44,128.04)(213.96,110.48) -- (217.09,117.85)(236.12,102.98) -- (238.12,110.72)(261.29,98.59) -- (262.03,106.56)(284.69,98.16) -- (284.24,106.14)(307.85,101.24) -- (306.18,109.07)(330.24,107.95) -- (327.31,115.4)(351.29,118.39) -- (347.12,125.22)(370.43,132.64) -- (365.08,138.59)(387.1,150.8) -- (380.72,155.63)(399.6,170.73) -- (392.49,174.39)(408.35,191.47) -- (400.77,194.03)(414.4,214.9) -- (406.54,216.37)(417.31,237.99) -- (409.32,238.54) ;
	%Shape: Circle [id:dp81385918280685] 
	\draw  [fill={rgb, 255:red, 0; green, 0; blue, 0 }  ,fill opacity=1 ] (271.25,158.25) .. controls (271.25,157.01) and (272.26,156) .. (273.5,156) .. controls (274.74,156) and (275.75,157.01) .. (275.75,158.25) .. controls (275.75,159.49) and (274.74,160.5) .. (273.5,160.5) .. controls (272.26,160.5) and (271.25,159.49) .. (271.25,158.25) -- cycle ;
	%Straight Lines [id:da1883230978266146] 
	\draw    (273.5,241.25) -- (366.67,137.92) ;
	%Shape: Arc [id:dp5171778667888058] 
	\draw  [draw opacity=0] (313.36,196.74) .. controls (325.52,207.64) and (333.2,223.45) .. (333.25,241.06) -- (273.5,241.25) -- cycle ; \draw   (313.36,196.74) .. controls (325.52,207.64) and (333.2,223.45) .. (333.25,241.06) ;
	%Shape: Arc [id:dp10120721800348642] 
	\draw  [draw opacity=0] (356.19,117.58) .. controls (361.5,121.14) and (366.57,125.03) .. (371.36,129.22) -- (273.5,241.25) -- cycle ; \draw   (356.19,117.58) .. controls (361.5,121.14) and (366.57,125.03) .. (371.36,129.22) ;

	% Text Node
	\draw (254.5,155.25) node    {$y_{cm}$};
	% Text Node
	\draw (372.5,110.25) node    {$d\vartheta $};
	% Text Node
	\draw (338,213.75) node    {$\vartheta $};
	% Text Node
	\draw (493.5,236.75) node    {$x$};
	% Text Node
	\draw (260,35.25) node    {$y$};
	% Text Node
	\draw (324,166.75) node    {$R$};


	\end{tikzpicture}
\end{figure}
\FloatBarrier
Si avrà:
\[
	\lambda = \frac{M}{\pi R}
\]
Si posiziona l'origine sul centro della circonferenza e si fissano due assi $x$ e $y$. Per ragioni di simmetria il baricentro dovrà per forza appartenere all'asse $y$. L'ascissa del centro di massa va a $0$. Se ne calcola l'ordinata:
\[
	\vec{r}_\text{CM} = \frac{\int_{\text{linea} } \vec{r} (P)dl}{L} \qquad y_{cm}=\frac{\int_{\text{linea} }y(P)dl}{L} = \frac{1}{\pi R} \int_{\text{linea} } y(P) dl
\]
Operare un integrale di linea vuol dire andarla a suddividere in tanti segmentini infinitesimi di lunghezza $dl$. Tutte le volte che si ha un corpo dalla forma caratterizzata da una certa simmetria radiale, conviene andare a lavorare con le coordinate radiali. Bisogna trasformare quell'integrale nella nuova variabile di integrazione angolare.
$dl$ in questo caso è il solito arco di circonferenza di ampiezza $d\vartheta$, $dl=R\,d\vartheta$
\[
	\frac{1}{\pi R}\int_{\text{linea}} y(R)Rd\vartheta = \frac{1}{\pi R}\int_{0\text{ linea} }^{\pi } \sin \vartheta R^2 d\vartheta = \frac{R}{\pi }[-\cos \vartheta ]_0^{\pi } = \frac{2R}{\pi}
\]

\subsection{Punto di applicazione di una forza}

Essendo il corpo esteso, si dovrà porre particolare attenzione al \emph{punto di applicazione della forza}, problema che non si presentava per il punto materiale. Se si ha una singola forza che spinge o che tira in un punto, il punto di applicazione coincide esso. Esistono tuttavia anche delle forze distribuite su tutto il corpo, come la \textbf{forza peso}. È necessario quindi calcolare la forza peso totale, capire quanto vale e dove applicarla.
Considerare la forza peso totale, significa fare la somma di tutte le forze peso associate alle masse infinitesime:
\[
	\vec{F}_{peso,tot} = \sum_i m_i\vec{g} = \int dm\,\vec{g} = \vec{g} \int dm = M\vec{g}
\]
Si calcola invece il momento totale rispetto ad un generico polo generato dalla forza peso:
\[
	\vec{M}^{(o)}_{F\,peso} = \sum_i \vec{r}_i \times m_i\,\vec{g} = \int_{\text{corpo}}\vec{r} (P)\times \vec{g} \,dm
\]
Si ha di nuovo che $\vec{g}$ è la stessa per tutti i termini e può essere portata fuori dall'integrale. Si ritrova così un integrale che coincide con il vettore posizione del centro di massa per la massa del corpo rigido.
\[
	M\left( \frac{\int\vec{r} (P)dm}{M} \right) \times \vec{g} = \vec{r}_{cm}^{(o)}\cdot M \times \vec{g} = \vec{r}_\text{CM}^{(o)} \times M\vec{g}
\]
Invece che dover calcolare l'azione di tante forze peso infinitesime,  è equivalente trovare il baricentro del corpo ed applicarci la forza peso.

Vi è in realtà un'altra forza distribuita per la quale a priori non è noto il punto di applicazione. Si supponga di avere un corpo appoggiato su un piano d'appoggio, esso sarà soggetto alla forza peso diretta verso il basso.
\begin{figure}[htpb]
	\centering
	

	\tikzset{every picture/.style={line width=0.75pt}} %set default line width to 0.75pt        

	\begin{tikzpicture}[x=0.75pt,y=0.75pt,yscale=-1,xscale=1]
	%uncomment if require: \path (0,300); %set diagram left start at 0, and has height of 300

	%Straight Lines [id:da7441220630131655] 
	\draw    (136,205) -- (491.5,205) ;
	%Curve Lines [id:da6787839247003224] 
	\draw [color={rgb, 255:red, 0; green, 0; blue, 0 }  ][line width=0.75] [line join = round][line cap = round]   (173.5,205) .. controls (176.3,202.66) and (184.38,199.39) .. (184.5,195.8) .. controls (184.87,184.61) and (176.8,169.04) .. (187.5,162.32) .. controls (197.44,156.09) and (212.35,180.51) .. (221.5,169.02) .. controls (228.07,160.77) and (223.59,148.26) .. (225.5,139.73) .. controls (226.67,134.5) and (234.11,127.83) .. (239.5,124.67) .. controls (258.62,113.47) and (250.54,139.74) .. (265.5,142.24) .. controls (267.17,142.52) and (268.92,143.61) .. (270.5,143.08) .. controls (284.86,138.27) and (270.39,102.87) .. (292.5,100.4) .. controls (307.82,98.69) and (318.37,102.18) .. (319.5,115.46) .. controls (319.92,120.43) and (315.72,134.15) .. (325.5,137.22) .. controls (338.13,141.18) and (338.98,132.78) .. (345.5,125.51) .. controls (352.75,117.42) and (365.79,113.32) .. (377.5,114.63) .. controls (395.49,116.64) and (378.12,141.09) .. (385.5,147.26) .. controls (393.78,154.19) and (414.04,144.46) .. (423.5,143.92) .. controls (429.82,143.55) and (436.2,143.36) .. (442.5,143.92) .. controls (443.87,144.04) and (443.41,146.11) .. (443.5,147.26) .. controls (443.79,150.88) and (443.98,154.54) .. (443.5,158.14) .. controls (442.55,165.26) and (430.41,169.71) .. (429.5,176.55) .. controls (428.17,186.54) and (437.99,183.65) .. (444.5,189.1) .. controls (448.25,192.24) and (447.5,200.4) .. (447.5,204.16) ;
	%Straight Lines [id:da19958187671322225] 
	\draw    (201.78,205.26) -- (201.78,181.26) ;
	\draw [shift={(201.78,178.26)}, rotate = 450] [fill={rgb, 255:red, 0; green, 0; blue, 0 }  ][line width=0.08]  [draw opacity=0] (10.72,-5.15) -- (0,0) -- (10.72,5.15) -- (7.12,0) -- cycle    ;
	%Straight Lines [id:da8361420970725475] 
	\draw    (244.28,205.26) -- (244.28,181.26) ;
	\draw [shift={(244.28,178.26)}, rotate = 450] [fill={rgb, 255:red, 0; green, 0; blue, 0 }  ][line width=0.08]  [draw opacity=0] (10.72,-5.15) -- (0,0) -- (10.72,5.15) -- (7.12,0) -- cycle    ;
	%Straight Lines [id:da07338761902668622] 
	\draw    (380.78,205.26) -- (380.78,181.26) ;
	\draw [shift={(380.78,178.26)}, rotate = 450] [fill={rgb, 255:red, 0; green, 0; blue, 0 }  ][line width=0.08]  [draw opacity=0] (10.72,-5.15) -- (0,0) -- (10.72,5.15) -- (7.12,0) -- cycle    ;
	%Straight Lines [id:da18986830705750823] 
	\draw    (415.78,205.26) -- (415.78,181.26) ;
	\draw [shift={(415.78,178.26)}, rotate = 450] [fill={rgb, 255:red, 0; green, 0; blue, 0 }  ][line width=0.08]  [draw opacity=0] (10.72,-5.15) -- (0,0) -- (10.72,5.15) -- (7.12,0) -- cycle    ;
	%Straight Lines [id:da14510448793214303] 
	\draw    (313.28,161.76) -- (313.28,234.26) ;
	\draw [shift={(313.28,237.26)}, rotate = 270] [fill={rgb, 255:red, 0; green, 0; blue, 0 }  ][line width=0.08]  [draw opacity=0] (10.72,-5.15) -- (0,0) -- (10.72,5.15) -- (7.12,0) -- cycle    ;
	%Shape: Circle [id:dp24028553452436285] 
	\draw  [fill={rgb, 255:red, 0; green, 0; blue, 0 }  ,fill opacity=1 ] (310.78,161.76) .. controls (310.78,160.38) and (311.9,159.26) .. (313.28,159.26) .. controls (314.66,159.26) and (315.78,160.38) .. (315.78,161.76) .. controls (315.78,163.14) and (314.66,164.26) .. (313.28,164.26) .. controls (311.9,164.26) and (310.78,163.14) .. (310.78,161.76) -- cycle ;
	%Shape: Circle [id:dp5831623738284168] 
	\draw  [fill={rgb, 255:red, 0; green, 0; blue, 0 }  ,fill opacity=1 ] (311.25,205) .. controls (311.25,203.62) and (312.37,202.5) .. (313.75,202.5) .. controls (315.13,202.5) and (316.25,203.62) .. (316.25,205) .. controls (316.25,206.38) and (315.13,207.5) .. (313.75,207.5) .. controls (312.37,207.5) and (311.25,206.38) .. (311.25,205) -- cycle ;

	% Text Node
	\draw (334,217) node    {$M\vec{g}$};


	\end{tikzpicture}
\end{figure}
\FloatBarrier
Sicuramente il piano d'appoggio genererà una forza di \textbf{reazione normale} a tale piano. Ogni punto di esso è soggetto a una certa reazione normale e quella complessiva è la somma di tutte queste forze. Si troveranno tanti vettori paralleli il cui punto di applicazione è a priori non noto. Si può dimostrare che la reazione normale in questo caso giace nel punto del piano d'appoggio passante per la verticale del centro di massa. Bisogna imporre che il corpo sia fermo, quindi che la risultante delle forze esterne sia nulla.
\[
	R_n-Mg=0
\]
Si impone poi che il corpo non ruoti: rispetto ad un certo polo i momenti delle forze sono uguali a zero. Come polo si sceglie il punto sul piano d'appoggio passante per la verticale.
\[
	M^{(o)} = R_n x = Mgx =0 \implies x=0
\]
La forza peso rispetto al polo non genera momento perché il raggio vettore è diretto parallelamente ad essa e si ha solo un effetto di compressione. Se si immagina di congiungere il punto di applicazione della forza con il polo tramite un asta rigida e l'asta rigida con la forza che sta agendo, si ottiene un effetto di rotazione. Si ha che la reazione normale genererà un momento concorde all'asse $z$ di intensità pari a $R_n\,\cdot x$.
Perché esso sia zero, bisogna necessariamente avere $x=0$. Se oltre al peso dell'oggetto si applica su di esso una forza che lo schiaccia verso il basso ma in una parte di destra, la porzione in questa zona del piano d'appoggio deve reagire con una reazione normale più intesa tale per cui, in presenza di $F_o$ e della forza peso, la risultante della reazione normale si sposti verso destra per permettere al corpo di rimanere in equilibrio.







































\section{Statica del corpo rigido}

Tutte queste informazioni servono per affrontare lo studio del corpo rigido, che si suddivide in statica e in dinamica.
Studiare la statica del corpo corpo rigido è importante perché dà le basi della statica di tutte le strutture. Si tratta di studiare quali sono le posizioni per cui un corpo, se inizialmente fermo, rimane tale. Ciò accade \textit{in primis} se la risultante di tutte le forze che agiscono su di esso si bilanciano. Questa condizione è \emph{necessaria ma non è sufficiente}. L'altra da aggiungere è che i momenti di tutte le forze agenti sul corpo rigido rispetto ad un certo polo si bilancino. Si può dimostrare che se la risultante delle forze è uguale a $0$ allora la sommatoria dei momenti di tutte le forze rispetto a un polo è uguale alla somma dei momenti delle forze rispetto a un altro polo.

\paragraph{Esempio} Si immagini di avere una stanza con un pavimento orizzontale e una parete verticale (come tutte le normali stanze).
\begin{figure}[htpb]
	\centering
	

	\tikzset{every picture/.style={line width=0.75pt}} %set default line width to 0.75pt        

	\begin{tikzpicture}[x=0.75pt,y=0.75pt,yscale=-0.8,xscale=0.8]
	%uncomment if require: \path (0,300); %set diagram left start at 0, and has height of 300

	%Shape: Axis 2D [id:dp5275048344026976] 
	\draw  (70,260.83) -- (368.5,260.83)(99.67,30) -- (99.67,288) (361.5,255.83) -- (368.5,260.83) -- (361.5,265.83) (94.67,37) -- (99.67,30) -- (104.67,37)  ;
	%Straight Lines [id:da09308237236536421] 
	\draw [line width=3]    (100,90) -- (220,260) ;
	%Shape: Circle [id:dp47627567183333586] 
	\draw  [fill={rgb, 255:red, 0; green, 0; blue, 0 }  ,fill opacity=1 ] (154.17,175) .. controls (154.17,171.78) and (156.78,169.17) .. (160,169.17) .. controls (163.22,169.17) and (165.83,171.78) .. (165.83,175) .. controls (165.83,178.22) and (163.22,180.83) .. (160,180.83) .. controls (156.78,180.83) and (154.17,178.22) .. (154.17,175) -- cycle ;
	%Straight Lines [id:da9640655037760903] 
	\draw [line width=1.5]    (160,175) -- (160,225.5) ;
	\draw [shift={(160,229.5)}, rotate = 270] [fill={rgb, 255:red, 0; green, 0; blue, 0 }  ][line width=0.08]  [draw opacity=0] (13.4,-6.43) -- (0,0) -- (13.4,6.44) -- (8.9,0) -- cycle    ;
	%Straight Lines [id:da43283520729120917] 
	\draw [line width=1.5]    (100,90) -- (143,90) ;
	\draw [shift={(147,90)}, rotate = 180] [fill={rgb, 255:red, 0; green, 0; blue, 0 }  ][line width=0.08]  [draw opacity=0] (13.4,-6.43) -- (0,0) -- (13.4,6.44) -- (8.9,0) -- cycle    ;
	%Straight Lines [id:da31900880190452674] 
	\draw [line width=1.5]    (220,260) -- (220,218.17) ;
	\draw [shift={(220,214.17)}, rotate = 450] [fill={rgb, 255:red, 0; green, 0; blue, 0 }  ][line width=0.08]  [draw opacity=0] (13.4,-6.43) -- (0,0) -- (13.4,6.44) -- (8.9,0) -- cycle    ;
	%Straight Lines [id:da0916139073976554] 
	\draw [line width=1.5]    (220,260) -- (174.75,260) ;
	\draw [shift={(170.75,260)}, rotate = 360] [fill={rgb, 255:red, 0; green, 0; blue, 0 }  ][line width=0.08]  [draw opacity=0] (13.4,-6.43) -- (0,0) -- (13.4,6.44) -- (8.9,0) -- cycle    ;
	%Shape: Arc [id:dp40138877934035766] 
	\draw  [draw opacity=0] (194.28,260.75) .. controls (194.71,252.04) and (199.13,244.38) .. (205.75,239.56) -- (222.13,262.13) -- cycle ; \draw   (194.28,260.75) .. controls (194.71,252.04) and (199.13,244.38) .. (205.75,239.56) ;

	% Text Node
	\draw (168,155) node    {$L$};
	% Text Node
	\draw (86.67,88.33) node    {$B$};
	% Text Node
	\draw (171,89) node    {$\vec{R}_{n,B}$};
	% Text Node
	\draw (225.33,191.33) node    {$\vec{R}_{n,A}$};
	% Text Node
	\draw (208,279.67) node    {$\vec{R}_{t,A}$};
	% Text Node
	\draw (294.17,247.83) node    {$\mu _{s}$};
	% Text Node
	\draw (86.67,33) node    {$y$};
	% Text Node
	\draw (378.67,262.33) node    {$x$};
	% Text Node
	\draw (188,238.5) node    {$\vartheta $};
	% Text Node
	\draw (90.17,269.83) node    {$0$};


	\end{tikzpicture}
\end{figure}
\FloatBarrier
Si ha una scala lunga $L$ appoggiata, inclinata di un certo angolo $\vartheta$ formato fra la scala e il pavimento orizzontale. È un corpo omogeneo, quindi il baricentro sta a metà dell'asta. Non è possibile far stare in piedi una scala in tale posizione se la parete e il pavimento sono perfettamente lisci. Sull'asta infatti agiscono la forza peso e le due reazioni normali. Siano $A$ e $B$ i rispettivi punti di appoggio della scala. Su $A$ agisce la reazione normale di intensità $\vec{R}_{n,A}$ e su quella verticale agisce una reazione normale diretta come $\vec{R}_{n,B}$. È il piano orizzontale che deve essere scabro, perché così su $A$ reagisce anche una reazione tangente al piano d'appoggio $\vec{R}_{t,A}$ diretta verso sinistra. Noto il coefficiente di attrito statico fra pavimento e scala, ci si chiede qual è l'angolo minimo oltre il quale l'asta non sta più in equilibrio.
Si impone l'equilibrio in direzione verticale e in direzione orizzontale. Si ottiene che:
\begin{gather*}
	\vec{R} = 0 \implies \left\{ \begin{array}{l}
	 	R_{n,B}-R_{t,A} = 0 \\
		Mg-R_{n,A} = 0
	\end{array} \right. \\
	\vec{M}^{(o)} = 0 \implies -R_{n,B}\,L\sin \vartheta + Mg\frac{L}{2}\cos \vartheta = 0
\end{gather*}
La scala tende a ruotare nel verso opposto sotto l'effetto della forza peso rispetto al momento generato da $\vec{R}_B$. Al posto di $\vec{R}_{n,B}$ si può sostituire la forza di attrito. Si impone la condizione per cui la forza di attrito generata dal piano d'appoggio sia minore del valore massimo generabile tra questo piano d'appoggio e il punto $A$.
\[
	\vec{R}_{t,A}= \frac{Mg}{2\tan \vartheta } < \mu_s Mg \implies \vartheta > \tan^{-1} \frac{1}{2\mu_s}
\]

\paragraph{Esempio} In questa situazione, si vuole trovare la condizione su $\vec{F}_o$ affinché il corpo non si muova, perché sia in equilibrio.
\begin{figure}[htpb]
	\centering
	

	% Pattern Info
	 
	\tikzset{
	pattern size/.store in=\mcSize, 
	pattern size = 5pt,
	pattern thickness/.store in=\mcThickness, 
	pattern thickness = 0.3pt,
	pattern radius/.store in=\mcRadius, 
	pattern radius = 1pt}
	\makeatletter
	\pgfutil@ifundefined{pgf@pattern@name@_fqqdc2siy}{
	\pgfdeclarepatternformonly[\mcThickness,\mcSize]{_fqqdc2siy}
	{\pgfqpoint{0pt}{-\mcThickness}}
	{\pgfpoint{\mcSize}{\mcSize}}
	{\pgfpoint{\mcSize}{\mcSize}}
	{
	\pgfsetcolor{\tikz@pattern@color}
	\pgfsetlinewidth{\mcThickness}
	\pgfpathmoveto{\pgfqpoint{0pt}{\mcSize}}
	\pgfpathlineto{\pgfpoint{\mcSize+\mcThickness}{-\mcThickness}}
	\pgfusepath{stroke}
	}}
	\makeatother
	\tikzset{every picture/.style={line width=0.75pt}} %set default line width to 0.75pt        

	\begin{tikzpicture}[x=0.75pt,y=0.75pt,yscale=-1,xscale=1]
	%uncomment if require: \path (0,300); %set diagram left start at 0, and has height of 300

	%Shape: Rectangle [id:dp17470329632127668] 
	\draw  [draw opacity=0][pattern=_fqqdc2siy,pattern size=6.5249999999999995pt,pattern thickness=0.75pt,pattern radius=0pt, pattern color={rgb, 255:red, 155; green, 155; blue, 155}] (174.5,224) -- (422,224) -- (422,251) -- (174.5,251) -- cycle ;
	%Straight Lines [id:da7360973066943703] 
	\draw    (174.5,224) -- (424.5,224) ;
	%Shape: Rectangle [id:dp9882332590443061] 
	\draw  [line width=1.5]  (235,77) -- (373.5,77) -- (373.5,224) -- (235,224) -- cycle ;
	%Straight Lines [id:da3503173706744964] 
	\draw    (208,81) -- (208,215.67) ;
	\draw [shift={(208,218.67)}, rotate = 270] [fill={rgb, 255:red, 0; green, 0; blue, 0 }  ][line width=0.08]  [draw opacity=0] (10.72,-5.15) -- (0,0) -- (10.72,5.15) -- (7.12,0) -- cycle    ;
	\draw [shift={(208,78)}, rotate = 90] [fill={rgb, 255:red, 0; green, 0; blue, 0 }  ][line width=0.08]  [draw opacity=0] (10.72,-5.15) -- (0,0) -- (10.72,5.15) -- (7.12,0) -- cycle    ;
	%Straight Lines [id:da19920558839985936] 
	\draw    (241.67,58) -- (363.67,58) ;
	\draw [shift={(366.67,58)}, rotate = 180] [fill={rgb, 255:red, 0; green, 0; blue, 0 }  ][line width=0.08]  [draw opacity=0] (10.72,-5.15) -- (0,0) -- (10.72,5.15) -- (7.12,0) -- cycle    ;
	\draw [shift={(238.67,58)}, rotate = 0] [fill={rgb, 255:red, 0; green, 0; blue, 0 }  ][line width=0.08]  [draw opacity=0] (10.72,-5.15) -- (0,0) -- (10.72,5.15) -- (7.12,0) -- cycle    ;
	%Straight Lines [id:da07804041759485103] 
	\draw    (374,100) -- (421,100) ;
	\draw [shift={(424,100)}, rotate = 180] [fill={rgb, 255:red, 0; green, 0; blue, 0 }  ][line width=0.08]  [draw opacity=0] (10.72,-5.15) -- (0,0) -- (10.72,5.15) -- (7.12,0) -- cycle    ;
	%Straight Lines [id:da3173765466687095] 
	\draw    (344.67,224) -- (344.67,181) ;
	\draw [shift={(344.67,178)}, rotate = 450] [fill={rgb, 255:red, 0; green, 0; blue, 0 }  ][line width=0.08]  [draw opacity=0] (10.72,-5.15) -- (0,0) -- (10.72,5.15) -- (7.12,0) -- cycle    ;
	%Straight Lines [id:da588107923183989] 
	\draw    (304.67,143.67) -- (304.67,182) ;
	\draw [shift={(304.67,185)}, rotate = 270] [fill={rgb, 255:red, 0; green, 0; blue, 0 }  ][line width=0.08]  [draw opacity=0] (10.72,-5.15) -- (0,0) -- (10.72,5.15) -- (7.12,0) -- cycle    ;
	%Straight Lines [id:da28424913410939934] 
	\draw    (330.5,230) -- (274.17,230) ;
	\draw [shift={(271.17,230)}, rotate = 360] [fill={rgb, 255:red, 0; green, 0; blue, 0 }  ][line width=0.08]  [draw opacity=0] (10.72,-5.15) -- (0,0) -- (10.72,5.15) -- (7.12,0) -- cycle    ;
	%Shape: Circle [id:dp3493283424309832] 
	\draw  [fill={rgb, 255:red, 0; green, 0; blue, 0 }  ,fill opacity=1 ] (302.83,143.67) .. controls (302.83,142.65) and (303.65,141.83) .. (304.67,141.83) .. controls (305.68,141.83) and (306.5,142.65) .. (306.5,143.67) .. controls (306.5,144.68) and (305.68,145.5) .. (304.67,145.5) .. controls (303.65,145.5) and (302.83,144.68) .. (302.83,143.67) -- cycle ;

	% Text Node
	\draw (194,143.33) node    {$H$};
	% Text Node
	\draw (356.67,94) node    {$B$};
	% Text Node
	\draw (438.67,95.33) node    {$\vec{F}_{o}$};
	% Text Node
	\draw (346.67,162) node    {$\vec{R}_{n}$};
	% Text Node
	\draw (319.67,206.33) node    {$\vec{R}_{t}$};
	% Text Node
	\draw (291.33,157) node    {$\vec{P}$};


	\end{tikzpicture}
\end{figure}
\FloatBarrier
È evidente che se il piano di appoggio è sufficientemente scabro, tirando il corpo non è detto che si muova. Il parallelepipedo è omogeneo quindi il centro di massa giace a metà della base e dell'altezza e qui si applica la forza peso. Su di esso agiscono anche la reazione normale e la reazione tangente. Si applica la reazione normale in un punto generico della base, $x$.
Risolvendo il problema, si troverà un certo valore per $x$, ma non tutti i valori sono possibili: la condizione limite per $x$ è che deve rientrare nella base. Applicando le condizioni di equilibrio, si ha:
\[
	\vec{R} = 0 \implies \left\{ \begin{array}{l}
	 	F_o - R_t   = 0 \\
		Mg-R_n = 0
	\end{array} \right.
	\qquad
	\left\{ \begin{array}{l}
	 	F_o = R_t \\
		Mg = R_n
	\end{array} \right.
\]
Si ha una prima condizione sulla forza $\vec{F}_o$ permessa:
\[
	R_t < \mu_s R_n = \mu_s Mg \implies F_o < \mu_s Mg
\]
Si impone poi la condizione per cui il corpo non ruoti, quindi per cui i momenti delle forze siano uguali a $0$. Si sceglie come polo il punto del piano di appoggio passante per la verticale del centro di massa. Si sceglie con verso di rotazione positivo un asse $z$ ortogonale al piano del foglio ed entrante. Si ottiene che la reazione normale rispetto a questo polo genera un momento, il cui braccio sarà $x$. Si impone che i due momenti si debbano bilanciare e si ottiene la condizione su $x$, trovando:
\[
	-R_n x + F_o H = 0 \implies  \frac{F_o H}{Mg} = x
\]
Bisogna imporre che $x<b/2$. Devono valere entrambe le condizioni, una sarà più stringente dell'altra:
\[
	F_o < \mu_s Mg \quad F_o < \frac{B}{2H}Mg
\]







































\section{Dinamica del corpo rigido}

Oltre a studiare l'equilibrio del corpo rigido, si affronta anche il suo possibile movimento, la sua dinamica. Esso si può muovere in tre possibile tipi di moto:
\begin{itemize}
	\item Traslazione
	\item Rotazione
	\item Rototraslazione
\end{itemize}

\subsection{Traslazione del corpo rigido}

Il moto di traslazione del corpo rigido è quello in cui tutti i punti del corpo si muovono su traiettorie parallele a quella del centro di massa in modo tale che l'oggetto in questione non cambi mai la sua orientazione. Essi avranno tutti la stessa velocità, in modulo, direzione e verso e questo va a semplificare di molto il problema.
\begin{figure}[htpb]
	\centering
	


	\tikzset{every picture/.style={line width=0.75pt}} %set default line width to 0.75pt        

	\begin{tikzpicture}[x=0.75pt,y=0.75pt,yscale=-1,xscale=1]
	%uncomment if require: \path (0,300); %set diagram left start at 0, and has height of 300

	%Shape: Can [id:dp6323252782775708] 
	\draw  [fill={rgb, 255:red, 255; green, 255; blue, 255 }  ,fill opacity=1 ] (295.74,99.68) -- (238.68,156.74) .. controls (236.91,158.52) and (230.66,155.16) .. (224.73,149.23) .. controls (218.81,143.3) and (215.44,137.06) .. (217.22,135.28) -- (274.28,78.22) .. controls (276.06,76.44) and (282.3,79.8) .. (288.23,85.73) .. controls (294.16,91.66) and (297.52,97.9) .. (295.74,99.68) .. controls (293.97,101.46) and (287.72,98.1) .. (281.79,92.17) .. controls (275.87,86.24) and (272.5,80) .. (274.28,78.22) ;
	%Shape: Axis 2D [id:dp7646233898089312] 
	\draw  (245,120.15) -- (333.5,120.15)(253.85,40.5) -- (253.85,129) (326.5,115.15) -- (333.5,120.15) -- (326.5,125.15) (248.85,47.5) -- (253.85,40.5) -- (258.85,47.5)  ;
	%Shape: Axis 2D [id:dp5228073483027118] 
	\draw [color={rgb, 255:red, 155; green, 155; blue, 155 }  ,draw opacity=1 ] (248.55,125.45) -- (301.56,72.44)(206.14,72.44) -- (259.15,125.45) (293.07,73.86) -- (301.56,72.44) -- (300.14,80.93) (207.56,80.93) -- (206.14,72.44) -- (214.63,73.86)  ;
	%Shape: Can [id:dp6975388475569626] 
	\draw  [fill={rgb, 255:red, 255; green, 255; blue, 255 }  ,fill opacity=1 ] (478.74,186.68) -- (421.68,243.74) .. controls (419.91,245.52) and (413.66,242.16) .. (407.73,236.23) .. controls (401.81,230.3) and (398.44,224.06) .. (400.22,222.28) -- (457.28,165.22) .. controls (459.06,163.44) and (465.3,166.8) .. (471.23,172.73) .. controls (477.16,178.66) and (480.52,184.9) .. (478.74,186.68) .. controls (476.97,188.46) and (470.72,185.1) .. (464.79,179.17) .. controls (458.87,173.24) and (455.5,167) .. (457.28,165.22) ;
	%Shape: Axis 2D [id:dp7028463504592724] 
	\draw  (428,207.15) -- (516.5,207.15)(436.85,127.5) -- (436.85,216) (509.5,202.15) -- (516.5,207.15) -- (509.5,212.15) (431.85,134.5) -- (436.85,127.5) -- (441.85,134.5)  ;
	%Shape: Axis 2D [id:dp5515393672066395] 
	\draw [color={rgb, 255:red, 155; green, 155; blue, 155 }  ,draw opacity=1 ] (431.55,212.45) -- (484.56,159.44)(389.14,159.44) -- (442.15,212.45) (476.07,160.86) -- (484.56,159.44) -- (483.14,167.93) (390.56,167.93) -- (389.14,159.44) -- (397.63,160.86)  ;
	%Curve Lines [id:da08058788370798475] 
	\draw    (123.5,137) .. controls (326.5,41) and (288.5,267) .. (514.5,192) ;

	% Text Node
	\draw (345,126) node    {$x$};
	% Text Node
	\draw (240,39) node    {$y$};
	% Text Node
	\draw (315,78) node  [color={rgb, 255:red, 155; green, 155; blue, 155 }  ,opacity=1 ]  {$x*$};
	% Text Node
	\draw (194,77) node  [color={rgb, 255:red, 155; green, 155; blue, 155 }  ,opacity=1 ]  {$y*$};
	% Text Node
	\draw (528,213) node    {$x$};
	% Text Node
	\draw (423,126) node    {$y$};
	% Text Node
	\draw (498,165) node  [color={rgb, 255:red, 155; green, 155; blue, 155 }  ,opacity=1 ]  {$x*$};
	% Text Node
	\draw (377,164) node  [color={rgb, 255:red, 155; green, 155; blue, 155 }  ,opacity=1 ]  {$y*$};


	\end{tikzpicture}
\end{figure}
\FloatBarrier
Studiare le equazioni cardinali della dinamica permette di ricavare il moto del sistema. Si impone anche la condizione che il corpo non ruoti.
\[
	\vec{R} = m\,\vec{a}_{cm} 	\quad \vec{M} = 0
\]
Il problema potrebbe essere studiato anche da un punto di vista energetico dicendo che il lavoro di tutte le forze è uguale alla variazione di energia cinetica subita dal corpo rigido.
\[
	\mathcal{L} = \Delta E_K \quad E_K = \sum_i \frac{1}{2}  m_i v_i^2  = \frac{1}{2} \sum_i m_i v_{cm}^2  = \frac{1}{2} M\,v_{cm}^2
\]
in caso di traslazione l'energia cinetica del corpo rigido è uguale all'energia cinetica del solo centro di massa. Si può sostituire questa relazione a una delle due equazioni cardinali della dinamica. Tale risultato non contraddice il teorema di K\"onig perché se si considera un osservatore sul centro di massa, esso non vede le parti intorno a sé che si muovono.

\paragraph{Esempio} Si consideri lo stesso problema affrontato in precedenza. Si ha lo stesso parallelepipedo che deve essere trascinato su un piano scabro. La velocità può essere una funzione del tempo, il corpo in generale non si muove di moto uniforme. Si utilizzano le leggi della dinamica e, definito un asse orizzontale $x$, si avrà:
\begin{gather*}
	\underbrace{F_o\,H - R_n x}_{\vec{M}} = 0 \\
	F_o - \mu_d R_n = Ma_{cm} \implies \frac{F}{M} - \mu_d g = a_{cm}
\end{gather*}
SI sceglie come polo $O$ il punto sulla verticale passante per il centro di massa, sul piano d'appoggio. Le uniche due forze che generano momento saranno $\vec{F}_o$ e $\vec{R}_n$.

\subsection{Rotazione del corpo rigido}

In tale tipo di moto, le varie parti del corpo rigido si muovono seguendo come traiettoria delle circonferenze che hanno centro sull'asse di rotazione e che giacciono su un piano ortogonale ad esso.
\begin{figure}[htpb]
	\centering
	

	\tikzset{every picture/.style={line width=0.75pt}} %set default line width to 0.75pt        

	\begin{tikzpicture}[x=0.75pt,y=0.75pt,yscale=-1,xscale=1]
	%uncomment if require: \path (0,300); %set diagram left start at 0, and has height of 300

	%Straight Lines [id:da50179012944499] 
	\draw    (240.5,201.67) -- (240.5,58) ;
	\draw [shift={(240.5,55)}, rotate = 450] [fill={rgb, 255:red, 0; green, 0; blue, 0 }  ][line width=0.08]  [draw opacity=0] (10.72,-5.15) -- (0,0) -- (10.72,5.15) -- (7.12,0) -- cycle    ;
	%Shape: Cube [id:dp757166379594971] 
	\draw   (240.5,87) -- (243.83,83.67) -- (300.5,83.67) -- (300.5,184.33) -- (297.17,187.67) -- (240.5,187.67) -- cycle ; \draw   (300.5,83.67) -- (297.17,87) -- (240.5,87) ; \draw   (297.17,87) -- (297.17,187.67) ;
	%Shape: Ellipse [id:dp23280864685067515] 
	\draw   (197.78,133.16) .. controls (197.78,126.14) and (216.83,120.45) .. (240.33,120.45) .. controls (263.84,120.45) and (282.89,126.14) .. (282.89,133.16) .. controls (282.89,140.18) and (263.84,145.87) .. (240.33,145.87) .. controls (216.83,145.87) and (197.78,140.18) .. (197.78,133.16) -- cycle ;
	\draw   (212.1,119.46) -- (221.95,121.69) -- (214.26,128.23) ;
	%Straight Lines [id:da3343461623953874] 
	\draw    (240.33,132.2) -- (279.72,132.2) ;
	\draw [shift={(282.72,132.2)}, rotate = 180] [fill={rgb, 255:red, 0; green, 0; blue, 0 }  ][line width=0.08]  [draw opacity=0] (10.72,-5.15) -- (0,0) -- (10.72,5.15) -- (7.12,0) -- cycle    ;

	% Text Node
	\draw (235.2,45.6) node    {$z$};
	% Text Node
	\draw (252,138) node  [font=\tiny]  {$R_{i}$};


	\end{tikzpicture}
\end{figure}
\FloatBarrier
I vari punti del corpo avranno velocità che è tangente alla circonferenza. Essa in generale non è sempre uguale perché i punti più all'esterno devono compiere in ugual tempo una circonferenza maggiore. La quantità che si mantiene uguale è $\omega$. $\vec{\omega}$ e $\vec{R}$ sono ortogonali tra di loro.
\[
	\vec{v}_i =\vec{\omega} \times \vec{R}_i \qquad v_i=\omega \,R_i
\]
Questa condizione è quella che caratterizza la rotazione del corpo rigido. Il problema si sposta nello studiare nel tempo come variano $\omega$ e $\alpha$, parametri univoci per ogni punto del corpo.

\paragraph{Rotazioni rigide attorno ad un asse fisso in un sistema di riferimento inerziale} Si studia il moto di rotazione di un corpo rigido intorno a un asse fisso $z$. La caratteristica fondamentale nel moto di rotazione del corpo rigido intorno ad un asse è che fondamentalmente tutti i punti di esso si vanno a muovere su delle circonferenze, poste su piani paralleli tra di loro, quindi complanari. I vari punti del corpo possiedono velocità diverse tra loro, ma hanno stessa velocità angolare. Quindi, i punti più lontani dall'asse dovranno avere velocità maggiore per poter percorrere nello stesso tempo una circonferenza di raggio maggiore. Il vettore velocità angolare, parallelo all'asse $z$, diventa la caratteristica cinematica che bisogna calcolare a partire dalla conoscenza delle forze, e in particolare del momento delle forze, applicate al corpo rigido. Si utilizzeranno le equazioni cardinali della dinamica.
\begin{gather*}
	v_i = \omega \, R_i \quad \vec{\omega}_i = \vec{\omega} \;\; \forall i \qquad \vec{\omega} \parallel \vec{u}_z \\
	\vec{M}_{tot}^{(o)} = \frac{ d\vec{L}_{tot}^{(o)}  }{dt} - \vec{v}_o\times M\vec{v}_{cm}
\end{gather*}
Il polo lo si prende sull'asse quindi la sua velocità vale zero: si può eliminare il secondo termine. Ora, bisogna trovare il modo per esprimere in maniera più semplice il momento angolare complessivo del sistema di punti. Si calcola $\vec{L}_\text{tot}$:
\[
	\vec{L}_{tot}^{(o)} = \sum_i^n \vec{r}\,^{(o)}_i \times m_i\vec{v}_i
\]
Il raggio vettore $\vec{r}_i$ che congiunge il polo $o$ alla massa $i$-esima è quello in figura. Contemporaneamente si può rilevare la direzione della velocità del corpo, se esso sta ruotando intorno all'asse $z$. Ad esempio,
nel punto $A$ la velocità avrà direzione entrante al piano del foglio. Si osserva che il raggio vettore può essere visto come la somma di due contributi. In particolare modo esso è dato dalla somma vettoriale fra $\vec{R}_i$ e un secondo vettore $\vec{z}_i$ e che identifica la quota a cui si trova la massa rispetto al polo:
\[
	\vec{r}_i = \vec{z}_i +\vec{R}_i
\]
Sostituendo si ha:
\[
	\vec{L}_{tot}^{(o)} = \sum_i \left( \vec{z}_i+\vec{R}_i\right) \times m_i\vec{v}_i
\]
Applicando la proprietà distributiva:
\[
	\sum_i \left( \vec{z}_i\times m_i\vec{v}_i    \right) + \sum_i \left( \vec{R}_i\times m_i\vec{v}_i \right)
\]
Nascono due termini che sono entrambi momenti angolari.
\begin{itemize}
	\item Primo termine. $\vec{z}_i$ è diretto verso l'alto, $\vec{v}_i$ è sempre entrante nel piano del foglio. Il risultato del prodotto vettoriale è un vettore che giace sempre nel piano ortogonale all'asse $z$ e che è sempre diretto con verso centripeto a tale asse. Tale termine nella sommatoria prende il nome di \emph{momento angolare ortogonale $i$-esimo}, o anche \emph{momento angolare radiale}, perché è parallelo al raggio.
	\item Secondo termine. $\norma{\vec{R}_i}$ è la distanza assiale. $\vec{v}_i$ nel disegno è entrante al piano del foglio. Il prodotto vettoriale è quindi un vettore parallelo all'asse $z$. Il termine dentro la sommatoria lo si chiama quindi \emph{momento angolare assiale $i$-esimo}.
\end{itemize}
Il generico punto $i$ sarà dotato di un momento angolare ortogonale e uno assiale. Si considerino le masse $m_i$ ed $m_j$ come in figura.
\begin{figure}[htpb]
	\centering
	

	\tikzset{every picture/.style={line width=0.75pt}} %set default line width to 0.75pt        

	\begin{tikzpicture}[x=0.75pt,y=0.75pt,yscale=-1,xscale=1]
	%uncomment if require: \path (0,376); %set diagram left start at 0, and has height of 376

	%Straight Lines [id:da5195660360442784] 
	\draw    (255.74,330.5) -- (255.74,39.34) ;
	\draw [shift={(255.74,36.34)}, rotate = 450] [fill={rgb, 255:red, 0; green, 0; blue, 0 }  ][line width=0.08]  [draw opacity=0] (10.72,-5.15) -- (0,0) -- (10.72,5.15) -- (7.12,0) -- cycle    ;
	%Shape: Can [id:dp8455999264544007] 
	\draw   (385.39,151.22) -- (211.11,325.49) .. controls (206.08,330.52) and (183.5,316.1) .. (160.67,293.27) .. controls (137.85,270.45) and (123.43,247.86) .. (128.46,242.83) -- (302.73,68.56) .. controls (307.76,63.53) and (330.35,77.95) .. (353.17,100.77) .. controls (376,123.6) and (390.42,146.18) .. (385.39,151.22) .. controls (380.35,156.25) and (357.77,141.83) .. (334.94,119) .. controls (312.12,96.18) and (297.7,73.59) .. (302.73,68.56) ;
	%Shape: Ellipse [id:dp9697851711980743] 
	\draw  [color={rgb, 255:red, 128; green, 128; blue, 128 }  ,draw opacity=1 ] (199.45,143.06) .. controls (199.45,133.83) and (224.51,126.34) .. (255.42,126.34) .. controls (286.33,126.34) and (311.39,133.83) .. (311.39,143.06) .. controls (311.39,152.3) and (286.33,159.78) .. (255.42,159.78) .. controls (224.51,159.78) and (199.45,152.3) .. (199.45,143.06) -- cycle ;
	\draw  [color={rgb, 255:red, 128; green, 128; blue, 128 }  ,draw opacity=1 ] (218.29,125.05) -- (231.24,127.98) -- (221.12,136.58) ;
	%Straight Lines [id:da36834618504300853] 
	\draw [line width=1.5]    (255.74,114.56) -- (255.74,75.98) ;
	\draw [shift={(255.74,71.98)}, rotate = 450] [fill={rgb, 255:red, 0; green, 0; blue, 0 }  ][line width=0.08]  [draw opacity=0] (13.4,-6.43) -- (0,0) -- (13.4,6.44) -- (8.9,0) -- cycle    ;
	%Straight Lines [id:da9652713306447409] 
	\draw [line width=0.75]    (255.74,307) -- (310.43,145.9) ;
	\draw [shift={(311.39,143.06)}, rotate = 468.75] [fill={rgb, 255:red, 0; green, 0; blue, 0 }  ][line width=0.08]  [draw opacity=0] (10.72,-5.15) -- (0,0) -- (10.72,5.15) -- (7.12,0) -- cycle    ;
	%Shape: Ellipse [id:dp3060968998104503] 
	\draw  [fill={rgb, 255:red, 0; green, 0; blue, 0 }  ,fill opacity=1 ] (251.62,307) .. controls (251.62,304.73) and (253.46,302.89) .. (255.74,302.89) .. controls (258.01,302.89) and (259.85,304.73) .. (259.85,307) .. controls (259.85,309.28) and (258.01,311.12) .. (255.74,311.12) .. controls (253.46,311.12) and (251.62,309.28) .. (251.62,307) -- cycle ;
	%Shape: Rectangle [id:dp5101136247809177] 
	\draw  [fill={rgb, 255:red, 0; green, 0; blue, 0 }  ,fill opacity=1 ] (307.86,136.93) -- (315.87,136.93) -- (315.87,143.87) -- (307.86,143.87) -- cycle ;
	%Straight Lines [id:da6918942322173436] 
	\draw [line width=0.75]    (311.87,140.4) -- (311.87,108.45) ;
	\draw [shift={(311.87,105.45)}, rotate = 450] [fill={rgb, 255:red, 0; green, 0; blue, 0 }  ][line width=0.08]  [draw opacity=0] (10.72,-5.15) -- (0,0) -- (10.72,5.15) -- (7.12,0) -- cycle    ;
	%Straight Lines [id:da48439494498668956] 
	\draw [line width=0.75]    (311.87,140.4) -- (280.32,140.4) ;
	\draw [shift={(277.32,140.4)}, rotate = 360] [fill={rgb, 255:red, 0; green, 0; blue, 0 }  ][line width=0.08]  [draw opacity=0] (10.72,-5.15) -- (0,0) -- (10.72,5.15) -- (7.12,0) -- cycle    ;
	%Shape: Ellipse [id:dp7304179034589999] 
	\draw  [color={rgb, 255:red, 128; green, 128; blue, 128 }  ,draw opacity=1 ] (199.45,221.02) .. controls (199.45,211.79) and (224.51,204.3) .. (255.42,204.3) .. controls (286.33,204.3) and (311.39,211.79) .. (311.39,221.02) .. controls (311.39,230.25) and (286.33,237.74) .. (255.42,237.74) .. controls (224.51,237.74) and (199.45,230.25) .. (199.45,221.02) -- cycle ;
	\draw  [color={rgb, 255:red, 128; green, 128; blue, 128 }  ,draw opacity=1 ] (260.22,198.72) -- (271.89,205.05) -- (259.82,210.59) ;
	%Straight Lines [id:da26215718910091446] 
	\draw [line width=0.75]    (255.74,307) -- (201.1,223.53) ;
	\draw [shift={(199.45,221.02)}, rotate = 416.78999999999996] [fill={rgb, 255:red, 0; green, 0; blue, 0 }  ][line width=0.08]  [draw opacity=0] (10.72,-5.15) -- (0,0) -- (10.72,5.15) -- (7.12,0) -- cycle    ;
	%Shape: Rectangle [id:dp4523941492080834] 
	\draw  [fill={rgb, 255:red, 0; green, 0; blue, 0 }  ,fill opacity=1 ] (195.45,217.55) -- (203.46,217.55) -- (203.46,224.49) -- (195.45,224.49) -- cycle ;
	%Straight Lines [id:da3397567748160464] 
	\draw [line width=0.75]    (199.45,221.02) -- (199.45,189.07) ;
	\draw [shift={(199.45,186.07)}, rotate = 450] [fill={rgb, 255:red, 0; green, 0; blue, 0 }  ][line width=0.08]  [draw opacity=0] (10.72,-5.15) -- (0,0) -- (10.72,5.15) -- (7.12,0) -- cycle    ;
	%Straight Lines [id:da002150672325501146] 
	\draw [line width=0.75]    (199.45,221.02) -- (227.73,221.02) ;
	\draw [shift={(230.73,221.02)}, rotate = 180] [fill={rgb, 255:red, 0; green, 0; blue, 0 }  ][line width=0.08]  [draw opacity=0] (10.72,-5.15) -- (0,0) -- (10.72,5.15) -- (7.12,0) -- cycle    ;

	% Text Node
	\draw (278.86,42.03) node    {$z$};
	% Text Node
	\draw (225.22,93.01) node    {$\vec{\omega }$};
	% Text Node
	\draw (277.53,303.02) node    {$O$};
	% Text Node
	\draw (308.54,185.12) node  [font=\tiny]  {$\vec{r}_{i}$};
	% Text Node
	\draw (325.02,121.71) node  [font=\tiny]  {$\vec{L}_{z,i}$};
	% Text Node
	\draw (267.84,141.85) node  [font=\tiny]  {$\vec{L}_{\bot ,i}$};
	% Text Node
	\draw (326.74,144.84) node  [font=\tiny]  {$m_{i}$};
	% Text Node
	\draw (218.24,267.9) node  [font=\tiny]  {$\vec{r}_{j}$};
	% Text Node
	\draw (183.6,203.67) node  [font=\tiny]  {$\vec{L}_{z,j}$};
	% Text Node
	\draw (242.02,219.47) node  [font=\tiny]  {$\vec{L}_{\bot ,j}$};
	% Text Node
	\draw (184.81,228.8) node  [font=\tiny]  {$m_{j}$};


	\end{tikzpicture}
\end{figure}
\FloatBarrier
Si osservi come cambiano le direzioni dei momenti. La velocità della massa $m_j$ sarà uscente dal piano del foglio. Il momento angolare assiale è concorde a quello precedente della massa $i$-esima. Anche il suo momento ortogonale punta verso l'asse. Se si considerano tutte le masse infinitesime, si avrà un momento angolare assiale dato dalla somma di tanti contributi tutti concordi fra di loro in direzione e verso. Quindi si dovrà semplicemente sommarli scalarmente, ottenendo $\vec{L}_z$, il \textbf{momento angolare assiale}. Il termine che nasce invece sommando tutti i momenti angolari ortogonali $i$-esimi, ossia il \textbf{momento angolare radiale}, lo si indica invece con $\vec{L}_\perp$. Il momento angolare totale rispetto al polo $O$ sarà dato dalla somma di $\vec{L}_z$ e $\vec{L}_\perp$.
\[
	\boxed{\vec{L}_{tot}^{(o)} = \vec{L}_z + \vec{L}_{\bot}}
\]

\paragraph{Momento angolare assiale}
\begin{itemize}
	\item I vettori sono tutti paralleli tra loro, diretti come $\vec{u}_z$ e concordi, il prodotto vettoriale sarà dato dal prodotto fra i moduli
	\[
		\vec{L}_z = \sum_i R_i m_i v_i \vec{u}_z
	\]
	\item La velocità scalare che possiede ogni punto materiale è legata alla velocità angolare, si può sostituire a $v_i$, $\omega R_i$
	\[
		\vec{L}_z = \sum_i \left(R_i m_i\omega R_i \right) \vec{u}_z
	\]
	\item A questo punto si porta fuori $\omega$ dalla sommatoria perché è in comune a tutti gli angoli.
	\[
		\vec{L}_z = \underbrace{\left( \sum_i m_i R_i^2 \right) \cdot \omega \vec{u}_z}_{\vec{\omega} \parallel \vec{u}_z \implies \omega \cdot \vec{u}_z = \vec{\omega}} = \sum_i \left( m_i R_i^2  \right) \cdot \vec{\omega}
	\]
\end{itemize}
Il termine evidenziato dipende dalla geometria del corpo e in particolare da quanta è la sua massa e come questa è distribuita rispetto all'asse $z$. Si tratta di una quantità caratteristica del corpo esteso e prende il nome di \textbf{momento di inerzia del corpo} rispetto all'asse $z$. Viene indicato con la lettera $I$ ed è una quantità scalare. Va sottolineato sempre a pedice qual è l'asse $z$ rispetto a cui si sta indicando questa proprietà del corpo. Infatti, se si fa ruotare il corpo in un modo, attorno a un certo asse, si avrà un certo momento di inerzia. Tuttavia, se lo si fa ruotare in un altra maniera, attorno ad un asse diverso, anche il momento di inerzia sarà diverso. Esso dipende quindi anche dall'asse di rotazione.
\[
	\boxed{I_z = \sum_i^n m_i\,R_i^2} \qquad \text{Momento d'inerzia}
\]

\paragraph{Momento angolare radiale} Finora fondamentalmente è stato detto che, senza più suddividere il corpo in tante parti, esso è dotato di una componente del momento angolare totale diretto parallelamente all'asse $z$. È possibile dire lo stesso del momento angolare radiale? Ossia, si può andare a calcolare la somma vettoriale di tutti i momenti ortogonali $i$-esimi e far vedere che le varie componenti della somma hanno la stessa direzione, in modo tale da dare come risultato a loro volta un vettore avente ugual direzione? La risposta ovviamente è negativa, e ciò complica notevolmente il calcolo di tale termine. Infatti, i momenti radiali $i$-esimi sono vettori che puntano sempre verso l'asse di rotazione.
\begin{figure}[htpb]
	\centering
	

	\tikzset{every picture/.style={line width=0.75pt}} %set default line width to 0.75pt        

	\begin{tikzpicture}[x=0.75pt,y=0.75pt,yscale=-1,xscale=1]
	%uncomment if require: \path (0,300); %set diagram left start at 0, and has height of 300

	%Straight Lines [id:da12459801471491572] 
	\draw [color={rgb, 255:red, 155; green, 155; blue, 155 }  ,draw opacity=1 ]   (287.25,90) -- (230.58,90) ;
	\draw [shift={(227.58,90)}, rotate = 360] [fill={rgb, 255:red, 155; green, 155; blue, 155 }  ,fill opacity=1 ][line width=0.08]  [draw opacity=0] (10.72,-5.15) -- (0,0) -- (10.72,5.15) -- (7.12,0) -- cycle    ;
	%Shape: Ellipse [id:dp23179179531682093] 
	\draw   (170,162.25) .. controls (170,97.49) and (222.49,45) .. (287.25,45) .. controls (352.01,45) and (404.5,97.49) .. (404.5,162.25) .. controls (404.5,227.01) and (352.01,279.5) .. (287.25,279.5) .. controls (222.49,279.5) and (170,227.01) .. (170,162.25) -- cycle ;
	%Shape: Ellipse [id:dp9965082237761831] 
	\draw   (215,162.25) .. controls (215,122.35) and (247.35,90) .. (287.25,90) .. controls (327.15,90) and (359.5,122.35) .. (359.5,162.25) .. controls (359.5,202.15) and (327.15,234.5) .. (287.25,234.5) .. controls (247.35,234.5) and (215,202.15) .. (215,162.25) -- cycle ;
	%Shape: Circle [id:dp055117462910539095] 
	\draw  [fill={rgb, 255:red, 0; green, 0; blue, 0 }  ,fill opacity=1 ] (282.92,90) .. controls (282.92,87.61) and (284.86,85.67) .. (287.25,85.67) .. controls (289.64,85.67) and (291.58,87.61) .. (291.58,90) .. controls (291.58,92.39) and (289.64,94.33) .. (287.25,94.33) .. controls (284.86,94.33) and (282.92,92.39) .. (282.92,90) -- cycle ;
	%Straight Lines [id:da041846034904831075] 
	\draw    (287.25,90) -- (287.25,141.92) ;
	\draw [shift={(287.25,144.92)}, rotate = 270] [fill={rgb, 255:red, 0; green, 0; blue, 0 }  ][line width=0.08]  [draw opacity=0] (10.72,-5.15) -- (0,0) -- (10.72,5.15) -- (7.12,0) -- cycle    ;
	%Straight Lines [id:da11385728118969496] 
	\draw [color={rgb, 255:red, 155; green, 155; blue, 155 }  ,draw opacity=1 ]   (359.35,161.13) -- (359.74,104.46) ;
	\draw [shift={(359.76,101.46)}, rotate = 450.39] [fill={rgb, 255:red, 155; green, 155; blue, 155 }  ,fill opacity=1 ][line width=0.08]  [draw opacity=0] (10.72,-5.15) -- (0,0) -- (10.72,5.15) -- (7.12,0) -- cycle    ;
	%Shape: Circle [id:dp8931699931661818] 
	\draw  [fill={rgb, 255:red, 0; green, 0; blue, 0 }  ,fill opacity=1 ] (359.38,156.8) .. controls (361.78,156.81) and (363.7,158.77) .. (363.69,161.16) .. controls (363.67,163.55) and (361.72,165.48) .. (359.32,165.46) .. controls (356.93,165.45) and (355,163.49) .. (355.02,161.1) .. controls (355.04,158.71) and (356.99,156.78) .. (359.38,156.8) -- cycle ;
	%Straight Lines [id:da7193380899952875] 
	\draw    (359.35,161.13) -- (307.44,160.78) ;
	\draw [shift={(304.44,160.76)}, rotate = 360.39] [fill={rgb, 255:red, 0; green, 0; blue, 0 }  ][line width=0.08]  [draw opacity=0] (10.72,-5.15) -- (0,0) -- (10.72,5.15) -- (7.12,0) -- cycle    ;

	% Text Node
	\draw (288.67,72.67) node    {$m_{j}$};
	% Text Node
	\draw (266.67,116) node    {$\vec{L}_{\bot ,j}$};
	% Text Node
	\draw (330.67,142) node    {$\vec{L}_{\bot ,i}$};
	% Text Node
	\draw (378,159.33) node    {$m_{i}$};


	\end{tikzpicture}
\end{figure}
\FloatBarrier
Si immagini che a fianco sia raffigurato il cilindro disegnato in precedenza, visto dall'alto. Il problema è che se bisogna sommare tutti i vettori della forma che si osserva a lato, essi non hanno stessa direzione e il calcolo è più complicato.

Otteniamo quindi che
\[
	\vec{L}_{\bot} = \sum_i z_i m_i v_i \vec{u}_{\bot,i} = \sum_i z_i m_i\omega R_i\vec{u}_{\bot,i} = \sum_i (z_i m_i R_i \vec{u}_{\bot,i})\omega
\]
Il calcolo è complesso, ma si può notare una cosa importante quando l'asse di rotazione è un asse di simmetria per il corpo, ed esso è omogeneo.
\begin{figure}[htpb]
	\centering
	

	\tikzset{every picture/.style={line width=0.75pt}} %set default line width to 0.75pt        

	\begin{tikzpicture}[x=0.75pt,y=0.75pt,yscale=-1,xscale=1]
	%uncomment if require: \path (0,300); %set diagram left start at 0, and has height of 300

	%Straight Lines [id:da4575961066685954] 
	\draw    (263.5,252) -- (263.5,85) ;
	\draw [shift={(263.5,82)}, rotate = 450] [fill={rgb, 255:red, 0; green, 0; blue, 0 }  ][line width=0.08]  [draw opacity=0] (10.72,-5.15) -- (0,0) -- (10.72,5.15) -- (7.12,0) -- cycle    ;
	%Shape: Can [id:dp23997666541951923] 
	\draw   (293,111) -- (293,229) .. controls (293,233.97) and (279.57,238) .. (263,238) .. controls (246.43,238) and (233,233.97) .. (233,229) -- (233,111) .. controls (233,106.03) and (246.43,102) .. (263,102) .. controls (279.57,102) and (293,106.03) .. (293,111) .. controls (293,115.97) and (279.57,120) .. (263,120) .. controls (246.43,120) and (233,115.97) .. (233,111) ;
	%Straight Lines [id:da9486642752628265] 
	\draw [line width=1.5]    (263.5,196.5) -- (263.5,158.5) ;
	\draw [shift={(263.5,154.5)}, rotate = 450] [fill={rgb, 255:red, 0; green, 0; blue, 0 }  ][line width=0.08]  [draw opacity=0] (13.4,-6.43) -- (0,0) -- (13.4,6.44) -- (8.9,0) -- cycle    ;

	% Text Node
	\draw (279.5,78.5) node    {$z$};
	% Text Node
	\draw (277,187) node    {$\vec{L}_{z}$};


	\end{tikzpicture}
\end{figure}
\FloatBarrier
Infatti, considerando soltanto la componente $\vec{L}_\perp$, presa la massa $i$-esima che avrà un certo momento ortogonale $i$-esimo, esisterà sempre una massa $m_j$ posta in posizione simmetrica alla massa $m_i$ tale per cui nascerà la componente radiale del momento angolare $j$-esima che andrà annullare quella $i$-esima. La massa $j$ infatti sarà alla stessa quota di $i$ (stesso $z_i$), avrà raggio $R_i$ e il suo momento $\vec{L}_j$ sarà uguale e opposto a $\vec{L}_i$. Allora Il momento angolare totale è formato soltanto dalla componente assiale.

Se $\vec{L}_\perp$ è diverso da zero, si può dimostrare che fondamentalmente è una componente che punterà sempre verso l'asse facendo la somma vettoriale totale. $\vec{L}_\text{tot}$ sarà allora diretto come in figura.
\begin{figure}[htpb]
	\centering
	

	\tikzset{every picture/.style={line width=0.75pt}} %set default line width to 0.75pt        

	\begin{tikzpicture}[x=0.75pt,y=0.75pt,yscale=-1,xscale=1]
	%uncomment if require: \path (0,300); %set diagram left start at 0, and has height of 300

	%Straight Lines [id:da3478243083512724] 
	\draw    (263.5,239) -- (263.5,67) ;
	\draw [shift={(263.5,64)}, rotate = 450] [fill={rgb, 255:red, 0; green, 0; blue, 0 }  ][line width=0.08]  [draw opacity=0] (10.72,-5.15) -- (0,0) -- (10.72,5.15) -- (7.12,0) -- cycle    ;
	%Shape: Can [id:dp3112940153689916] 
	\draw   (325.93,149.49) -- (242.49,232.93) .. controls (238.98,236.45) and (226.63,229.8) .. (214.92,218.08) .. controls (203.2,206.37) and (196.55,194.02) .. (200.07,190.51) -- (283.51,107.07) .. controls (287.02,103.55) and (299.37,110.2) .. (311.08,121.92) .. controls (322.8,133.63) and (329.45,145.98) .. (325.93,149.49) .. controls (322.42,153.01) and (310.07,146.36) .. (298.36,134.64) .. controls (286.64,122.93) and (279.99,110.58) .. (283.51,107.07) ;
	%Straight Lines [id:da11426354012586315] 
	\draw [line width=1.5]    (263.5,178.5) -- (263.5,140.5) ;
	\draw [shift={(263.5,136.5)}, rotate = 450] [fill={rgb, 255:red, 0; green, 0; blue, 0 }  ][line width=0.08]  [draw opacity=0] (13.4,-6.43) -- (0,0) -- (13.4,6.44) -- (8.9,0) -- cycle    ;
	%Straight Lines [id:da11117414316202523] 
	\draw [line width=1.5]    (290.43,142.57) -- (328.91,104.09) ;
	\draw [shift={(331.74,101.26)}, rotate = 495] [fill={rgb, 255:red, 0; green, 0; blue, 0 }  ][line width=0.08]  [draw opacity=0] (13.4,-6.43) -- (0,0) -- (13.4,6.44) -- (8.9,0) -- cycle    ;
	%Straight Lines [id:da966548557557317] 
	\draw [line width=1.5]    (190.67,183.93) -- (253.07,183.93) ;
	\draw [shift={(257.07,183.93)}, rotate = 180] [fill={rgb, 255:red, 0; green, 0; blue, 0 }  ][line width=0.08]  [draw opacity=0] (13.4,-6.43) -- (0,0) -- (13.4,6.44) -- (8.9,0) -- cycle    ;
	%Straight Lines [id:da2931254655420723] 
	\draw [line width=1.5]    (263.5,109.67) -- (263.5,85.83) ;
	\draw [shift={(263.5,81.83)}, rotate = 450] [fill={rgb, 255:red, 0; green, 0; blue, 0 }  ][line width=0.08]  [draw opacity=0] (13.4,-6.43) -- (0,0) -- (13.4,6.44) -- (8.9,0) -- cycle    ;

	% Text Node
	\draw (277.5,60.5) node    {$z$};
	% Text Node
	\draw (281.67,152) node    {$\vec{L}_{z}$};
	% Text Node
	\draw (344.33,121) node    {$\vec{L}_{\text{tot}}$};
	% Text Node
	\draw (197,165.67) node    {$\vec{L}_{\bot }$};
	% Text Node
	\draw (245.83,95.17) node    {$\vec{\omega }$};


	\end{tikzpicture}
\end{figure}
\FloatBarrier
In generale, il momento angolare complessivo del corpo, non è costante, ma ruota, si muove durante il moto. L'informazione data dal momento angolare totale non è utile per studiare come varia nel tempo la velocità angolare ma lo è solo per capire l'effetto delle forze di vincolo che permettono al corpo di ruotare rimanendo vincolato all'asse.

Concentrandosi sul momento angolare assiale in particolare, a questo punto bisogna collegare questa proprietà cinematica del corpo all'effetto dinamico delle forze, con la seconda equazione cardinale della dinamica. Si considererà soltanto i momenti delle forze assiali, cioè quelle che danno luogo a un momento parallelo all'asse $z$, tali per cui vale la seconda equazione cardinale della dinamica. Si deriva di tale quantità.
\[
	\vec{M}^{(o)}_z = \frac{d\vec{L}_z }{dt} = \frac{d(I_z\vec{\omega}) }{dt} = I_z \vec{\alpha}
\]
Derivare è semplice perché $I_z$ non varia mentre il corpo si muove, perciò si può portarlo fuori dalla derivata. Rimane la derivata del vettore velocità angolare nel tempo che è pari al vettore accelerazione angolare. Questa allora diventa la relazione della dinamica. Si può quindi calcolare i momenti delle forze diretti lungo l'asse $z$ e a questo punto si sarà in grado di ricavare qual è l'accelerazione angolare del corpo. Fondamentalmente è l'equivalente di utilizzare la seconda legge della dinamica per un punto materiale.
\[
	\boxed{\vec{M}_z = I_z\vec{\alpha}} \qquad \text{Legge fondamentale della dinamica}
\]
\begin{table}[H]
		\centering
		\begin{tabular}{c|c}
			Punto materiale & Corpo rigido in rotazione \\
			\hline
			$m$ & $I_z$ \\
			$\vec{p}$ & $\vec{L}$ \\
			$\vec{F} = \frac{d\vec{p} }{dt} = m\vec{a}$ & $\vec{M}^{(o)} = \frac{d\vec{L} }{dt}$ \\
			$\vec{F}_t = m\vec{a}_t$ & $\vec{M}_z=I_z\vec{\alpha}$ \\
			$\vec{F}_n = m\vec{a}_n$ & $\vec{M}_{\bot} = \frac{d\vec{L}_{\bot} }{dt}$
		\end{tabular}
	\end{table}
È utile fare un'analogia fra le proprietà introdotte per studiare il moto del punto materiale e quelle per il moto del corpo rigido in rotazione.
\begin{table}[H]
	\centering
	\begin{tabular}{|p{6.5cm}|p{6.5cm}|}
		\hline
		Punto materiale & Corpo rigido in rotazione \\
		\hline
		Il punto materiale lo si caratterizza con la \emph{massa inerziale}. Essa rappresenta l'inerzia del sistema, ossia la fatica che fa il corpo a cambiare il suo stato di moto rettilineo uniforme. & Ciò che caratterizza la fatica del corpo a essere messo in rotazione o, se si muove di moto uniforme, a far variare la velocità di rotazione, è il \emph{momento di inerzia}. Non basta solo la massa, ma conta come essa è distribuita intorno all'asse $x$. Dipende dalla forma del corpo e da come essa è posta rispetto all'asse $z$. \\
		\hline
		La \emph{quantità di moto} è l'inerzia di un corpo per la sua velocità. & La proprietà cinematica fondamentale di un corpo rigido in rotazione è il \emph{momento angolare}.\\
		\hline
		La \emph{seconda legge della dinamica} dice che la risultante delle forze provoca una variazione nel tempo della quantità di moto. & La variazione di $\vec{L}$ nel tempo è data dalla \emph{seconda equazione cardinale della dinamica}. Bisogna applicare una forza con momento tale da far variare il momento angolare, per far ruotare il corpo.\\
		\hline
		L'effetto delle forze che agiscono in direzione parallela alla traiettoria è diverso da quelle che agiscono in direzione tangente alla traiettoria. Le \emph{forze tangenziali} provocano un accelerazione tangente, fanno variare il modulo della velocità. & I momenti delle forze hanno effetti diversi a seconda di come sono diretti, in particolare i \emph{momenti assiali} delle forze sono quelli che hanno l'effetto di far variare il modulo della velocità angolare. \\
		\hline
		Le \emph{forze centripete} non fanno variare il modulo della velocità, ma ne fanno variare la direzione, permettono al corpo di percorrere una traiettoria curva. & Quelle forze che danno luogo a un \emph{momento ortogonale} all'asse $z$ provocano la variazione del momento angolare ortogonale nel tempo. Infatti, il momento angolare totale non è costante in direzione ma cambia.\\
		\hline
	\end{tabular}
\end{table}
Si vede ora come si calcola il momento di inerzia di un corpo.  Dato un corpo di forma estesa e definito un asse $z$ intorno a cui lo si vede ruotare, il momento di inerzia è definito come la somma di tutte le masse $i$-esime per il quadrato della distanza scalare fra la massa $i$-esima e l'asse $z$.
\[
	\sum_i m_i R_i^2  = I_z
\]
Questa è la forma che si userebbe se il corpo fosse una composizione discreta di masse. Se invece è un continuo, bisogna fare la somma continua fra tutte le masse infinitesime $dm$ per il quadrato della loro distanza. Questa somma continua la si estende a tutto il volume del corpo. Si avrà:
\[
	I_z = \sum_i m_i R_i^2  = \int_{\text{volume} }dm\,R^2 = \left\{ \begin{array}{l}
	 	\int_{\text{volume}} \rho R^2 dV \\
		\int_{\text{superficie}} \sigma R^2 dS \\
		\int_{\text{linea}} \lambda R^2 dl
	\end{array} \right.
\]
Dimensionalmente, il momento di inerzia è una massa per una distanza al quadrato. Per corpi di forme note o standard, esistono delle tabelle che riportano il momento di inerzia di corpi di forme standard per possibili assi $z$. Richiamiamo il momento di inerzia per alcuni di questi corpi.

\paragraph{Momento di inerzia di un anello} Si immagini di avere un anello di raggio $R$ e massa $M$.
\begin{figure}[htpb]
	\centering
	

	\tikzset{every picture/.style={line width=0.75pt}} %set default line width to 0.75pt        

	\begin{tikzpicture}[x=0.75pt,y=0.75pt,yscale=-1,xscale=1]
	%uncomment if require: \path (0,300); %set diagram left start at 0, and has height of 300

	%Shape: Circle [id:dp5232216239703855] 
	\draw  [line width=2.25]  (228,151.75) .. controls (228,107.15) and (264.15,71) .. (308.75,71) .. controls (353.35,71) and (389.5,107.15) .. (389.5,151.75) .. controls (389.5,196.35) and (353.35,232.5) .. (308.75,232.5) .. controls (264.15,232.5) and (228,196.35) .. (228,151.75) -- cycle ;
	%Straight Lines [id:da03854700022051927] 
	\draw    (308.75,151.75) -- (367.5,98) ;

	% Text Node
	\draw (334,111) node    {$R$};


	\end{tikzpicture}
\end{figure}
\FloatBarrier
Essa è distribuita solo sulla circonferenza. Bisogna capire qual è il suo momento di inerzia rispetto all'asse ortogonale al piano del foglio e passante per il centro dell'anello.
\[
	I_z = \int_{\text{linea}} dm\,R^2 = R^2 \int_{\text{linea}}dm = MR^2
\]

\paragraph{Momento di inerzia di un disco} Si ha un disco di massa $M$ e raggio $R$. Il momento di inerzia è, presa una massa infinitesima posta sul disco:
\[
	I_z = \int dm\,r^2
\]
Non si ha più $R$, perché in generale la distanza di ogni punto dal centro è diversa. Ogni punto dell'anello è alla stessa distanza dall'asse di rotazione.
\begin{figure}[htpb]
	\centering
	

	\tikzset{every picture/.style={line width=0.75pt}} %set default line width to 0.75pt        

	\begin{tikzpicture}[x=0.75pt,y=0.75pt,yscale=-1,xscale=1]
	%uncomment if require: \path (0,300); %set diagram left start at 0, and has height of 300

	%Shape: Circle [id:dp42752566168779715] 
	\draw  [color={rgb, 255:red, 0; green, 0; blue, 0 }  ,draw opacity=1 ][fill={rgb, 255:red, 222; green, 222; blue, 222 }  ,fill opacity=1 ][line width=2.25]  (228,151.75) .. controls (228,107.15) and (264.15,71) .. (308.75,71) .. controls (353.35,71) and (389.5,107.15) .. (389.5,151.75) .. controls (389.5,196.35) and (353.35,232.5) .. (308.75,232.5) .. controls (264.15,232.5) and (228,196.35) .. (228,151.75) -- cycle ;
	%Straight Lines [id:da7503358431498248] 
	\draw    (308.75,151.75) -- (308.75,71) ;
	%Straight Lines [id:da6185009717920751] 
	\draw    (308.75,151.75) -- (346.25,119.5) ;
	%Shape: Circle [id:dp4717670858472307] 
	\draw  [color={rgb, 255:red, 128; green, 128; blue, 128 }  ,draw opacity=1 ][line width=3]  (258.75,151.75) .. controls (258.75,124.14) and (281.14,101.75) .. (308.75,101.75) .. controls (336.36,101.75) and (358.75,124.14) .. (358.75,151.75) .. controls (358.75,179.36) and (336.36,201.75) .. (308.75,201.75) .. controls (281.14,201.75) and (258.75,179.36) .. (258.75,151.75) -- cycle ;

	% Text Node
	\draw (318,80.5) node    {$R$};
	% Text Node
	\draw (334,141) node    {$r$};


	\end{tikzpicture}
\end{figure}
\FloatBarrier
È necessario in questo caso risolvere un integrale di superficie, quindi doppio. Si può quindi ragionare come segue. Si pensa il disco come costituito da tante corone circolari (anelli di spessore infinitesimo) di raggio $r$ a cui è associato il momento di inerzia:
\[
	I_\text{z, corona}=M_\text{corona}r^2
\]
La massa su una corona circolare è data da:
\[
	\frac{\text{Area}_\text{corona}}{\text{Area}_\text{tot}}\, M=\frac{2\pi r\,dr}{\pi R^2} M
\]
Quindi:
\[
	I_\text{z, corona}=\frac{2 r\,dr}{R^2}M r^2
\]
Il momento di inerzia del disco, essendo una quantità additiva, è dato dalla somma dei momenti di inerzia associati alle singole corone, i cui raggi variano con continuità da 0 a $R$:
\[
	i_z=\int_0^R \frac{2}{R^2}r^3M\,dr=\frac{2}{R^2}M\biggl[\frac{r^4}{4} \biggr]_0^R=\frac{1}{2}MR^2
\]
La cosa interessante di questo risultato è che a pari massa e a pari raggio un disco presenta un momento di inerzia inferiore di un anello. Questo vuol dire che a pari forza applicata, o a pari momento di forze applicate, il disco accelererà con accelerazione maggiore dell'anello. Questo lo si può dimostrare. Se si prende un piano inclinato e ci si fa scendere un anello di massa $m$ e raggio $R$ e poi un disco di stessa massa e raggio, il disco arriva più velocemente dell'anello proprio perché ha un'inerzia inferiore.
Nell'anello la massa è tutta concentrata sul perimetro. Lo si fa muovere con una certa $\omega$, tutti i punti di esso possiedono velocità lineare pari a $\omega R$, si dovranno applicare delle forze con momento tale da riuscire a dare ad ognuno di questi punti tale velocità. Se si prende il disco, la massa è più distribuita.  Un punto all'estremo del disco deve avere velocità $\omega R$. Ma la massa in corrispondenza di $R/2$ avrà velocità pari a $\omega R/2$. Quella al centro addirittura non ha velocità. Questo vuole dire che l'oggetto ha un'inerzia minore, perché avendo meno massa più lontana dall'asse di rotazione, le forze devono agire per imporre velocità nei vari punti che sono minori di quelli che vanno imposte ai punti del perimetro. L'inerzia di un corpo rigido in rotazione, è tanto più maggiore quanto la massa è distribuita lontano dall'asse di rotazione.

\paragraph{Momento di inerzia di un'asta} Si ha un'asta di lunghezza $D$ che si vuole mettere in rotazione attorno a un asse $z$ passante per il suo centro e ortogonale a $D$.
\begin{figure}[htpb]
	\centering


	\tikzset{every picture/.style={line width=0.75pt}} %set default line width to 0.75pt        

	\begin{tikzpicture}[x=0.75pt,y=0.75pt,yscale=-1,xscale=1]
	%uncomment if require: \path (0,300); %set diagram left start at 0, and has height of 300

	%Straight Lines [id:da6939231860886572] 
	\draw    (161,211) -- (477.5,211) ;
	\draw [shift={(480.5,211)}, rotate = 180] [fill={rgb, 255:red, 0; green, 0; blue, 0 }  ][line width=0.08]  [draw opacity=0] (10.72,-5.15) -- (0,0) -- (10.72,5.15) -- (7.12,0) -- cycle    ;
	%Straight Lines [id:da49746209378373907] 
	\draw    (319,204) -- (319,219) ;
	%Shape: Rectangle [id:dp023827527134408832] 
	\draw  [draw opacity=0][fill={rgb, 255:red, 222; green, 222; blue, 222 }  ,fill opacity=1 ][line width=1.5]  (196.67,156.67) -- (440.67,156.67) -- (440.67,173.33) -- (196.67,173.33) -- cycle ;
	%Straight Lines [id:da04830795019727763] 
	\draw [color={rgb, 255:red, 155; green, 155; blue, 155 }  ,draw opacity=1 ]   (196.67,143.67) -- (440.67,143.67) ;
	\draw [shift={(440.67,143.67)}, rotate = 180] [color={rgb, 255:red, 155; green, 155; blue, 155 }  ,draw opacity=1 ][line width=0.75]    (0,5.59) -- (0,-5.59)   ;
	\draw [shift={(196.67,143.67)}, rotate = 180] [color={rgb, 255:red, 155; green, 155; blue, 155 }  ,draw opacity=1 ][line width=0.75]    (0,5.59) -- (0,-5.59)   ;
	%Straight Lines [id:da9433065630599604] 
	\draw    (319,157) -- (319,173.29) ;
	%Shape: Rectangle [id:dp9291189988064272] 
	\draw  [draw opacity=0][fill={rgb, 255:red, 128; green, 128; blue, 128 }  ,fill opacity=1 ] (375.4,156.67) -- (389.13,156.67) -- (389.13,173.33) -- (375.4,173.33) -- cycle ;
	%Straight Lines [id:da7586638771789609] 
	\draw [color={rgb, 255:red, 155; green, 155; blue, 155 }  ,draw opacity=1 ]   (318.67,184.8) -- (375.67,184.8) ;
	\draw [shift={(375.67,184.8)}, rotate = 180] [color={rgb, 255:red, 155; green, 155; blue, 155 }  ,draw opacity=1 ][line width=0.75]    (0,5.59) -- (0,-5.59)   ;
	\draw [shift={(318.67,184.8)}, rotate = 180] [color={rgb, 255:red, 155; green, 155; blue, 155 }  ,draw opacity=1 ][line width=0.75]    (0,5.59) -- (0,-5.59)   ;
	%Shape: Rectangle [id:dp7061108287025681] 
	\draw  [line width=1.5]  (196.67,156.67) -- (440.67,156.67) -- (440.67,173.33) -- (196.67,173.33) -- cycle ;

	% Text Node
	\draw (318,227.67) node    {$O$};
	% Text Node
	\draw (487.33,218.33) node    {$x$};
	% Text Node
	\draw (318.5,132.17) node    {$D$};
	% Text Node
	\draw (350.63,190.67) node  [font=\normalsize]  {$x$};
	% Text Node
	\draw (177.5,165.17) node    {$M$};


	\end{tikzpicture}
\end{figure}
\FloatBarrier
Per definire la distanza assiale di una generica massa $i$-esima si definisce un asse parallelo alla lunghezza dell'asta. L'origine la si pone in corrispondente all'asse. La distanza assiale è proprio la variabile $x$. Si ha $dm=\lambda dx$.
\[
	I_z = \int_{-D/2}^{D/2} dm\,x^2 = \int_{-D/2}^{D/2} \lambda x^2 dx
\]
Diamo per scontato che l'asta sia omogenea, quindi con densità lineare costante $\lambda=M/D$.
\[
	I_z = \frac{M}{D} \left[ \frac{x^3 }{3} \right]_{-D/2}^{D/2} = \frac{M}{3D}\frac{D^3 }{8}2 = \frac{1}{12}MD^2
\]
Se la stessa asta la si fa ruotare intorno a un asse $z$ passante per l'estremo dell'asta, $I_z$ aumenta perché si ha più massa lontana dall'asse di rotazione.
\[
	I_z = \int_0^D dm\,x^2 = \frac{M}{D}\left[ \frac{x^2 }{3} \right]_0^D = \frac{M}{D}\frac{D^3 }{3}= \frac{1}{3} MD^2
\]

\paragraph{La macchina di Atwood} Si cerca in questo esempio di capire come delle forze provocano l'accelerazione angolare del corpo rigido.
\begin{figure}[htpb]
	\centering
	

	\tikzset{every picture/.style={line width=0.75pt}} %set default line width to 0.75pt        

	\begin{tikzpicture}[x=0.75pt,y=0.75pt,yscale=-1,xscale=1]
	%uncomment if require: \path (0,476); %set diagram left start at 0, and has height of 476

	%Shape: Circle [id:dp287190602871209] 
	\draw  [draw opacity=0][fill={rgb, 255:red, 209; green, 209; blue, 209 }  ,fill opacity=1 ] (231,187) .. controls (231,148.34) and (262.34,117) .. (301,117) .. controls (339.66,117) and (371,148.34) .. (371,187) .. controls (371,225.66) and (339.66,257) .. (301,257) .. controls (262.34,257) and (231,225.66) .. (231,187) -- cycle ;
	%Shape: Circle [id:dp577091868880393] 
	\draw  [fill={rgb, 255:red, 155; green, 155; blue, 155 }  ,fill opacity=1 ] (341,347) .. controls (341,330.43) and (354.43,317) .. (371,317) .. controls (387.57,317) and (401,330.43) .. (401,347) .. controls (401,363.57) and (387.57,377) .. (371,377) .. controls (354.43,377) and (341,363.57) .. (341,347) -- cycle ;
	%Shape: Circle [id:dp7005266442079385] 
	\draw  [fill={rgb, 255:red, 155; green, 155; blue, 155 }  ,fill opacity=1 ] (211,307) .. controls (211,295.95) and (219.95,287) .. (231,287) .. controls (242.05,287) and (251,295.95) .. (251,307) .. controls (251,318.05) and (242.05,327) .. (231,327) .. controls (219.95,327) and (211,318.05) .. (211,307) -- cycle ;
	%Straight Lines [id:da31695538097494946] 
	\draw    (231,187) -- (231,287) ;
	%Straight Lines [id:da7585268171258928] 
	\draw    (371,187) -- (371,317) ;
	%Straight Lines [id:da32083901818240124] 
	\draw [line width=1.5]    (301,187) -- (301,91) ;
	\draw [shift={(301,87)}, rotate = 450] [fill={rgb, 255:red, 0; green, 0; blue, 0 }  ][line width=0.08]  [draw opacity=0] (13.4,-6.43) -- (0,0) -- (13.4,6.44) -- (8.9,0) -- cycle    ;
	%Straight Lines [id:da5376464248095933] 
	\draw [line width=1.5]    (301,187) -- (301,283) ;
	\draw [shift={(301,287)}, rotate = 270] [fill={rgb, 255:red, 0; green, 0; blue, 0 }  ][line width=0.08]  [draw opacity=0] (13.4,-6.43) -- (0,0) -- (13.4,6.44) -- (8.9,0) -- cycle    ;
	%Shape: Circle [id:dp10748121142795108] 
	\draw  [fill={rgb, 255:red, 0; green, 0; blue, 0 }  ,fill opacity=1 ] (296,187) .. controls (296,184.24) and (298.24,182) .. (301,182) .. controls (303.76,182) and (306,184.24) .. (306,187) .. controls (306,189.76) and (303.76,192) .. (301,192) .. controls (298.24,192) and (296,189.76) .. (296,187) -- cycle ;
	%Straight Lines [id:da3684136627430463] 
	\draw [line width=1.5]    (371,187) -- (371,243) ;
	\draw [shift={(371,247)}, rotate = 270] [fill={rgb, 255:red, 0; green, 0; blue, 0 }  ][line width=0.08]  [draw opacity=0] (13.4,-6.43) -- (0,0) -- (13.4,6.44) -- (8.9,0) -- cycle    ;
	%Straight Lines [id:da18692534139299388] 
	\draw [line width=1.5]    (371,317) -- (371,261) ;
	\draw [shift={(371,257)}, rotate = 450] [fill={rgb, 255:red, 0; green, 0; blue, 0 }  ][line width=0.08]  [draw opacity=0] (13.4,-6.43) -- (0,0) -- (13.4,6.44) -- (8.9,0) -- cycle    ;
	%Straight Lines [id:da10085015388066632] 
	\draw [line width=1.5]    (231,187) -- (231,223) ;
	\draw [shift={(231,227)}, rotate = 270] [fill={rgb, 255:red, 0; green, 0; blue, 0 }  ][line width=0.08]  [draw opacity=0] (13.4,-6.43) -- (0,0) -- (13.4,6.44) -- (8.9,0) -- cycle    ;
	%Straight Lines [id:da8881358901703795] 
	\draw [line width=1.5]    (231,287) -- (231,261) ;
	\draw [shift={(231,257)}, rotate = 450] [fill={rgb, 255:red, 0; green, 0; blue, 0 }  ][line width=0.08]  [draw opacity=0] (13.4,-6.43) -- (0,0) -- (13.4,6.44) -- (8.9,0) -- cycle    ;
	%Straight Lines [id:da4824151405818853] 
	\draw [line width=1.5]    (231,353) -- (231,327) ;
	\draw [shift={(231,357)}, rotate = 270] [fill={rgb, 255:red, 0; green, 0; blue, 0 }  ][line width=0.08]  [draw opacity=0] (13.4,-6.43) -- (0,0) -- (13.4,6.44) -- (8.9,0) -- cycle    ;
	%Straight Lines [id:da17712756501560833] 
	\draw [line width=1.5]    (371,433) -- (371,377) ;
	\draw [shift={(371,437)}, rotate = 270] [fill={rgb, 255:red, 0; green, 0; blue, 0 }  ][line width=0.08]  [draw opacity=0] (13.4,-6.43) -- (0,0) -- (13.4,6.44) -- (8.9,0) -- cycle    ;
	%Straight Lines [id:da049600524167557225] 
	\draw [color={rgb, 255:red, 155; green, 155; blue, 155 }  ,draw opacity=1 ][line width=0.75]    (181,287) -- (181,240) ;
	\draw [shift={(181,237)}, rotate = 450] [fill={rgb, 255:red, 155; green, 155; blue, 155 }  ,fill opacity=1 ][line width=0.08]  [draw opacity=0] (10.72,-5.15) -- (0,0) -- (10.72,5.15) -- (7.12,0) -- cycle    ;
	%Straight Lines [id:da43365541795855966] 
	\draw [color={rgb, 255:red, 155; green, 155; blue, 155 }  ,draw opacity=1 ][line width=0.75]    (441,369) -- (441,322) ;
	\draw [shift={(441,372)}, rotate = 270] [fill={rgb, 255:red, 155; green, 155; blue, 155 }  ,fill opacity=1 ][line width=0.08]  [draw opacity=0] (10.72,-5.15) -- (0,0) -- (10.72,5.15) -- (7.12,0) -- cycle    ;
	%Shape: Arc [id:dp041555453597121206] 
	\draw  [draw opacity=0] (323.9,110.33) .. controls (345.63,116.81) and (363.48,132.27) .. (373.15,152.4) -- (301,187) -- cycle ; \draw   (323.9,110.33) .. controls (345.63,116.81) and (363.48,132.27) .. (373.15,152.4) ;
	\draw  [fill={rgb, 255:red, 0; green, 0; blue, 0 }  ,fill opacity=1 ] (374.12,139.89) -- (375.81,155.25) -- (362.51,147.38) -- (372.06,149.45) -- cycle ;
	%Shape: Circle [id:dp6438603490091646] 
	\draw  [fill={rgb, 255:red, 0; green, 0; blue, 0 }  ,fill opacity=1 ] (229,187) .. controls (229,185.9) and (229.9,185) .. (231,185) .. controls (232.1,185) and (233,185.9) .. (233,187) .. controls (233,188.1) and (232.1,189) .. (231,189) .. controls (229.9,189) and (229,188.1) .. (229,187) -- cycle ;
	%Shape: Circle [id:dp25257009832339206] 
	\draw  [fill={rgb, 255:red, 0; green, 0; blue, 0 }  ,fill opacity=1 ] (369,187) .. controls (369,185.9) and (369.9,185) .. (371,185) .. controls (372.1,185) and (373,185.9) .. (373,187) .. controls (373,188.1) and (372.1,189) .. (371,189) .. controls (369.9,189) and (369,188.1) .. (369,187) -- cycle ;
	%Straight Lines [id:da996236735362686] 
	\draw [line width=0.75]  [dash pattern={on 0.84pt off 2.51pt}]  (251.5,137.5) -- (301,187) ;

	% Text Node
	\draw (339.4,157) node    {$M$};
	% Text Node
	\draw (284.5,93) node    {$\vec{\Phi }$};
	% Text Node
	\draw (319.5,279.8) node    {$\vec{P}_{3}$};
	% Text Node
	\draw (390,214.5) node    {$\vec{T}_{2}$};
	% Text Node
	\draw (390,291.5) node    {$\vec{T}_{2}$};
	% Text Node
	\draw (214,212.5) node    {$\vec{T}_{1}$};
	% Text Node
	\draw (215,267.5) node    {$\vec{T}_{1}$};
	% Text Node
	\draw (231.5,307) node    {$m_{1}$};
	% Text Node
	\draw (372.5,347) node    {$m_{2}$};
	% Text Node
	\draw (215,340.5) node    {$\vec{P}_{1}$};
	% Text Node
	\draw (354,398.5) node    {$\vec{P}_{2}$};
	% Text Node
	\draw (171,261) node    {$\vec{a}$};
	% Text Node
	\draw (431,346) node    {$\vec{a}$};
	% Text Node
	\draw (364,117) node    {$\alpha $};
	% Text Node
	\draw (215.2,180.4) node    {$A$};
	% Text Node
	\draw (385.6,180) node    {$B$};
	% Text Node
	\draw (263.4,163.8) node    {$R$};


	\end{tikzpicture}
\end{figure}
\FloatBarrier
La macchina di Atwood è semplicemente una carrucola: essa è costituita da due oggetti di massa $m_1$ e $m_2$ connessi da un filo inestensibile posto sopra una carrucola. Essa è pesante, ha una sua inerzia e per essere messa in rotazione richiede una certa forza. La carrucola è un disco di massa $M$ e raggio $R$ che viene incerniata rispetto al suo centro e può ruotare intorno ad esso. Questo vuole dire che allora essa potrà ruotare intorno all'asse $z$ ortogonale al piano del foglio passante per il centro della carrucola. Essa da ferma si mette ad accelerare in verso orario. $m_1$ sale e $m_2$ scende. La fune è di massa trascurabile e quindi l'accelerazione per tutti i punti, anche quella dei due corpi, è la stessa. Il corpo di massa $m_1$ vorrebbe cadere verso il basso a causa della forza peso ma è sostenuto dalla tensione della fune che lo porta verso l'altro.
\[
	T_1 - m_1 g = m_1 a
\]
Analogamente $m_2$ è mossa verso il basso dalla forza peso e sostenuta dalla tensione della fune. Si avrà:
\[
	m_2 g - T_2 = m_2 a
\]
$m_1$ genera la tensione $\vec{T}_1$. Si sa che essa si propaga inalterata fino al punto A della carrucola. Essa vorrebbe essere spinta verso il basso dalla tensione $\vec{T}_1$. D'altra parte anche sul corpo $m_2$ agisce $\vec{T}_2$ che si propaga inalterata fino a $B$. Per il principio di azione e reazione si vede che la carrucola è soggetta anche ad una tensione $\vec{T}_2$. Si consideri nel disegno solo le forze applicate alla carrucola. C'è una reazione vincolare generata dall'asse, dal sostegno che fa in modo che la carrucola rimanga ferma, non cada. Queste sono le uniche forze applicate ad essa. Si noti che sono state impostate le tensioni diverse. Se le tensioni fossero uguali la carrucola non ruoterebbe. Se il corpo in rotazione è dotato di massa, la sua inerzia fa variare la tensione della fune.

Si impone il fatto che le forze si devono bilanciare in verticale:
\[
	R_{tot} = 0 \implies \Phi - T_1 - T_2 - Mg = 0
\]
Si calcolano i momenti di tutte le forze in direzione assiale.
\[
	M_z = T_2 R - T_1 R = I_z\alpha
\]
Il legame che manca fra le relazioni scritte per poter risolvere il problema è che l'accelerazione angolare non è indipendente dall'accelerazione linerare con cui si muovono i corpi. Aggiungendo l'informazione che $a=\alpha R$, si risolve il sistema.
\begin{gather*}
	T_2\frac{R}{R} - T_1 \frac{R}{R} =  \frac{I_z \alpha  }{R} \\
	m_2 g - m_1 g = (m_1+m_2)\alpha R + \frac{I_z\alpha  }{R}
\end{gather*}
\[
	\boxed{\alpha = \frac{g(m_2-m_1)}{m_1+m_2+ \frac{I_z }{R^2}}\frac{1}{R}}
\]
Il sistema ha una accelerazione angolare $\alpha$ pari a quella trovata.

Si immagini di avere un corpo esteso che questa volta si guarda dall'alto in modo tale che l'asse attorno a cui ruota è ortogonale al piano del foglio.
\begin{figure}[htpb]
	\centering
	

	\tikzset{every picture/.style={line width=0.75pt}} %set default line width to 0.75pt        

	\begin{tikzpicture}[x=0.75pt,y=0.75pt,yscale=-1,xscale=1]
	%uncomment if require: \path (0,300); %set diagram left start at 0, and has height of 300

	%Shape: Circle [id:dp07436427005931256] 
	\draw  [dash pattern={on 0.84pt off 2.51pt}] (193,144.75) .. controls (193,107.33) and (223.33,77) .. (260.75,77) .. controls (298.17,77) and (328.5,107.33) .. (328.5,144.75) .. controls (328.5,182.17) and (298.17,212.5) .. (260.75,212.5) .. controls (223.33,212.5) and (193,182.17) .. (193,144.75) -- cycle ;
	%Straight Lines [id:da9965818127873434] 
	\draw [line width=1.5]    (193,144.75) -- (193,95) ;
	\draw [shift={(193,91)}, rotate = 450] [fill={rgb, 255:red, 0; green, 0; blue, 0 }  ][line width=0.08]  [draw opacity=0] (13.4,-6.43) -- (0,0) -- (13.4,6.44) -- (8.9,0) -- cycle    ;
	\draw  [fill={rgb, 255:red, 0; green, 0; blue, 0 }  ,fill opacity=1 ] (290.73,150.29) -- (279.14,160.52) -- (277.91,145.12) -- (281.73,154.11) -- cycle ;
	%Shape: Arc [id:dp8802504072717734] 
	\draw  [draw opacity=0] (242.78,158.98) .. controls (239.68,155.07) and (237.83,150.12) .. (237.83,144.75) .. controls (237.83,132.09) and (248.09,121.83) .. (260.75,121.83) .. controls (273.41,121.83) and (283.67,132.09) .. (283.67,144.75) .. controls (283.67,147.72) and (283.1,150.55) .. (282.08,153.15) -- (260.75,144.75) -- cycle ; \draw   (242.78,158.98) .. controls (239.68,155.07) and (237.83,150.12) .. (237.83,144.75) .. controls (237.83,132.09) and (248.09,121.83) .. (260.75,121.83) .. controls (273.41,121.83) and (283.67,132.09) .. (283.67,144.75) .. controls (283.67,147.72) and (283.1,150.55) .. (282.08,153.15) ;

	% Text Node
	\draw (172,119.17) node    {$\omega R_{i}$};
	% Text Node
	\draw (260.75,144.75) node    {$\vec{\omega }$};


	\end{tikzpicture}
\end{figure}
\FloatBarrier
In questo tipo di moto tutti i punti si muovono con la stessa velocità angolare $\omega$ che sarà un vettore parallelo all'asse $z$. Si sottolinea che se si prende un generico punto $i$ del corpo, esso ha velocità vettoriale tangente alla circonferenza e con modulo pari a $\omega R_i$. Questa informazione è importante perché se il corpo è collegato ad altri corpi, bisogna capire come la sua velocità si collega alle altre. Se si vuole che un corpo che sta ruotando con una certa $\omega$ vari la sua velocità angolare, è necessario che agiscano su di esso delle forze che danno luogo a dei momenti diretti parallelamente all'asse di rotazione. Il legame tra l'azione dinamica e l'effetto cinematico è la seconda equazione cardinale della dinamica per un corpo in rotazione intorno a un asse $z$.
\[
	M_z = I_z \alpha
\]
Dove il momento di inerzia rappresenta l'inerzia che il corpo oppone alla variazione della sua velocità angolare. Se si vuole che $\omega$ sia una funzione del tempo, è necessario che esistano dei momenti assiali. Se questi momenti non esistono, o si compensano a vicenda, si ha che il momento angolare totale, e in particolare quello assiale, sarà costante. Questo vuole dire che, presi due istanti di tempo successivi, ci sono due possibilità:
\begin{itemize}
	\item il corpo ha momento di inerzia che non varia durante il movimento e dunque, se i momenti assiali sono nulli o si bilanciano, esso si muove di moto rotatorio uniforme.
	\item se durante il moto per qualche motivo varia il momento di inerzia del corpo, si mantiene costante il prodotto tra il momento di inerzia e la velocità angolare. Se per qualche motivo il corpo aumenta il momento di inerzia, avviene che diminuisce la sua velocità angolare. Se per un punto materiale non è così naturale pensare che cambi la sua massa, per un corpo può accadere che il momento di inerzia vari perché la massa viene spostata da una parte all'altra del corpo.
	\[
		M_z^{(o)} = 0 \implies \vec{L}^{(o)} = \text{costante} = I_z\omega
	\]
\end{itemize}
Ora ci si chiede se la prima equazione cardinale della dinamica serva per studiare il moto di rotazione di un corpo intorno a un asse fisso. Si noti che se l'asse $z$ non passa per il centro di massa, quest'ultimo non è fermo ma deve ruotare anche lui su una circonferenza con raggio $R$ e centro sull'asse $z$. Mi aspetto che la prima equazione cardinale della dinamica mi dia un valore non nullo, che ci siano forze che fanno muovere il centro di massa. Se invece si fa coincidere l'asse $z$ con il centro di massa, ci si aspetta che il centro di massa sia fermo e che tale equazione dica che la risultante delle forze è nulla. Si può studiare il moto del corpo usando le due equazioni cardinali della dinamica o sostituendo una delle due con il teorema dell'energia cinetica.







































\section{Energia cinetica di un corpo in rotazione}

L'energia cinetica è data da:
\[
	\sum_i \frac{1}{2} m_i v_i^2 = \sum_i \frac{1}{2} m_i\omega^2 R_i^2  = \frac{1}{2} \omega^2 \left( \sum_i m_i^2 R_i^2 \right)
\]
Il termine tra parentesi è proprio il momento di inerzia. Quindi si ottiene un'espressione che definisce l'energia cinetica di un corpo che ruota intorno all'asse $z$. Bisogna fare attenzione al fatto che il momento di inerzia è quello del corpo \emph{rispetto all'asse di rotazione}, \emph{non} rispetto al centro di massa.
\[
	E_K=\frac{1}{2} I_z\omega^2
\]
Se si hanno delle forze che generano dei momenti, bisogna capire come si scrive il lavoro delle forze quando queste invece che far traslare un punto del sistema lo fanno ruotare, quindi il lavoro dei momenti delle forze.
\begin{figure}[htpb]
	\centering
	

	\tikzset{every picture/.style={line width=0.75pt}} %set default line width to 0.75pt        

	\begin{tikzpicture}[x=0.75pt,y=0.75pt,yscale=-1,xscale=1]
	%uncomment if require: \path (0,300); %set diagram left start at 0, and has height of 300

	%Straight Lines [id:da7037828027851607] 
	\draw [color={rgb, 255:red, 155; green, 155; blue, 155 }  ,draw opacity=1 ]   (356.55,188.52) -- (412.24,133.53) ;
	\draw [shift={(414.38,131.42)}, rotate = 495.36] [fill={rgb, 255:red, 155; green, 155; blue, 155 }  ,fill opacity=1 ][line width=0.08]  [draw opacity=0] (10.72,-5.15) -- (0,0) -- (10.72,5.15) -- (7.12,0) -- cycle    ;
	%Shape: Regular Polygon [id:dp7869409182664271] 
	\draw   (341.52,204.77) .. controls (321.24,214.19) and (230.99,221.59) .. (251.56,202.17) .. controls (272.13,182.76) and (272.41,172.76) .. (253.29,142.2) .. controls (234.16,111.64) and (324.12,114.23) .. (343.25,144.79) .. controls (362.38,175.35) and (361.8,195.35) .. (341.52,204.77) -- cycle ;
	%Shape: Circle [id:dp009598075955776997] 
	\draw  [dash pattern={on 0.84pt off 2.51pt}] (171.11,187.72) .. controls (154.13,154.38) and (167.39,113.58) .. (200.74,96.6) .. controls (234.08,79.62) and (274.87,92.88) .. (291.85,126.22) .. controls (308.84,159.57) and (295.57,200.36) .. (262.23,217.34) .. controls (228.89,234.32) and (188.09,221.06) .. (171.11,187.72) -- cycle ;
	%Shape: Circle [id:dp5400712793990026] 
	\draw  [fill={rgb, 255:red, 0; green, 0; blue, 0 }  ,fill opacity=1 ] (228.81,158.33) .. controls (228.06,156.86) and (228.65,155.05) .. (230.12,154.3) .. controls (231.6,153.55) and (233.41,154.13) .. (234.16,155.61) .. controls (234.91,157.09) and (234.32,158.89) .. (232.85,159.65) .. controls (231.37,160.4) and (229.56,159.81) .. (228.81,158.33) -- cycle ;
	%Straight Lines [id:da11086399310222395] 
	\draw    (231.48,156.97) -- (353.64,187.79) ;
	\draw [shift={(356.55,188.52)}, rotate = 194.16] [fill={rgb, 255:red, 0; green, 0; blue, 0 }  ][line width=0.08]  [draw opacity=0] (10.72,-5.15) -- (0,0) -- (10.72,5.15) -- (7.12,0) -- cycle    ;
	%Shape: Circle [id:dp663064867078651] 
	\draw  [fill={rgb, 255:red, 0; green, 0; blue, 0 }  ,fill opacity=1 ] (294.92,174.53) .. controls (294.17,173.05) and (294.76,171.25) .. (296.24,170.5) .. controls (297.71,169.74) and (299.52,170.33) .. (300.27,171.81) .. controls (301.02,173.28) and (300.44,175.09) .. (298.96,175.84) .. controls (297.48,176.59) and (295.68,176.01) .. (294.92,174.53) -- cycle ;
	%Shape: Circle [id:dp02478333726937576] 
	\draw  [fill={rgb, 255:red, 0; green, 0; blue, 0 }  ,fill opacity=1 ] (353.88,189.89) .. controls (353.12,188.41) and (353.71,186.6) .. (355.19,185.85) .. controls (356.66,185.1) and (358.47,185.69) .. (359.22,187.16) .. controls (359.97,188.64) and (359.39,190.45) .. (357.91,191.2) .. controls (356.43,191.95) and (354.63,191.36) .. (353.88,189.89) -- cycle ;
	%Straight Lines [id:da6728968637253481] 
	\draw    (356.55,188.52) -- (393.15,198.21) ;
	\draw [shift={(396.05,198.98)}, rotate = 194.83] [fill={rgb, 255:red, 0; green, 0; blue, 0 }  ][line width=0.08]  [draw opacity=0] (10.72,-5.15) -- (0,0) -- (10.72,5.15) -- (7.12,0) -- cycle    ;
	%Straight Lines [id:da7741385010148079] 
	\draw    (356.55,188.52) -- (374.95,123.83) ;
	\draw [shift={(375.77,120.95)}, rotate = 465.88] [fill={rgb, 255:red, 0; green, 0; blue, 0 }  ][line width=0.08]  [draw opacity=0] (10.72,-5.15) -- (0,0) -- (10.72,5.15) -- (7.12,0) -- cycle    ;
	%Straight Lines [id:da8723315686748525] 
	\draw  [dash pattern={on 0.84pt off 2.51pt}]  (375.77,120.95) -- (415.27,131.4) ;
	%Straight Lines [id:da6197450263799631] 
	\draw  [dash pattern={on 0.84pt off 2.51pt}]  (395.15,199) -- (414.38,131.42) ;

	% Text Node
	\draw (219.4,150.58) node    {$z$};
	% Text Node
	\draw (306.62,189.09) node  [font=\footnotesize]  {$CM$};
	% Text Node
	\draw (336.78,168.15) node    {$\vec{R}$};
	% Text Node
	\draw (356.88,203.11) node    {$A$};
	% Text Node
	\draw (359.45,113.74) node    {$\vec{F}_{t}$};
	% Text Node
	\draw (426.15,126.55) node    {$\vec{F}$};
	% Text Node
	\draw (409.89,203.42) node    {$\vec{F}_{n}$};


	\end{tikzpicture}
\end{figure}
\FloatBarrier
Si immagini che per far ruotare il corpo agisca su $A$ la forza $\vec{F}$. Tale punto si muove sull'archetto di circonferenza con centro sull'asse $z$, di lunghezza $ds$. La componente della forza che compie lavoro è solo quella tangente alla circonferenza. Si può trasformare lo spostamento lineare del punto in uno spostamento angolare $d\vartheta$, dove $ds = R\,d\vartheta$. Ma $\vec{F}_t R$ non è altro che l'intensità del momento generato dalla forza $F$ rispetto al polo $z$. Lo si chiama $\vec{M}_z$, ed è il momento assiale che permette al corpo di ruotare. Con queste informazioni, si scrive il lavoro compiuto da $\vec{F}$:
\[
	\mathcal{L} = \int \vec{F}_t \cdot d\vec{s} = \int F_t Rd\vartheta = \int M_z d\vartheta
\]
Note le intensità dei momenti assiali, il lavoro dei momenti si calcola proprio come: $\mathcal{L} = \int M_z d\vartheta$. Con queste ultime due informazioni trovate, si può applicare il teorema dell'energia cinetica, che fa vedere l'effetto dal punto di vista energetico del lavoro compiuto dalle forze. Si andrà a sostituire alla relazione $M=I\alpha$ perché collega l'effetto dei momenti delle forze all'accelerazione del corpo o alla variazione di energia cinetica.

\textbf{I principali momenti di inerzia}
\begin{itemize}

	\item \textbf{Anello} che ruota attorno al suo centro di massa $\boxed{I_z = MR^2}$

	\item \textbf{Disco} che ruota attorno al suo centro $\boxed{I_z=\frac{1}{2} MR^2}$

	\item \textbf{Cilindro} che ruota attorno al suo asse. Il suo momento di inerzia è esattamente lo stesso di quello del disco. Questo perché la massa non ha cambiato la sua distribuzione rispetto all'asse $z$ ma è solo stata distribuita lungo una direzione. $\boxed{I_z = \frac{1}{2} MR^2}$

	\item \textbf{Guscio cilindrico sottile} che ruota attorno al suo asse. Analogo discorso varrebbe per l'anello. Se si considera un guscio cilindrico sottile, il suo momento sarà sempre $\boxed{I_z = MR^2}$

	\item \textbf{Sfera} di raggio $R$. Una sfera, fatta ruotare per un qualunque asse passante per il suo centro, avrà momento di inerzia inferiore a quello del cilindro perché ho maggiore massa distribuita più lontano dall'asse $z$ nel caso del cilindro. $\boxed{I_z = \frac{2}{5}MR^2}$

	\item \textbf{Asta} di lunghezza $L$ che ruota intorno ad un asse $z$ passante per il suo centro $\boxed{I_z = \frac{1}{12}ML^2}$

	\item \textbf{Lastra}  di lati $A$ e $B$ che ruota attorno al suo centro: $\boxed{I_z = \frac{1}{12}M(A^2+B^2)}$
\end{itemize}







































\section{Teorema di Huygens-Steiner}

Se si fa ruotare il disco rispetto a un altro asse, non passante per forza per il centro di massa,  viene in aiuto il \textbf{teorema degli assi paralleli}, o teorema di Huygens-Steiner. Si supponga di avere un corpo di forma qualunque di cui è noto il momento di inerzia rispetto a un asse $z'$ passante per il centro di massa. Il teorema degli assi paralleli permette di calcolare il momento di inerzia del corpo rispetto a un qualunque altro asse $z \parallel z'$. Il teorema afferma che nota la distanza assiale tra i due assi, $D$, e la massa $M$ del corpo, il momento di inerzia di esso rispetto all'asse $z$ sarà calcolabile partendo dal momento di inerzia rispetto a $z'$, in aggiunta a una quantità sempre positiva data dalla massa del corpo per il quadrato della distanza fra i due corpi:
\[
	\boxed{I_z = I_{z'} + MD^2, \quad I_z > I_{z'}}
\]
Qualunque sia l'asse $z$, il momento di inerzia è sempre maggiore rispetto a quello rispetto all'asse passante per il centro di massa, $z'$.
\begin{figure}[htpb]
	\centering
	

	\tikzset{every picture/.style={line width=0.75pt}} %set default line width to 0.75pt        

	\begin{tikzpicture}[x=0.75pt,y=0.75pt,yscale=-0.9,xscale=0.9]
	%uncomment if require: \path (0,300); %set diagram left start at 0, and has height of 300

	%Shape: Axis 2D [id:dp8273336417410213] 
	\draw  (148.9,220.64) -- (224.37,220.64)(156.44,152.71) -- (156.44,228.18) (217.37,215.64) -- (224.37,220.64) -- (217.37,225.64) (151.44,159.71) -- (156.44,152.71) -- (161.44,159.71)  ;
	%Shape: Axis 2D [id:dp46760875552896186] 
	\draw  (299.21,141.61) -- (397.2,141.61)(306.82,50.91) -- (306.82,148.9) (390.2,136.61) -- (397.2,141.61) -- (390.2,146.61) (301.82,57.91) -- (306.82,50.91) -- (311.82,57.91)  ;
	%Shape: Circle [id:dp33560493835936445] 
	\draw  [fill={rgb, 255:red, 0; green, 0; blue, 0 }  ,fill opacity=1 ] (302.95,141.36) .. controls (302.95,139.25) and (304.66,137.55) .. (306.76,137.55) .. controls (308.86,137.55) and (310.57,139.25) .. (310.57,141.36) .. controls (310.57,143.46) and (308.86,145.16) .. (306.76,145.16) .. controls (304.66,145.16) and (302.95,143.46) .. (302.95,141.36) -- cycle ;
	%Shape: Ellipse [id:dp07615770474265449] 
	\draw  [fill={rgb, 255:red, 0; green, 0; blue, 0 }  ,fill opacity=1 ] (338.26,59.54) .. controls (338.26,57.44) and (339.96,55.73) .. (342.07,55.73) .. controls (344.17,55.73) and (345.87,57.44) .. (345.87,59.54) .. controls (345.87,61.64) and (344.17,63.34) .. (342.07,63.34) .. controls (339.96,63.34) and (338.26,61.64) .. (338.26,59.54) -- cycle ;
	%Straight Lines [id:da39168454837719757] 
	\draw    (156.44,220.64) -- (300.27,145) ;
	\draw [shift={(302.92,143.61)}, rotate = 512.26] [fill={rgb, 255:red, 0; green, 0; blue, 0 }  ][line width=0.08]  [draw opacity=0] (10.72,-5.15) -- (0,0) -- (10.72,5.15) -- (7.12,0) -- cycle    ;
	%Straight Lines [id:da9139176154510034] 
	\draw    (156.44,220.64) -- (336.26,65.56) ;
	\draw [shift={(338.54,63.6)}, rotate = 499.22] [fill={rgb, 255:red, 0; green, 0; blue, 0 }  ][line width=0.08]  [draw opacity=0] (10.72,-5.15) -- (0,0) -- (10.72,5.15) -- (7.12,0) -- cycle    ;
	%Shape: Ellipse [id:dp30294253521760495] 
	\draw   (27,156) .. controls (27,88.35) and (134.68,33.5) .. (267.5,33.5) .. controls (400.32,33.5) and (508,88.35) .. (508,156) .. controls (508,223.65) and (400.32,278.5) .. (267.5,278.5) .. controls (134.68,278.5) and (27,223.65) .. (27,156) -- cycle ;
	%Straight Lines [id:da30583034209389814] 
	\draw    (306.82,141.61) -- (339.22,68.87) ;
	\draw [shift={(340.44,66.13)}, rotate = 474.01] [fill={rgb, 255:red, 0; green, 0; blue, 0 }  ][line width=0.08]  [draw opacity=0] (10.72,-5.15) -- (0,0) -- (10.72,5.15) -- (7.12,0) -- cycle    ;
	%Straight Lines [id:da6744668858852887] 
	\draw    (156.44,246.01) -- (306.76,246.01) ;
	\draw [shift={(306.76,246.01)}, rotate = 180] [color={rgb, 255:red, 0; green, 0; blue, 0 }  ][line width=0.75]    (0,5.59) -- (0,-5.59)   ;
	\draw [shift={(156.44,246.01)}, rotate = 180] [color={rgb, 255:red, 0; green, 0; blue, 0 }  ][line width=0.75]    (0,5.59) -- (0,-5.59)   ;
	%Straight Lines [id:da8072781954268866] 
	\draw    (344.81,220.64) -- (344.88,145.41) ;
	\draw [shift={(344.88,145.41)}, rotate = 450.05] [color={rgb, 255:red, 0; green, 0; blue, 0 }  ][line width=0.75]    (0,5.59) -- (0,-5.59)   ;
	\draw [shift={(344.81,220.64)}, rotate = 450.05] [color={rgb, 255:red, 0; green, 0; blue, 0 }  ][line width=0.75]    (0,5.59) -- (0,-5.59)   ;

	% Text Node
	\draw (238.32,219.3) node    {$x$};
	% Text Node
	\draw (141.92,145.73) node    {$y$};
	% Text Node
	\draw (143.82,226.28) node    {$z$};
	% Text Node
	\draw (408.94,142.56) node    {$x'$};
	% Text Node
	\draw (293.5,50.59) node    {$y'$};
	% Text Node
	\draw (314.37,153.28) node    {$z'$};
	% Text Node
	\draw (264.87,180.58) node    {$\vec{D}$};
	% Text Node
	\draw (365.17,57.57) node    {$m_{i}$};
	% Text Node
	\draw (285.26,133.68) node  [font=\footnotesize]  {$CM$};
	% Text Node
	\draw (223.74,132.41) node    {$\vec{R}_{i}$};
	% Text Node
	\draw (339.17,107.68) node    {$\vec{R} '_{i}$};
	% Text Node
	\draw (235.79,258.63) node    {$a$};
	% Text Node
	\draw (358.83,177.44) node    {$b$};
	% Text Node
	\draw (132.17,72.57) node    {$M$};

	\end{tikzpicture}
\end{figure}
\FloatBarrier

\paragraph{Dimostrazione} Si definisce un sistema di riferimento centrato sul centro di massa con assi $x', y'$ e $z'$ e un sistema di riferimento centrato sul punto rispetto al quale si vuole calcolare il momento di inerzia, $x, y, z$.
Nota $D$, siano:
\begin{itemize}
	\item $a=$ ascissa della distanza fra i due sistemi di riferimento;
	\item $b=$ distanza in termini di ordinata fra i due sistemi di riferimento.
\end{itemize}
Per prima cosa si applica la definizione di momento di inerzia, utilizzando la notazione discreta. Si immagini di suddividere il corpo in tante masse infinitesime. Il momento di inerzia sarà la somma della massa $i$-esima per il quadrato della distanza assiale fra la massa $i$-esima e l'asse. $R_i$ si può riscrivere come:
\[
	R_i^2 = x_i^2 + y_i^2
\]
Dove $x_i$ e $y_i$ sono l'ascissa e l'ordinata della massa $i$-esima rispetto al sistema di riferimento centrato nell'asse $z$. Siano $x_i'$ e $y_i'$ le coordinate della massa $i$-esima rispetto all'altro sistema di riferimento.
\[
	\sum_i m_i x_i^2  = \sum_i m_i (x_i'+a)^2 = \sum_i m_i x_i'^2 + \sum_i m_i a^2 + \sum_i 2 a m_i x_i'
\]
Per quanto riguarda l'ultimo termine, si noti che $2a$ lo posso portare fuori dalla sommatoria. La sommatoria darebbe luogo a:
\[
	\sum_i m_i x_i'
\]
Si tratta della massa del sistema per l'ascissa del centro di massa nel sistema di riferimento del centro di massa, pertanto non può che essere $0$. Si usa lo stesso concetto utilizzato per il teorema di K\"onig.
\begin{equation*}
	\begin{aligned}
		\sum_i m_i x_i^2 &= \sum_i m_i x_i'^2 + \sum_i m_i a^2 \\
		\sum_i m_i y_i^2 &= \sum_i m_i y_i'^2 + \sum_i m_i b^2 \\
		\\
		I_z &= \sum_i m_i R_i^2 \\
		&= \sum_i m_i x_i^2 + \sum_i m_i y_i^2 \\
		&=\sum_i m_i x_i'^2 + \sum_i m_i a^2 + \sum_i m_i y_i'^2 + \sum_i m_i b^2 \\
		&= \sum_i m_i R_i'^2 +\sum_i m_i D^2 \\
		&= I_z' + MD^2 \\
	\end{aligned}
\end{equation*}
Si ottiene così ciò che si voleva dimostrare.

\subsection{Confronto fra il teorema di Huygens Steiner e di K\"onig}

Dal momento che si ha utilizzato un ragionamento analogo per il teorema di K\"onig, si deduce che esso ha fondamentalmente lo stesso significato e fornisce le stesse informazioni. L'energia cinetica del corpo rigido in rotazione sarà data da:
\[
	E_K = \frac{1}{2} \omega^2 I_z
\]
Si definisce un asse $z'$ parallelo a $z$ e passante per il centro di massa e si va a sostituire le informazioni fornite dal teorema di Huygens-Steiner.
\[
	E_K = \frac{1}{2} \omega^2 (I_z' + MD^2) = \frac{1}{2} \omega^2 I_z' + \frac{1}{2} \omega^2 MD^2
\]
Il primo termine è l'energia cinetica di un corpo che sta ruotando intorno all'asse $z'$ rispetto al centro di massa. Per quanto riguarda il secondo termine, si ha $\omega D$, che dimensionalmente è una velocità lineare di un punto che ruota su una circonferenza di raggio $D$. In particolare esso sarà il centro di massa. Quindi il secondo termine si può scrivere meglio come:
\[
	\frac{1}{2} M(D\omega)^2 = \frac{1}{2} Mv_{cm}^2
\]
Dunque:
\[
	E_K = \underbrace{\frac{1}{2} \omega^2 I_z'}_{E_K\text{ rotazione }cm  } + \underbrace{\frac{1}{2} Mv_{cm}^2}_{E_K\text{ traslazione } cm }
\]
Questi due contributi sono quelli che sono nati quando è stato enunciato e dimostrato il teorema di K\"onig.

\paragraph{Esempio di applicazione del teorema dell'energia cinetica} Si immagini di avere un disco che è fissato intorno al suo centro in modo tale che possa ruotare intorno ad esso.
\begin{figure}[htpb]
	\centering
	

	\tikzset{every picture/.style={line width=0.75pt}} %set default line width to 0.75pt        

	\begin{tikzpicture}[x=0.75pt,y=0.75pt,yscale=-1,xscale=1]
	%uncomment if require: \path (0,300); %set diagram left start at 0, and has height of 300

	%Shape: Circle [id:dp37449881724298395] 
	\draw  [draw opacity=0][fill={rgb, 255:red, 155; green, 155; blue, 155 }  ,fill opacity=1 ] (237,136.75) .. controls (237,98.23) and (268.23,67) .. (306.75,67) .. controls (345.27,67) and (376.5,98.23) .. (376.5,136.75) .. controls (376.5,175.27) and (345.27,206.5) .. (306.75,206.5) .. controls (268.23,206.5) and (237,175.27) .. (237,136.75) -- cycle ;
	%Straight Lines [id:da3587624434409662] 
	\draw    (306.75,136.75) -- (306.75,231) ;
	\draw [shift={(306.75,234)}, rotate = 270] [fill={rgb, 255:red, 0; green, 0; blue, 0 }  ][line width=0.08]  [draw opacity=0] (10.72,-5.15) -- (0,0) -- (10.72,5.15) -- (7.12,0) -- cycle    ;
	%Straight Lines [id:da3594545535993927] 
	\draw    (306.75,42.5) -- (306.75,136.75) ;
	\draw [shift={(306.75,39.5)}, rotate = 90] [fill={rgb, 255:red, 0; green, 0; blue, 0 }  ][line width=0.08]  [draw opacity=0] (10.72,-5.15) -- (0,0) -- (10.72,5.15) -- (7.12,0) -- cycle    ;
	%Shape: Circle [id:dp6298373522875735] 
	\draw  [draw opacity=0][fill={rgb, 255:red, 0; green, 0; blue, 0 }  ,fill opacity=1 ] (303.92,136.75) .. controls (303.92,135.19) and (305.19,133.92) .. (306.75,133.92) .. controls (308.31,133.92) and (309.58,135.19) .. (309.58,136.75) .. controls (309.58,138.31) and (308.31,139.58) .. (306.75,139.58) .. controls (305.19,139.58) and (303.92,138.31) .. (303.92,136.75) -- cycle ;
	%Straight Lines [id:da6804063707082968] 
	\draw    (376.5,136.75) -- (376.5,213) ;
	%Shape: Rectangle [id:dp7253125564821739] 
	\draw  [fill={rgb, 255:red, 0; green, 0; blue, 0 }  ,fill opacity=1 ] (368.83,206.33) -- (384.17,206.33) -- (384.17,219.67) -- (368.83,219.67) -- cycle ;
	%Straight Lines [id:da5070125564426535] 
	\draw [line width=1.5]    (376.5,136.75) -- (376.5,165.8) ;
	\draw [shift={(376.5,169.8)}, rotate = 270] [fill={rgb, 255:red, 0; green, 0; blue, 0 }  ][line width=0.08]  [draw opacity=0] (13.4,-6.43) -- (0,0) -- (13.4,6.44) -- (8.9,0) -- cycle    ;
	%Straight Lines [id:da3741592651594814] 
	\draw [line width=1.5]    (376.5,178.88) -- (376.5,207.93) ;
	\draw [shift={(376.5,174.88)}, rotate = 90] [fill={rgb, 255:red, 0; green, 0; blue, 0 }  ][line width=0.08]  [draw opacity=0] (13.4,-6.43) -- (0,0) -- (13.4,6.44) -- (8.9,0) -- cycle    ;
	%Straight Lines [id:da8878387068216638] 
	\draw  [dash pattern={on 0.84pt off 2.51pt}]  (356.07,87.43) -- (306.75,136.75) ;
	%Straight Lines [id:da25441565817973344] 
	\draw    (419.75,256.75) -- (193.75,256.75) ;
	%Straight Lines [id:da588260612497469] 
	\draw    (437.75,220.67) -- (437.75,256.75) ;
	\draw [shift={(437.75,256.75)}, rotate = 270] [color={rgb, 255:red, 0; green, 0; blue, 0 }  ][line width=0.75]    (0,5.59) -- (0,-5.59)   ;
	\draw [shift={(437.75,220.67)}, rotate = 270] [color={rgb, 255:red, 0; green, 0; blue, 0 }  ][line width=0.75]    (0,5.59) -- (0,-5.59)   ;

	% Text Node
	\draw (297,132.5) node    {$z$};
	% Text Node
	\draw (322.5,45.5) node    {$\vec{\Phi }$};
	% Text Node
	\draw (326.5,225.5) node    {$M\vec{g}$};
	% Text Node
	\draw (395.1,211.1) node    {$m$};
	% Text Node
	\draw (390.3,153.7) node    {$\vec{T}$};
	% Text Node
	\draw (389.9,182.9) node    {$\vec{T}$};
	% Text Node
	\draw (331,93.83) node    {$R$};
	% Text Node
	\draw (268.5,174.17) node    {$M$};
	% Text Node
	\draw (451.83,235.5) node    {$h$};


	\end{tikzpicture}
\end{figure}
\FloatBarrier
Intorno al cilindro è avvolto un filo al cui estremo è applicata la massa $m$, a una quota $h$ dal suolo: inizialmente essa è sostenuta. Il sostegno viene rimosso e si vuole sapere con che velocità esso arriva al suolo.
Sul rocchetto agiranno sicuramente la forza peso, il vincolo che non lo fa cadere a terra e la tensione del filo. Mentre queste due forze non compiono lavoro, la tensione lo farà. In particolare darà un momento $TR$. Esso deve servire per far variare l'energia cinetica del rocchetto. Il tutto lo si lega alla variazione di quota di $m$, che arriverà al suolo con un'energia potenziale nulla. $mgh$ non si trasforma interamente in energia cinetica perché così la massa si comporterebbe come se fosse libera. Ma non è così, la massa per cadere verso il basso deve muovere questo disco che ha una sua inerzia. C'è una forza che agisce sulla massa che è la tensione del filo.

\paragraph{Esempio del pendolo composto} È un esempio interessante del moto di rotazione di un corpo rigido attorno a un asse.
\begin{figure}[htpb]
	\centering
	

	\tikzset{every picture/.style={line width=0.75pt}} %set default line width to 0.75pt        

	\begin{tikzpicture}[x=0.75pt,y=0.75pt,yscale=-1,xscale=1]
	%uncomment if require: \path (0,300); %set diagram left start at 0, and has height of 300

	%Shape: Arc [id:dp25219748707798706] 
	\draw  [draw opacity=0][dash pattern={on 0.84pt off 2.51pt}] (365.43,199.39) .. controls (345.71,212.57) and (322,220.25) .. (296.5,220.25) -- (296.5,96) -- cycle ; \draw  [dash pattern={on 0.84pt off 2.51pt}] (365.43,199.39) .. controls (345.71,212.57) and (322,220.25) .. (296.5,220.25) ;
	%Straight Lines [id:da6360552694564299] 
	\draw    (296.5,96) -- (296.5,221) ;
	%Straight Lines [id:da6954033748497943] 
	\draw    (296.5,96) -- (365.43,199.39) ;
	%Shape: Arc [id:dp5035601333805273] 
	\draw  [draw opacity=0] (319.66,130.74) .. controls (313.03,135.17) and (305.07,137.75) .. (296.5,137.75) -- (296.5,96) -- cycle ; \draw   (319.66,130.74) .. controls (313.03,135.17) and (305.07,137.75) .. (296.5,137.75) ;
	%Shape: Circle [id:dp45991567118054166] 
	\draw  [draw opacity=0][fill={rgb, 255:red, 128; green, 128; blue, 128 }  ,fill opacity=1 ] (285.25,220.25) .. controls (285.25,214.04) and (290.29,209) .. (296.5,209) .. controls (302.71,209) and (307.75,214.04) .. (307.75,220.25) .. controls (307.75,226.46) and (302.71,231.5) .. (296.5,231.5) .. controls (290.29,231.5) and (285.25,226.46) .. (285.25,220.25) -- cycle ;
	%Shape: Circle [id:dp9187145095922711] 
	\draw  [draw opacity=0][fill={rgb, 255:red, 128; green, 128; blue, 128 }  ,fill opacity=1 ] (354.18,199.39) .. controls (354.18,193.18) and (359.22,188.14) .. (365.43,188.14) .. controls (371.64,188.14) and (376.68,193.18) .. (376.68,199.39) .. controls (376.68,205.61) and (371.64,210.64) .. (365.43,210.64) .. controls (359.22,210.64) and (354.18,205.61) .. (354.18,199.39) -- cycle ;
	%Shape: Arc [id:dp06123254472319295] 
	\draw  [draw opacity=0] (353.97,217.36) .. controls (343.53,222.32) and (332.34,225.96) .. (320.62,228.09) -- (296.5,96) -- cycle ; \draw   (353.97,217.36) .. controls (343.53,222.32) and (332.34,225.96) .. (320.62,228.09) ;
	\draw   (331.53,231.13) -- (319.45,228.63) -- (328.7,220.47) ;

	% Text Node
	\draw (309.5,147) node    {$\vartheta $};


	\end{tikzpicture}
\end{figure}
\FloatBarrier
Nel pendolo composto la massa non è puntiforme ma è un corpo esteso che viene fatto oscillare, vincolato ad un certo punto. Si ha che la posizione di equilibrio del corpo è sulla verticale. Il movimento sarà dato sempre dalla forza peso, che tende a riportare il corpo nella posizione di equilibrio. Bisogna però risolvere il problema
utilizzando l'equazione dei momenti e tenendo conto dell'inerzia del sistema.







































\section{Il moto di rototraslazione}

Il moto di rototraslazione è quello caratteristico di un corpo che ruota e intanto trasla. Il modo più comodo per studiarlo è quello di vederlo come combinazione della rotazione del corpo attorno al centro di massa più la traslazione di quest'ultimo. Si può anche considerarlo come la rotazione attorno a un qualsiasi punto del corpo più la traslazione di questo. Si utilizza la prima equazione cardinale della dinamica per studiare il moto di traslazione del centro di massa e la seconda equazione cardinale della dinamica per studiare la rotazione.

\paragraph{Moto di scivolamento senza slittamento} Si tratta di un particolare caso di rototraslazione. È il moto di un oggetto che può rotolare su un piano in modo tale che mentre rotola, il punto di contatto con il piano d'appoggio non slitta mai. Si consideri un disco di raggio $R$ che rotola senza strisciare su un piano d'appoggio orizzontale.
\begin{figure}[htpb]
	\centering
	

	% Pattern Info
	 
	\tikzset{
	pattern size/.store in=\mcSize, 
	pattern size = 5pt,
	pattern thickness/.store in=\mcThickness, 
	pattern thickness = 0.3pt,
	pattern radius/.store in=\mcRadius, 
	pattern radius = 1pt}
	\makeatletter
	\pgfutil@ifundefined{pgf@pattern@name@_mwikyhy1d}{
	\pgfdeclarepatternformonly[\mcThickness,\mcSize]{_mwikyhy1d}
	{\pgfqpoint{0pt}{0pt}}
	{\pgfpoint{\mcSize+\mcThickness}{\mcSize+\mcThickness}}
	{\pgfpoint{\mcSize}{\mcSize}}
	{
	\pgfsetcolor{\tikz@pattern@color}
	\pgfsetlinewidth{\mcThickness}
	\pgfpathmoveto{\pgfqpoint{0pt}{0pt}}
	\pgfpathlineto{\pgfpoint{\mcSize+\mcThickness}{\mcSize+\mcThickness}}
	\pgfusepath{stroke}
	}}
	\makeatother
	\tikzset{every picture/.style={line width=0.75pt}} %set default line width to 0.75pt        

	\begin{tikzpicture}[x=0.75pt,y=0.75pt,yscale=-1,xscale=1]
	%uncomment if require: \path (0,300); %set diagram left start at 0, and has height of 300

	%Shape: Rectangle [id:dp05787029687722711] 
	\draw  [draw opacity=0][pattern=_mwikyhy1d,pattern size=6pt,pattern thickness=0.75pt,pattern radius=0pt, pattern color={rgb, 255:red, 155; green, 155; blue, 155}] (180,210) -- (462.75,210) -- (462.75,225.5) -- (180,225.5) -- cycle ;
	%Straight Lines [id:da9575884425325174] 
	\draw    (180,210) -- (470.5,210) ;
	\draw [shift={(473.5,210)}, rotate = 180] [fill={rgb, 255:red, 0; green, 0; blue, 0 }  ][line width=0.08]  [draw opacity=0] (10.72,-5.15) -- (0,0) -- (10.72,5.15) -- (7.12,0) -- cycle    ;
	%Shape: Circle [id:dp6173822162539673] 
	\draw   (210,165) .. controls (210,140.15) and (230.15,120) .. (255,120) .. controls (279.85,120) and (300,140.15) .. (300,165) .. controls (300,189.85) and (279.85,210) .. (255,210) .. controls (230.15,210) and (210,189.85) .. (210,165) -- cycle ;
	%Shape: Circle [id:dp31313922768748026] 
	\draw  [dash pattern={on 0.84pt off 2.51pt}] (340,165) .. controls (340,140.15) and (360.15,120) .. (385,120) .. controls (409.85,120) and (430,140.15) .. (430,165) .. controls (430,189.85) and (409.85,210) .. (385,210) .. controls (360.15,210) and (340,189.85) .. (340,165) -- cycle ;
	%Straight Lines [id:da010485267518391517] 
	\draw    (255,165) -- (281.17,128.33) ;
	%Shape: Arc [id:dp5132535481287805] 
	\draw  [draw opacity=0] (264.99,111.31) .. controls (276.67,113.47) and (287.05,119.35) .. (294.85,127.68) -- (255,165) -- cycle ; \draw   (264.99,111.31) .. controls (276.67,113.47) and (287.05,119.35) .. (294.85,127.68) ;
	\draw   (291.33,117.34) -- (295.67,128.25) -- (284.34,125.17) ;
	%Shape: Circle [id:dp15806679768486953] 
	\draw  [fill={rgb, 255:red, 0; green, 0; blue, 0 }  ,fill opacity=1 ] (252.25,210) .. controls (252.25,208.48) and (253.48,207.25) .. (255,207.25) .. controls (256.52,207.25) and (257.75,208.48) .. (257.75,210) .. controls (257.75,211.52) and (256.52,212.75) .. (255,212.75) .. controls (253.48,212.75) and (252.25,211.52) .. (252.25,210) -- cycle ;
	%Shape: Circle [id:dp45395702869368915] 
	\draw  [fill={rgb, 255:red, 0; green, 0; blue, 0 }  ,fill opacity=1 ] (382.25,210) .. controls (382.25,208.48) and (383.48,207.25) .. (385,207.25) .. controls (386.52,207.25) and (387.75,208.48) .. (387.75,210) .. controls (387.75,211.52) and (386.52,212.75) .. (385,212.75) .. controls (383.48,212.75) and (382.25,211.52) .. (382.25,210) -- cycle ;

	% Text Node
	\draw (263.2,136) node    {$R$};
	% Text Node
	\draw (283.45,101.75) node    {$\vec{\omega }$};
	% Text Node
	\draw (246.5,192.5) node    {$C$};
	% Text Node
	\draw (377.5,194) node    {$C$};


	\end{tikzpicture}
\end{figure}
\FloatBarrier
I vari punti sulla circonferenza via via nel movimento andranno a contatto con il suolo. Quello che succede è che il punto del corpo che era a contatto con il suolo, $C$, che ovviamente cambierà istante per istante, perché non strisci deve avere istantaneamente velocità nulla. Se il corpo rotola senza strisciare, le varie parti del corpo avranno una velocità, ma $C$ istantaneamente si ferma. Si noti che il corpo non sta ruotando attorno all'asse passante per il centro di massa ma attorno all'asse $z$ passante per il punto di contatto, che via via via cambia durante il moto. I vari punti del corpo, apparte il centro di massa, che si sposta parallelamente al piano di appoggio, seguono traiettorie più complicate di quelle viste finora.
\begin{figure}[htpb]
	\centering
	

	% Pattern Info
	 
	\tikzset{
	pattern size/.store in=\mcSize, 
	pattern size = 5pt,
	pattern thickness/.store in=\mcThickness, 
	pattern thickness = 0.3pt,
	pattern radius/.store in=\mcRadius, 
	pattern radius = 1pt}
	\makeatletter
	\pgfutil@ifundefined{pgf@pattern@name@_ghnoqj5wk}{
	\pgfdeclarepatternformonly[\mcThickness,\mcSize]{_ghnoqj5wk}
	{\pgfqpoint{0pt}{0pt}}
	{\pgfpoint{\mcSize+\mcThickness}{\mcSize+\mcThickness}}
	{\pgfpoint{\mcSize}{\mcSize}}
	{
	\pgfsetcolor{\tikz@pattern@color}
	\pgfsetlinewidth{\mcThickness}
	\pgfpathmoveto{\pgfqpoint{0pt}{0pt}}
	\pgfpathlineto{\pgfpoint{\mcSize+\mcThickness}{\mcSize+\mcThickness}}
	\pgfusepath{stroke}
	}}
	\makeatother
	\tikzset{every picture/.style={line width=0.75pt}} %set default line width to 0.75pt        

	\begin{tikzpicture}[x=0.75pt,y=0.75pt,yscale=-1,xscale=1]
	%uncomment if require: \path (0,300); %set diagram left start at 0, and has height of 300

	%Shape: Rectangle [id:dp14507808096786023] 
	\draw  [draw opacity=0][pattern=_ghnoqj5wk,pattern size=6pt,pattern thickness=0.75pt,pattern radius=0pt, pattern color={rgb, 255:red, 155; green, 155; blue, 155}] (130,210) -- (485.75,210) -- (485.75,226) -- (130,226) -- cycle ;
	%Straight Lines [id:da9852040206440102] 
	\draw    (130,210) -- (497,210) ;
	\draw [shift={(500,210)}, rotate = 180] [fill={rgb, 255:red, 0; green, 0; blue, 0 }  ][line width=0.08]  [draw opacity=0] (10.72,-5.15) -- (0,0) -- (10.72,5.15) -- (7.12,0) -- cycle    ;
	%Shape: Circle [id:dp13866972663201427] 
	\draw   (160,175) .. controls (160,155.67) and (175.67,140) .. (195,140) .. controls (214.33,140) and (230,155.67) .. (230,175) .. controls (230,194.33) and (214.33,210) .. (195,210) .. controls (175.67,210) and (160,194.33) .. (160,175) -- cycle ;
	%Shape: Circle [id:dp6026450717164555] 
	\draw   (270,175) .. controls (270,155.67) and (285.67,140) .. (305,140) .. controls (324.33,140) and (340,155.67) .. (340,175) .. controls (340,194.33) and (324.33,210) .. (305,210) .. controls (285.67,210) and (270,194.33) .. (270,175) -- cycle ;
	%Shape: Circle [id:dp0024488836516574075] 
	\draw   (380,175) .. controls (380,155.67) and (395.67,140) .. (415,140) .. controls (434.33,140) and (450,155.67) .. (450,175) .. controls (450,194.33) and (434.33,210) .. (415,210) .. controls (395.67,210) and (380,194.33) .. (380,175) -- cycle ;
	%Curve Lines [id:da1351197624789504] 
	\draw [line width=1.5]    (195,140) .. controls (260.75,140.5) and (290.25,180.5) .. (305,210) ;
	%Curve Lines [id:da2535479529176272] 
	\draw [line width=1.5]    (415,140) .. controls (351.25,140) and (320.25,180.5) .. (305,210) ;
	%Shape: Circle [id:dp12598589167302587] 
	\draw  [fill={rgb, 255:red, 0; green, 0; blue, 0 }  ,fill opacity=1 ] (302.25,209) .. controls (302.25,207.48) and (303.48,206.25) .. (305,206.25) .. controls (306.52,206.25) and (307.75,207.48) .. (307.75,209) .. controls (307.75,210.52) and (306.52,211.75) .. (305,211.75) .. controls (303.48,211.75) and (302.25,210.52) .. (302.25,209) -- cycle ;

	% Text Node
	\draw (304,188) node    {$C$};


	\end{tikzpicture}
\end{figure}
\FloatBarrier
In un moto di rototraslazione, la velocità di un generico punto del corpo si può sempre vedere come la velocità del centro di massa, più la rotazione del punto intorno al centro di massa:
\[
	\vec{v} (P) = \vec{v}_{cm} + \vec{\omega}\times \vec{r}_{cm}(P)
\]
Si sfrutta questa relazione per capire il legame tra la velocità del punto di contatto $C$ e la velocità del centro di massa.
\[
	\vec{v} (C) = \vec{v}_{cm} + \vec{\omega}\times \vec{r}_{cm}(C)
\]
$\vec{\omega}\times \vec{r}_{cm}(C)$ è un un vettore parallelo al piano d'appoggio diretto a sinistra (il senso è orario).
\[
	\vec{\omega} \times \vec{r}(C) = -\omega R \vec{u}_x
\]
La velocità di $C$ in un moto di rotolamento senza strisciamento vale zero. Imporre che la velocità del punto di contatto sia uguale a zero istantaneamente, vuole dire imporre che il centro di massa abbia una velocità strettamente legata alla velocità angolare di rotazione perché vale $\omega R$.
\[
	\boxed{v_{cm} = \omega R}
\]
Se si pensava di avere due incognite in un moto di moto traslazione, in realtà questa relazione fanno avere una sola incognita,  basta trovare $\omega$ e poi la velocità con cui si muove il centro di massa. Nel moto di puro rotolamento infatti, la velocità di rotazione e quella di traslazione non sono indipendenti.
Si capisce perché il moto può essere visto come il moto di rotazione attorno al punto di contatto. I punti che devono compiere spazio maggiore sono quelli sul perimetro, il centro di massa si muove meno, il punto di contatto invece non si muove. Il modo più conveniente è quello di vederlo come un moto di rotazione attorno al punto di contatto, istantaneamente fermo.

\paragraph{Esempio} Per fare in modo che le ruote di un'automobile quando ruotano non slittino, è necessario che ci sia un piano d'appoggio con sufficiente attrito da bloccare il punto di contatto, che deve essere istantaneamente fermo. Si tratterà di forza di attrito statico, e sarà diretta in direzione tangente al moto, facendo ruotare il corpo. Ad esempio, se si applica una forza $\vec{F}_o$ al centro del disco di intensità costante e tangente al piano, l'oggetto non riesce a slittare ma grazie all'attrito può rotolare. Potrebbe anche succedere che, invece che cercare di far muovere l'oggetto tirandolo, gli si applica un motore di rotazione, una coppia motrice. Si tratta di una coppia di forze uguali e opposte che non ha l'effetto di traslare ma di mettere in rotazione. In questo caso l'oggetto tenderà a ruotare indietro e allora la forza di attrito statico risulterà diretto in avanti. Le forze che agiscono sul corpo sono quelle
in figura.
\begin{figure}[htpb]
	\centering
	

	\tikzset{every picture/.style={line width=0.75pt}} %set default line width to 0.75pt        

	\begin{tikzpicture}[x=0.75pt,y=0.75pt,yscale=-1,xscale=1]
	%uncomment if require: \path (0,300); %set diagram left start at 0, and has height of 300

	%Straight Lines [id:da6140945885120805] 
	\draw    (100,240) -- (437,240) ;
	\draw [shift={(440,240)}, rotate = 180] [fill={rgb, 255:red, 0; green, 0; blue, 0 }  ][line width=0.08]  [draw opacity=0] (10.72,-5.15) -- (0,0) -- (10.72,5.15) -- (7.12,0) -- cycle    ;
	%Shape: Circle [id:dp2872045225708375] 
	\draw  [draw opacity=0][fill={rgb, 255:red, 216; green, 216; blue, 216 }  ,fill opacity=1 ] (190,160) .. controls (190,115.82) and (225.82,80) .. (270,80) .. controls (314.18,80) and (350,115.82) .. (350,160) .. controls (350,204.18) and (314.18,240) .. (270,240) .. controls (225.82,240) and (190,204.18) .. (190,160) -- cycle ;
	%Straight Lines [id:da507849495249439] 
	\draw [line width=1.5]    (270,160) -- (270,64) ;
	\draw [shift={(270,60)}, rotate = 450] [fill={rgb, 255:red, 0; green, 0; blue, 0 }  ][line width=0.08]  [draw opacity=0] (13.4,-6.43) -- (0,0) -- (13.4,6.44) -- (8.9,0) -- cycle    ;
	%Shape: Circle [id:dp4881888290360934] 
	\draw  [fill={rgb, 255:red, 0; green, 0; blue, 0 }  ,fill opacity=1 ] (266,160) .. controls (266,157.79) and (267.79,156) .. (270,156) .. controls (272.21,156) and (274,157.79) .. (274,160) .. controls (274,162.21) and (272.21,164) .. (270,164) .. controls (267.79,164) and (266,162.21) .. (266,160) -- cycle ;
	%Straight Lines [id:da21324584280345382] 
	\draw [line width=1.5]    (270,256) -- (270,160) ;
	\draw [shift={(270,260)}, rotate = 270] [fill={rgb, 255:red, 0; green, 0; blue, 0 }  ][line width=0.08]  [draw opacity=0] (13.4,-6.43) -- (0,0) -- (13.4,6.44) -- (8.9,0) -- cycle    ;
	%Straight Lines [id:da9034182639606498] 
	\draw [line width=1.5]    (204,240) -- (270,240) ;
	\draw [shift={(200,240)}, rotate = 0] [fill={rgb, 255:red, 0; green, 0; blue, 0 }  ][line width=0.08]  [draw opacity=0] (13.4,-6.43) -- (0,0) -- (13.4,6.44) -- (8.9,0) -- cycle    ;
	%Straight Lines [id:da7554370242670971] 
	\draw [line width=1.5]    (270,80) -- (336,80) ;
	\draw [shift={(340,80)}, rotate = 180] [fill={rgb, 255:red, 0; green, 0; blue, 0 }  ][line width=0.08]  [draw opacity=0] (13.4,-6.43) -- (0,0) -- (13.4,6.44) -- (8.9,0) -- cycle    ;
	%Curve Lines [id:da21076727837307252] 
	\draw    (360,120) .. controls (368.32,141.76) and (372.34,167.82) .. (361.44,187.57) ;
	\draw [shift={(360,190)}, rotate = 302.35] [fill={rgb, 255:red, 0; green, 0; blue, 0 }  ][line width=0.08]  [draw opacity=0] (10.72,-5.15) -- (0,0) -- (10.72,5.15) -- (7.12,0) -- cycle    ;
	%Straight Lines [id:da050615307129076026] 
	\draw [line width=1.5]    (270,240) -- (356.67,240) ;
	\draw [shift={(360.67,240)}, rotate = 180] [fill={rgb, 255:red, 0; green, 0; blue, 0 }  ][line width=0.08]  [draw opacity=0] (13.4,-6.43) -- (0,0) -- (13.4,6.44) -- (8.9,0) -- cycle    ;
	%Shape: Circle [id:dp11678220559527897] 
	\draw  [fill={rgb, 255:red, 0; green, 0; blue, 0 }  ,fill opacity=1 ] (266,240) .. controls (266,237.79) and (267.79,236) .. (270,236) .. controls (272.21,236) and (274,237.79) .. (274,240) .. controls (274,242.21) and (272.21,244) .. (270,244) .. controls (267.79,244) and (266,242.21) .. (266,240) -- cycle ;
	%Straight Lines [id:da6964104143677115] 
	\draw [line width=0.75]  [dash pattern={on 0.84pt off 2.51pt}]  (270,160) -- (326.57,103.43) ;

	% Text Node
	\draw (450.67,238.33) node    {$x$};
	% Text Node
	\draw (290,255) node    {$M\vec{g}$};
	% Text Node
	\draw (254,113.67) node    {$\vec{R}_{n}$};
	% Text Node
	\draw (359.67,222) node    {$\vec{R}_{t}$};
	% Text Node
	\draw (352.67,77.67) node    {$\vec{F}_{o}$};
	% Text Node
	\draw (192.67,223) node    {$\vec{F}_{o}$};
	% Text Node
	\draw (305.67,141.33) node    {$R$};


	\end{tikzpicture}
\end{figure}
\FloatBarrier
Applicare una coppia di forze di questo tipo è equivalente a applicare un momento $M_z$ esterno di intensità $2 F_o R$. Applicando la prima e la seconda legge cardinale della dinamica:
\[
	\begin{array}{rl}
		\text{I equazione} & \left\{\begin{array}{l}
									 	x:\quad R_t=Ma_{cm}  \\
										y:\quad Mg = R_n
									\end{array} \right. \\ \\
		\text{II equazione} & M_z = I_z\alpha
	\end{array}
\]
Si cosideri un asse $z$ passante per $C$, lo si sceglie con verso entrante nel piano del foglio. Rispetto al polo $C$, l'unica forza che genera momento è $\vec{F}_o$ (sopra). Il suo momento sarà:
\[
	M_z = F_o\cdot 2R = I_z\alpha \qquad \alpha = \frac{F_o 2 R}{I_z }
\]
Il momento di inerzia non è riferito all'asse passante per il centro di massa, ma per il teorema di Huygens-Steiner, sarà:
\[
	I_z = \frac{1}{2} MR^2 + MR^2 = \frac{3}{2} MR^2 \qquad \alpha = \frac{F_o 4 R}{3MR^2} = \frac{4}{3}\frac{F_o }{MR}
\]
Il piano d'appoggio deve generare un attrito sufficiente a bloccare quel punto di contatto. Una tipica domanda è: qual è il coefficiente di attrito statico minimo affinché il corpo ruoti senza scivolare?
\[
	R_t = Ma_{cm} = M\alpha R < \mu_s Mg \implies \mu_s>\frac{\alpha R}{g}
\]

\paragraph{Forza di attrito volvente} La forza di attrito statico non andrà a dissipare via via l'energia cinetica del sistema perché non provoca movimento e quindi non compie alcun lavoro. Agisce su un punto di applicazione istantaneamente fermo. Il moto che si ottiene va avanti indefinitamente nel tempo. Se non ci fosse nient'altro che fa rallentare il corpo, esso non si fermerebbe mai, tuttavia nella realtà ciò non accade.

Per capire questo concetto, si prende un disco che, sulla sommità di un cuneo inclinato di un angolo $\vartheta$, rotola lungo un piano e si osserva l'andamento della sua velocità angolare. Si immagini che vada avanti su un piano orizzontale scabro. Durante il moto di rotolamento senza strisciamento il corpo pian piano rallenta,  perché entra anche in gioco l'attrito volvente. Durante il moto il corpo si deforma, schiacciandosi in corrispondenza del punto di contatto. La reazione normale esercitata dal piano d'appoggio non è veramente applicata sul punto passante per la verticale al centro di massa ma un po' spostata verso destra. Tale reazione normale tende a generare un momento opposto al verso di rotazione che rallenta il moto. Per vincere il momento dovuto all'attrito volvente, si deve applicare al corpo di forma circolare una forza di trazione.
\begin{figure}[htpb]
	\centering
	

	\tikzset{every picture/.style={line width=0.75pt}} %set default line width to 0.75pt        

	\begin{tikzpicture}[x=0.75pt,y=0.75pt,yscale=-1,xscale=1]
	%uncomment if require: \path (0,300); %set diagram left start at 0, and has height of 300

	%Straight Lines [id:da5764642361967112] 
	\draw    (120,260) -- (457,260) ;
	\draw [shift={(460,260)}, rotate = 180] [fill={rgb, 255:red, 0; green, 0; blue, 0 }  ][line width=0.08]  [draw opacity=0] (10.72,-5.15) -- (0,0) -- (10.72,5.15) -- (7.12,0) -- cycle    ;
	%Shape: Circle [id:dp31649283228684366] 
	\draw  [draw opacity=0][fill={rgb, 255:red, 216; green, 216; blue, 216 }  ,fill opacity=1 ] (210,180) .. controls (210,135.82) and (245.82,100) .. (290,100) .. controls (334.18,100) and (370,135.82) .. (370,180) .. controls (370,224.18) and (334.18,260) .. (290,260) .. controls (245.82,260) and (210,224.18) .. (210,180) -- cycle ;
	%Straight Lines [id:da8430112699742474] 
	\draw [line width=1.5]    (290,260) -- (290,164) ;
	\draw [shift={(290,160)}, rotate = 450] [fill={rgb, 255:red, 0; green, 0; blue, 0 }  ][line width=0.08]  [draw opacity=0] (13.4,-6.43) -- (0,0) -- (13.4,6.44) -- (8.9,0) -- cycle    ;
	%Curve Lines [id:da9237086565736599] 
	\draw    (380,140) .. controls (388.32,161.76) and (392.34,187.82) .. (381.44,207.57) ;
	\draw [shift={(380,210)}, rotate = 302.35] [fill={rgb, 255:red, 0; green, 0; blue, 0 }  ][line width=0.08]  [draw opacity=0] (10.72,-5.15) -- (0,0) -- (10.72,5.15) -- (7.12,0) -- cycle    ;

	% Text Node
	\draw (314,205.67) node    {$\vec{R}_{n}$};
	% Text Node
	\draw (409.67,177.67) node    {$\vec{\omega }$};


	\end{tikzpicture}
\end{figure}























































































































\chapter{Statica dei fluidi}

\section{Generalità sui fluidi: densità e pressione}

Si passa ora dallo studio dei solidi, a quello di un fluido, ossia di un materiale in fase liquida o gassosa. Ci si limiterà a studiarne le condizioni di equilibrio statico. Nei corpi in forma solida, i legami molecolari sono talmente forti da rendere difficile deformarli, anche se in realtà essi hanno sempre un grado di deformabilità, tanto che un corpo ha anche proprietà elastiche.

Un materiale indeformabile può essere allungato in maniera elastica, per poi tornare quindi alla posizione di partenza o in maniera plastica e quindi permanente. Sicuramente però quando si va a studiare un fluido c'è una grossa differenza in termini di legami molecolari. Fondamentalmente i fluidi, mentre possono resistere ad un tentativo di compressione, non sono in grado di sopportare gli sforzi di taglio. Se si prende un fluido, gli si può applicare una forza che sia ortogonale alla superficie cercando di comprimerlo e non si comprimerà. Se si va a spingere a lato il fluido, cosa che si traduce dicendo che gli si sta \emph{applicando uno sforzo di taglio}, esso trasla. I fluidi non riescono a opporsi alle forze di taglio e quindi sono messi in movimento. Dal punto di vista microscopico, questo è dovuto al fatto che i legami molecolari tendono ad essere molto deboli nel piano del liquido, quindi si verifica uno scorrimento delle molecole.
Se ci si vuole limitare a studiare un fluido in condizione statiche, bisogna immaginare di essere in una situazione in cui le forze esterne sul fluido sono normali alla superficie, ed entrerà in gioco il concetto di pressione. Presa una superficie $dS$ all'interno del fluido, orientata in maniera casuale, su di essa agirà una forza che dovrà essere necessariamente normale ad essa, altrimenti sarebbe di taglio e metterebbe $dS$ in moto.
\begin{figure}[htpb]
	\centering
	

	\tikzset{every picture/.style={line width=0.75pt}} %set default line width to 0.75pt        

	\begin{tikzpicture}[x=0.75pt,y=0.75pt,yscale=-1,xscale=1]
	%uncomment if require: \path (0,300); %set diagram left start at 0, and has height of 300

	%Shape: Rectangle [id:dp3713564616734577] 
	\draw  [draw opacity=0][fill={rgb, 255:red, 212; green, 212; blue, 212 }  ,fill opacity=1 ] (170,130) -- (370,130) -- (370,220) -- (170,220) -- cycle ;
	%Shape: Rectangle [id:dp8448984996265976] 
	\draw  [draw opacity=0][fill={rgb, 255:red, 155; green, 155; blue, 155 }  ,fill opacity=1 ] (329.71,170.63) -- (339.37,180.06) -- (330.29,189.37) -- (320.63,179.94) -- cycle ;
	%Straight Lines [id:da35624833529467814] 
	\draw    (170,110) -- (170,220) ;
	%Straight Lines [id:da9756231879480504] 
	\draw    (370,110) -- (370,220) ;
	%Straight Lines [id:da6992538058902569] 
	\draw    (170,220) -- (370,220) ;
	%Straight Lines [id:da08382077498641105] 
	\draw    (290,100) -- (290,117) ;
	\draw [shift={(290,120)}, rotate = 270] [fill={rgb, 255:red, 0; green, 0; blue, 0 }  ][line width=0.08]  [draw opacity=0] (10.72,-5.15) -- (0,0) -- (10.72,5.15) -- (7.12,0) -- cycle    ;
	%Straight Lines [id:da5990049494414285] 
	\draw    (190,160) -- (257,160) ;
	\draw [shift={(260,160)}, rotate = 180] [fill={rgb, 255:red, 0; green, 0; blue, 0 }  ][line width=0.08]  [draw opacity=0] (10.72,-5.15) -- (0,0) -- (10.72,5.15) -- (7.12,0) -- cycle    ;
	%Straight Lines [id:da2443778154609575] 
	\draw    (300.25,151) -- (327.85,177.91) ;
	\draw [shift={(330,180)}, rotate = 224.27] [fill={rgb, 255:red, 0; green, 0; blue, 0 }  ][line width=0.08]  [draw opacity=0] (10.72,-5.15) -- (0,0) -- (10.72,5.15) -- (7.12,0) -- cycle    ;

	% Text Node
	\draw (229.5,174.5) node   [align=left] {sforzo di taglio};
	% Text Node
	\draw (322,148.5) node    {$\vec{F}_{n}$};
	% Text Node
	\draw (345,188.5) node    {$dS$};


	\end{tikzpicture}
\end{figure}
\FloatBarrier
Questa forza, che si può immaginare per unità di superficie, prende il nome di \textbf{pressione} agente nel fluido in quel punto. È una grandezza scalare, si sa già infatti che è frutto di una forza che agisce ortogonalmente alla superficie, è sufficiente darne il valore.
La pressione $p$ è l'intensità della forza infinitesima che agisce su una superficie infinitesima:
\[
	p(P) = \frac{|d\vec{F}_n|}{dS}
\]
Essa può variare di punto in punto del fluido. Dimensionalmente è una forza per unità di superficie e quindi si misura in: $N/m^2$, a cui si da il nome di \emph{Pascal}, $Pa$. Esistono altre unità di misura molto usate per dare il valore di una pressione. Ad esempio, l'atmosfera è un fluido e ci si aspetta che in ogni punto di essa ci sia una certa pressione. Sulla superficie del mare, essa ha un preciso valore identificato da un'unità di misura che prende il nome di atmosfera:
\[
	1\,atm = 1.013 \times 10^5\,Pa
\]
Altre note unità di misura per la pressione sono:
\begin{itemize}
	\item il bar $1\,bar = 10^5\,Pa$
	\item millimetri di mercurio (in termini di quota di una colonna riempita di mercurio liquido): $1\,atm = 760\,mmHg$
\end{itemize}
Si può dimostrare che la pressione in un fluido, in un certo suo punto, è indipendente da come è orientata la superficie su cui la si sta misurando. Si dice che è una grandezza isotropa, che cioè non dipende appunto da come è orientata la superficie ma soltanto dal punto considerato.

Per caratterizzare un fluido è bene definirne anche la densità. Dato un volumetto di fluido infinitesimo si va a definire la \textbf{densità del fluido} in quel punto come il rapporto:
\[
	p(P) = \frac{dm}{dV} \implies p_{H_2 O, liq } = \frac{1\,kg}{1\,dm^3 } = 10^3\,kg/m^3
\]







































\section{Equilibrio statico di un fluido e legge di Stevino}

A questo punto ci si pone come obbiettivo quello di capire come varia la pressione all'interno di un fluido quando è sottoposto all'azione di forze esterne, ad esempio la forza peso. Si otterrà una legge generale che potrà informare su come varia la pressione con la profondità.
Si supponga di avere un fluido in condizioni statiche, cioè tale per cui, preso un volumetto di esso, questo sarà in equilibrio rispetto alle forze che agiscono. Bisogna però distinguere le forze in due categorie:
\begin{itemize}
	\item \textbf{Forze di volume}: quelle forze tali per cui la loro intensità è proporzionale al volume.
	\item \textbf{Forze di superficie}: si tratta di forze che agiscono su ognuna delle faccine del volumetto, la loro intensità è proporzionale alla superficie su cui stanno agendo e sono legate alla pressione del fluido.
\end{itemize}
\begin{figure}[htpb]
	\centering
	

	\tikzset{every picture/.style={line width=0.75pt}} %set default line width to 0.75pt        

	\begin{tikzpicture}[x=0.75pt,y=0.75pt,yscale=-1,xscale=1]
	%uncomment if require: \path (0,300); %set diagram left start at 0, and has height of 300

	%Shape: Rectangle [id:dp7439777710631521] 
	\draw  [draw opacity=0][fill={rgb, 255:red, 212; green, 212; blue, 212 }  ,fill opacity=1 ] (160,143) -- (331.5,143) -- (331.5,271) -- (160,271) -- cycle ;
	%Straight Lines [id:da12104530843999184] 
	\draw    (160,121.58) -- (160,271) ;
	%Straight Lines [id:da7693512673530147] 
	\draw    (331.5,121.58) -- (331.5,271) ;
	%Straight Lines [id:da2919712740516276] 
	\draw    (160,271) -- (331.5,271) ;
	%Straight Lines [id:da8045135759389666] 
	\draw    (240,170) -- (240,197.17) ;
	\draw [shift={(240,200.17)}, rotate = 270] [fill={rgb, 255:red, 0; green, 0; blue, 0 }  ][line width=0.08]  [draw opacity=0] (10.72,-5.15) -- (0,0) -- (10.72,5.15) -- (7.12,0) -- cycle    ;
	%Straight Lines [id:da5794484438449439] 
	\draw    (350,270.1) -- (350,120) ;
	\draw [shift={(350,117)}, rotate = 450] [fill={rgb, 255:red, 0; green, 0; blue, 0 }  ][line width=0.08]  [draw opacity=0] (10.72,-5.15) -- (0,0) -- (10.72,5.15) -- (7.12,0) -- cycle    ;
	%Straight Lines [id:da935822571715307] 
	\draw    (176,200) -- (213.87,162.28) ;
	\draw [shift={(216,160.17)}, rotate = 495.12] [fill={rgb, 255:red, 0; green, 0; blue, 0 }  ][line width=0.08]  [draw opacity=0] (10.72,-5.15) -- (0,0) -- (10.72,5.15) -- (7.12,0) -- cycle    ;
	%Shape: Cube [id:dp6006406406068177] 
	\draw   (227.14,203.89) -- (231.03,200) -- (250,200) -- (250,216.11) -- (246.11,220) -- (227.14,220) -- cycle ; \draw   (250,200) -- (246.11,203.89) -- (227.14,203.89) ; \draw   (246.11,203.89) -- (246.11,220) ;
	%Straight Lines [id:da3179623071123807] 
	\draw    (240,222.83) -- (240,250) ;
	\draw [shift={(240,219.83)}, rotate = 90] [fill={rgb, 255:red, 0; green, 0; blue, 0 }  ][line width=0.08]  [draw opacity=0] (10.72,-5.15) -- (0,0) -- (10.72,5.15) -- (7.12,0) -- cycle    ;
	%Straight Lines [id:da7311859668862664] 
	\draw  [dash pattern={on 0.84pt off 2.51pt}]  (246.11,203.89) -- (350,203.89) ;
	%Straight Lines [id:da042259797735868965] 
	\draw  [dash pattern={on 0.84pt off 2.51pt}]  (246.11,220) -- (350,220) ;

	% Text Node
	\draw (176.5,164.5) node    {$\vec{F}_{v}$};
	% Text Node
	\draw (365,117) node    {$z$};
	% Text Node
	\draw (283.5,181.5) node    {$\vec{F}_{s}( z+dz)$};
	% Text Node
	\draw (268.5,234.5) node    {$\vec{F}_{s}( z)$};
	% Text Node
	\draw (360.33,218) node    {$z$};
	% Text Node
	\draw (376.33,198.67) node    {$z+dz$};


	\end{tikzpicture}
\end{figure}
\FloatBarrier
Si immagini di avere all'interno del fluido una certa forza di volume, che ha una direzione qualunque rispetto a un sistema di riferimento $xyz$ come in figura. Essa ha una componente lungo i tre assi $x, y$ e $z$. La faccia più bassa del volumetto è alla quota $z$, mentre quella più alta alla quota $z+dz$. Si tratta infatti di un volume molto piccolo di lati $dx, dy, dz$. In condizioni di equilibrio statico la forza di superficie, cioè la pressione di tutte le altre parti del fluido, varierà in modo tale da equilibrare la forza di volume.  Si studia ora il problema solo lungo l'asse $z$ per poi generalizzare. Lungo $z$ si avrà una pressione che agisce sulla faccia più alta del volume e una sulla faccia più bassa (analogamente si ha la pressione che agisce nelle altre direzioni). Si impone che le forze totali sul fluido debbano equilibrarsi fra di loro:
\begin{gather*}
	\vec{F}_s + \vec{F}_v = 0 \quad \text{scompongo lungo z}\\
	z:\quad p(z)dS - p(z+dz)dS + F_{v,z} = 0
\end{gather*}
$dS$ è un elemento di superficie che si può scrivere come: $dS=dx\,dy$.
Si definisce la forza di volume che sta agendo in un certo volume del fluido introducendo il concetto di \textbf{densità della forza di volume}. Essa è il rapporto:
\[
	\vec{f}_v = \frac{\vec{F}_v }{dm} \]
Si ha quindi:
\[
	\quad \vec{F}_v = \vec{f}_v\,dm = \vec{f}_v\rho dV = \vec{f}_v\rho\,dx\,dy\,dz
\]
\begin{equation*}
	\begin{aligned}
		z&:\quad p(z)dx\,dy - p(z+dz)dx\,dy + f_v\,dx\,dy\,dz = 0 \\
		z&:\quad p(z) - p(z+dz) + f_v dz = 0
	\end{aligned}
\end{equation*}
Ora bisogna semplicemente riscrivere il secondo termine sfruttando le proprietà delle derivate. La derivata di una funzione infatti non è altro che il limite del rapporto incrementale:
\begin{gather*}
	\frac{dp}{dz} = \frac{p(z+dz)-p(z)}{dz} \implies p(z+dz)=\frac{dp}{dz}dz + p(z) \\
	z:\quad p(z)-p(z) - \frac{dp}{dz}dz + \vec{f}_{v,z}\rho dz =0 \\
	\frac{dp}{dz} = \vec{f}_{v,z}\rho
\end{gather*}
Data una certa direzione, la pressione del fluido varierà con la posizione esattamente in maniera tale da essere pari alla densità della forza di volume per la densità del fluido. Questo risultato è stato ottenuto per la componente in $z$, ma un risultato del tutto analogo (la forza di volume è generale in questo caso) lo si otterrà sulle altre direzioni:
\[
	\frac{\partial p}{\partial z}  = f_{v,z}\rho \qquad \frac{\partial p}{\partial y}  = f_{v,y}\rho \qquad \frac{\partial p}{\partial x}  = f_{v,x}\rho
\]
La pressione in un fluido può variare in $x, y, z$ in maniera tale da seguire l'intensità della componente della forza di volume che agisce sul fluido. Questo viene scritto in modo molto più compatto introducento l'operatore \textbf{gradiente}.  È un operatore (per definizione un vettore) che trasforma una grandezza scalare in una grandezza vettoriale ed è la variazione della funzione scalare a cui è applicato nelle tre direzioni dello spazio, ognuna moltiplicata per il suo versore.  Si dimostra che esso indica la direzione di massima variazione della funzione a cui è applicato. Si scrive anche come \emph{nabla} della grandezza.
\[
	\text{grad}p = \rho \vec{f}_v = \left\{ \begin{array}{r}
	 	\frac{\partial p}{\partial z} = f_{v,z}\rho \\
	 	\frac{\partial p}{\partial y} = f_{v,y}\rho \\
	 	\frac{\partial p}{\partial x} = f_{v,x}\rho
	\end{array} \right.
\]
In maniera compatta la relazione si può scrivere dicendo che in un punto di condizioni statiche, la pressione deve variare da punto a punto in modo tale che il suo gradiente sia equiverso e parallelo alla direzione della forza di volume, e in intensità deve essere pari all'intensità di forza di volume per la densità del fluido.
\[
	\text{grad}p = \underbrace{\vec{\nabla}}_\text{`nabla'} p = \frac{\partial p}{\partial x} \vec{u}_x + \frac{\partial p}{\partial y} \vec{u}_y + \frac{\partial p}{\partial z} \vec{u}_z
\]

\paragraph{Esempio} Si immagini che la forza di volume in questione sia la forza peso.
\[
	\vec{F}_v = dm\,\vec{g} = \rho\,dV\vec{g}  \quad \vec{f}_v = \vec{g} = -g\vec{u}_z
\]
Si va a sostituire quello che si ha in questo caso:
\[
	\vec{\nabla} p = \left\{ \begin{array}{l}
	 	\frac{\partial p}{\partial z} = -\rho\,g \\
	 	\frac{\partial p}{\partial y} = 0 \\
	 	\frac{\partial p}{\partial x} = 0
	\end{array} \right.
\]
Si immagini di avere un fluido in cui si pone l'origine dell'asse $z$ sul pelo dell'acqua in modo tale da vedere come varia la pressione con la profondità del fluido. Essa aumenta perché il volume di fluido deve sopportare il peso di tutta la colonna d'acqua che c'è sopra di lui, altrimenti l'acqua si muoverebbe verso il basso. Se la densità del fluido ha densità costante:
\[
	\int_{p_0 }^{p(z)} dp = \int_0^{-z} -p\,g\,dz= p(-z) - p_0= -\rho g(-z) \implies p(-z) ) p_0 + \rho gz
\]
Questa prende il nome di \textbf{legge di Stevino}. Essa informa su come varia la pressione in un fluido al variare della profondità. La pressione alla profondità $H$ è la pressione che si avrebbe sul pelo libero dell'acqua più $\rho g H$.
\[
	\boxed{p(H) = p_0 + \rho gH}
\]
La pressione è aumentata del peso della colonna di fluido che sta al di sopra del punto considerato.
\begin{figure}[htpb]
	\centering
	

	\tikzset{every picture/.style={line width=0.75pt}} %set default line width to 0.75pt        

	\begin{tikzpicture}[x=0.75pt,y=0.75pt,yscale=-1,xscale=1]
	%uncomment if require: \path (0,300); %set diagram left start at 0, and has height of 300

	%Shape: Rectangle [id:dp9257457317027504] 
	\draw  [draw opacity=0][fill={rgb, 255:red, 212; green, 212; blue, 212 }  ,fill opacity=1 ] (160,135.25) -- (331.5,135.25) -- (331.5,242.67) -- (160,242.67) -- cycle ;
	%Straight Lines [id:da45733836926557014] 
	\draw    (160,105.08) -- (160,242.67) ;
	%Straight Lines [id:da27101038653804976] 
	\draw    (331.5,105.08) -- (331.5,242.67) ;
	%Straight Lines [id:da2452169725190556] 
	\draw    (160,242.67) -- (331.5,242.67) ;
	%Straight Lines [id:da1603614064589991] 
	\draw    (350,242) -- (350,103) ;
	\draw [shift={(350,100)}, rotate = 450] [fill={rgb, 255:red, 0; green, 0; blue, 0 }  ][line width=0.08]  [draw opacity=0] (10.72,-5.15) -- (0,0) -- (10.72,5.15) -- (7.12,0) -- cycle    ;
	%Straight Lines [id:da8367477743150609] 
	\draw    (253.33,160.33) -- (253.33,221.33) ;
	\draw [shift={(253.33,224.33)}, rotate = 270] [fill={rgb, 255:red, 0; green, 0; blue, 0 }  ][line width=0.08]  [draw opacity=0] (10.72,-5.15) -- (0,0) -- (10.72,5.15) -- (7.12,0) -- cycle    ;
	%Straight Lines [id:da6069175472492154] 
	\draw    (345.67,135.67) -- (355,135.67) ;
	%Straight Lines [id:da5622008752683028] 
	\draw    (345.67,215.67) -- (355,215.67) ;

	% Text Node
	\draw (243.17,186.5) node    {$\vec{g}$};
	% Text Node
	\draw (363,90) node    {$z$};
	% Text Node
	\draw (366.33,214) node    {$z$};
	% Text Node
	\draw (366.33,132.67) node    {$0$};


	\end{tikzpicture}
\end{figure}
\FloatBarrier
Si vuole ora capire come varia la pressione con la quota nell'atmosfera. Ci si aspetta che essa diminuisca, perché se si va in quota, sopra si ha una colonna d'aria minore di quella che vi è sopra il pelo dell'acqua. L'atmosfera ha una densità che via via diminuisce mano a mano che ci si sposta in quota e questo rende la relazione è più complicata. Si ha un integrale in cui $\rho$ varia con la quota, la densità dell'aria è proporzionale alla pressione stessa. Si può risolvere e si trova che la $p$ diminuisce esponenzialmente con la quota.
\begin{figure}[htpb]
	\centering
	

	\tikzset{every picture/.style={line width=0.75pt}} %set default line width to 0.75pt        

	\begin{tikzpicture}[x=0.75pt,y=0.75pt,yscale=-1,xscale=1]
	%uncomment if require: \path (0,300); %set diagram left start at 0, and has height of 300

	%Shape: Polygon [id:ds7354615261716488] 
	\draw  [draw opacity=0][fill={rgb, 255:red, 212; green, 212; blue, 212 }  ,fill opacity=1 ] (252.13,256.9) -- (172.67,183.92) -- (319.33,183.92) -- cycle ;
	%Straight Lines [id:da20099196556604415] 
	\draw    (258.54,76.19) -- (165.61,177.64) ;
	%Straight Lines [id:da2835286186128565] 
	\draw    (345.06,155.44) -- (252.13,256.9) ;
	%Straight Lines [id:da7481027563971243] 
	\draw    (165.61,177.64) -- (252.13,256.9) ;
	%Straight Lines [id:da8587576102155454] 
	\draw    (261.33,198) -- (261.33,223.33) ;
	\draw [shift={(261.33,226.33)}, rotate = 270] [fill={rgb, 255:red, 0; green, 0; blue, 0 }  ][line width=0.08]  [draw opacity=0] (10.72,-5.15) -- (0,0) -- (10.72,5.15) -- (7.12,0) -- cycle    ;

	% Text Node
	\draw (247.17,205.17) node    {$\vec{g}$};


	\end{tikzpicture}
\end{figure}
\FloatBarrier
La legge di Stevino dice che la pressione nel fluido varia massimamente nella direzione in cui sta agendo la forza di volume. Ad esempio, se si ha un bicchiere messo come in figura, il pelo dell'acqua si distribuisce così in conseguenza della legge generale di equilibrio in un fluido. Per non sentire variazione di volume ci si deve muovere ortogonalmente alla forza di volume. Le superfici ortogonali alla forza di volume che sta agendo sono superfici isobare.
Il pelo libero dell'acqua è una superficie isobara. Tutti i suoi punti sono soggetti allo stesso valore della superficie atmosferica. Ecco perché il pelo libero dell'acqua si dispone sempre in modo da essere ortogonale alla forza di volume. Le superfici isobare sono sempre ortogonali all'accelerazione di gravità.







































\section{Applicazioni della legge di Stevino}

Ci sono alcuni risultati importanti conseguenza della legge di Stevino.

\paragraph{Legge dei vasi comunicanti} Si hanno due vasi riempiti dello stesso liquido che comunicano tra di loro tramite un tubo: il livello del pelo del liquido, sotto l'ipotesi che sia a contatto con l'atmosfera, sarà lo stesso.
\begin{figure}[htpb]
	\centering
	

	\tikzset{every picture/.style={line width=0.75pt}} %set default line width to 0.75pt        

	\begin{tikzpicture}[x=0.75pt,y=0.75pt,yscale=-1,xscale=1]
	%uncomment if require: \path (0,300); %set diagram left start at 0, and has height of 300

	%Shape: Rectangle [id:dp27704513154870636] 
	\draw  [draw opacity=0][fill={rgb, 255:red, 212; green, 212; blue, 212 }  ,fill opacity=1 ] (160,130) -- (300,130) -- (300,240) -- (160,240) -- cycle ;
	%Shape: Rectangle [id:dp4584260031687635] 
	\draw  [draw opacity=0][fill={rgb, 255:red, 212; green, 212; blue, 212 }  ,fill opacity=1 ] (390,130) -- (420,130) -- (420,240) -- (390,240) -- cycle ;
	%Shape: Rectangle [id:dp0668709439513262] 
	\draw  [draw opacity=0][fill={rgb, 255:red, 212; green, 212; blue, 212 }  ,fill opacity=1 ] (300,200) -- (390,200) -- (390,240) -- (300,240) -- cycle ;
	%Straight Lines [id:da4468593700445236] 
	\draw    (160,100) -- (160,240) ;
	%Straight Lines [id:da38055217362304306] 
	\draw    (300,100) -- (300,200) ;
	%Straight Lines [id:da049168839863139135] 
	\draw    (160,240) -- (420,240) ;
	%Straight Lines [id:da6961574404465656] 
	\draw    (210,173) -- (210,227) ;
	\draw [shift={(210,230)}, rotate = 270] [fill={rgb, 255:red, 0; green, 0; blue, 0 }  ][line width=0.08]  [draw opacity=0] (10.72,-5.15) -- (0,0) -- (10.72,5.15) -- (7.12,0) -- cycle    ;
	\draw [shift={(210,170)}, rotate = 90] [fill={rgb, 255:red, 0; green, 0; blue, 0 }  ][line width=0.08]  [draw opacity=0] (10.72,-5.15) -- (0,0) -- (10.72,5.15) -- (7.12,0) -- cycle    ;
	%Straight Lines [id:da8280487569760746] 
	\draw    (420,100) -- (420,240) ;
	%Straight Lines [id:da8954022549832723] 
	\draw    (390,100) -- (390,200) ;
	%Straight Lines [id:da05136683391980745] 
	\draw    (390,200) -- (300,200) ;
	%Straight Lines [id:da1644519278898211] 
	\draw    (160,240) -- (420,240) ;
	%Straight Lines [id:da6179116634021047] 
	\draw  [dash pattern={on 0.84pt off 2.51pt}]  (160,170) -- (420,170) ;
	%Straight Lines [id:da16023166347142093] 
	\draw  [dash pattern={on 0.84pt off 2.51pt}]  (160,230) -- (420,230) ;

	% Text Node
	\draw (221.5,200) node    {$H$};
	% Text Node
	\draw (226.5,151) node    {$p_{1} =p_{0} +\rho gH$};
	% Text Node
	\draw (476.5,151) node    {$p_{2} =p_{0} +\rho gH$};
	% Text Node
	\draw (482.5,226.5) node   [align=left] {superfici isobare};


	\end{tikzpicture}
\end{figure}
\FloatBarrier
Infatti, se si considera una superficie ad una certa profondità del fluido, quella superficie tratteggiata deve essere isobara. Ma il salto di pressione tra la pressione nel punto considerato e la superficie, è dato dalla colonna alta $H$ di fluido, quindi le due altezze devono essere uguali. Questo oggetto lo si potrebbe disegnare come un \emph{manometro} a U, che serve per misurare la densità di un fluido rispetto ad un altro di densità nota. Riempiendolo con due liquidi immiscibili, di densità $\rho_1$ e $\rho_2$, i peli si dispongono in maniera tale da non essere più alla stessa quota. Il liquido di densità minore ha bisogno del peso di una colonna più alta.
\begin{figure}[htpb]
	\centering
	

	\tikzset{every picture/.style={line width=0.75pt}} %set default line width to 0.75pt        

	\begin{tikzpicture}[x=0.75pt,y=0.75pt,yscale=-1,xscale=1]
	%uncomment if require: \path (0,300); %set diagram left start at 0, and has height of 300

	%Shape: Rectangle [id:dp44840532884233686] 
	\draw  [draw opacity=0][fill={rgb, 255:red, 212; green, 212; blue, 212 }  ,fill opacity=1 ] (270,210) -- (330,210) -- (330,240) -- (270,240) -- cycle ;
	%Shape: Rectangle [id:dp7423661924962384] 
	\draw  [draw opacity=0][fill={rgb, 255:red, 212; green, 212; blue, 212 }  ,fill opacity=1 ] (330,130) -- (360,130) -- (360,240) -- (330,240) -- cycle ;
	%Shape: Rectangle [id:dp7738438895244188] 
	\draw  [draw opacity=0][fill={rgb, 255:red, 155; green, 155; blue, 155 }  ,fill opacity=1 ] (270,210) -- (270,240) -- (190,240) -- (190,210) -- cycle ;
	%Shape: Rectangle [id:dp9832647459808672] 
	\draw  [draw opacity=0][fill={rgb, 255:red, 155; green, 155; blue, 155 }  ,fill opacity=1 ] (160,180) -- (190,180) -- (190,240) -- (160,240) -- cycle ;
	%Straight Lines [id:da4591492750465662] 
	\draw    (160,100) -- (160,240) ;
	%Straight Lines [id:da3814059419644684] 
	\draw    (190,100) -- (190,210) ;
	%Straight Lines [id:da7914670550753795] 
	\draw    (160,240) -- (360,240) ;
	%Straight Lines [id:da1987832309048725] 
	\draw    (360,100) -- (360,240) ;
	%Straight Lines [id:da31302221508785455] 
	\draw    (330,100) -- (330,210) ;
	%Straight Lines [id:da009014055394372722] 
	\draw    (330,210) -- (190,210) ;
	%Straight Lines [id:da10964146931263641] 
	\draw    (160,240) -- (360,240) ;
	%Straight Lines [id:da6091679957128837] 
	\draw  [dash pattern={on 0.84pt off 2.51pt}]  (160,230) -- (360,230) ;
	%Straight Lines [id:da27691635574222273] 
	\draw    (120,180) -- (120,230) ;
	\draw [shift={(120,230)}, rotate = 270] [color={rgb, 255:red, 0; green, 0; blue, 0 }  ][line width=0.75]    (0,5.59) -- (0,-5.59)   ;
	\draw [shift={(120,180)}, rotate = 270] [color={rgb, 255:red, 0; green, 0; blue, 0 }  ][line width=0.75]    (0,5.59) -- (0,-5.59)   ;
	%Straight Lines [id:da8085267888906842] 
	\draw    (390,130) -- (390,230) ;
	\draw [shift={(390,230)}, rotate = 270] [color={rgb, 255:red, 0; green, 0; blue, 0 }  ][line width=0.75]    (0,5.59) -- (0,-5.59)   ;
	\draw [shift={(390,130)}, rotate = 270] [color={rgb, 255:red, 0; green, 0; blue, 0 }  ][line width=0.75]    (0,5.59) -- (0,-5.59)   ;

	% Text Node
	\draw (145,227) node    {$p_{x}$};
	% Text Node
	\draw (175,188) node    {$p_{1}$};
	% Text Node
	\draw (176,105) node    {$p_{0}$};
	% Text Node
	\draw (345,106) node    {$p_{0}$};
	% Text Node
	\draw (345,138) node    {$p_{2}$};
	% Text Node
	\draw (107.5,205) node    {$H_{1}$};
	% Text Node
	\draw (489,158) node    {$p_{1}  >p_{2}$};
	% Text Node
	\draw (405.5,205) node    {$H_{2}$};


	\end{tikzpicture}
\end{figure}
\FloatBarrier
Per capire la densità $\rho_1$ si guarda la differenza di quota.
\[
	p_x = p_0 + \rho_1 g H_1 = p_0 + \rho_2 gH_2 \implies \boxed{\rho_2 = \rho_1\frac{H_1 }{H_2 }  }
\]
Questa volta si riempie il manometro a U con un fluido di densità nota, l'acqua. Un braccio lo si mette a contatto con l'atmosfera e l'altro con una camera contenente un fluido non miscibile con quello nel manometro.
\begin{figure}[htpb]
	\centering
	

	\tikzset{every picture/.style={line width=0.75pt}} %set default line width to 0.75pt        

	\begin{tikzpicture}[x=0.75pt,y=0.75pt,yscale=-1,xscale=1]
	%uncomment if require: \path (0,300); %set diagram left start at 0, and has height of 300

	%Shape: Rectangle [id:dp3959953820308113] 
	\draw  [draw opacity=0][fill={rgb, 255:red, 212; green, 212; blue, 212 }  ,fill opacity=1 ] (160,130) -- (180,130) -- (180,240) -- (160,240) -- cycle ;
	%Shape: Rectangle [id:dp02069253486010303] 
	\draw  [draw opacity=0][fill={rgb, 255:red, 212; green, 212; blue, 212 }  ,fill opacity=1 ] (180,210) -- (300,210) -- (300,240) -- (180,240) -- cycle ;
	%Shape: Rectangle [id:dp24772278939984171] 
	\draw  [draw opacity=0][fill={rgb, 255:red, 212; green, 212; blue, 212 }  ,fill opacity=1 ] (300,100) -- (320,100) -- (320,240) -- (300,240) -- cycle ;
	%Straight Lines [id:da1464620974589843] 
	\draw    (160,100) -- (160,240) ;
	%Straight Lines [id:da5521606911463739] 
	\draw    (180,100) -- (180,210) ;
	%Straight Lines [id:da3987971312365681] 
	\draw    (160,240) -- (320,240) ;
	%Straight Lines [id:da7965655478878608] 
	\draw    (320,100) -- (320,240) ;
	%Straight Lines [id:da39133260111362134] 
	\draw    (300,100) -- (300,210) ;
	%Straight Lines [id:da5089136574199133] 
	\draw    (300,210) -- (180,210) ;
	%Straight Lines [id:da2020120424295544] 
	\draw    (160,240) -- (320,240) ;
	%Straight Lines [id:da21354002102079073] 
	\draw  [dash pattern={on 0.84pt off 2.51pt}]  (160,130) -- (320,130) ;
	%Straight Lines [id:da5797138891459994] 
	\draw    (320,100) -- (300,100) ;
	%Straight Lines [id:da5184545696166589] 
	\draw  [dash pattern={on 0.84pt off 2.51pt}]  (290,100) -- (320,100) ;
	%Straight Lines [id:da39132803166157104] 
	\draw    (290,103) -- (290,127) ;
	\draw [shift={(290,130)}, rotate = 270] [fill={rgb, 255:red, 0; green, 0; blue, 0 }  ][line width=0.08]  [draw opacity=0] (10.72,-5.15) -- (0,0) -- (10.72,5.15) -- (7.12,0) -- cycle    ;
	\draw [shift={(290,100)}, rotate = 90] [fill={rgb, 255:red, 0; green, 0; blue, 0 }  ][line width=0.08]  [draw opacity=0] (10.72,-5.15) -- (0,0) -- (10.72,5.15) -- (7.12,0) -- cycle    ;
	%Straight Lines [id:da1952310583687571] 
	\draw    (120,50) -- (120,110) ;
	%Straight Lines [id:da9633082911892565] 
	\draw    (220,50) -- (220,110) ;
	%Straight Lines [id:da49174603564191055] 
	\draw    (120,50) -- (220,50) ;
	%Straight Lines [id:da20105406269485737] 
	\draw    (120,110) -- (160,110) ;
	%Straight Lines [id:da4127788726192494] 
	\draw    (180,110) -- (220,110) ;
	%Shape: Circle [id:dp20478396804391652] 
	\draw  [draw opacity=0][fill={rgb, 255:red, 212; green, 212; blue, 212 }  ,fill opacity=1 ] (125.2,61.1) .. controls (125.2,57.18) and (128.38,54) .. (132.3,54) .. controls (136.22,54) and (139.4,57.18) .. (139.4,61.1) .. controls (139.4,65.02) and (136.22,68.2) .. (132.3,68.2) .. controls (128.38,68.2) and (125.2,65.02) .. (125.2,61.1) -- cycle ;
	%Shape: Circle [id:dp5800302968048918] 
	\draw  [draw opacity=0][fill={rgb, 255:red, 212; green, 212; blue, 212 }  ,fill opacity=1 ] (149.6,65.5) .. controls (149.6,61.58) and (152.78,58.4) .. (156.7,58.4) .. controls (160.62,58.4) and (163.8,61.58) .. (163.8,65.5) .. controls (163.8,69.42) and (160.62,72.6) .. (156.7,72.6) .. controls (152.78,72.6) and (149.6,69.42) .. (149.6,65.5) -- cycle ;
	%Shape: Circle [id:dp08611580268843988] 
	\draw  [draw opacity=0][fill={rgb, 255:red, 212; green, 212; blue, 212 }  ,fill opacity=1 ] (136.4,78.3) .. controls (136.4,74.38) and (139.58,71.2) .. (143.5,71.2) .. controls (147.42,71.2) and (150.6,74.38) .. (150.6,78.3) .. controls (150.6,82.22) and (147.42,85.4) .. (143.5,85.4) .. controls (139.58,85.4) and (136.4,82.22) .. (136.4,78.3) -- cycle ;
	%Shape: Circle [id:dp3107456928013643] 
	\draw  [draw opacity=0][fill={rgb, 255:red, 212; green, 212; blue, 212 }  ,fill opacity=1 ] (158.8,83.9) .. controls (158.8,79.98) and (161.98,76.8) .. (165.9,76.8) .. controls (169.82,76.8) and (173,79.98) .. (173,83.9) .. controls (173,87.82) and (169.82,91) .. (165.9,91) .. controls (161.98,91) and (158.8,87.82) .. (158.8,83.9) -- cycle ;
	%Shape: Circle [id:dp957081040610827] 
	\draw  [draw opacity=0][fill={rgb, 255:red, 212; green, 212; blue, 212 }  ,fill opacity=1 ] (183.6,65.5) .. controls (183.6,61.58) and (186.78,58.4) .. (190.7,58.4) .. controls (194.62,58.4) and (197.8,61.58) .. (197.8,65.5) .. controls (197.8,69.42) and (194.62,72.6) .. (190.7,72.6) .. controls (186.78,72.6) and (183.6,69.42) .. (183.6,65.5) -- cycle ;
	%Shape: Circle [id:dp8412082706771953] 
	\draw  [draw opacity=0][fill={rgb, 255:red, 212; green, 212; blue, 212 }  ,fill opacity=1 ] (196.8,79.1) .. controls (196.8,75.18) and (199.98,72) .. (203.9,72) .. controls (207.82,72) and (211,75.18) .. (211,79.1) .. controls (211,83.02) and (207.82,86.2) .. (203.9,86.2) .. controls (199.98,86.2) and (196.8,83.02) .. (196.8,79.1) -- cycle ;
	%Shape: Circle [id:dp8823666291882055] 
	\draw  [draw opacity=0][fill={rgb, 255:red, 212; green, 212; blue, 212 }  ,fill opacity=1 ] (180,87.5) .. controls (180,83.58) and (183.18,80.4) .. (187.1,80.4) .. controls (191.02,80.4) and (194.2,83.58) .. (194.2,87.5) .. controls (194.2,91.42) and (191.02,94.6) .. (187.1,94.6) .. controls (183.18,94.6) and (180,91.42) .. (180,87.5) -- cycle ;
	%Shape: Circle [id:dp2598521058646015] 
	\draw  [draw opacity=0][fill={rgb, 255:red, 212; green, 212; blue, 212 }  ,fill opacity=1 ] (194.8,97.9) .. controls (194.8,93.98) and (197.98,90.8) .. (201.9,90.8) .. controls (205.82,90.8) and (209,93.98) .. (209,97.9) .. controls (209,101.82) and (205.82,105) .. (201.9,105) .. controls (197.98,105) and (194.8,101.82) .. (194.8,97.9) -- cycle ;

	% Text Node
	\draw (137.8,91.4) node    {$p_{x}$};
	% Text Node
	\draw (265.5,114) node    {$\Delta H$};
	% Text Node
	\draw (336,97) node    {$p_{0}$};


	\end{tikzpicture}
\end{figure}
\FloatBarrier
Si vuole capire quale sia la pressione $p_x$. Il fluido si dispone in modo tale da far avere uno sbilanciamento fra i due bracci pari a $\Delta H$. $p_x$ sarà maggiore di $p_o$ perché anche nell'altro braccio alla stessa quota si avrà la stessa pressione di $p_x$, e questa quota sta sotto $p_0$.
\[
	p_x = p_0 + \rho g\Delta H
\]

\paragraph{Il barometro di torricelli} Si ha un recipiente riempito di mercurio. Si usa questo materiale perché è un liquido dalla densità moto elevata. Ora si considera un capillare molto lungo, sopra chiuso e sotto aperto. Si immagina di aspirare completamente tutta l'atmosfera che c'è nel capillare, si dice realizzare il vuoto dentro il capillare. All'intero di esso c'è pressione di 0 $Pa$. Si immerge immediatamente il capillare nel mercurio. Esso risucchia quest'ultimo perché sul pelo del mercurio c'è una pressione atmosferica, alta, che manda il mercurio a riempire il capillare fino ad una certa quota, esattamente pari a 760mm.
\begin{figure}[htpb]
	\centering
	

	\tikzset{every picture/.style={line width=0.75pt}} %set default line width to 0.75pt        

	\begin{tikzpicture}[x=0.75pt,y=0.75pt,yscale=-1,xscale=1]
	%uncomment if require: \path (0,300); %set diagram left start at 0, and has height of 300

	%Shape: Rectangle [id:dp4663593906154735] 
	\draw  [draw opacity=0][fill={rgb, 255:red, 212; green, 212; blue, 212 }  ,fill opacity=1 ] (230,110) -- (250,110) -- (250,210) -- (230,210) -- cycle ;
	%Shape: Rectangle [id:dp5997570980829663] 
	\draw  [draw opacity=0][fill={rgb, 255:red, 212; green, 212; blue, 212 }  ,fill opacity=1 ] (130,200) -- (350,200) -- (350,230) -- (130,230) -- cycle ;
	%Straight Lines [id:da38362604979167303] 
	\draw    (130,190) -- (130,230) ;
	%Straight Lines [id:da06020640806887423] 
	\draw    (230,100) -- (230,210) ;
	%Straight Lines [id:da6687943667253919] 
	\draw    (130,230) -- (350,230) ;
	%Straight Lines [id:da4050142485373689] 
	\draw    (350,190) -- (350,230) ;
	%Straight Lines [id:da049442261547041566] 
	\draw    (250,100) -- (250,210) ;
	%Straight Lines [id:da6671297622804289] 
	\draw  [dash pattern={on 0.84pt off 2.51pt}]  (210,110) -- (270,110) ;
	%Straight Lines [id:da8928874617764384] 
	\draw    (270,113) -- (270,197) ;
	\draw [shift={(270,200)}, rotate = 270] [fill={rgb, 255:red, 0; green, 0; blue, 0 }  ][line width=0.08]  [draw opacity=0] (10.72,-5.15) -- (0,0) -- (10.72,5.15) -- (7.12,0) -- cycle    ;
	\draw [shift={(270,110)}, rotate = 90] [fill={rgb, 255:red, 0; green, 0; blue, 0 }  ][line width=0.08]  [draw opacity=0] (10.72,-5.15) -- (0,0) -- (10.72,5.15) -- (7.12,0) -- cycle    ;
	%Straight Lines [id:da20014899847149237] 
	\draw    (230,100) -- (250,100) ;
	%Straight Lines [id:da5254081865605136] 
	\draw  [dash pattern={on 0.84pt off 2.51pt}]  (210,200) -- (270,200) ;

	% Text Node
	\draw (329,150) node    {$\Delta H=760mm$};
	% Text Node
	\draw (198,186) node    {$p_{0}$};
	% Text Node
	\draw (149,213) node    {$Hg$};


	\end{tikzpicture}
\end{figure}
\FloatBarrier
La pressione atmosferica è allora pari al peso di una colonnina di Mercurio alta 760mm.
\[
	\rho g\Delta H + 0 \,atm = p_0 \implies p_0 = \rho g\Delta H
\]
A partire da questo risultato, si utilizzano i $mmHg$ come unità di misura per la pressione.

All'interno del questo concetto espresso dalla legge di Stevino è contenuto il \textbf{principio di Pascal}. Esso afferma che se si applica una variazione di pressione sul fluido, essa si propaga inalterata in tutti i suoi punti.

\paragraph{La pressa idraulica} Si può utilizzare tale principio per realizzare la cosiddetta pressa idraulica. Essa è un oggetto che si utilizza per moltiplicare l'intensità di una forza, al fine ad esempio di sollevare o caricare carichi molto pesanti.
\begin{figure}[htpb]
	\centering
	

	\tikzset{every picture/.style={line width=0.75pt}} %set default line width to 0.75pt        

	\begin{tikzpicture}[x=0.75pt,y=0.75pt,yscale=-1,xscale=1]
	%uncomment if require: \path (0,300); %set diagram left start at 0, and has height of 300

	%Shape: Rectangle [id:dp3367893604283956] 
	\draw  [draw opacity=0][fill={rgb, 255:red, 212; green, 212; blue, 212 }  ,fill opacity=1 ] (160,130) -- (300,130) -- (300,240) -- (160,240) -- cycle ;
	%Shape: Rectangle [id:dp4131208207535777] 
	\draw  [draw opacity=0][fill={rgb, 255:red, 212; green, 212; blue, 212 }  ,fill opacity=1 ] (390,130) -- (420,130) -- (420,240) -- (390,240) -- cycle ;
	%Shape: Rectangle [id:dp1340729136265908] 
	\draw  [draw opacity=0][fill={rgb, 255:red, 212; green, 212; blue, 212 }  ,fill opacity=1 ] (300,220) -- (390,220) -- (390,240) -- (300,240) -- cycle ;
	%Straight Lines [id:da7123106707750482] 
	\draw    (160,100) -- (160,240) ;
	%Straight Lines [id:da6483304733188897] 
	\draw    (300,100) -- (300,220) ;
	%Straight Lines [id:da5551274557326262] 
	\draw    (160,240) -- (420,240) ;
	%Straight Lines [id:da4554982630064912] 
	\draw    (190,163) -- (190,227) ;
	\draw [shift={(190,230)}, rotate = 270] [fill={rgb, 255:red, 0; green, 0; blue, 0 }  ][line width=0.08]  [draw opacity=0] (10.72,-5.15) -- (0,0) -- (10.72,5.15) -- (7.12,0) -- cycle    ;
	\draw [shift={(190,160)}, rotate = 90] [fill={rgb, 255:red, 0; green, 0; blue, 0 }  ][line width=0.08]  [draw opacity=0] (10.72,-5.15) -- (0,0) -- (10.72,5.15) -- (7.12,0) -- cycle    ;
	%Straight Lines [id:da797810460779933] 
	\draw    (420,100) -- (420,240) ;
	%Straight Lines [id:da24408752045223792] 
	\draw    (390,100) -- (390,220) ;
	%Straight Lines [id:da9437332171240931] 
	\draw    (390,220) -- (300,220) ;
	%Straight Lines [id:da8620671123863763] 
	\draw    (160,240) -- (420,240) ;
	%Straight Lines [id:da5713717636782858] 
	\draw  [dash pattern={on 0.84pt off 2.51pt}]  (160,160) -- (420,160) ;
	%Straight Lines [id:da7323275755860827] 
	\draw  [dash pattern={on 0.84pt off 2.51pt}]  (160,230) -- (420,230) ;
	%Straight Lines [id:da518096766255123] 
	\draw [line width=2.25]    (160,130) -- (300,130) ;
	%Straight Lines [id:da7470580939680109] 
	\draw [line width=2.25]    (390,130) -- (420,130) ;
	%Straight Lines [id:da11589481225675469] 
	\draw    (230,73) -- (230,130) ;
	\draw [shift={(230,70)}, rotate = 90] [fill={rgb, 255:red, 0; green, 0; blue, 0 }  ][line width=0.08]  [draw opacity=0] (10.72,-5.15) -- (0,0) -- (10.72,5.15) -- (7.12,0) -- cycle    ;
	%Straight Lines [id:da9690223161704998] 
	\draw    (400,100) -- (400,127) ;
	\draw [shift={(400,130)}, rotate = 270] [fill={rgb, 255:red, 0; green, 0; blue, 0 }  ][line width=0.08]  [draw opacity=0] (10.72,-5.15) -- (0,0) -- (10.72,5.15) -- (7.12,0) -- cycle    ;

	% Text Node
	\draw (206.5,197) node    {$H$};
	% Text Node
	\draw (183,113) node    {$A_{2}$};
	% Text Node
	\draw (434,119) node    {$A_{1}$};
	% Text Node
	\draw (245.5,102.5) node    {$\vec{F}_{2}$};
	% Text Node
	\draw (405.5,84.5) node    {$\vec{F}_{1}$};


	\end{tikzpicture}
\end{figure}
\FloatBarrier
È un macchinario molto utilizzato nelle officine meccaniche per sollevare le automobili.  Si ha una specie di capillare a U in cui uno dei due bracci ha una sezione molto più piccola dell'altra. Si riempie il capillare di un certo fluido di densità nota. A questo punto i due bracci sono in equilibrio perché su entrambi agisce la pressione atmosferica. Ad un certo punto si applica a sinistra una forza $\vec{F}_1$ che è equivalente ad applicare una variazione di pressione pari a $\norma{\vec{F}_1}/A_1$. Questa variazione di pressione si propaga inalterata in tutti i punti del fluido, fino al braccio destro. Si avrà a destra una forza $\vec{F}_2$ moltiplicata di un fattore proporzionale al rapporto fra le due aree. Il risultato è che il meccanico spinge sul pedale applicando una forza non troppo forte, che viene moltiplica di un fattore molto alto pari al rapporto tra l'area $A_2$ e $A_1$, riuscendo così a sollevare l'automobile.
\[
	p_0+ \frac{F_1 }{A_1 } = p_0 + \frac{F_2 }{A_2 } \qquad F_2 = F_1\frac{A_2 }{A_1 } \quad (\gg F_1 )
\]
Questo aumento della forza è andato a scapito del fatto che il meccanico ha spinto sul pedale andando ad abbassare la colonnina di fluido di un $\Delta x_1$ molto maggiore di quanto si è sollevato il pistone su cui è la macchina, per la conservazione della massa. Deve essere valido:
\[
	A_1\Delta x_1 = A_2\Delta x_2
\]
perché se spingendo il pedale il volume di fludio nel braccio a sinistra diminuisce di una certa quantità, della stessa quantità deve poi aumentare nel braccio di destra.

Il lavoro della forza $\vec{F}_1$ è esattamente uguale a quello fatto dalla $\vec{F}_2$.
\[
	\mathcal{L}_{F_1 } = F_1 \Delta x_1 \quad \mathcal{L}_{F_2 } = F_2 \Delta x_2 = \underbrace{F_1\frac{A_2}{A_1} }_{F_2} \underbrace{\frac{A_1}{A_2}\Delta x_1}_{\Delta x_2 } = F_1 \Delta x_1 = \mathcal{L}_{F_1}
\]

\paragraph{Superfici isobare e superfici equipotenziali} Si è visto che il gradiente della pressione è sempre diretto come la forza di volume che sta agendo. Se ad esempio il fluido è soggetto soltanto alla forza peso, per cui la forza di volume per unità di massa non è altro che l'accelerazione di gravità, la pressione dovrà variare spostandosi verso il fondo del recipiente. Questo perché le superfici ortogonali al gradiente della pressione sono isobare. È importante notare che l'energia potenziale associata alla forza peso è $mgz$, che aumenta con la quota. La superfici equipotenziali sono allora le superfici a $z$ costante. Si potrebbe dimostrare che quando si ha un fluido in condizioni statiche le superficie isobare sono anche equipotenziali. Non vale solo nel caso della forza peso ma anche in quello in cui la forza di volume è un'altra.

Si immagini di avere un bicchiere in condizioni statiche e di metterlo su un carrellino, che si muove di moto accelerato uniforme.
\begin{figure}[htpb]
	\centering
	

	\tikzset{every picture/.style={line width=0.75pt}} %set default line width to 0.75pt        

	\begin{tikzpicture}[x=0.75pt,y=0.75pt,yscale=-1,xscale=1]
	%uncomment if require: \path (0,300); %set diagram left start at 0, and has height of 300

	%Shape: Polygon [id:ds5968286621281482] 
	\draw  [draw opacity=0][fill={rgb, 255:red, 212; green, 212; blue, 212 }  ,fill opacity=1 ] (300,220) -- (300,240) -- (160,240) -- (160,130) -- cycle ;
	%Straight Lines [id:da8695446790694943] 
	\draw    (160,120) -- (160,240) ;
	%Straight Lines [id:da6436112391399056] 
	\draw    (300,120) -- (300,240) ;
	%Straight Lines [id:da4690519581967658] 
	\draw    (160,240) -- (300,240) ;
	%Straight Lines [id:da3136281429797212] 
	\draw    (377,220) -- (330,220) ;
	\draw [shift={(380,220)}, rotate = 180] [fill={rgb, 255:red, 0; green, 0; blue, 0 }  ][line width=0.08]  [draw opacity=0] (10.72,-5.15) -- (0,0) -- (10.72,5.15) -- (7.12,0) -- cycle    ;
	%Shape: Rectangle [id:dp7713581388669724] 
	\draw  [draw opacity=0][fill={rgb, 255:red, 74; green, 74; blue, 74 }  ,fill opacity=1 ] (120,240) -- (340,240) -- (340,250) -- (120,250) -- cycle ;
	%Shape: Circle [id:dp5064776968626046] 
	\draw  [draw opacity=0][fill={rgb, 255:red, 74; green, 74; blue, 74 }  ,fill opacity=1 ] (150,256) .. controls (150,250.48) and (154.48,246) .. (160,246) .. controls (165.52,246) and (170,250.48) .. (170,256) .. controls (170,261.52) and (165.52,266) .. (160,266) .. controls (154.48,266) and (150,261.52) .. (150,256) -- cycle ;
	%Shape: Circle [id:dp47319546248089894] 
	\draw  [draw opacity=0][fill={rgb, 255:red, 74; green, 74; blue, 74 }  ,fill opacity=1 ] (290,256) .. controls (290,250.48) and (294.48,246) .. (300,246) .. controls (305.52,246) and (310,250.48) .. (310,256) .. controls (310,261.52) and (305.52,266) .. (300,266) .. controls (294.48,266) and (290,261.52) .. (290,256) -- cycle ;
	%Shape: Rectangle [id:dp4245897025459362] 
	\draw  [draw opacity=0][fill={rgb, 255:red, 128; green, 128; blue, 128 }  ,fill opacity=1 ] (210,180) -- (230,180) -- (230,200) -- (210,200) -- cycle ;
	%Straight Lines [id:da0738617207736405] 
	\draw    (220,227) -- (220,190) ;
	\draw [shift={(220,230)}, rotate = 270] [fill={rgb, 255:red, 0; green, 0; blue, 0 }  ][line width=0.08]  [draw opacity=0] (10.72,-5.15) -- (0,0) -- (10.72,5.15) -- (7.12,0) -- cycle    ;
	%Straight Lines [id:da25227835718417047] 
	\draw    (191.8,227.6) -- (220,190) ;
	\draw [shift={(190,230)}, rotate = 306.87] [fill={rgb, 255:red, 0; green, 0; blue, 0 }  ][line width=0.08]  [draw opacity=0] (10.72,-5.15) -- (0,0) -- (10.72,5.15) -- (7.12,0) -- cycle    ;
	%Straight Lines [id:da579938384724676] 
	\draw    (193,190) -- (220,190) ;
	\draw [shift={(190,190)}, rotate = 0] [fill={rgb, 255:red, 0; green, 0; blue, 0 }  ][line width=0.08]  [draw opacity=0] (10.72,-5.15) -- (0,0) -- (10.72,5.15) -- (7.12,0) -- cycle    ;

	% Text Node
	\draw (354,203.5) node    {$\vec{a}_{0}$};
	% Text Node
	\draw (237,211) node    {$m\vec{g}$};
	% Text Node
	\draw (183.5,171.5) node    {$-m\vec{a}_{0}$};


	\end{tikzpicture}
\end{figure}
\FloatBarrier
Si può sfruttare il concetto di sistemi di riferimento non inerziali. Si immagini di osservare quello che succede al fluido dal punto di vista di un osservatore posto sopra il carrellino, e che quindi accelera con esso: per lui tale fluido è fermo. Un elemento di fluido di volume $dV$ e massa $dm$ sarà soggetto alle forze reali (la forza peso $dm\,g$) e alle forze apparenti legate al fatto che il sistema di riferimento accelera in avanti. Ogni massa del fluido sarà soggetta a una forza di volume pari alla somma vettoriale dell'accelerazione di gravità e $-a_o$. In tal caso nel bicchiere il gradiente della pressione dovrà variare esattamente in questa direzione, sarà parallelo ed equiverso alla forza di volume per unità di massa. La superficie libera dell'acqua diventa quella in figura.

\paragraph{Liquido in rotazione} Esattamente lo stesso discorso si può applicare se invece il bicchiere lo si fa ruotare rispetto ad un asse $z$ di simmetria del fluido. Nasce un vettore velocità angolare diretto verso l'alto, che si immagina costante.
Considerato un elemento infinitesimo generico del fluido, esso è soggetto all'accelerazione di gravità e ad una forza centrifuga diretta verso l'esterno pari a $\omega_0 2r$ (dove $r$ e la distanza dell'elemento di fluido dall'asse).
Mentre l'azione dell'accelerazione di gravità è la stessa su tutti i volumetti del fluido, l'accelerazione centrifuga è diversa in ogni punto. Mano a mano che ci si allontana dall'asse l'accelerazione centrifuga aumenta. Si ha un esempio in cui la forza totale di volume varia in direzione e in intensità di punto a punto. Anche il gradiente non cambia mano a mano che ci si sposta nei punti del fluido. Il pelo libero dell'acqua, in conseguenza a questa situazione, assume la forma che si osserva in figura.
\begin{figure}[htpb]
	\centering
	

	\tikzset{every picture/.style={line width=0.75pt}} %set default line width to 0.75pt        

	\begin{tikzpicture}[x=0.75pt,y=0.75pt,yscale=-1,xscale=1]
	%uncomment if require: \path (0,300); %set diagram left start at 0, and has height of 300

	%Shape: Polygon Curved [id:ds20096883102093122] 
	\draw  [draw opacity=0][fill={rgb, 255:red, 212; green, 212; blue, 212 }  ,fill opacity=1 ] (160,80) .. controls (220.33,196.67) and (280.33,196) .. (340,80) .. controls (340.33,162.67) and (340.33,200) .. (340,240) .. controls (278.33,239.33) and (206.33,240) .. (160,240) .. controls (160.33,198.67) and (160.33,184) .. (160,80) -- cycle ;
	%Straight Lines [id:da9654921326134607] 
	\draw    (160,70) -- (160,240) ;
	%Straight Lines [id:da6205540832353014] 
	\draw    (340,70) -- (340,240) ;
	%Straight Lines [id:da2940200728809077] 
	\draw    (160,240) -- (340,240) ;
	%Straight Lines [id:da876455808886512] 
	\draw    (250,70) -- (250,167) ;
	\draw [shift={(250,67)}, rotate = 90] [fill={rgb, 255:red, 0; green, 0; blue, 0 }  ][line width=0.08]  [draw opacity=0] (10.72,-5.15) -- (0,0) -- (10.72,5.15) -- (7.12,0) -- cycle    ;
	%Shape: Rectangle [id:dp6950850612097643] 
	\draw  [draw opacity=0][fill={rgb, 255:red, 128; green, 128; blue, 128 }  ,fill opacity=1 ] (190,170) -- (210,170) -- (210,190) -- (190,190) -- cycle ;
	%Straight Lines [id:da48466613038417683] 
	\draw    (200,217) -- (200,180) ;
	\draw [shift={(200,220)}, rotate = 270] [fill={rgb, 255:red, 0; green, 0; blue, 0 }  ][line width=0.08]  [draw opacity=0] (10.72,-5.15) -- (0,0) -- (10.72,5.15) -- (7.12,0) -- cycle    ;
	%Straight Lines [id:da4514717750458741] 
	\draw    (171.8,217.6) -- (200,180) ;
	\draw [shift={(170,220)}, rotate = 306.87] [fill={rgb, 255:red, 0; green, 0; blue, 0 }  ][line width=0.08]  [draw opacity=0] (10.72,-5.15) -- (0,0) -- (10.72,5.15) -- (7.12,0) -- cycle    ;
	%Straight Lines [id:da8756408615323261] 
	\draw    (173,180) -- (200,180) ;
	\draw [shift={(170,180)}, rotate = 0] [fill={rgb, 255:red, 0; green, 0; blue, 0 }  ][line width=0.08]  [draw opacity=0] (10.72,-5.15) -- (0,0) -- (10.72,5.15) -- (7.12,0) -- cycle    ;
	%Curve Lines [id:da4799020204729234] 
	\draw    (250,110) .. controls (230.67,110) and (229.33,130.67) .. (250,130) ;
	\draw [shift={(235,120.4)}, rotate = 291.72] [fill={rgb, 255:red, 0; green, 0; blue, 0 }  ][line width=0.08]  [draw opacity=0] (10.72,-5.15) -- (0,0) -- (10.72,5.15) -- (7.12,0) -- cycle    ;
	%Shape: Boxed Bezier Curve [id:dp013842827744275477] 
	\draw    (250.08,130) .. controls (269.41,129.93) and (270.66,109.25) .. (250,110) ;
	%Shape: Rectangle [id:dp22661450660004268] 
	\draw  [draw opacity=0][fill={rgb, 255:red, 128; green, 128; blue, 128 }  ,fill opacity=1 ] (240,180) -- (260,180) -- (260,200) -- (240,200) -- cycle ;
	%Straight Lines [id:da029217868553261583] 
	\draw    (250,227) -- (250,190) ;
	\draw [shift={(250,230)}, rotate = 270] [fill={rgb, 255:red, 0; green, 0; blue, 0 }  ][line width=0.08]  [draw opacity=0] (10.72,-5.15) -- (0,0) -- (10.72,5.15) -- (7.12,0) -- cycle    ;

	% Text Node
	\draw (268,89) node    {$\vec{\omega }$};
	% Text Node
	\draw (267,209) node    {$\vec{g}$};


	\end{tikzpicture}
\end{figure}
\FloatBarrier
La superficie andando verso il perimetro del cerchio comincia a inclinarsi e si inclina sempre di più. Si ha una parabola ruotata attorno all'asse di simmetria, che prende il nome di \emph{paraboloide di rotazione}.







































\section{Il principio di Archimede}

L'ultimo concetto legato alla statica dei fluidi, è lo studio di un corpo immerso in un fluido in equilibrio statico. Si prenda un corpo di forma qualunque e lo si immerga dentro a un fluido, di densità uniforme $\rho$. Si vuole sapere se affonderà, se tenderà a galleggiare o se rimarrà in equilibrio in quel punto.

Il corpo nel fluido generalmente non rimane in equilibrio, non bisogna imporre su di esso la condizione di staticità. Si applica al centro di massa la forza peso totale. Il corpo è circondato da un fluido che genererà una pressione e quindi delle forze di superficie su ogni porzione di superficie del corpo. A esso si applica la prima equazione cardinale della dinamica: la somma vettoriale della forza peso più le forze di superficie, danno luogo alla traslazione del centro di massa del corpo.
\[
	\text{corpo:} \quad M\vec{g} + \vec{F}_s = m\vec{a}_{cm}
\]
Per risolvere questo problema bisogna capire come sono dirette e quanto sono intense le forze di superficie. Per questo si prende il fluido, si elimina il corpo e si va a considerare una porzione del fluido esattamente identifica a quella che era occupata dal solido. Tale porzione è in condizioni statiche, perché è una parte isolata di un fluido in condizioni statiche. Essa dovrà essere soggetta a una risultante delle forze che è uguale a zero, queste sono esattamente le stesse di prima: la forza peso e le forze di superficie che agiscono sulla superficie di questa porzione di fluido. Queste ultime sono le stesse che agiscono sul corpo, il fluido intorno si comporta sempre alla stessa maniera, che ci sia immerso il corpo o meno.
\begin{figure}[htpb]
	\centering
	

	\tikzset{every picture/.style={line width=0.75pt}} %set default line width to 0.75pt        

	\begin{tikzpicture}[x=0.75pt,y=0.75pt,yscale=-1,xscale=1]
	%uncomment if require: \path (0,300); %set diagram left start at 0, and has height of 300

	%Shape: Rectangle [id:dp2924271807521268] 
	\draw  [draw opacity=0][fill={rgb, 255:red, 212; green, 212; blue, 212 }  ,fill opacity=1 ] (37,139) -- (217,139) -- (217,219) -- (37,219) -- cycle ;
	%Shape: Rectangle [id:dp268230096545264] 
	\draw  [draw opacity=0][fill={rgb, 255:red, 212; green, 212; blue, 212 }  ,fill opacity=1 ] (267,139) -- (447,139) -- (447,219) -- (267,219) -- cycle ;
	%Straight Lines [id:da9333188435120883] 
	\draw    (37,119) -- (37,219) ;
	%Straight Lines [id:da25664817053366384] 
	\draw    (217,119) -- (217,219) ;
	%Straight Lines [id:da8751662146914414] 
	\draw    (37,219) -- (217,219) ;
	%Straight Lines [id:da4092839702494766] 
	\draw    (127,156) -- (127,119) ;
	\draw [shift={(127,159)}, rotate = 270] [fill={rgb, 255:red, 0; green, 0; blue, 0 }  ][line width=0.08]  [draw opacity=0] (10.72,-5.15) -- (0,0) -- (10.72,5.15) -- (7.12,0) -- cycle    ;
	%Straight Lines [id:da49174964102349517] 
	\draw    (171.12,167.23) -- (194.33,158) ;
	\draw [shift={(168.33,168.33)}, rotate = 338.33] [fill={rgb, 255:red, 0; green, 0; blue, 0 }  ][line width=0.08]  [draw opacity=0] (10.72,-5.15) -- (0,0) -- (10.72,5.15) -- (7.12,0) -- cycle    ;
	%Shape: Regular Polygon [id:dp8459420545468057] 
	\draw  [draw opacity=0][fill={rgb, 255:red, 128; green, 128; blue, 128 }  ,fill opacity=1 ] (127.59,203.33) .. controls (104.98,194.13) and (82.85,145.16) .. (106.8,160.21) .. controls (130.75,175.25) and (139.36,150.04) .. (158.31,164.17) .. controls (177.27,178.31) and (150.2,212.52) .. (127.59,203.33) -- cycle ;
	%Straight Lines [id:da10166248651917864] 
	\draw    (165.3,196.93) -- (181.67,210.67) ;
	\draw [shift={(163,195)}, rotate = 40.01] [fill={rgb, 255:red, 0; green, 0; blue, 0 }  ][line width=0.08]  [draw opacity=0] (10.72,-5.15) -- (0,0) -- (10.72,5.15) -- (7.12,0) -- cycle    ;
	%Straight Lines [id:da41918106832275703] 
	\draw    (93.46,177.86) -- (73,184) ;
	\draw [shift={(96.33,177)}, rotate = 163.3] [fill={rgb, 255:red, 0; green, 0; blue, 0 }  ][line width=0.08]  [draw opacity=0] (10.72,-5.15) -- (0,0) -- (10.72,5.15) -- (7.12,0) -- cycle    ;
	%Straight Lines [id:da7859413090803753] 
	\draw    (267,119) -- (267,219) ;
	%Straight Lines [id:da8364425132614817] 
	\draw    (447,119) -- (447,219) ;
	%Straight Lines [id:da28095595212447666] 
	\draw    (267,219) -- (447,219) ;
	%Straight Lines [id:da6624570155311353] 
	\draw    (357,156) -- (357,119) ;
	\draw [shift={(357,159)}, rotate = 270] [fill={rgb, 255:red, 0; green, 0; blue, 0 }  ][line width=0.08]  [draw opacity=0] (10.72,-5.15) -- (0,0) -- (10.72,5.15) -- (7.12,0) -- cycle    ;
	%Straight Lines [id:da6892077093994595] 
	\draw    (401.12,167.23) -- (424.33,158) ;
	\draw [shift={(398.33,168.33)}, rotate = 338.33] [fill={rgb, 255:red, 0; green, 0; blue, 0 }  ][line width=0.08]  [draw opacity=0] (10.72,-5.15) -- (0,0) -- (10.72,5.15) -- (7.12,0) -- cycle    ;
	%Shape: Regular Polygon [id:dp3956362618417233] 
	\draw  [color={rgb, 255:red, 0; green, 0; blue, 0 }  ,draw opacity=1 ][dash pattern={on 0.84pt off 2.51pt}] (357.59,203.33) .. controls (334.98,194.13) and (312.85,145.16) .. (336.8,160.21) .. controls (360.75,175.25) and (369.36,150.04) .. (388.31,164.17) .. controls (407.27,178.31) and (380.2,212.52) .. (357.59,203.33) -- cycle ;
	%Straight Lines [id:da3421447885651081] 
	\draw    (395.3,196.93) -- (411.67,210.67) ;
	\draw [shift={(393,195)}, rotate = 40.01] [fill={rgb, 255:red, 0; green, 0; blue, 0 }  ][line width=0.08]  [draw opacity=0] (10.72,-5.15) -- (0,0) -- (10.72,5.15) -- (7.12,0) -- cycle    ;
	%Straight Lines [id:da6201930847496877] 
	\draw    (323.46,177.86) -- (303,184) ;
	\draw [shift={(326.33,177)}, rotate = 163.3] [fill={rgb, 255:red, 0; green, 0; blue, 0 }  ][line width=0.08]  [draw opacity=0] (10.72,-5.15) -- (0,0) -- (10.72,5.15) -- (7.12,0) -- cycle    ;

	% Text Node
	\draw (146,113.5) node    {$\vec{F}_{s}$};
	% Text Node
	\draw (65,200.17) node    {$\rho _{\text{fluido}}$};
	% Text Node
	\draw (376,113.5) node    {$\vec{F}_{s}$};
	% Text Node
	\draw (295,200.17) node    {$\rho _{\text{fluido}}$};


	\end{tikzpicture}
\end{figure}
\FloatBarrier
Quindi:
\[
	\text{fluido:} \quad M_{\text{fluido} }\vec{g} +\vec{F}_s = 0 \implies \vec{F}_s = -M_{\text{fluido} }\vec{g} = -\rho_{\text{fluido} }V\vec{g}
\]
Si indica un versore $\vec{u}_z$ diretto verso l'alto.
\[
	\vec{F}_s = \rho_{\text{fluido} }Vg\vec{u}_z = \vec{F}_{\text{Archimede} }
\]
Questa quantità si chiama \textbf{spinta di Archimede}. Il \textbf{principio di Archimede} afferma che quando si ha un corpo immerso in un fluido, questo riceve una spinta verso l'alto di intensità pari al peso del volume di fluido occupata dal corpo. Il volume sarà solo la porzione immersa nel fluido perché si trascura la spinta di Archimede dell' atmosfera. Invece di disegnare tutte le forze di superficie, se ne disegna la risultante: è una forza diretta come $\vec{g}$ ma verso l'alto. Se la spinta di Archimede è minore del peso il corpo affonda. Se le due cose sono uguali il corpo fluttua. Se essa è invece maggiore del peso, il corpo galleggia.
Si dimostra che la spinta di Archimede, essendo in generale i corpi esterni estesi, si applica nel centro di massa della parte immersa. Solo se tutto il corpo è immerso, esso coincide con il baricentro del corpo.
\begin{gather*}
	\text{corpo tutto immerso:} \quad M\vec{g} +\vec{F}_s = M\vec{a}_{cm} \\
	-\rho Vg\vec{u}_z + \rho_{\text{fluido} }Vg\vec{u}_z = (\rho_{\text{fluido} }- \rho)Vg\vec{u}_z
\end{gather*}
Si ottiene che il corpo si muoverà verso l'alto.
\[
	\rho_{\text{fluido} }-\rho \left\{ \begin{array}{l}
	 	\rho > \rho_{\text{fluido}} \implies \text{il corpo affonda} \\
	 	\rho = \rho_{\text{fluido}} \implies \text{il corpo fluttua} \\
	 	\rho < \rho_{\text{fluido}} \implies \text{il corpo galleggia} \\
	\end{array} \right.
\]
Il fatto che il corpo galleggi significa che esso, totalmente immerso, risale verso l'alto fino a quando la spinta di Archimede non arriva a bilanciare la forza peso. La spinta di Archimede diminuisce perché quando il corpo incomincia a fuoriuscire dall'acqua diminuisce il volume immerso.
\[
	-\rho Vg\vec{u}_z + \rho_{\text{fluido} }\widetilde{V}g\vec{u}_z=0
\]
Non solo le forze si devono bilanciare fra di loro, ma si devono bilanciare anche i momenti delle forze. Questa è una condizione necessaria per il galleggiamento delle barche. Quando soltanto una parte del corpo galleggia, il baricentro del corpo e quello della parte immersa non coincidono, il loro momento è zero finché si trovano sulla stessa retta parallela a $\vec{g}$.
Quando la barca si inclina, ad esempio a causa delle onde, gli scafi delle barche devono essere progettati in maniera tale che essa reagisca riportandosi in posizione di equilibrio (si veda la figura).
\begin{figure}[htpb]
	\centering
	

	% Pattern Info
	 
	\tikzset{
	pattern size/.store in=\mcSize, 
	pattern size = 5pt,
	pattern thickness/.store in=\mcThickness, 
	pattern thickness = 0.3pt,
	pattern radius/.store in=\mcRadius, 
	pattern radius = 1pt}
	\makeatletter
	\pgfutil@ifundefined{pgf@pattern@name@_31vtuqmc4}{
	\pgfdeclarepatternformonly[\mcThickness,\mcSize]{_31vtuqmc4}
	{\pgfqpoint{0pt}{0pt}}
	{\pgfpoint{\mcSize+\mcThickness}{\mcSize+\mcThickness}}
	{\pgfpoint{\mcSize}{\mcSize}}
	{
	\pgfsetcolor{\tikz@pattern@color}
	\pgfsetlinewidth{\mcThickness}
	\pgfpathmoveto{\pgfqpoint{0pt}{0pt}}
	\pgfpathlineto{\pgfpoint{\mcSize+\mcThickness}{\mcSize+\mcThickness}}
	\pgfusepath{stroke}
	}}
	\makeatother

	% Pattern Info
	 
	\tikzset{
	pattern size/.store in=\mcSize, 
	pattern size = 5pt,
	pattern thickness/.store in=\mcThickness, 
	pattern thickness = 0.3pt,
	pattern radius/.store in=\mcRadius, 
	pattern radius = 1pt}
	\makeatletter
	\pgfutil@ifundefined{pgf@pattern@name@_q84sces0b}{
	\pgfdeclarepatternformonly[\mcThickness,\mcSize]{_q84sces0b}
	{\pgfqpoint{0pt}{0pt}}
	{\pgfpoint{\mcSize+\mcThickness}{\mcSize+\mcThickness}}
	{\pgfpoint{\mcSize}{\mcSize}}
	{
	\pgfsetcolor{\tikz@pattern@color}
	\pgfsetlinewidth{\mcThickness}
	\pgfpathmoveto{\pgfqpoint{0pt}{0pt}}
	\pgfpathlineto{\pgfpoint{\mcSize+\mcThickness}{\mcSize+\mcThickness}}
	\pgfusepath{stroke}
	}}
	\makeatother
	\tikzset{every picture/.style={line width=0.75pt}} %set default line width to 0.75pt        

	\begin{tikzpicture}[x=0.75pt,y=0.75pt,yscale=-1,xscale=1]
	%uncomment if require: \path (0,300); %set diagram left start at 0, and has height of 300

	%Shape: Rectangle [id:dp47829763359688604] 
	\draw  [draw opacity=0][fill={rgb, 255:red, 212; green, 212; blue, 212 }  ,fill opacity=1 ] (70.5,141) -- (269.5,141) -- (269.5,201) -- (70.5,201) -- cycle ;
	%Flowchart: Delay [id:dp29197696959440433] 
	\draw  [color={rgb, 255:red, 128; green, 128; blue, 128 }  ,draw opacity=1 ][fill={rgb, 255:red, 128; green, 128; blue, 128 }  ,fill opacity=1 ] (215,123.67) -- (215,149) .. controls (215,162.99) and (194.78,174.33) .. (169.83,174.33) .. controls (144.89,174.33) and (124.67,162.99) .. (124.67,149) -- (124.67,123.67) -- cycle ;
	%Shape: Circle [id:dp8821713706044376] 
	\draw  [fill={rgb, 255:red, 0; green, 0; blue, 0 }  ,fill opacity=1 ] (167.33,153.5) .. controls (167.33,152.12) and (168.45,151) .. (169.83,151) .. controls (171.21,151) and (172.33,152.12) .. (172.33,153.5) .. controls (172.33,154.88) and (171.21,156) .. (169.83,156) .. controls (168.45,156) and (167.33,154.88) .. (167.33,153.5) -- cycle ;
	%Curve Lines [id:da7112974575143318] 
	\draw    (175.62,93.91) .. controls (201.37,86.54) and (231.23,100.38) .. (237.82,125.8) ;
	\draw [shift={(238.44,128.61)}, rotate = 259.51] [fill={rgb, 255:red, 0; green, 0; blue, 0 }  ][line width=0.08]  [draw opacity=0] (10.72,-5.15) -- (0,0) -- (10.72,5.15) -- (7.12,0) -- cycle    ;
	%Straight Lines [id:da2837377357891766] 
	\draw    (70.5,131) -- (70.5,201) ;
	%Straight Lines [id:da2684328230332256] 
	\draw    (269.5,131) -- (269.5,201) ;
	%Straight Lines [id:da7800389284596869] 
	\draw    (269.5,201) -- (70.5,201) ;
	%Shape: Path Data [id:dp45697782539174714] 
	\draw  [color={rgb, 255:red, 0; green, 0; blue, 0 }  ,draw opacity=1 ][pattern=_31vtuqmc4,pattern size=6pt,pattern thickness=0.75pt,pattern radius=0pt, pattern color={rgb, 255:red, 0; green, 0; blue, 0}] (215,149) .. controls (215,162.99) and (194.78,174.33) .. (169.83,174.33) .. controls (144.89,174.33) and (124.67,162.99) .. (124.67,149) -- (124.67,141) -- (215,141) -- (215,149) -- cycle ;
	%Shape: Circle [id:dp30364608297317264] 
	\draw  [fill={rgb, 255:red, 0; green, 0; blue, 0 }  ,fill opacity=1 ] (167.33,133.5) .. controls (167.33,132.12) and (168.45,131) .. (169.83,131) .. controls (171.21,131) and (172.33,132.12) .. (172.33,133.5) .. controls (172.33,134.88) and (171.21,136) .. (169.83,136) .. controls (168.45,136) and (167.33,134.88) .. (167.33,133.5) -- cycle ;
	%Shape: Rectangle [id:dp2713282993239199] 
	\draw  [draw opacity=0][fill={rgb, 255:red, 212; green, 212; blue, 212 }  ,fill opacity=1 ] (319.5,141) -- (518.5,141) -- (518.5,201) -- (319.5,201) -- cycle ;
	%Flowchart: Delay [id:dp4272201970702536] 
	\draw  [color={rgb, 255:red, 128; green, 128; blue, 128 }  ,draw opacity=1 ][fill={rgb, 255:red, 128; green, 128; blue, 128 }  ,fill opacity=1 ] (470.58,148.92) -- (459.08,171.49) .. controls (452.74,183.96) and (429.57,184.89) .. (407.34,173.58) .. controls (385.11,162.26) and (372.23,142.98) .. (378.58,130.51) -- (390.08,107.93) -- cycle ;
	%Curve Lines [id:da141599797063642] 
	\draw    (434.84,89.44) .. controls (460.75,83.83) and (489.61,98.83) .. (494.44,124.94) ;
	\draw [shift={(431.62,90.25)}, rotate = 344.02] [fill={rgb, 255:red, 0; green, 0; blue, 0 }  ][line width=0.08]  [draw opacity=0] (10.72,-5.15) -- (0,0) -- (10.72,5.15) -- (7.12,0) -- cycle    ;
	%Straight Lines [id:da26918340914974825] 
	\draw    (319.5,131) -- (319.5,201) ;
	%Straight Lines [id:da6034750913598959] 
	\draw    (518.5,131) -- (518.5,201) ;
	%Straight Lines [id:da3957935253978764] 
	\draw    (518.5,201) -- (319.5,201) ;
	%Shape: Circle [id:dp23398629144511873] 
	\draw  [fill={rgb, 255:red, 0; green, 0; blue, 0 }  ,fill opacity=1 ] (417.9,152.7) .. controls (418.63,151.69) and (420.05,151.46) .. (421.06,152.19) .. controls (422.08,152.93) and (422.31,154.34) .. (421.57,155.36) .. controls (420.84,156.37) and (419.42,156.6) .. (418.41,155.87) .. controls (417.39,155.14) and (417.16,153.72) .. (417.9,152.7) -- cycle ;
	%Shape: Circle [id:dp8816370101911546] 
	\draw  [fill={rgb, 255:red, 0; green, 0; blue, 0 }  ,fill opacity=1 ] (428.61,161.24) .. controls (429.32,160.24) and (430.71,160.02) .. (431.7,160.74) .. controls (432.69,161.45) and (432.91,162.84) .. (432.2,163.83) .. controls (431.48,164.82) and (430.1,165.04) .. (429.11,164.33) .. controls (428.11,163.61) and (427.89,162.23) .. (428.61,161.24) -- cycle ;
	%Straight Lines [id:da38157126120264007] 
	\draw    (419.74,191.03) -- (419.74,154.03) ;
	\draw [shift={(419.74,194.03)}, rotate = 270] [fill={rgb, 255:red, 0; green, 0; blue, 0 }  ][line width=0.08]  [draw opacity=0] (10.72,-5.15) -- (0,0) -- (10.72,5.15) -- (7.12,0) -- cycle    ;
	%Straight Lines [id:da36464594177504006] 
	\draw    (430.4,162.53) -- (430.4,125.53) ;
	\draw [shift={(430.4,122.53)}, rotate = 450] [fill={rgb, 255:red, 0; green, 0; blue, 0 }  ][line width=0.08]  [draw opacity=0] (10.72,-5.15) -- (0,0) -- (10.72,5.15) -- (7.12,0) -- cycle    ;
	%Shape: Path Data [id:dp5684196922078801] 
	\draw  [color={rgb, 255:red, 0; green, 0; blue, 0 }  ,draw opacity=1 ][pattern=_q84sces0b,pattern size=6pt,pattern thickness=0.75pt,pattern radius=0pt, pattern color={rgb, 255:red, 0; green, 0; blue, 0}] (459.08,171.49) .. controls (452.74,183.96) and (429.57,184.89) .. (407.34,173.58) .. controls (390.71,165.11) and (379.32,152.19) .. (377.29,141) -- (455.03,141) -- (470.58,148.92) ;

	% Text Node
	\draw (396.67,184.83) node    {$M\vec{g}$};
	% Text Node
	\draw (446.67,112.83) node    {$\vec{F}_{A}$};


	\end{tikzpicture}
\end{figure}
\FloatBarrier
Questo si verifica se la spinta di Archimede tende a far ruotare la barca in senso opposto a quello della perturbazione compensando lo sbilanciamento.
\begin{figure}[htpb]
	\centering
	

	% Pattern Info
	 
	\tikzset{
	pattern size/.store in=\mcSize, 
	pattern size = 5pt,
	pattern thickness/.store in=\mcThickness, 
	pattern thickness = 0.3pt,
	pattern radius/.store in=\mcRadius, 
	pattern radius = 1pt}
	\makeatletter
	\pgfutil@ifundefined{pgf@pattern@name@_v6p0c9ovg}{
	\pgfdeclarepatternformonly[\mcThickness,\mcSize]{_v6p0c9ovg}
	{\pgfqpoint{0pt}{0pt}}
	{\pgfpoint{\mcSize+\mcThickness}{\mcSize+\mcThickness}}
	{\pgfpoint{\mcSize}{\mcSize}}
	{
	\pgfsetcolor{\tikz@pattern@color}
	\pgfsetlinewidth{\mcThickness}
	\pgfpathmoveto{\pgfqpoint{0pt}{0pt}}
	\pgfpathlineto{\pgfpoint{\mcSize+\mcThickness}{\mcSize+\mcThickness}}
	\pgfusepath{stroke}
	}}
	\makeatother

	% Pattern Info
	 
	\tikzset{
	pattern size/.store in=\mcSize, 
	pattern size = 5pt,
	pattern thickness/.store in=\mcThickness, 
	pattern thickness = 0.3pt,
	pattern radius/.store in=\mcRadius, 
	pattern radius = 1pt}
	\makeatletter
	\pgfutil@ifundefined{pgf@pattern@name@_xvr8v4s9a}{
	\pgfdeclarepatternformonly[\mcThickness,\mcSize]{_xvr8v4s9a}
	{\pgfqpoint{0pt}{0pt}}
	{\pgfpoint{\mcSize+\mcThickness}{\mcSize+\mcThickness}}
	{\pgfpoint{\mcSize}{\mcSize}}
	{
	\pgfsetcolor{\tikz@pattern@color}
	\pgfsetlinewidth{\mcThickness}
	\pgfpathmoveto{\pgfqpoint{0pt}{0pt}}
	\pgfpathlineto{\pgfpoint{\mcSize+\mcThickness}{\mcSize+\mcThickness}}
	\pgfusepath{stroke}
	}}
	\makeatother
	\tikzset{every picture/.style={line width=0.75pt}} %set default line width to 0.75pt        

	\begin{tikzpicture}[x=0.75pt,y=0.75pt,yscale=-1,xscale=1]
	%uncomment if require: \path (0,300); %set diagram left start at 0, and has height of 300

	%Shape: Rectangle [id:dp9862095167922351] 
	\draw  [draw opacity=0][fill={rgb, 255:red, 212; green, 212; blue, 212 }  ,fill opacity=1 ] (296.5,160) -- (463.5,160) -- (463.5,220) -- (296.5,220) -- cycle ;
	%Straight Lines [id:da447179809049856] 
	\draw    (296.5,150) -- (296.5,220) ;
	%Straight Lines [id:da3781711975130897] 
	\draw    (463.5,150) -- (463.5,220) ;
	%Straight Lines [id:da9710787433089609] 
	\draw    (463.5,220) -- (296.5,220) ;
	%Flowchart: Delay [id:dp34858695131676476] 
	\draw  [color={rgb, 255:red, 128; green, 128; blue, 128 }  ,draw opacity=1 ][fill={rgb, 255:red, 128; green, 128; blue, 128 }  ,fill opacity=1 ] (417.15,148.4) -- (396.66,176.78) .. controls (385.35,192.45) and (370.73,201.23) .. (364.01,196.38) .. controls (357.3,191.53) and (361.02,174.89) .. (372.34,159.22) -- (392.83,130.84) -- cycle ;
	%Shape: Circle [id:dp47969536261711365] 
	\draw  [fill={rgb, 255:red, 0; green, 0; blue, 0 }  ,fill opacity=1 ] (384.56,164.04) .. controls (385.3,163.02) and (386.71,162.79) .. (387.73,163.53) .. controls (388.75,164.26) and (388.97,165.68) .. (388.24,166.69) .. controls (387.51,167.71) and (386.09,167.94) .. (385.08,167.2) .. controls (384.06,166.47) and (383.83,165.05) .. (384.56,164.04) -- cycle ;
	%Shape: Circle [id:dp9500480670392788] 
	\draw  [fill={rgb, 255:red, 0; green, 0; blue, 0 }  ,fill opacity=1 ] (375.61,176.57) .. controls (376.32,175.58) and (377.71,175.35) .. (378.7,176.07) .. controls (379.69,176.79) and (379.91,178.17) .. (379.2,179.16) .. controls (378.48,180.15) and (377.1,180.38) .. (376.11,179.66) .. controls (375.11,178.94) and (374.89,177.56) .. (375.61,176.57) -- cycle ;
	%Straight Lines [id:da4084746695198622] 
	\draw    (386.4,202.37) -- (386.4,165.37) ;
	\draw [shift={(386.4,205.37)}, rotate = 270] [fill={rgb, 255:red, 0; green, 0; blue, 0 }  ][line width=0.08]  [draw opacity=0] (10.72,-5.15) -- (0,0) -- (10.72,5.15) -- (7.12,0) -- cycle    ;
	%Straight Lines [id:da608832121120398] 
	\draw    (377.4,177.87) -- (377.4,140.87) ;
	\draw [shift={(377.4,137.87)}, rotate = 450] [fill={rgb, 255:red, 0; green, 0; blue, 0 }  ][line width=0.08]  [draw opacity=0] (10.72,-5.15) -- (0,0) -- (10.72,5.15) -- (7.12,0) -- cycle    ;
	%Curve Lines [id:da11003730679167623] 
	\draw    (370.29,97.91) .. controls (396.04,90.54) and (425.9,104.38) .. (432.48,129.8) ;
	\draw [shift={(433.11,132.61)}, rotate = 259.51] [fill={rgb, 255:red, 0; green, 0; blue, 0 }  ][line width=0.08]  [draw opacity=0] (10.72,-5.15) -- (0,0) -- (10.72,5.15) -- (7.12,0) -- cycle    ;
	%Shape: Path Data [id:dp8117456064987676] 
	\draw  [color={rgb, 255:red, 0; green, 0; blue, 0 }  ,draw opacity=1 ][pattern=_v6p0c9ovg,pattern size=6pt,pattern thickness=0.75pt,pattern radius=0pt, pattern color={rgb, 255:red, 0; green, 0; blue, 0}] (396.66,176.78) .. controls (385.35,192.45) and (370.73,201.23) .. (364.01,196.38) .. controls (357.41,191.61) and (360.9,175.44) .. (371.78,160) -- (408.78,160) -- (396.66,176.78) -- cycle ;
	%Shape: Rectangle [id:dp7984287448127838] 
	\draw  [draw opacity=0][fill={rgb, 255:red, 212; green, 212; blue, 212 }  ,fill opacity=1 ] (79.5,160) -- (246.5,160) -- (246.5,220) -- (79.5,220) -- cycle ;
	%Flowchart: Delay [id:dp9187309202524181] 
	\draw  [color={rgb, 255:red, 128; green, 128; blue, 128 }  ,draw opacity=1 ][fill={rgb, 255:red, 128; green, 128; blue, 128 }  ,fill opacity=1 ] (176,130) -- (176,165) .. controls (176,184.33) and (169.28,200) .. (161,200) .. controls (152.72,200) and (146,184.33) .. (146,165) -- (146,130) -- cycle ;
	%Shape: Circle [id:dp9403857998134741] 
	\draw  [fill={rgb, 255:red, 0; green, 0; blue, 0 }  ,fill opacity=1 ] (158.5,161.75) .. controls (158.5,160.37) and (159.62,159.25) .. (161,159.25) .. controls (162.38,159.25) and (163.5,160.37) .. (163.5,161.75) .. controls (163.5,163.13) and (162.38,164.25) .. (161,164.25) .. controls (159.62,164.25) and (158.5,163.13) .. (158.5,161.75) -- cycle ;
	%Curve Lines [id:da5917708601819769] 
	\draw    (132.29,97.91) .. controls (158.04,90.54) and (187.9,104.38) .. (194.48,129.8) ;
	\draw [shift={(195.11,132.61)}, rotate = 259.51] [fill={rgb, 255:red, 0; green, 0; blue, 0 }  ][line width=0.08]  [draw opacity=0] (10.72,-5.15) -- (0,0) -- (10.72,5.15) -- (7.12,0) -- cycle    ;
	%Straight Lines [id:da4782038123895451] 
	\draw    (79.5,150) -- (79.5,220) ;
	%Straight Lines [id:da039222109655102866] 
	\draw    (246.5,150) -- (246.5,220) ;
	%Straight Lines [id:da352891107189514] 
	\draw    (246.5,220) -- (79.5,220) ;
	%Shape: Path Data [id:dp5449531698933052] 
	\draw  [color={rgb, 255:red, 0; green, 0; blue, 0 }  ,draw opacity=1 ][pattern=_xvr8v4s9a,pattern size=6pt,pattern thickness=0.75pt,pattern radius=0pt, pattern color={rgb, 255:red, 0; green, 0; blue, 0}] (176,165) .. controls (176,184.33) and (169.28,200) .. (161,200) .. controls (152.72,200) and (146,184.33) .. (146,165) -- (146,160) -- (176,160) -- (176,165) -- cycle ;

	% Text Node
	\draw (360.67,128.83) node    {$\vec{F}_{A}$};
	% Text Node
	\draw (409.67,197.83) node    {$M\vec{g}$};


	\end{tikzpicture}
\end{figure}
\FloatBarrier
Se si ha uno scafo molto più lungo e molto più stretto, la parte immersa è un po' più spostata verso sinistra. Uno scafo di questo tipo tende a ribaltare ulteriormente la barca.























































































































\chapter{Termodinamica}

\section{Introduzione generale}

\paragraph{Esempio} Secondo il teorema dell'energia meccanica essa varia se agiscono forze non conservative che compiono lavoro. Ci sono casi in cui l'energia meccanica si dissipa, per esempio a causa dell'effetto della forza d'attrito. Si consideri la situazione in figura, in cui un corpo è fermo in cima ad un piano liscio inclinato che termina con un piano orizzontale scabro. Quando esso viene lasciato libero di scivolare, la sua energia potenziale si trasforma in energia cinetica che lo fa scendere verso il basso. Poi il corpo striscia sul piano con attrito fino a fermarsi, arrivando ad avere energia cinetica nulla. Se si ripete tante volte l'esperimento, si sente che il tavolo si è scaldato. Si comincia a capire che l'energia non è svanita ma si è trasformata in un altra forma. L'effetto dello strisciamento fa si che le molecole del piano si mettano a vibrare dando luogo a un moto di agitazione termica. Il moto ordinato si trasforma in un moto disordinato delle molecole che stanno sulla superficie del tavolo e vengono agitate.
Tramite gli urti fra il reticolo cristallino del materiale e il piano scabro, l' energia cinetica si è trasformata in calore. Quindi, il lavoro fatto da un corpo si può trasformare in calore ceduto.

Uno degli argomenti principali della termodinamica quindi è proprio l'esame del bilancio energetico complessivo di un processo fisico, esteso a scambi di energia che non sono meccanici nel senso macroscopico finora descritto.
Storicamente la termodinamica nasce con una valenza concreta, che è quella di riuscire a realizzare delle macchine che invertano questo processo descritto nell'esempio. Le \textbf{macchine termiche} (un esempio ne è quella a vapore) sono proprio dei dispositivi che sfruttando il calore ceduto da un corpo caldo, riuscendo a trasformarne una parte in lavoro meccanico. Non è infatti possibile trasformare tutto il calore disponibile, perché c'è sempre una parte di energia che diventa termica. Altro obiettivo della termodinamica è quello in realizzare \textbf{macchine frigorifere}, dispositivi che permettono il passaggio di calore da un corpo più freddo a un corpo più caldo.
In termodinamica non si mantiene lo stesso approccio microscopico utilizzato per lo studio della meccanica. Si immagini di avere un certo gas in una stanza. Esso è chiuso da un coperchio che si può muovere, sopra il quale è posta una massa $m$. Si vuole realizzare un processo per sollevarla. Si può fare ciò mettendo il contenitore su un fornello che cede il calore al gas: esso scaldandosi tende a espandersi e arriva a sollevare il pistone.
\begin{figure}[htpb]
	\centering
	

	% Pattern Info
	 
	\tikzset{
	pattern size/.store in=\mcSize, 
	pattern size = 5pt,
	pattern thickness/.store in=\mcThickness, 
	pattern thickness = 0.3pt,
	pattern radius/.store in=\mcRadius, 
	pattern radius = 1pt}
	\makeatletter
	\pgfutil@ifundefined{pgf@pattern@name@_w62qczmze}{
	\pgfdeclarepatternformonly[\mcThickness,\mcSize]{_w62qczmze}
	{\pgfqpoint{0pt}{0pt}}
	{\pgfpoint{\mcSize}{\mcSize}}
	{\pgfpoint{\mcSize}{\mcSize}}
	{
	\pgfsetcolor{\tikz@pattern@color}
	\pgfsetlinewidth{\mcThickness}
	\pgfpathmoveto{\pgfqpoint{0pt}{\mcSize}}
	\pgfpathlineto{\pgfpoint{\mcSize+\mcThickness}{-\mcThickness}}
	\pgfpathmoveto{\pgfqpoint{0pt}{0pt}}
	\pgfpathlineto{\pgfpoint{\mcSize+\mcThickness}{\mcSize+\mcThickness}}
	\pgfusepath{stroke}
	}}
	\makeatother
	\tikzset{every picture/.style={line width=0.75pt}} %set default line width to 0.75pt        

	\begin{tikzpicture}[x=0.75pt,y=0.75pt,yscale=-1,xscale=1]
	%uncomment if require: \path (0,300); %set diagram left start at 0, and has height of 300

	%Shape: Rectangle [id:dp8385749660235218] 
	\draw  [fill={rgb, 255:red, 155; green, 155; blue, 155 }  ,fill opacity=1 ] (183,99) -- (213,99) -- (213,119) -- (183,119) -- cycle ;
	%Straight Lines [id:da657731717925097] 
	\draw    (150,112) -- (150,220) ;
	%Straight Lines [id:da7283267698919824] 
	\draw    (240,112) -- (240,220) ;
	%Straight Lines [id:da4943759947161752] 
	\draw    (150,220) -- (240,220) ;
	%Straight Lines [id:da3363410918516512] 
	\draw [line width=2.25]    (150,120) -- (240,120) ;
	%Shape: Circle [id:dp1531034878009534] 
	\draw  [draw opacity=0][fill={rgb, 255:red, 212; green, 212; blue, 212 }  ,fill opacity=1 ] (159,139.13) .. controls (159,134.64) and (162.64,131) .. (167.13,131) .. controls (171.61,131) and (175.25,134.64) .. (175.25,139.13) .. controls (175.25,143.61) and (171.61,147.25) .. (167.13,147.25) .. controls (162.64,147.25) and (159,143.61) .. (159,139.13) -- cycle ;
	%Shape: Circle [id:dp16555656889441606] 
	\draw  [draw opacity=0][fill={rgb, 255:red, 212; green, 212; blue, 212 }  ,fill opacity=1 ] (185.5,140.63) .. controls (185.5,136.14) and (189.14,132.5) .. (193.63,132.5) .. controls (198.11,132.5) and (201.75,136.14) .. (201.75,140.63) .. controls (201.75,145.11) and (198.11,148.75) .. (193.63,148.75) .. controls (189.14,148.75) and (185.5,145.11) .. (185.5,140.63) -- cycle ;
	%Shape: Circle [id:dp2325841443862715] 
	\draw  [draw opacity=0][fill={rgb, 255:red, 212; green, 212; blue, 212 }  ,fill opacity=1 ] (206,156.13) .. controls (206,151.64) and (209.64,148) .. (214.13,148) .. controls (218.61,148) and (222.25,151.64) .. (222.25,156.13) .. controls (222.25,160.61) and (218.61,164.25) .. (214.13,164.25) .. controls (209.64,164.25) and (206,160.61) .. (206,156.13) -- cycle ;
	%Shape: Circle [id:dp23279454300641023] 
	\draw  [draw opacity=0][fill={rgb, 255:red, 212; green, 212; blue, 212 }  ,fill opacity=1 ] (167.5,157.13) .. controls (167.5,152.64) and (171.14,149) .. (175.63,149) .. controls (180.11,149) and (183.75,152.64) .. (183.75,157.13) .. controls (183.75,161.61) and (180.11,165.25) .. (175.63,165.25) .. controls (171.14,165.25) and (167.5,161.61) .. (167.5,157.13) -- cycle ;
	%Shape: Circle [id:dp1344282787670925] 
	\draw  [draw opacity=0][fill={rgb, 255:red, 212; green, 212; blue, 212 }  ,fill opacity=1 ] (185,169.63) .. controls (185,165.14) and (188.64,161.5) .. (193.13,161.5) .. controls (197.61,161.5) and (201.25,165.14) .. (201.25,169.63) .. controls (201.25,174.11) and (197.61,177.75) .. (193.13,177.75) .. controls (188.64,177.75) and (185,174.11) .. (185,169.63) -- cycle ;
	%Shape: Circle [id:dp3125843154048371] 
	\draw  [draw opacity=0][fill={rgb, 255:red, 212; green, 212; blue, 212 }  ,fill opacity=1 ] (161,180.63) .. controls (161,176.14) and (164.64,172.5) .. (169.13,172.5) .. controls (173.61,172.5) and (177.25,176.14) .. (177.25,180.63) .. controls (177.25,185.11) and (173.61,188.75) .. (169.13,188.75) .. controls (164.64,188.75) and (161,185.11) .. (161,180.63) -- cycle ;
	%Shape: Circle [id:dp2629546156324918] 
	\draw  [draw opacity=0][fill={rgb, 255:red, 212; green, 212; blue, 212 }  ,fill opacity=1 ] (212,179.63) .. controls (212,175.14) and (215.64,171.5) .. (220.13,171.5) .. controls (224.61,171.5) and (228.25,175.14) .. (228.25,179.63) .. controls (228.25,184.11) and (224.61,187.75) .. (220.13,187.75) .. controls (215.64,187.75) and (212,184.11) .. (212,179.63) -- cycle ;
	%Shape: Circle [id:dp8643090492517644] 
	\draw  [draw opacity=0][fill={rgb, 255:red, 212; green, 212; blue, 212 }  ,fill opacity=1 ] (191.5,190.63) .. controls (191.5,186.14) and (195.14,182.5) .. (199.63,182.5) .. controls (204.11,182.5) and (207.75,186.14) .. (207.75,190.63) .. controls (207.75,195.11) and (204.11,198.75) .. (199.63,198.75) .. controls (195.14,198.75) and (191.5,195.11) .. (191.5,190.63) -- cycle ;
	%Shape: Circle [id:dp9763411767245602] 
	\draw  [draw opacity=0][fill={rgb, 255:red, 212; green, 212; blue, 212 }  ,fill opacity=1 ] (211,205.13) .. controls (211,200.64) and (214.64,197) .. (219.13,197) .. controls (223.61,197) and (227.25,200.64) .. (227.25,205.13) .. controls (227.25,209.61) and (223.61,213.25) .. (219.13,213.25) .. controls (214.64,213.25) and (211,209.61) .. (211,205.13) -- cycle ;
	%Shape: Circle [id:dp04886257473073541] 
	\draw  [draw opacity=0][fill={rgb, 255:red, 212; green, 212; blue, 212 }  ,fill opacity=1 ] (172,204.63) .. controls (172,200.14) and (175.64,196.5) .. (180.13,196.5) .. controls (184.61,196.5) and (188.25,200.14) .. (188.25,204.63) .. controls (188.25,209.11) and (184.61,212.75) .. (180.13,212.75) .. controls (175.64,212.75) and (172,209.11) .. (172,204.63) -- cycle ;
	%Shape: Trapezoid [id:dp5293639465287374] 
	\draw  [pattern=_w62qczmze,pattern size=3pt,pattern thickness=0.75pt,pattern radius=0pt, pattern color={rgb, 255:red, 222; green, 222; blue, 222}] (161.5,220) -- (165.93,234.75) -- (224.08,234.75) -- (228.5,220) -- cycle ;
	%Shape: Circle [id:dp9592351510731913] 
	\draw  [draw opacity=0][fill={rgb, 255:red, 212; green, 212; blue, 212 }  ,fill opacity=1 ] (216,136.13) .. controls (216,131.64) and (219.64,128) .. (224.13,128) .. controls (228.61,128) and (232.25,131.64) .. (232.25,136.13) .. controls (232.25,140.61) and (228.61,144.25) .. (224.13,144.25) .. controls (219.64,144.25) and (216,140.61) .. (216,136.13) -- cycle ;

	% Text Node
	\draw (197.5,106) node    {$m$};


	\end{tikzpicture}
\end{figure}
\FloatBarrier
Tramite energia termica si è compiuto lavoro meccanico. Per usare un approccio microscopico si dovrebbero considerare tutte le innumerevoli molecole del gas e vedere come si muovono sotto l'azione dell'energia termica, cosa ovviamente impossibile. Si vanno invece a definire delle \textbf{variabili termodinamiche}, che permettono di descrivere nel complesso quello che succede al gas, basandosi sul suo comportamento osservato da un punto di vista macroscopico. Il sapere come si muovono in moto disordinato, anche detto \emph{browoniano}, le particelle del gas non è molto utile per capire come si muove il pistone verso l'alto. Insomma, non si studierà più il moto di punti materiali, ma i passaggi energetici da un corpo a un altro.

La definizione stessa di stato termodinamico è concettualmente diversa da quella di stato meccanico, per il quale, in linea di principio, si presuppone la conoscenza di posizione e velocità di ciascuno dei punti. In questi termini un sistema termodinamico non è definibile. In effetti, se è noto lo stato termodinamico, non lo è in generale quello meccanico, anzi, a un dato stato termodinamico possono corrispondere moltissimi stati meccanici diversi. La descrizione termodinamica di un sistema e dei suoi scambi energetici, tramite le variabili termodinamiche, porta a conclusioni di grande generalità, applicabili a sistemi molto diversi tra loro. Tale descrizione però non può fornire informazioni sulle caratteristiche microscopiche del sistema, che vanno studiate concettualmente in modo diverso.







































\section{I sistemi termodinamici}

\subsection{Definizione e caratteristiche generali}

Un \textbf{sistema termodinamico} è una porzione di spazio materiale, separata dal resto dell'ambiente esterno mediante una superficie di controllo. La parte o l'insieme delle parti con cui il sistema può interagire, prende il nome di \textbf{ambiente termodinamico}. Esso, nel caso prima citato è costituito dal fornello, dalla massa sul pistone e dallo spazio attorno ad esso. L'insieme sistema più ambiente si chiama \textbf{universo termodinamico}. La superficie che separa sistema termodinamico può essere reale o immaginaria e, a seconda dei passaggi che essa permette, il sistema termodinamico può essere classificato in tre tipologie:
\begin{itemize}
	\item Aperto: il confine permette il passaggio sia di energia sia di materia.
	\item Chiuso: il confine permette scambi di energia ma non di materia.
	\item Isolato: nessuno scambio è permesso.
\end{itemize}
Come già anticipato, per descrivere macroscopicamente un sistema termodinamico si introducono delle variabili termodinamiche che vanno a descrivere lo stato del sistema. Esse sono la pressione, la temperatura, il volume e la quantità di materia presente, ossia il numero di moli $n$.
Le variabili termodinamiche si distinguono in:
\begin{itemize}
	\item Variabili \textbf{intensive} (o locali): sono funzione della posizione (come la pressione o la temperatura);
	\item Variabili \textbf{estensive} (o globali): danno un'informazione generale, appunto, globale sul sistema. Esse sono ad esempio il volume o la quantità di materia.
\end{itemize}
Quando il sistema termodinamico è in equilibrio, esistono delle relazioni matematiche tra le variabili termodinamiche che prendono il nome di \textbf{funzioni di stato}: esse sussistono \emph{solo nella situazione in cui il sistema è in equilibrio}.
\[
	f(p,T,V,n) = 0
\]

\paragraph{Esempio} Quando un gas ideale viene mantenuto a temperatura costante e si espande in volume, la pressione diminuisce. Se la temperatura è costante, anche il prodotto pressione e volume si mantiene costante. Questo risultato è una funzione di stato nota come \textbf{legge di Boyle}.
\[
	p_1 V_1=p_2 V_2
\]
Le funzioni di stato sono caratteristiche del particolare sistema termodinamico che si sta considerando.

\subsection{Equilibrio termodinamico}

Perché un sistema termodinamico sia in equilibrio termodinamico devono sussistere contemporaneamente tre equilibri:
\begin{itemize}
	\item \emph{equilibrio termico}: il sistema ha una temperatura uniforme in tutti i suoi punti. Se così non fosse esso tenderebbe spontaneamente a scambiare calore da una parte all'altra fino a raggiungere l'equilibrio termico. In più, quando il confine lo permette, un sistema in equilibrio termico si trova in equilibrio con la temperatura dell'ambiente circostante. Ci sono infatti delle situazioni in cui il sistema termodinamico è costituito da un confine che non permette il passaggio del calore. Ad esempio il termos del caffè ha delle pareti isolanti tali per cui se anche fuori fa freddo il caffè rimane caldo. Esso allora sarà in equilibrio termico quando la temperatura è la stessa in tutti i suoi punti ma non si può imporre un equilibrio termico anche con l'ambiente circostante.
	\item \emph{equilibrio meccanico}: un sistema termodinamico è in equilibrio meccanico quando la pressione all'interno del sistema è la stessa in tutti i suoi punti: essa tende a bilanciare quella esterna se il confine lo permette. Il coperchio nell'esempio si solleva perché quando la pressione aumenta in modo tale da rendere il gas più rarefatto e andare a diminuirla fino a che essa non ritorna a bilanciare quella esterna. Questo è un esempio di equilibrio meccanico. La pressione deve essere uguale in tutti i punti e, se il confine lo permette, pari alla pressione esterna.
	\item \emph{equilibrio chimico}: le parti del sistema non stanno eseguendo nessuna reazione chimica e non stanno facendo alcun cambiamento di fase.
\end{itemize}

\subsection{Trasformazioni termodinamiche}

L'evoluzione del sistema da uno stato di equilibrio iniziale a uno finale, in cui le variabili termodinamiche sono cambiate, prende il nome di \textbf{trasformazione termodinamica}. Nello stato di equilibrio iniziale e finale le variabili termodinamiche rispettano la funzione di stato. In mezzo, durante la trasformazione termodinamica, questa cosa non avviene. Si osserva che le variabili indipendenti che cambiano durante una trasformazione sono le coordinate termodinamiche, in funzione delle quali si esprimono tutte le proprietà del sistema.
\begin{figure}[htpb]
	\centering
	

	\tikzset{every picture/.style={line width=0.75pt}} %set default line width to 0.75pt        

	\begin{tikzpicture}[x=0.75pt,y=0.75pt,yscale=-1,xscale=1]
	%uncomment if require: \path (0,300); %set diagram left start at 0, and has height of 300

	%Shape: Rectangle [id:dp14035139223234783] 
	\draw  [fill={rgb, 255:red, 155; green, 155; blue, 155 }  ,fill opacity=1 ] (290,120) -- (320,120) -- (320,140) -- (290,140) -- cycle ;
	%Straight Lines [id:da49681278948526053] 
	\draw    (130,92) -- (130,200) ;
	%Straight Lines [id:da4994473620807358] 
	\draw    (220,92) -- (220,200) ;
	%Straight Lines [id:da6387671993341548] 
	\draw    (130,200) -- (220,200) ;
	%Straight Lines [id:da08286092123994537] 
	\draw [line width=2.25]    (130,100) -- (220,100) ;
	%Straight Lines [id:da8841436304629093] 
	\draw    (260,92) -- (260,200) ;
	%Straight Lines [id:da868104119488712] 
	\draw    (350,92) -- (350,200) ;
	%Straight Lines [id:da6604958276683548] 
	\draw    (260,200) -- (350,200) ;
	%Straight Lines [id:da4733301466444775] 
	\draw [line width=2.25]    (260,140) -- (350,140) ;
	%Shape: Circle [id:dp847297195728026] 
	\draw  [draw opacity=0][fill={rgb, 255:red, 212; green, 212; blue, 212 }  ,fill opacity=1 ] (139,119.13) .. controls (139,114.64) and (142.64,111) .. (147.13,111) .. controls (151.61,111) and (155.25,114.64) .. (155.25,119.13) .. controls (155.25,123.61) and (151.61,127.25) .. (147.13,127.25) .. controls (142.64,127.25) and (139,123.61) .. (139,119.13) -- cycle ;
	%Shape: Circle [id:dp5074006468657941] 
	\draw  [draw opacity=0][fill={rgb, 255:red, 212; green, 212; blue, 212 }  ,fill opacity=1 ] (165.5,120.63) .. controls (165.5,116.14) and (169.14,112.5) .. (173.63,112.5) .. controls (178.11,112.5) and (181.75,116.14) .. (181.75,120.63) .. controls (181.75,125.11) and (178.11,128.75) .. (173.63,128.75) .. controls (169.14,128.75) and (165.5,125.11) .. (165.5,120.63) -- cycle ;
	%Shape: Circle [id:dp24582200119657371] 
	\draw  [draw opacity=0][fill={rgb, 255:red, 212; green, 212; blue, 212 }  ,fill opacity=1 ] (186,136.13) .. controls (186,131.64) and (189.64,128) .. (194.13,128) .. controls (198.61,128) and (202.25,131.64) .. (202.25,136.13) .. controls (202.25,140.61) and (198.61,144.25) .. (194.13,144.25) .. controls (189.64,144.25) and (186,140.61) .. (186,136.13) -- cycle ;
	%Shape: Circle [id:dp8148506207242139] 
	\draw  [draw opacity=0][fill={rgb, 255:red, 212; green, 212; blue, 212 }  ,fill opacity=1 ] (147.5,137.13) .. controls (147.5,132.64) and (151.14,129) .. (155.63,129) .. controls (160.11,129) and (163.75,132.64) .. (163.75,137.13) .. controls (163.75,141.61) and (160.11,145.25) .. (155.63,145.25) .. controls (151.14,145.25) and (147.5,141.61) .. (147.5,137.13) -- cycle ;
	%Shape: Circle [id:dp5047766488519425] 
	\draw  [draw opacity=0][fill={rgb, 255:red, 212; green, 212; blue, 212 }  ,fill opacity=1 ] (165,149.63) .. controls (165,145.14) and (168.64,141.5) .. (173.13,141.5) .. controls (177.61,141.5) and (181.25,145.14) .. (181.25,149.63) .. controls (181.25,154.11) and (177.61,157.75) .. (173.13,157.75) .. controls (168.64,157.75) and (165,154.11) .. (165,149.63) -- cycle ;
	%Shape: Circle [id:dp5204784668116911] 
	\draw  [draw opacity=0][fill={rgb, 255:red, 212; green, 212; blue, 212 }  ,fill opacity=1 ] (141,160.63) .. controls (141,156.14) and (144.64,152.5) .. (149.13,152.5) .. controls (153.61,152.5) and (157.25,156.14) .. (157.25,160.63) .. controls (157.25,165.11) and (153.61,168.75) .. (149.13,168.75) .. controls (144.64,168.75) and (141,165.11) .. (141,160.63) -- cycle ;
	%Shape: Circle [id:dp4567328025212567] 
	\draw  [draw opacity=0][fill={rgb, 255:red, 212; green, 212; blue, 212 }  ,fill opacity=1 ] (192,159.63) .. controls (192,155.14) and (195.64,151.5) .. (200.13,151.5) .. controls (204.61,151.5) and (208.25,155.14) .. (208.25,159.63) .. controls (208.25,164.11) and (204.61,167.75) .. (200.13,167.75) .. controls (195.64,167.75) and (192,164.11) .. (192,159.63) -- cycle ;
	%Shape: Circle [id:dp7972438623150406] 
	\draw  [draw opacity=0][fill={rgb, 255:red, 212; green, 212; blue, 212 }  ,fill opacity=1 ] (171.5,170.63) .. controls (171.5,166.14) and (175.14,162.5) .. (179.63,162.5) .. controls (184.11,162.5) and (187.75,166.14) .. (187.75,170.63) .. controls (187.75,175.11) and (184.11,178.75) .. (179.63,178.75) .. controls (175.14,178.75) and (171.5,175.11) .. (171.5,170.63) -- cycle ;
	%Shape: Circle [id:dp2794407603749758] 
	\draw  [draw opacity=0][fill={rgb, 255:red, 212; green, 212; blue, 212 }  ,fill opacity=1 ] (191,185.13) .. controls (191,180.64) and (194.64,177) .. (199.13,177) .. controls (203.61,177) and (207.25,180.64) .. (207.25,185.13) .. controls (207.25,189.61) and (203.61,193.25) .. (199.13,193.25) .. controls (194.64,193.25) and (191,189.61) .. (191,185.13) -- cycle ;
	%Shape: Circle [id:dp5580122495464388] 
	\draw  [draw opacity=0][fill={rgb, 255:red, 212; green, 212; blue, 212 }  ,fill opacity=1 ] (268.5,154.13) .. controls (268.5,149.64) and (272.14,146) .. (276.63,146) .. controls (281.11,146) and (284.75,149.64) .. (284.75,154.13) .. controls (284.75,158.61) and (281.11,162.25) .. (276.63,162.25) .. controls (272.14,162.25) and (268.5,158.61) .. (268.5,154.13) -- cycle ;
	%Shape: Circle [id:dp49522420372510156] 
	\draw  [draw opacity=0][fill={rgb, 255:red, 212; green, 212; blue, 212 }  ,fill opacity=1 ] (301,153.13) .. controls (301,148.64) and (304.64,145) .. (309.13,145) .. controls (313.61,145) and (317.25,148.64) .. (317.25,153.13) .. controls (317.25,157.61) and (313.61,161.25) .. (309.13,161.25) .. controls (304.64,161.25) and (301,157.61) .. (301,153.13) -- cycle ;
	%Shape: Circle [id:dp966021302774063] 
	\draw  [draw opacity=0][fill={rgb, 255:red, 212; green, 212; blue, 212 }  ,fill opacity=1 ] (318,166.63) .. controls (318,162.14) and (321.64,158.5) .. (326.13,158.5) .. controls (330.61,158.5) and (334.25,162.14) .. (334.25,166.63) .. controls (334.25,171.11) and (330.61,174.75) .. (326.13,174.75) .. controls (321.64,174.75) and (318,171.11) .. (318,166.63) -- cycle ;
	%Shape: Circle [id:dp11574108347735068] 
	\draw  [draw opacity=0][fill={rgb, 255:red, 212; green, 212; blue, 212 }  ,fill opacity=1 ] (285,167.63) .. controls (285,163.14) and (288.64,159.5) .. (293.13,159.5) .. controls (297.61,159.5) and (301.25,163.14) .. (301.25,167.63) .. controls (301.25,172.11) and (297.61,175.75) .. (293.13,175.75) .. controls (288.64,175.75) and (285,172.11) .. (285,167.63) -- cycle ;

	% Text Node
	\draw (304.5,127) node    {$m$};
	% Text Node
	\draw (154,181) node    {$p_{atm}$};
	% Text Node
	\draw (305,182) node    {$p_{atm} +mgS$};


	\end{tikzpicture}
\end{figure}
\FloatBarrier
Il tempo non compare esplicitamente, a differenza di quanto avviene in meccanica. Durante la trasformazione termodinamica, il sistema scambierà energia con l'ambiente circostante.
Le trasformazioni termodinamiche si distinguono in due tipologie:
\begin{itemize}
	\item reversibili: sono quelle trasformazioni in cui è possibile tornare indietro allo stato di partenza scambiando esattamente le stesse quantità di energia ribaltate di segno. Si tratta in realtà di processi ideali.
	\item Irreversibili: sono quelle trasformazioni in cui non si verificano le condizioni precedenti, ossia che passano attraverso stati di non equilibrio e/o avvengono in presenza di forze dissipative o entrambe le cose.
\end{itemize}
Per realizzare una trasformazione in maniera reversibile essa deve essere prima di tutto \textbf{quasi statica}. Ciò significa che viene fatta avvenire per passaggi tra continui stati di equilibrio. La si fa avvenire così lentamente che il sistema va in un nuovo stato di equilibrio, lo si smuove un po' e lo si fa passare per un altro stato di equilibrio e via dicendo. In ogni punto della trasformazione quasi statica vale la funzione di stato. In aggiunta, perché essa sia reversibile, è necessario che durante la trasformazione tutti gli effetti dissipativi, ossia tutti le forze d'attrito, siano rimossi. Questo accade perché bisogna poter tornare indietro scambiando le stessa quantità di energia, quindi essa non può andare dissipata.
In una trasformazione reversibile, poiché passa per continui stati di equilibrio, i valori delle variabili termodinamiche sono noti in tutti i punti perché dati dalla funzione di stato, che è, proprio per la quasi staticità, valida in ogni istante. Ecco perché essa può essere graficata nel cosiddetto \textbf{piano di Clapeyron} come una certa funzione nelle variabili.
\begin{figure}[htpb]
	\centering
	

	\tikzset{every picture/.style={line width=0.75pt}} %set default line width to 0.75pt        

	\begin{tikzpicture}[x=0.75pt,y=0.75pt,yscale=-1,xscale=1]
	%uncomment if require: \path (0,300); %set diagram left start at 0, and has height of 300

	%Shape: Axis 2D [id:dp3310228244561544] 
	\draw  (148.5,177.74) -- (378.5,177.74)(159.52,101) -- (159.52,186) (371.5,172.74) -- (378.5,177.74) -- (371.5,182.74) (154.52,108) -- (159.52,101) -- (164.52,108)  ;

	% Text Node
	\draw (147,104.5) node    {$p$};
	% Text Node
	\draw (392.5,178) node    {$V$};


	\end{tikzpicture}
\end{figure}
\FloatBarrier
In una trasformazione irreversibile, sono noti stato iniziale e finale perché ivi è soddisfatta l'equazione di stato, ma in mezzo il gas assume valori caotici di volume pressione e temperatura, non noti. In questo caso si disegna la trasformazione tratteggiata.

\subsection{Lavoro meccanico}

Data una certa trasformazione, per passare da uno stato di equilibrio all'altro il sistema scambierà delle energie con l'ambiente circostanze. Esse potranno essere un lavoro meccanico, elettrico o un'energia termica.
La pressione è quella variabile che permette di capire se il sistema è in equilibrio meccanico con l'ambiente circostante. La mancanza di equilibrio meccanico comporta la realizzazione di un lavoro meccanico $\mathcal{L}$ chiamato lavoro della pressione.
Si immagini di voler calcolare il lavoro fatto dalla pressione di un gas in un cilindro mentre esso si espande di un tratto $z$, dalla situazione iniziale fino a quella finale.
\begin{figure}[htpb]
	\centering
	

	\tikzset{every picture/.style={line width=0.75pt}} %set default line width to 0.75pt        

	\begin{tikzpicture}[x=0.75pt,y=0.75pt,yscale=-1,xscale=1]
	%uncomment if require: \path (0,300); %set diagram left start at 0, and has height of 300

	%Straight Lines [id:da46135479872599006] 
	\draw    (130,100) -- (130,200) ;
	%Straight Lines [id:da5457765796068568] 
	\draw    (220,100) -- (220,200) ;
	%Straight Lines [id:da6748695752601523] 
	\draw    (130,200) -- (220,200) ;
	%Straight Lines [id:da5142681174243358] 
	\draw [line width=2.25]    (130,123) -- (220,123) ;
	%Shape: Circle [id:dp9738521381170515] 
	\draw  [draw opacity=0][fill={rgb, 255:red, 212; green, 212; blue, 212 }  ,fill opacity=1 ] (138.5,139.13) .. controls (138.5,134.64) and (142.14,131) .. (146.63,131) .. controls (151.11,131) and (154.75,134.64) .. (154.75,139.13) .. controls (154.75,143.61) and (151.11,147.25) .. (146.63,147.25) .. controls (142.14,147.25) and (138.5,143.61) .. (138.5,139.13) -- cycle ;
	%Shape: Circle [id:dp6699148200404468] 
	\draw  [draw opacity=0][fill={rgb, 255:red, 212; green, 212; blue, 212 }  ,fill opacity=1 ] (186,136.13) .. controls (186,131.64) and (189.64,128) .. (194.13,128) .. controls (198.61,128) and (202.25,131.64) .. (202.25,136.13) .. controls (202.25,140.61) and (198.61,144.25) .. (194.13,144.25) .. controls (189.64,144.25) and (186,140.61) .. (186,136.13) -- cycle ;
	%Shape: Circle [id:dp06833473002119006] 
	\draw  [draw opacity=0][fill={rgb, 255:red, 212; green, 212; blue, 212 }  ,fill opacity=1 ] (140.5,165.13) .. controls (140.5,160.64) and (144.14,157) .. (148.63,157) .. controls (153.11,157) and (156.75,160.64) .. (156.75,165.13) .. controls (156.75,169.61) and (153.11,173.25) .. (148.63,173.25) .. controls (144.14,173.25) and (140.5,169.61) .. (140.5,165.13) -- cycle ;
	%Shape: Circle [id:dp4991812434962075] 
	\draw  [draw opacity=0][fill={rgb, 255:red, 212; green, 212; blue, 212 }  ,fill opacity=1 ] (165,149.63) .. controls (165,145.14) and (168.64,141.5) .. (173.13,141.5) .. controls (177.61,141.5) and (181.25,145.14) .. (181.25,149.63) .. controls (181.25,154.11) and (177.61,157.75) .. (173.13,157.75) .. controls (168.64,157.75) and (165,154.11) .. (165,149.63) -- cycle ;
	%Shape: Circle [id:dp1471771351523674] 
	\draw  [draw opacity=0][fill={rgb, 255:red, 212; green, 212; blue, 212 }  ,fill opacity=1 ] (139,187.63) .. controls (139,183.14) and (142.64,179.5) .. (147.13,179.5) .. controls (151.61,179.5) and (155.25,183.14) .. (155.25,187.63) .. controls (155.25,192.11) and (151.61,195.75) .. (147.13,195.75) .. controls (142.64,195.75) and (139,192.11) .. (139,187.63) -- cycle ;
	%Shape: Circle [id:dp4476354313067108] 
	\draw  [draw opacity=0][fill={rgb, 255:red, 212; green, 212; blue, 212 }  ,fill opacity=1 ] (192,159.63) .. controls (192,155.14) and (195.64,151.5) .. (200.13,151.5) .. controls (204.61,151.5) and (208.25,155.14) .. (208.25,159.63) .. controls (208.25,164.11) and (204.61,167.75) .. (200.13,167.75) .. controls (195.64,167.75) and (192,164.11) .. (192,159.63) -- cycle ;
	%Shape: Circle [id:dp3622533136691808] 
	\draw  [draw opacity=0][fill={rgb, 255:red, 212; green, 212; blue, 212 }  ,fill opacity=1 ] (168,179.13) .. controls (168,174.64) and (171.64,171) .. (176.13,171) .. controls (180.61,171) and (184.25,174.64) .. (184.25,179.13) .. controls (184.25,183.61) and (180.61,187.25) .. (176.13,187.25) .. controls (171.64,187.25) and (168,183.61) .. (168,179.13) -- cycle ;
	%Shape: Circle [id:dp3062247486090506] 
	\draw  [draw opacity=0][fill={rgb, 255:red, 212; green, 212; blue, 212 }  ,fill opacity=1 ] (191,185.13) .. controls (191,180.64) and (194.64,177) .. (199.13,177) .. controls (203.61,177) and (207.25,180.64) .. (207.25,185.13) .. controls (207.25,189.61) and (203.61,193.25) .. (199.13,193.25) .. controls (194.64,193.25) and (191,189.61) .. (191,185.13) -- cycle ;
	%Straight Lines [id:da25081709908699445] 
	\draw [line width=0.75]    (130,110) -- (220,110)(130,113) -- (220,113) ;
	%Straight Lines [id:da5784681561768281] 
	\draw    (229.5,110) -- (229.5,128.5) ;
	\draw [shift={(229.5,107)}, rotate = 90] [fill={rgb, 255:red, 0; green, 0; blue, 0 }  ][line width=0.08]  [draw opacity=0] (10.72,-5.15) -- (0,0) -- (10.72,5.15) -- (7.12,0) -- cycle    ;
	%Straight Lines [id:da8294208051947267] 
	\draw    (147.5,96) -- (147.5,122.5) ;
	\draw [shift={(147.5,93)}, rotate = 90] [fill={rgb, 255:red, 0; green, 0; blue, 0 }  ][line width=0.08]  [draw opacity=0] (10.72,-5.15) -- (0,0) -- (10.72,5.15) -- (7.12,0) -- cycle    ;
	%Straight Lines [id:da9530121924947981] 
	\draw    (175,96.5) -- (175,123) ;
	\draw [shift={(175,93.5)}, rotate = 90] [fill={rgb, 255:red, 0; green, 0; blue, 0 }  ][line width=0.08]  [draw opacity=0] (10.72,-5.15) -- (0,0) -- (10.72,5.15) -- (7.12,0) -- cycle    ;
	%Straight Lines [id:da5497785411174465] 
	\draw    (201,96) -- (201,122.5) ;
	\draw [shift={(201,93)}, rotate = 90] [fill={rgb, 255:red, 0; green, 0; blue, 0 }  ][line width=0.08]  [draw opacity=0] (10.72,-5.15) -- (0,0) -- (10.72,5.15) -- (7.12,0) -- cycle    ;

	% Text Node
	\draw (245.5,115.5) node    {$dz$};


	\end{tikzpicture}
\end{figure}
\FloatBarrier
La forza che compie il lavoro è quella generata dalla pressione. In quanto tale, è sempre ortogonale al pistone, alla superficie su cui agisce. Il prodotto scalare $\vec{F}\cdot d\vec{r}$ sarà semplicemente il prodotto della forza normale per lo spostamento infinitesimo $dz$.
\[
	\mathcal{L} = \int \vec{F} d\vec{r} = \int F_n dz = \int p_{\text{gas} }Sdz
\]
$S$ è la superficie del pistone, e quindi $S\,dz$ è la variazione di volume infinitesima che subisce il gas mentre si espande. Il lavoro della pressione si può scrivere semplicemente come:
\[
	\boxed{\mathcal{L} = \int p_{\text{gas}}dV}
\]
La pressione del gas non è necessariamente costante. Ma se esso si espande la pressione diminuirà. Inoltre non è una variabile sempre nota durante la trasformazione. Lo è se la trasformazione è quasi statica. Quando non si ha un equilibrio delle pressioni, il gas fa sempre un lavoro meccanico legato alla pressione. In termodinamica ci si pone dal punto di vista del sistema termodinamico, cioè del gas. Il lavoro sarà positivo quando il gas si espande, è negativo altrimenti.

\subsection{Calore scambiato}

L'altra grandezza energetica che scambia il sistema termodinamico è il calore (si misura in Joule) e il suo scambio comporta movimenti macroscopici. La variabile termodinamica coinvolta è la temperatura. Essa è indice dell'equilibro termodinamico tra due corpi o fra il sistema termodinamico e l'ambiente circostante. Quando due corpi sono alla stessa temperatura, si dice che sono in equilibrio termico, altrimenti vi è passaggio di calore dal corpo più caldo al corpo più freddo. La scala termometrica che si usa nel sistema internazionale delle unità di misura è la scala Kelvin. Molto utilizzata è anche la scala Celsius, che è definita tramite due punti di riferimento: il punto di congelamento dell'acqua alla pressione atmosferica $(0 \degree \text{C})$ e quello della sua ebollizione (sempre alla pressione atmosferica, $100 \degree \text{C}$). Si prendono questi due punti, li si divide in cento parti e si ottiene la definizione di grado centigrado. La scala in gradi Kelvin invece è costruita in maniera tale che $0 \degree \text{C} =273.15 K$. Vi è poi una temperatura minima a cui un corpo può solo tendere che è proprio $0 K (-273.15 \degree \text{C})$. Le temperature in Kelvin sono quindi sempre positive. Quando le grandezze sono funzione del salto termico è equivalente usare i gradi centigradi o i Kelvin: aumentare o diminuire la temperatura di un grado Kelvin o Celsius è infatti la stessa cosa.

Quando due corpi sono a temperature diverse si può verificare uno scambio di calore che dipende da come essi sono fatti. Per caratterizzare la capacità che ha un materiale di variare la sua temperatura al variare dell'energia termica che assorbe, viene definita una grandezza caratteristica del materiale che prende il nome di \textbf{calore specifico} ($c$) e che fondamentalmente è un energia per unità di massa, per unità di temperatura. È definito come la quantità di energia termica, di calore, che bisogna fornire a un corpo per unità di massa per fargli variare la sua temperatura di una quantità infinitesima $dT$. È evidente che maggiore è il calore specifico, maggiore calore deve essere ceduto a pari salto termico.
\[
	c = \frac{1}{m} \frac{dQ}{dT}
\]
Calore specifico elevato vuole dire bassa capacità di variare la propria temperatura, bassa conducibilità termina. Basso calore specifico è tipico invece dei metalli. Dimensionalmente il calore specifico è:
\begin{gather*}
	c = \frac{[E]}{[m][T]} = \frac{J}{kg\cdot K} \\
	c_{H_2 O} = \frac{1\,cal }{g\, \degree \text{C} } = \frac{4.18 \, J}{g\, \degree \text{C}} = \frac{4180J}{kg\, \degree \text{C}}
\end{gather*}
Preso un grammo d'acqua, se si vuole innalzare la sua temperatura da $17.5 \degree \text{C}$ a $18.5 \degree \text{C}$ bisogna fornirgli una caloria o $4.18 \, J$.
Si usa la definizione di $c$ per ottenere quella di calore trasferito da un corpo all'altro. È possibile ribaltare la relazione e dire che l'energia termica che deve assorbire o cedere un corpo di massa $m$ e calore specifico $c$, a cui si vuol far aumentare la temperatura di una variazione infinitesima $dT$ è pari a:
\[
	dQ = m\,c\,dT
\]
Questa è una quantità infinitesima. Se si vuole la quantità totale di calore o di energia termica che è necessario fornire a un corpo per fargli variare la temperatura, si sommano in maniera continua tutti i calori infinitesimi, si calcola un integrale.
\[
	Q = \int_{T_A}^{T_B} m\,c\,dT
\]
Questa definizione è valida tutte le volte che il corpo non sta cambiando di fase. Quando una sostanza cambia fase non varia la sua temperatura, ma il calore specifico è diverso. Teoricamente bisognerebbe conoscere la funzione che descrive come varia $c$ al variare di $T$. Si ipotizzerà sempre costante il calore specifico nell'intervallo termico considerato. È evidente che quando il corpo si scalda l'energia sarà assorbita e quindi positiva.

Si supponga di avere una quantità di latte (1) che si vuole mischiare con del caffè liquido (2) in una tazza isolante, tale per cui gli scambi di calore con l'ambiente esterno sono trascurabili.
\begin{figure}[htpb]
	\centering
	

	\tikzset{every picture/.style={line width=0.75pt}} %set default line width to 0.75pt        

	\begin{tikzpicture}[x=0.75pt,y=0.75pt,yscale=-1,xscale=1]
	%uncomment if require: \path (0,300); %set diagram left start at 0, and has height of 300

	%Shape: Regular Polygon [id:dp4011525611692206] 
	\draw  [fill={rgb, 255:red, 240; green, 239; blue, 239 }  ,fill opacity=1 ] (276.21,139.12) .. controls (259.96,147.3) and (186.65,155.91) .. (202.83,139.65) .. controls (219.02,123.4) and (218.96,115.33) .. (202.48,91.25) .. controls (186,67.16) and (259.37,66.63) .. (275.85,90.71) .. controls (292.33,114.79) and (292.45,130.93) .. (276.21,139.12) -- cycle ;
	%Shape: Regular Polygon [id:dp3359698689255244] 
	\draw  [fill={rgb, 255:red, 100; green, 100; blue, 100 }  ,fill opacity=1 ] (407.22,136.12) .. controls (390.97,144.31) and (317.63,152.92) .. (333.82,136.66) .. controls (350.01,120.4) and (349.95,112.33) .. (333.46,88.24) .. controls (316.98,64.15) and (390.37,63.61) .. (406.86,87.7) .. controls (423.35,111.79) and (423.47,127.93) .. (407.22,136.12) -- cycle ;
	%Straight Lines [id:da851480115804305] 
	\draw    (345.25,112) .. controls (343.58,113.67) and (341.92,113.67) .. (340.25,112) .. controls (338.58,110.33) and (336.92,110.33) .. (335.25,112) .. controls (333.58,113.67) and (331.92,113.67) .. (330.25,112) .. controls (328.58,110.33) and (326.92,110.33) .. (325.25,112) .. controls (323.58,113.67) and (321.92,113.67) .. (320.25,112) .. controls (318.58,110.33) and (316.92,110.33) .. (315.25,112) .. controls (313.58,113.67) and (311.92,113.67) .. (310.25,112) .. controls (308.58,110.33) and (306.92,110.33) .. (305.25,112) .. controls (303.58,113.67) and (301.92,113.67) .. (300.25,112) -- (299.25,112) -- (296.25,112)(345.25,109) .. controls (343.58,110.67) and (341.92,110.67) .. (340.25,109) .. controls (338.58,107.33) and (336.92,107.33) .. (335.25,109) .. controls (333.58,110.67) and (331.92,110.67) .. (330.25,109) .. controls (328.58,107.33) and (326.92,107.33) .. (325.25,109) .. controls (323.58,110.67) and (321.92,110.67) .. (320.25,109) .. controls (318.58,107.33) and (316.92,107.33) .. (315.25,109) .. controls (313.58,110.67) and (311.92,110.67) .. (310.25,109) .. controls (308.58,107.33) and (306.92,107.33) .. (305.25,109) .. controls (303.58,110.67) and (301.92,110.67) .. (300.25,109) -- (299.25,109) -- (296.25,109) ;
	\draw [shift={(287.25,110.5)}, rotate = 360] [fill={rgb, 255:red, 0; green, 0; blue, 0 }  ][line width=0.08]  [draw opacity=0] (10.72,-5.15) -- (0,0) -- (10.72,5.15) -- (7.12,0) -- cycle    ;
	%Shape: Regular Polygon [id:dp19706115012076286] 
	\draw  [fill={rgb, 255:red, 212; green, 212; blue, 212 }  ,fill opacity=1 ] (276.21,269.12) .. controls (259.96,277.3) and (186.65,285.91) .. (202.83,269.65) .. controls (219.02,253.4) and (218.96,245.33) .. (202.48,221.25) .. controls (186,197.16) and (259.37,196.63) .. (275.85,220.71) .. controls (292.33,244.79) and (292.45,260.93) .. (276.21,269.12) -- cycle ;
	%Shape: Regular Polygon [id:dp013310898503069435] 
	\draw  [fill={rgb, 255:red, 212; green, 212; blue, 212 }  ,fill opacity=1 ] (407.22,266.12) .. controls (390.97,274.31) and (317.63,282.92) .. (333.82,266.66) .. controls (350.01,250.4) and (349.95,242.33) .. (333.46,218.24) .. controls (316.98,194.15) and (390.37,193.61) .. (406.86,217.7) .. controls (423.35,241.79) and (423.47,257.93) .. (407.22,266.12) -- cycle ;

	% Text Node
	\draw (228,88.25) node    {$m_{1}$};
	% Text Node
	\draw (260,102) node    {$c_{1}$};
	% Text Node
	\draw (235.5,123.5) node    {$T_{1}$};
	% Text Node
	\draw (361.5,88.25) node    {$m_{2}$};
	% Text Node
	\draw (391.5,102.5) node    {$c_{2}$};
	% Text Node
	\draw (367,124) node    {$T_{2}$};
	% Text Node
	\draw (218,54) node   [align=left] {Stato A};
	% Text Node
	\draw (225,214) node    {$m_{1}$};
	% Text Node
	\draw (260,232) node    {$c_{1}$};
	% Text Node
	\draw (235.5,253.5) node    {$T_{eq}$};
	% Text Node
	\draw (356.5,214.5) node    {$m_{2}$};
	% Text Node
	\draw (391.5,232.5) node    {$c_{2}$};
	% Text Node
	\draw (367,254) node    {$T_{eq}$};
	% Text Node
	\draw (218,184) node   [align=left] {Stato B};


	\end{tikzpicture}
\end{figure}
\FloatBarrier
L'unico scambio termico che avviene è il passaggio di calore dal corpo più caldo al corpo più freddo. Si ipotizza che non ci sia lavoro meccanico perché ne il caffè ne il latte si espandono. Per capire come il sistema termodinamico arriva, dopo un certo tempo, in una situazione finale in cui i due corpi sono in equilibrio basta imporre che il calore totale scambiato nel sistema termodinamico è zero. Il corpo più freddo assorbe una quantità di calore $Q_1$ esattamente uguale a quella ceduta dal corpo più caldo, $Q_2$.
\begin{gather*}
	m_1 c_1 (T_{eq} - T_1 ) + m_2 c_2(T_{eq}-T_2) = 0 \\
	\boxed{T_{eq} = \frac{m_1 c_1 T_1 + m_2 c_2 T_2}{m_1 c_1+m_2 c_2}}
\end{gather*}
Questo risultato si ottiene nell'ipotesi che i calori specifici siano costanti al variare della temperatura e che durante il processo non avvenga nessun cambiamento di fase.
Insieme a $c$ si introduce il concetto di \textbf{capacità termica} di un oggetto. Essa è definita come il calore specifico del materiale di cui il corpo è costituito, per la sua massa. Dimensionalmente è un'energia su una temperatura e si misura in $J/K$.

È possibile che avvenga uno scambio di energia termica anche quando il corpo non cambia temperatura. Questo avviene quando sta cambiando di fase.
\begin{table}[htpb]
	\centering
	\begin{tabular}{ll}
		fusione & solido $\to$ liquido \\
		solidificazione & liquido $\to$ solido \\
		evaporazione & liquido $\to$ gassoso \\
		condensazione & gassoso $\to$ liquido \\
		sublimazione & solido $\to$ gassoso \\
		brinamento & gassoso $\to$ solido \\
	\end{tabular}
\end{table}
Se si va a guardare il microscopico, quando si ha un cambiamento di fase, ad esempio la fusione, l'energia fornita va a rompere alcuni legami molecolari. Per una sostanza in fase solida infatti, si hanno dei legami molecolari molto più rigidi di quelli che caratterizzerebbero la stessa sostanza liquida. Fino a che tutte le particelle della sostanza non sono state trasformate nella nuova fase, il corpo non cambia di temperatura. Si parla di processi ipotermici.
Non si avrà dipendenza da temperatura come in precedenza, ma solo dalla quantità di sostanza a cui si vuole far cambiare la fase e da un parametro detto \textbf{calore latente}. Esso dipende dal materiale in questione e dalla transizione di fase da seguire.
\[
	Q = m \lambda \qquad \lambda_s,\lambda_f,\lambda_v
\]
Cambiando verso $\lambda$ non cambia se non di segno. Se si vuole fondere una sostanza solida bisognerà fornire energia e quindi $\lambda$ sarà positivo, negativo altrimenti. Il calore latente dimensionalmente è un'energia per unità di massa e si misura come $J/kg$.

\paragraph{Osservazione} Una caratteristica molto importante dei cambiamenti di fase è di essere, in opportune condizioni, trasformazioni praticamente reversibili.

Si va a ridefinire il calore scambiato quando la sostanza è un gas: non si ha più il calore specifico per unità di massa ma il \textbf{calore specifico molare}, per unità di mole. Esso è la quantità di energia termica che deve assorbire o cedere un materiale durante la variazione di temperatura per unità di temperatura per numero di moli. Quando si ha a che fare con una sostanza in fase gassosa infatti la quantità di materia che c'è nel recipiente che contiene il gas viene data in numero di moli, quindi in numero di particelle contenute. Quest'ultimo generalmente è enorme. Una mole è un numero di particelle pari al \textbf{numero di Avogadro}, corrispondente al numero di atomi contenuti in $12 g$ dell'isotopo 12 del carbonio.
\[
	N_A = 6.022 \times 10^{23} \quad \text{particelle/mol}
\]
A seconda della specie molecolare, parlare di mole vuole dire avere un certo numero di atomi o di molecole.
In realtà il calore specifico è una grandezza che dipende dalla trasformazione $\Gamma$ che segue il gas, quando varia la sua temperatura. Si ha un certo valore del calore specifico quando si scalda un corpo a volume costante, oppure a pressione costante. In particolare si imparerà a definire il calore specifico di un gas quando viene scaldato a volume costante, $c_v$, mentre quello scambiato a pressione costante prende il nome di $c_p$. Si ha $c_p > c_v$ perché è più difficile scaldare un gas mantenendo la pressione costante piuttosto che a volume costante.

Ad esempio, per un gas monoatomico, specie stabile quando l'atomo è da solo, il calore specifico a volume costante è pari a $3/2\;R$ mentre a pressione costante è pari a $5/2\;R$ (dove $R$ è la costante universale dei gas). Per molti gas biatomici, soprattutto per quelle molecole disposte su una linea si ha:
\[
	c_v = \frac{5}{2}R \qquad c_p = \frac{7}{2}R
\]
\begin{table}[htpb]
	\centering
	\begin{tabular}{|c|c|c|}
		\hline
		gas & $c_p$ & $c_v$ \\
		\hline
		mono & $5/2 \;R$ & $3/2 \;R$ \\
		\hline
		bi & $7/2 \;R$ & $5/2 \;R$ \\
		\hline
	\end{tabular}
\end{table}
In realtà queste considerazioni su $c_p$ e $c_v$ sono vere anche per i liquidi e per i soldi, solo che quando una sostanza solida viene scaldata, si da per scontato che quella trasformazione stia avvenendo a pressione costante. In questo caso avviene una trasformazione isobara. Quello che si può notare al massimo è una leggera espansione del corpo. Ribaltando la definizione di calore specifico, si può calcolare il calore che scambia un grammo come l'integrale:
\[
	Q=\int n\,c_{\Gamma}dT = n\,c_{\Gamma}\Delta T
\]







































\section{I principio della termodinamica}

Si consideri un sistema termodinamico costituito da un gas nello stato di equilibrio iniziale $A$ che in qualche maniera arriva a un nuovo stato di equilibrio finale $B$ seguendo un certo tipo di trasformazione $\Gamma_1$. Il gas scambierà una certa quantità di calore e una certa quantità di lavoro meccanico che dipenderanno dalla trasformazione seguita: si potrebbe infatti andare allo stesso stato finale seguendo diverse trasformazioni, a seconda delle quali esso scambierà determinate quantità di calore $Q$ e lavoro $\mathcal{L}$.
\begin{figure}[htpb]
	\centering
	

	\tikzset{every picture/.style={line width=0.75pt}} %set default line width to 0.75pt        

	\begin{tikzpicture}[x=0.75pt,y=0.75pt,yscale=-1,xscale=1]
	%uncomment if require: \path (0,300); %set diagram left start at 0, and has height of 300

	%Curve Lines [id:da6010533575144312] 
	\draw    (94,123) .. controls (134,93) and (174.5,256) .. (225.5,147) .. controls (276.5,38) and (291.5,176) .. (406.5,124) ;
	%Shape: Circle [id:dp19119229533280602] 
	\draw  [fill={rgb, 255:red, 0; green, 0; blue, 0 }  ,fill opacity=1 ] (91.75,123) .. controls (91.75,121.76) and (92.76,120.75) .. (94,120.75) .. controls (95.24,120.75) and (96.25,121.76) .. (96.25,123) .. controls (96.25,124.24) and (95.24,125.25) .. (94,125.25) .. controls (92.76,125.25) and (91.75,124.24) .. (91.75,123) -- cycle ;
	%Shape: Circle [id:dp532302755488802] 
	\draw  [fill={rgb, 255:red, 0; green, 0; blue, 0 }  ,fill opacity=1 ] (404.25,124) .. controls (404.25,122.76) and (405.26,121.75) .. (406.5,121.75) .. controls (407.74,121.75) and (408.75,122.76) .. (408.75,124) .. controls (408.75,125.24) and (407.74,126.25) .. (406.5,126.25) .. controls (405.26,126.25) and (404.25,125.24) .. (404.25,124) -- cycle ;

	% Text Node
	\draw (91,97) node    {$\text{stato di equilibrio} \ A$};
	% Text Node
	\draw (390,100) node    {$\text{stato di equilibrio} \ B$};
	% Text Node
	\draw (236,182) node    {$Q_{1} ,\mathcal{L}_{1}$};
	% Text Node
	\draw (264,89) node    {$\Gamma _{1}$};


	\end{tikzpicture}
\end{figure}
\FloatBarrier
Tuttavia, se si considera la quantità $Q-\mathcal{L}$, questa non dipende dalla trasformazione, ma solo dallo stato iniziale e finale.
Ciò significa che tale quantità è una funzione di stato, analogamente a come lo è l'energia potenziale. Si potrà sempre trovare una funzione $f$ tale per cui $Q-\mathcal{L}$ si può vedere come la variazione di $f$ valutata nello stato iniziale e in quello finale. La funzione prende il nome di \textbf{energia interna del sistema}.  Essa è definita a meno di una costante additiva, la cui presenza è irrilevante, in quanto dal punto di vista fisico è importante solo la sua variazione, come d'altronde nel caso dell'energia potenziale.
\begin{equation}
	\label{principio}
	\boxed{U(B)-U(A) = Q_p - \mathcal{L}_p \qquad U(m,T,p,V)}
\end{equation}
Si arriva così a enunciare il \textbf{primo principio della termodinamica}: esiste una funzione delle coordinate termodinamiche di un sistema, chiamata energia interna $U$, le cui variazioni generano gli scambi energetici del sistema con l'ambiente che lo circonda.

Tale principio è puramente sperimentale e rappresenta un bilancio energetico di un processo termodinamico. Non è altro che un principio di conservazione dell'energia in forma molto generale. L'energia intera accumulata in un sistema termodinamico è quella che può trasformarsi in lavoro meccanico della pressione, il quale a sua volta potrà trasformarsi in calore ceduto all'ambiente circostante. Essa può variare in modo da dare luogo a degli scambi di energia con l'ambiente circostante. Fisicamente essa racchiude tutti i contributi di energia cinetica e potenziale che ha il sistema termodinamico.
A livello microscopico, Un gas è costituito da tante particelle che si muovono secondo un moto di agitazione caotico che prende il nome di \textbf{agitazione termica}. Tali particelle forniscono un contributo di energia cinetica a $U$. L'agitazione termica è tanto più elevata quanto più è elevata la temperatura.
In più le molecole sono legate fra di loro da un legame che si può approssimare a una molla, e che figura in $U$ come contributo di energia potenziale della molla.
Si può collegare l'energia interna macroscopicamente al valore delle variabili termodinamiche e dire che quando un gas ha quei valori delle variabili ha quell'energia interna.
~\eqref{principio} è una relazione che collega un stato iniziale e finale distanti fra di loro, non infinitesima. La stessa relazione là si può scrivere anche come quantità infinitesime, andando a vedere quello che accade fra due stati molto vicini fra di loro.
\[
	dU = \delta Q - \delta\mathcal{L} \implies \int_A^B dU = \int_{\Gamma} \delta Q - \int_{\Gamma} \delta\mathcal{L}
\]
Questa cosa comunica che la quantità $dU$ infinitesima, a differenza di $\delta Q$ e $\delta\mathcal{L}$, è un \emph{differenziale esatto}. Significa che quando la si integra, si ha un integrale che non dipende dalla trasformazione.

\paragraph{Riassunto dei punti principali riguardo al primo principio della termodinamica}
\begin{itemize}
	\item Sperimentalmente si trova sempre verificato questo risultato: se il sistema compie una trasformazione dallo stato $A$ allo stato $B$, scambiando calore e lavoro con l'ambiente, $Q$ e $\mathcal{L}$ dipendono dalla trasformazione che congiunge i due stati termodinamici, mentre la loro differenza risulta indipendente dalla trasformazione.
	\item esiste una funzione delle coordinate termodinamiche del sistema o, come si dice, una funzione di stato, chiamata energia interna, le cui variazioni danno luogo a scambi energetici del sistema con l'ambiente che lo circonda durante una trasformazione.
	\item Quando, durante una trasformazione, si fornisce energia a un sistema, sia tramite un lavoro meccanico che con uno scambio di calore, questa resta immagazzinata sottoforma di energia interna e può essere successivamente
	utilizzata.
	\item Il termine energia interna indica che non si tratta dell'energia cinetica del sistema nel suo complesso, o dell'energia potenziale, bensì di un'energia legata alla proprietà interne del sistema, come moto molecolare o forze intermolecolari, che non dipendono dallo stato complessivo di moto, ma piuttosto dalla temperatura del sistema, dalla pressione a cui è sottoposto o dal volume che occupa.
	\item Il primo principio mette in evidenza l'esistenza di un meccanismo di scambio di energia, che non è esprimibile come lavoro meccanico macroscopico: a questo si da il nome di calore ed è ancora riconducibile a fenomeni meccanici, ma a livello microscopico. Il primo principio della termodinamica fornisce la definizione più generale di calore, sia concettualmente che dal punto di vista del calcolo.
\end{itemize}
Finora è stato definito come esprimere l'energia termica scambiata dal sistema termodinamico quando esso varia la sua temperatura o quando subisce un cambio di fase, oppure il lavoro se è di tipo meccanico, e quindi subito dalla pressione del gas. In realtà si potrebbe avere anche lavoro elettrico. Ad esempio se all'interno di un recipiente contenete un gas si fa passare una resistenza elettrica percorsa da corrente, viene compiuto lavoro elettrico dovuto al passaggio della corrente nel conduttore. La relazione comprende tutti i possibili lavori fatti o subiti dal gas.

Se si considera una trasformazione termodinamica ciclica, secondo il primo principio della dinamica:
\[
	\Delta U_{\text{ciclo} } = Q_{\Gamma,\text{ciclo} } -\mathcal{L}_{\Gamma,\text{ciclo} }  =0 \implies \mathcal{L}=Q
\]
\begin{figure}[htpb]
	\centering
	

	\tikzset{every picture/.style={line width=0.75pt}} %set default line width to 0.75pt        

	\begin{tikzpicture}[x=0.75pt,y=0.75pt,yscale=-1,xscale=1]
	%uncomment if require: \path (0,300); %set diagram left start at 0, and has height of 300

	%Shape: Axis 2D [id:dp6876434402516973] 
	\draw  (128.5,214.72) -- (358.5,214.72)(139.52,82) -- (139.52,229) (351.5,209.72) -- (358.5,214.72) -- (351.5,219.72) (134.52,89) -- (139.52,82) -- (144.52,89)  ;
	%Shape: Polygon Curved [id:ds34099068300713564] 
	\draw   (234.5,133) .. controls (261.5,148) and (322.5,87) .. (318.5,111) .. controls (314.5,135) and (259.5,148) .. (279.5,178) .. controls (299.5,208) and (221.5,181) .. (174.5,193) .. controls (127.5,205) and (207.5,118) .. (234.5,133) -- cycle ;
	\draw  [fill={rgb, 255:red, 0; green, 0; blue, 0 }  ,fill opacity=1 ] (187.3,144.36) -- (198.08,143.27) -- (192.49,152.55) -- (193.99,145.86) -- cycle ;

	% Text Node
	\draw (235,118) node    {$\Gamma _{\text{ciclo}}$};
	% Text Node
	\draw (127,84.5) node    {$p$};
	% Text Node
	\draw (370.5,218) node    {$V$};


	\end{tikzpicture}
\end{figure}
\FloatBarrier
L'energia interna è funzione solo dello stato iniziale, quindi qualunque processo si segua, la variazione dell'energia interna è sempre uguale a zero durante una trasformazione ciclica. Durante il ciclo il sistema termodinamico scambia calore e lavoro ma le due quantità energetiche scambiate si bilanciano, il lavoro scambiato è esattamente uguale al lavoro compiuto.







































\section{I gas perfetti}

\subsection{Definizione e equazione di stato}

Un \textbf{gas perfetto} è un gas tale per cui le sue particelle interagiscono molto poco le une con le altre, non sono così vicine da essere attratte per attrazione elettrostatica o gravitazionale. Le uniche interazioni che subiscono saranno urti fra di loro e con le pareti del contenitore presente. Dal punto di vista delle variabili termodinamiche, per avere un gas perfetto bisogna essere in condizioni di pressione e temperatura tali per cui si è sufficientemente lontani dal punto di condensazione del gas. Le particelle sono tante ma c'è tanto spazio vuoto tra una molecola e l'altra. In questo paragrafo si definiscono delle relazioni che valgono soltanto quando si ha un gas perfetto o ideale.

Avendo a che fare con una sostanza in fase gassosa, le variabile termodinamiche adatte per descrivere lo stato di un gas sono:
\begin{itemize}
	\item La quantità di molecole o di particelle contenuta in un certo recipiente: il numero di moli
	\item La pressione
	\item La temperatura
	\item Il volume occupato dal gas
\end{itemize}
Vari esperimenti condotti da scienziati su gas perfetti dedussero quattro leggi sperimentali:
\begin{itemize}
	\item \textbf{Legge di Boyle}: se un gas viene mantenuto a temperatura costante, il prodotto pressione volume si mantiene constante;
	\[ \boxed{p_1 V_1=p_2 V_2} \]
	\item \textbf{Legge di espansione isobara, o di Volta-Gay-Lussac}. Se si mantiene il gas a pressione costante, scaldandolo il volume aumenta proporzionalmente con la sua temperatura.
	\[ \boxed{V=V_0(1+\alpha T) } \quad \alpha = \text{coefficiente di dilatazione termica} \]
	\item \textbf{Legge isocora di Gay-Lussac}: se si mantiene il volume costante, quindi si blocca il recipiente in cui è contenuto il gas, e lo si scalda, aumenterà la pressione. Il suo aumento è lineare con l'aumento di temperatura.
	\[ \boxed{p=p_0(1+\beta T) } \quad \beta = \text{coefficiente di espansione dei gas} \]
	\item \textbf{Legge di Avogadro}: volumi e uguali di gas diversi, alla stessa temperatura e pressione, contengono lo stesso numero di molecole.
\end{itemize}
Da quest'ultima legge deriva la definizione di \textbf{volume molare}. Se si ha un gas qualunque e lo si porta alla pressione di un atmosfera e alla temperatura di $0 \degree \text{C}$, allora una mole di gas occuperà sempre un volume pari a $22.4$ litri. Questo risultato non dipende dal tipo di gas perché essendo perfetto è così rarefatto che lo spazio occupato dalle molecole è ininfluente. Il volume occupato dipende solo dal numero di moli.

Si definisce, sulla base delle tre leggi elementari e della legge di Avogadro, come gas ideale un sistema le cui coordinate termodinamiche in uno stato di equilibrio obbediscono all'\textbf{equazione di stato di un gas ideale}:
\[
	\boxed{pV=nRT} \qquad R=\frac{[p][L]^3}{[mol][K]}
\]
Dove $R$ è costante e vale $8.31 J/(K\,\text{mol})$.
La relazione la si può usare tutte le volte che il gas perfetto è in condizioni di equilibrio termodinamico.

\subsection{Trasformazioni termodinamiche di gas perfetti}

Si consideri una trasformazione che va da $A$ a $B$, stati di equilibrio. In tali punti le variabili termodinamiche del gas soddisferanno sempre l'equazione di stato. Nei punti intermedi della relazione non si può imporre la validità dell'equazione di stato.
La relazione mostra che le quattro variabili non sono indipendenti. Noto il numero di moli, allora la temperatura è automaticamente fissata, e bastano due variabili termodinamiche per conoscere completamente lo stato del sistema. Ecco perché è possibile utilizzare il piano di Clapeyron (che ha due soli assi) per definire uno stato termodinamico.

È stato detto che quando il gas si espande o si comprime, il lavoro fatto da esso lungo la trasformazione $\Gamma$ è quello dovuto alla pressione. La sua espressione è la seguente:
\[
	\mathcal{L}_{\Gamma} = \int_{\Gamma }\rho_{\text{gas} }dV
\]
Se la trasformazione è quasi statica, lungo tutti gli stati intermedi della trasformazione il gas soddisfa l'equazione di stato. La pressione del gas potrà essere riscritta in base alla relazione $pV=nRT$.
\[
	\mathcal{L}_{\Gamma }= \int \frac{nRT}{V}dV
\]
C'è un modo più semplice per calcolare il lavoro compiuto dal gas quando si espande secondo una trasformazione quasi statica. Se il piano di Clapeyron mette in relazione pressione volume, il lavoro è rappresentato dall'area sottesa dalla trasformazione, con segno positivo o negativo a seconda di espansione o compressione.
Viceversa, se la trasformazione non è quasi statica e quindi gli stati intermedi durante la trasformazione non sono noti, non si può imporre la validità dell'equazione di stato. Si sfrutta il fatto che se il gas nel pistone si sta espandendo, il lavoro che compie sarà subito dall'ambiente circostante. Tutta l'atmosfera subisce il lavoro, essa si trova ad una certa pressione e viene compressa. La pressione dell'atmosfera è sempre la stessa e quindi il lavoro sarà:
\[
	-\mathcal{L}_{est} = - p_{est}\Delta V
\]
Questo è un metodo molto efficace per calcolare il lavoro quando la trasformazione non è reversibile.

In seguito una classifica di alcune delle trasformazioni più importanti che si incontreranno.

\paragraph{Trasformazione isoterma} È una trasformazione che avviene a temperatura costante.
\[
	T=\text{costante} \implies pV=\text{costante}
\]
$pV=$cost rappresenta un'iperbole.
\begin{figure}[htpb]
	\centering
	

	\tikzset{every picture/.style={line width=0.75pt}} %set default line width to 0.75pt        

	\begin{tikzpicture}[x=0.75pt,y=0.75pt,yscale=-1,xscale=1]
	%uncomment if require: \path (0,300); %set diagram left start at 0, and has height of 300

	%Shape: Axis 2D [id:dp3762397448703849] 
	\draw  (128.5,214.82) -- (358.5,214.82)(139.52,83) -- (139.52,229) (351.5,209.82) -- (358.5,214.82) -- (351.5,219.82) (134.52,90) -- (139.52,83) -- (144.52,90)  ;
	%Straight Lines [id:da7833866831656486] 
	\draw    (257.49,113.06) -- (184.75,182) ;
	\draw [shift={(259.67,111)}, rotate = 136.54] [fill={rgb, 255:red, 0; green, 0; blue, 0 }  ][line width=0.08]  [draw opacity=0] (10.72,-5.15) -- (0,0) -- (10.72,5.15) -- (7.12,0) -- cycle    ;
	%Curve Lines [id:da08665820907179822] 
	\draw    (169.67,133) .. controls (177.67,169.67) and (195.67,192.33) .. (244.33,193.67) ;
	%Curve Lines [id:da14814853449635912] 
	\draw    (187.67,129.67) .. controls (195.67,166.33) and (195.67,177) .. (244.33,178.33) ;
	%Curve Lines [id:da8569087873264798] 
	\draw    (206.33,128.33) .. controls (207,154.33) and (214.33,163.67) .. (242.33,164.33) ;

	% Text Node
	\draw (127,84.5) node    {$p$};
	% Text Node
	\draw (370.5,218) node    {$V$};
	% Text Node
	\draw (263,132.5) node    {$T$};


	\end{tikzpicture}
\end{figure}
\FloatBarrier
Questa, che porta dallo stato $A$ allo stato $B$ rappresenta la trasformazione isoterma di un gas, si va ad aumentare il volume a scapito della pressione che diminuisce. Se la si disegna continua la si intende reversibile, se la si fa tratteggiata è irreversibile. Aumentando la temperatura dell'isoterma, l'iperbole si sposta verso destra (si veda la figura). Tale osservazione aiuta molto a capire in altre trasformazioni se il gas si sta scaldando o raffreddando perché disegnare una famiglia di isoterme significa visualizzare sul piano di Clapeyron i luoghi dei punti a temperatura costante.

\paragraph{Trasformazione isobara} Si tratta di trasformazioni che avvengono a pressione costante. Si parte da uno stato $A$ e si arriva a uno stato $B$ tramite una retta. Il gas è tale per cui dallo stato $A$ passerà nella isoterma a sinistra, nello stato $B$ passerà per in isoterma più a destra, quindi il gas si scalda.
\begin{figure}[htpb]
	\centering
	

	\tikzset{every picture/.style={line width=0.75pt}} %set default line width to 0.75pt        

	\begin{tikzpicture}[x=0.75pt,y=0.75pt,yscale=-1,xscale=1]
	%uncomment if require: \path (0,300); %set diagram left start at 0, and has height of 300

	%Shape: Axis 2D [id:dp33331390425563545] 
	\draw  (128.5,214.82) -- (358.5,214.82)(139.52,83) -- (139.52,229) (351.5,209.82) -- (358.5,214.82) -- (351.5,219.82) (134.52,90) -- (139.52,83) -- (144.52,90)  ;
	%Straight Lines [id:da8859151290369149] 
	\draw    (323.25,141.5) -- (179.75,141.5) ;
	\draw [shift={(251.5,141.5)}, rotate = 180] [fill={rgb, 255:red, 0; green, 0; blue, 0 }  ][line width=0.08]  [draw opacity=0] (10.72,-5.15) -- (0,0) -- (10.72,5.15) -- (7.12,0) -- cycle    ;
	%Shape: Circle [id:dp5063629402975616] 
	\draw  [fill={rgb, 255:red, 0; green, 0; blue, 0 }  ,fill opacity=1 ] (177.25,141.5) .. controls (177.25,140.12) and (178.37,139) .. (179.75,139) .. controls (181.13,139) and (182.25,140.12) .. (182.25,141.5) .. controls (182.25,142.88) and (181.13,144) .. (179.75,144) .. controls (178.37,144) and (177.25,142.88) .. (177.25,141.5) -- cycle ;
	%Shape: Circle [id:dp2755167660691369] 
	\draw  [fill={rgb, 255:red, 0; green, 0; blue, 0 }  ,fill opacity=1 ] (320.75,141.5) .. controls (320.75,140.12) and (321.87,139) .. (323.25,139) .. controls (324.63,139) and (325.75,140.12) .. (325.75,141.5) .. controls (325.75,142.88) and (324.63,144) .. (323.25,144) .. controls (321.87,144) and (320.75,142.88) .. (320.75,141.5) -- cycle ;

	% Text Node
	\draw (127,84.5) node    {$p$};
	% Text Node
	\draw (370.5,218) node    {$V$};
	% Text Node
	\draw (176.5,157) node    {$A$};
	% Text Node
	\draw (322.5,155.5) node    {$B$};


	\end{tikzpicture}
\end{figure}
\FloatBarrier

\paragraph{Trasformazione isocora} Si tratta di trasformazioni che avvengono a volume costante e sono graficate da una retta. Aumentando la pressione, aiutandosi sempre con i grafici delle isoterme, si vede che la temperatura aumenta e quindi il gas si riscalda (\emph{riscaldamento isocoro}).
\begin{figure}[htpb]
	\centering
	

	\tikzset{every picture/.style={line width=0.75pt}} %set default line width to 0.75pt        

	\begin{tikzpicture}[x=0.75pt,y=0.75pt,yscale=-1,xscale=1]
	%uncomment if require: \path (0,300); %set diagram left start at 0, and has height of 300

	%Shape: Axis 2D [id:dp9666306813741952] 
	\draw  (128.5,214.91) -- (358.5,214.91)(139.52,84) -- (139.52,229) (351.5,209.91) -- (358.5,214.91) -- (351.5,219.91) (134.52,91) -- (139.52,84) -- (144.52,91)  ;
	%Straight Lines [id:da2063233018113888] 
	\draw    (242.2,109) -- (242.2,197.25) ;
	\draw [shift={(242.2,153.12)}, rotate = 90] [fill={rgb, 255:red, 0; green, 0; blue, 0 }  ][line width=0.08]  [draw opacity=0] (10.72,-5.15) -- (0,0) -- (10.72,5.15) -- (7.12,0) -- cycle    ;
	%Shape: Circle [id:dp9362939801432519] 
	\draw  [fill={rgb, 255:red, 0; green, 0; blue, 0 }  ,fill opacity=1 ] (242.21,199.75) .. controls (240.82,199.75) and (239.7,198.63) .. (239.7,197.25) .. controls (239.7,195.87) and (240.82,194.75) .. (242.2,194.75) .. controls (243.58,194.75) and (244.7,195.87) .. (244.7,197.25) .. controls (244.7,198.63) and (243.59,199.75) .. (242.21,199.75) -- cycle ;
	%Shape: Circle [id:dp5098416031796373] 
	\draw  [fill={rgb, 255:red, 0; green, 0; blue, 0 }  ,fill opacity=1 ] (242.21,111.5) .. controls (240.82,111.5) and (239.7,110.38) .. (239.7,109) .. controls (239.7,107.62) and (240.82,106.5) .. (242.2,106.5) .. controls (243.58,106.5) and (244.7,107.62) .. (244.7,109) .. controls (244.7,110.38) and (243.59,111.5) .. (242.21,111.5) -- cycle ;

	% Text Node
	\draw (127,84.5) node    {$p$};
	% Text Node
	\draw (370.5,218) node    {$V$};
	% Text Node
	\draw (225.17,195.67) node    {$A$};
	% Text Node
	\draw (227.17,106.17) node    {$B$};


	\end{tikzpicture}
\end{figure}
\FloatBarrier

\paragraph{Trasformazione adiabatica} Sono quelle trasformazioni che avvengono senza scambi di calore con l'ambiente. In esse non viene bloccata una costante ma si impone $Q=0$.

Queste ovviamente non sono le uniche trasformazione che può subire il gas.

\subsection{Espressione dell'energia interna di un gas perfetto}

\paragraph{Espansione libera di un gas perfetto eseguita da Joule} Si prende un recipiente con pareti rigide e adiabatiche, ossia indeformabili e perfettamente isolanti. Il recipiente è diviso in due parti da un setto rigido che ha un'apertura chiusa da un rubinetto. In una delle camere si inserisce una certa quantità di un gas perfetto.
Nella seconda camera viene fatto il vuoto, si aspira tutta l'aria che eventualmente era presente, così da eliminare qualunque molecola: la pressione in tale camera è $0 atm$. Si apre il rubinetto e si lascia espandere il gas contro il vuoto. \emph{Espansione libera} si riferisce proprio al fatto che l'espansione avviene contro il nulla. Sperimentalmente si può andare a misurare quali sono i valori di volume pressione e temperatura e si trova che, se il gas è perfetto, comunque sia fatto l'esperimento, la temperatura finale del gas è esattamente uguale a quella iniziale. Per tale trasformazione si scrive allora il primo principio della termodinamica.
\[
	Q-\mathcal{L}=0 \implies \Delta U=0
\]
Il sistema infatti non scambia energia con l'ambiente, essendo le pareti rigide e adiabatiche. In tale processo termodinamico l'energia interna del gas non varia. Quella accumulata nello stato iniziale è uguale a quella accumulata nello stato finale. Si deduce quindi che $U$ non è funzione della pressione e del volume perché è rimasta la stessa nonostante essi sian variati. Allora è evidente che l'energia interna in un gas perfetto è funzione della temperatura, perché $T$ non è variata e di conseguenza nemmeno $U$.
\begin{figure}[htpb]
	\centering
	

	\tikzset{every picture/.style={line width=0.75pt}} %set default line width to 0.75pt        

	\begin{tikzpicture}[x=0.75pt,y=0.75pt,yscale=-1,xscale=1]
	%uncomment if require: \path (0,300); %set diagram left start at 0, and has height of 300

	%Shape: Rectangle [id:dp684426269344268] 
	\draw  [line width=1.5]  (132,71) -- (339.5,71) -- (339.5,158) -- (132,158) -- cycle ;
	%Straight Lines [id:da3693322295005539] 
	\draw [line width=3.75]    (235,70) -- (235,158) ;
	%Shape: Circle [id:dp2789327757534068] 
	\draw  [draw opacity=0][fill={rgb, 255:red, 212; green, 212; blue, 212 }  ,fill opacity=1 ] (144.67,86.83) .. controls (144.67,81.95) and (148.62,78) .. (153.5,78) .. controls (158.38,78) and (162.33,81.95) .. (162.33,86.83) .. controls (162.33,91.71) and (158.38,95.67) .. (153.5,95.67) .. controls (148.62,95.67) and (144.67,91.71) .. (144.67,86.83) -- cycle ;
	%Shape: Circle [id:dp4620531125360525] 
	\draw  [draw opacity=0][fill={rgb, 255:red, 212; green, 212; blue, 212 }  ,fill opacity=1 ] (184.5,86.33) .. controls (184.5,81.45) and (188.45,77.5) .. (193.33,77.5) .. controls (198.21,77.5) and (202.17,81.45) .. (202.17,86.33) .. controls (202.17,91.21) and (198.21,95.17) .. (193.33,95.17) .. controls (188.45,95.17) and (184.5,91.21) .. (184.5,86.33) -- cycle ;
	%Shape: Circle [id:dp2905872442111941] 
	\draw  [draw opacity=0][fill={rgb, 255:red, 212; green, 212; blue, 212 }  ,fill opacity=1 ] (208.5,107.5) .. controls (208.5,102.62) and (212.45,98.67) .. (217.33,98.67) .. controls (222.21,98.67) and (226.17,102.62) .. (226.17,107.5) .. controls (226.17,112.38) and (222.21,116.33) .. (217.33,116.33) .. controls (212.45,116.33) and (208.5,112.38) .. (208.5,107.5) -- cycle ;
	%Shape: Circle [id:dp14827622187931477] 
	\draw  [draw opacity=0][fill={rgb, 255:red, 212; green, 212; blue, 212 }  ,fill opacity=1 ] (150.5,112.67) .. controls (150.5,107.79) and (154.45,103.83) .. (159.33,103.83) .. controls (164.21,103.83) and (168.17,107.79) .. (168.17,112.67) .. controls (168.17,117.55) and (164.21,121.5) .. (159.33,121.5) .. controls (154.45,121.5) and (150.5,117.55) .. (150.5,112.67) -- cycle ;
	%Shape: Circle [id:dp02711217543111255] 
	\draw  [draw opacity=0][fill={rgb, 255:red, 212; green, 212; blue, 212 }  ,fill opacity=1 ] (167.17,100.33) .. controls (167.17,95.45) and (171.12,91.5) .. (176,91.5) .. controls (180.88,91.5) and (184.83,95.45) .. (184.83,100.33) .. controls (184.83,105.21) and (180.88,109.17) .. (176,109.17) .. controls (171.12,109.17) and (167.17,105.21) .. (167.17,100.33) -- cycle ;
	%Shape: Circle [id:dp09428837669637646] 
	\draw  [draw opacity=0][fill={rgb, 255:red, 212; green, 212; blue, 212 }  ,fill opacity=1 ] (185.17,115.33) .. controls (185.17,110.45) and (189.12,106.5) .. (194,106.5) .. controls (198.88,106.5) and (202.83,110.45) .. (202.83,115.33) .. controls (202.83,120.21) and (198.88,124.17) .. (194,124.17) .. controls (189.12,124.17) and (185.17,120.21) .. (185.17,115.33) -- cycle ;
	%Shape: Rectangle [id:dp45584182924648853] 
	\draw  [line width=1.5]  (392,71) -- (599.5,71) -- (599.5,158) -- (392,158) -- cycle ;
	%Shape: Circle [id:dp23771206303308867] 
	\draw  [draw opacity=0][fill={rgb, 255:red, 212; green, 212; blue, 212 }  ,fill opacity=1 ] (404.67,86.83) .. controls (404.67,81.95) and (408.62,78) .. (413.5,78) .. controls (418.38,78) and (422.33,81.95) .. (422.33,86.83) .. controls (422.33,91.71) and (418.38,95.67) .. (413.5,95.67) .. controls (408.62,95.67) and (404.67,91.71) .. (404.67,86.83) -- cycle ;
	%Shape: Circle [id:dp015976345398689862] 
	\draw  [draw opacity=0][fill={rgb, 255:red, 212; green, 212; blue, 212 }  ,fill opacity=1 ] (474,96.33) .. controls (474,91.45) and (477.95,87.5) .. (482.83,87.5) .. controls (487.71,87.5) and (491.67,91.45) .. (491.67,96.33) .. controls (491.67,101.21) and (487.71,105.17) .. (482.83,105.17) .. controls (477.95,105.17) and (474,101.21) .. (474,96.33) -- cycle ;
	%Shape: Circle [id:dp7875101498064774] 
	\draw  [draw opacity=0][fill={rgb, 255:red, 212; green, 212; blue, 212 }  ,fill opacity=1 ] (551,132) .. controls (551,127.12) and (554.95,123.17) .. (559.83,123.17) .. controls (564.71,123.17) and (568.67,127.12) .. (568.67,132) .. controls (568.67,136.88) and (564.71,140.83) .. (559.83,140.83) .. controls (554.95,140.83) and (551,136.88) .. (551,132) -- cycle ;
	%Shape: Circle [id:dp33350367514751666] 
	\draw  [draw opacity=0][fill={rgb, 255:red, 212; green, 212; blue, 212 }  ,fill opacity=1 ] (410.5,112.67) .. controls (410.5,107.79) and (414.45,103.83) .. (419.33,103.83) .. controls (424.21,103.83) and (428.17,107.79) .. (428.17,112.67) .. controls (428.17,117.55) and (424.21,121.5) .. (419.33,121.5) .. controls (414.45,121.5) and (410.5,117.55) .. (410.5,112.67) -- cycle ;
	%Shape: Circle [id:dp8197882534729368] 
	\draw  [draw opacity=0][fill={rgb, 255:red, 212; green, 212; blue, 212 }  ,fill opacity=1 ] (444.67,130.33) .. controls (444.67,125.45) and (448.62,121.5) .. (453.5,121.5) .. controls (458.38,121.5) and (462.33,125.45) .. (462.33,130.33) .. controls (462.33,135.21) and (458.38,139.17) .. (453.5,139.17) .. controls (448.62,139.17) and (444.67,135.21) .. (444.67,130.33) -- cycle ;
	%Shape: Circle [id:dp2785467821873784] 
	\draw  [draw opacity=0][fill={rgb, 255:red, 212; green, 212; blue, 212 }  ,fill opacity=1 ] (499.17,127.33) .. controls (499.17,122.45) and (503.12,118.5) .. (508,118.5) .. controls (512.88,118.5) and (516.83,122.45) .. (516.83,127.33) .. controls (516.83,132.21) and (512.88,136.17) .. (508,136.17) .. controls (503.12,136.17) and (499.17,132.21) .. (499.17,127.33) -- cycle ;
	%Shape: Circle [id:dp2617671292101982] 
	\draw  [draw opacity=0][fill={rgb, 255:red, 212; green, 212; blue, 212 }  ,fill opacity=1 ] (561.5,101.5) .. controls (561.5,96.62) and (565.45,92.67) .. (570.33,92.67) .. controls (575.21,92.67) and (579.17,96.62) .. (579.17,101.5) .. controls (579.17,106.38) and (575.21,110.33) .. (570.33,110.33) .. controls (565.45,110.33) and (561.5,106.38) .. (561.5,101.5) -- cycle ;
	%Shape: Circle [id:dp2213669106831615] 
	\draw  [draw opacity=0][fill={rgb, 255:red, 212; green, 212; blue, 212 }  ,fill opacity=1 ] (509.5,88.5) .. controls (509.5,83.62) and (513.45,79.67) .. (518.33,79.67) .. controls (523.21,79.67) and (527.17,83.62) .. (527.17,88.5) .. controls (527.17,93.38) and (523.21,97.33) .. (518.33,97.33) .. controls (513.45,97.33) and (509.5,93.38) .. (509.5,88.5) -- cycle ;
	%Shape: Circle [id:dp6842733419109637] 
	\draw  [draw opacity=0][fill={rgb, 255:red, 212; green, 212; blue, 212 }  ,fill opacity=1 ] (509.5,88.5) .. controls (509.5,83.62) and (513.45,79.67) .. (518.33,79.67) .. controls (523.21,79.67) and (527.17,83.62) .. (527.17,88.5) .. controls (527.17,93.38) and (523.21,97.33) .. (518.33,97.33) .. controls (513.45,97.33) and (509.5,93.38) .. (509.5,88.5) -- cycle ;
	%Straight Lines [id:da2513436004439846] 
	\draw    (349,114) -- (381.5,114) ;
	\draw [shift={(384.5,114)}, rotate = 180] [fill={rgb, 255:red, 0; green, 0; blue, 0 }  ][line width=0.08]  [draw opacity=0] (10.72,-5.15) -- (0,0) -- (10.72,5.15) -- (7.12,0) -- cycle    ;

	% Text Node
	\draw (183.5,138.5) node    {$n,p_{0} ,V_{0} ,T_{0}$};
	% Text Node
	\draw (286.5,115) node    {$p=0\ atm$};


	\end{tikzpicture}
\end{figure}
\FloatBarrier
Visto questo esempio, che ha portato a dedurre che $U=f(T)$, si considerino nel piano di Clapeyron due stati. Si vuole calcolare la variazione di energia interna fra essi. Lo si può fare lungo una qualsiasi trasformazione perché $U$ è una funzione di stato, quindi $\Delta U$ sarà sempre la stessa. Si sceglie quindi una trasformazione semplice che dallo stato finale fa comprimere il gas fino a che non arriva al volume dello stato iniziale.
\begin{figure}[htpb]
	\centering
	

	\tikzset{every picture/.style={line width=0.75pt}} %set default line width to 0.75pt        

	\begin{tikzpicture}[x=0.75pt,y=0.75pt,yscale=-1,xscale=1]
	%uncomment if require: \path (0,300); %set diagram left start at 0, and has height of 300

	%Shape: Axis 2D [id:dp26852410186154585] 
	\draw  (128.5,213.85) -- (358.5,213.85)(139.52,73) -- (139.52,229) (351.5,208.85) -- (358.5,213.85) -- (351.5,218.85) (134.52,80) -- (139.52,73) -- (144.52,80)  ;
	%Straight Lines [id:da33150538466512813] 
	\draw    (213,89.67) -- (213,197.67) ;
	\draw [shift={(213,143.67)}, rotate = 90] [fill={rgb, 255:red, 0; green, 0; blue, 0 }  ][line width=0.08]  [draw opacity=0] (10.72,-5.15) -- (0,0) -- (10.72,5.15) -- (7.12,0) -- cycle    ;
	%Curve Lines [id:da23260166306084673] 
	\draw    (213,89.67) .. controls (221,126.33) and (239,149) .. (287.67,150.33) ;
	\draw [shift={(237.96,135.66)}, rotate = 221.21] [fill={rgb, 255:red, 0; green, 0; blue, 0 }  ][line width=0.08]  [draw opacity=0] (10.72,-5.15) -- (0,0) -- (10.72,5.15) -- (7.12,0) -- cycle    ;
	%Shape: Circle [id:dp7685148264751998] 
	\draw  [fill={rgb, 255:red, 0; green, 0; blue, 0 }  ,fill opacity=1 ] (213,200.17) .. controls (211.62,200.17) and (210.5,199.05) .. (210.5,197.67) .. controls (210.5,196.29) and (211.62,195.17) .. (213,195.17) .. controls (214.38,195.17) and (215.5,196.28) .. (215.5,197.67) .. controls (215.5,199.05) and (214.38,200.17) .. (213,200.17) -- cycle ;
	%Shape: Circle [id:dp6135691859180823] 
	\draw  [fill={rgb, 255:red, 0; green, 0; blue, 0 }  ,fill opacity=1 ] (213,92.17) .. controls (211.62,92.17) and (210.5,91.05) .. (210.5,89.67) .. controls (210.5,88.29) and (211.62,87.17) .. (213,87.17) .. controls (214.38,87.17) and (215.5,88.28) .. (215.5,89.67) .. controls (215.5,91.05) and (214.38,92.17) .. (213,92.17) -- cycle ;
	%Shape: Circle [id:dp757889220486204] 
	\draw  [fill={rgb, 255:red, 0; green, 0; blue, 0 }  ,fill opacity=1 ] (287.67,152.83) .. controls (286.29,152.83) and (285.17,151.72) .. (285.17,150.33) .. controls (285.17,148.95) and (286.28,147.83) .. (287.67,147.83) .. controls (289.05,147.83) and (290.17,148.95) .. (290.17,150.33) .. controls (290.17,151.71) and (289.05,152.83) .. (287.67,152.83) -- cycle ;

	% Text Node
	\draw (127,74.5) node    {$p$};
	% Text Node
	\draw (370.5,218) node    {$V$};
	% Text Node
	\draw (301,143.17) node    {$B$};
	% Text Node
	\draw (199,195.17) node    {$A$};
	% Text Node
	\draw (201.67,87.83) node    {$C$};


	\end{tikzpicture}
\end{figure}
\FloatBarrier
Questo stato lo si chiama $C$. Poi si riscalda il gas in maniera isocora, cioè a volume costante. Fondamentalmente si parte dallo stato iniziale $A$, si riscaldo il gas fino allo stato $C$, e con una isoterma si arriva a $B$.
Si può quindi vedere la variazione dell'energia interna come:
\[
	\Delta U = U(B)-U(A) = [U(C)-U(A)] + [U(B)-U(C)]
\]
La seconda variazione di energia interna vale zero perché la temperatura in $B$ e in $C$ è la stessa. Quindi rimane soltanto da calcolare il primo termine. Da $A$ a $C$ non c'è lavoro compiuto dal gas perché non c'è variazione di pressione. Si calcola il calore scambiato sfruttando la definizione di calore specifico molare, con queste informazioni:
\[
	\Delta U_{A\to C} = Q_{\text{isocora} } - \mathcal{L}_{\text{isocora} } = nc_v(T_C-T_A  )
\]
$T_C$ è uguale a $T_B$ visto che la seconda trasformazione è isoterma. Si ottiene che la variazione di energia interna dallo stato iniziale $A$ allo stato finale $B$ è esprimibile come:
\[
	\Delta U = nc_v(T_B-T_A  )
\]
La variazione di energia interna per un gas perfetto sarà sempre la stessa. Questa espressione vale per qualunque trasformazione.

\paragraph{Lavoro in un'isoterma} Si immagini di far espandere il gas lungo una certa trasformazione isoterma. Si ha un recipiente contenente un certo numero di moli ad una certa pressione, volume e temperatura. Il contenitore è mantenuto a contatto con un serbatoio di calore la cui temperatura è sempre $T_0$ e anche se cede o assorbe calore non la varia. Sul pistone mobile è appoggiata una certa massa $m$, per cui il gas inizialmente deve essere alla stessa pressione che sta all'esterno. Essa è la pressione atmosferica più la pressione data dalla massa. Mantenendo il contatto con il serbatoio si rimuove la massa. Il gas si espande perché deve andare in equilibrio con la pressione esterna,  atmosferica.
\begin{figure}[htpb]
	\centering
	

	% Pattern Info
	 
	\tikzset{
	pattern size/.store in=\mcSize, 
	pattern size = 5pt,
	pattern thickness/.store in=\mcThickness, 
	pattern thickness = 0.3pt,
	pattern radius/.store in=\mcRadius, 
	pattern radius = 1pt}
	\makeatletter
	\pgfutil@ifundefined{pgf@pattern@name@_1xlpa3eg8}{
	\pgfdeclarepatternformonly[\mcThickness,\mcSize]{_1xlpa3eg8}
	{\pgfqpoint{0pt}{-\mcThickness}}
	{\pgfpoint{\mcSize}{\mcSize}}
	{\pgfpoint{\mcSize}{\mcSize}}
	{
	\pgfsetcolor{\tikz@pattern@color}
	\pgfsetlinewidth{\mcThickness}
	\pgfpathmoveto{\pgfqpoint{0pt}{\mcSize}}
	\pgfpathlineto{\pgfpoint{\mcSize+\mcThickness}{-\mcThickness}}
	\pgfusepath{stroke}
	}}
	\makeatother

	% Pattern Info
	 
	\tikzset{
	pattern size/.store in=\mcSize, 
	pattern size = 5pt,
	pattern thickness/.store in=\mcThickness, 
	pattern thickness = 0.3pt,
	pattern radius/.store in=\mcRadius, 
	pattern radius = 1pt}
	\makeatletter
	\pgfutil@ifundefined{pgf@pattern@name@_7ausj6b50}{
	\pgfdeclarepatternformonly[\mcThickness,\mcSize]{_7ausj6b50}
	{\pgfqpoint{0pt}{-\mcThickness}}
	{\pgfpoint{\mcSize}{\mcSize}}
	{\pgfpoint{\mcSize}{\mcSize}}
	{
	\pgfsetcolor{\tikz@pattern@color}
	\pgfsetlinewidth{\mcThickness}
	\pgfpathmoveto{\pgfqpoint{0pt}{\mcSize}}
	\pgfpathlineto{\pgfpoint{\mcSize+\mcThickness}{-\mcThickness}}
	\pgfusepath{stroke}
	}}
	\makeatother
	\tikzset{every picture/.style={line width=0.75pt}} %set default line width to 0.75pt        

	\begin{tikzpicture}[x=0.75pt,y=0.75pt,yscale=-1,xscale=1]
	%uncomment if require: \path (0,300); %set diagram left start at 0, and has height of 300

	%Straight Lines [id:da09277888142735424] 
	\draw    (130,100) -- (130,200) ;
	%Straight Lines [id:da4610100840831337] 
	\draw    (220,100) -- (220,200) ;
	%Straight Lines [id:da7210267727104362] 
	\draw    (130,200) -- (220,200) ;
	%Straight Lines [id:da2693208501217441] 
	\draw [line width=2.25]    (130,141) -- (220,141) ;
	%Shape: Circle [id:dp1542212791327302] 
	\draw  [draw opacity=0][fill={rgb, 255:red, 212; green, 212; blue, 212 }  ,fill opacity=1 ] (158.5,156.13) .. controls (158.5,151.64) and (162.14,148) .. (166.63,148) .. controls (171.11,148) and (174.75,151.64) .. (174.75,156.13) .. controls (174.75,160.61) and (171.11,164.25) .. (166.63,164.25) .. controls (162.14,164.25) and (158.5,160.61) .. (158.5,156.13) -- cycle ;
	%Shape: Circle [id:dp8517752557618017] 
	\draw  [draw opacity=0][fill={rgb, 255:red, 212; green, 212; blue, 212 }  ,fill opacity=1 ] (184,153.13) .. controls (184,148.64) and (187.64,145) .. (192.13,145) .. controls (196.61,145) and (200.25,148.64) .. (200.25,153.13) .. controls (200.25,157.61) and (196.61,161.25) .. (192.13,161.25) .. controls (187.64,161.25) and (184,157.61) .. (184,153.13) -- cycle ;
	%Shape: Circle [id:dp415834616408854] 
	\draw  [draw opacity=0][fill={rgb, 255:red, 212; green, 212; blue, 212 }  ,fill opacity=1 ] (140.5,167.63) .. controls (140.5,163.14) and (144.14,159.5) .. (148.63,159.5) .. controls (153.11,159.5) and (156.75,163.14) .. (156.75,167.63) .. controls (156.75,172.11) and (153.11,175.75) .. (148.63,175.75) .. controls (144.14,175.75) and (140.5,172.11) .. (140.5,167.63) -- cycle ;
	%Shape: Circle [id:dp9194075293630759] 
	\draw  [draw opacity=0][fill={rgb, 255:red, 212; green, 212; blue, 212 }  ,fill opacity=1 ] (174,170.63) .. controls (174,166.14) and (177.64,162.5) .. (182.13,162.5) .. controls (186.61,162.5) and (190.25,166.14) .. (190.25,170.63) .. controls (190.25,175.11) and (186.61,178.75) .. (182.13,178.75) .. controls (177.64,178.75) and (174,175.11) .. (174,170.63) -- cycle ;
	%Shape: Circle [id:dp5830961561003043] 
	\draw  [draw opacity=0][fill={rgb, 255:red, 212; green, 212; blue, 212 }  ,fill opacity=1 ] (139,187.63) .. controls (139,183.14) and (142.64,179.5) .. (147.13,179.5) .. controls (151.61,179.5) and (155.25,183.14) .. (155.25,187.63) .. controls (155.25,192.11) and (151.61,195.75) .. (147.13,195.75) .. controls (142.64,195.75) and (139,192.11) .. (139,187.63) -- cycle ;
	%Shape: Circle [id:dp17728094173428888] 
	\draw  [draw opacity=0][fill={rgb, 255:red, 212; green, 212; blue, 212 }  ,fill opacity=1 ] (199.5,164.63) .. controls (199.5,160.14) and (203.14,156.5) .. (207.63,156.5) .. controls (212.11,156.5) and (215.75,160.14) .. (215.75,164.63) .. controls (215.75,169.11) and (212.11,172.75) .. (207.63,172.75) .. controls (203.14,172.75) and (199.5,169.11) .. (199.5,164.63) -- cycle ;
	%Shape: Circle [id:dp6783241133698021] 
	\draw  [draw opacity=0][fill={rgb, 255:red, 212; green, 212; blue, 212 }  ,fill opacity=1 ] (159.5,187.63) .. controls (159.5,183.14) and (163.14,179.5) .. (167.63,179.5) .. controls (172.11,179.5) and (175.75,183.14) .. (175.75,187.63) .. controls (175.75,192.11) and (172.11,195.75) .. (167.63,195.75) .. controls (163.14,195.75) and (159.5,192.11) .. (159.5,187.63) -- cycle ;
	%Shape: Circle [id:dp8098719490652282] 
	\draw  [draw opacity=0][fill={rgb, 255:red, 212; green, 212; blue, 212 }  ,fill opacity=1 ] (191,185.13) .. controls (191,180.64) and (194.64,177) .. (199.13,177) .. controls (203.61,177) and (207.25,180.64) .. (207.25,185.13) .. controls (207.25,189.61) and (203.61,193.25) .. (199.13,193.25) .. controls (194.64,193.25) and (191,189.61) .. (191,185.13) -- cycle ;
	%Shape: Rectangle [id:dp04965632327927327] 
	\draw  [draw opacity=0][fill={rgb, 255:red, 74; green, 74; blue, 74 }  ,fill opacity=1 ] (161.5,124.5) -- (188.25,124.5) -- (188.25,139.5) -- (161.5,139.5) -- cycle ;
	%Rounded Rect [id:dp5179782246702531] 
	\draw  [pattern=_1xlpa3eg8,pattern size=6pt,pattern thickness=0.75pt,pattern radius=0pt, pattern color={rgb, 255:red, 155; green, 155; blue, 155}] (120.25,204) .. controls (120.25,201.79) and (122.04,200) .. (124.25,200) -- (226.75,200) .. controls (228.96,200) and (230.75,201.79) .. (230.75,204) -- (230.75,216) .. controls (230.75,218.21) and (228.96,220) .. (226.75,220) -- (124.25,220) .. controls (122.04,220) and (120.25,218.21) .. (120.25,216) -- cycle ;
	%Straight Lines [id:da5802141561376881] 
	\draw    (260,100) -- (260,200) ;
	%Straight Lines [id:da19175209152388706] 
	\draw    (350,100) -- (350,200) ;
	%Straight Lines [id:da12090357925191442] 
	\draw    (260,200) -- (350,200) ;
	%Straight Lines [id:da4269274783694801] 
	\draw [line width=2.25]    (260,107) -- (350,107) ;
	%Shape: Circle [id:dp831118661671322] 
	\draw  [draw opacity=0][fill={rgb, 255:red, 212; green, 212; blue, 212 }  ,fill opacity=1 ] (266.5,123.46) .. controls (266.5,118.97) and (270.14,115.33) .. (274.63,115.33) .. controls (279.11,115.33) and (282.75,118.97) .. (282.75,123.46) .. controls (282.75,127.95) and (279.11,131.58) .. (274.63,131.58) .. controls (270.14,131.58) and (266.5,127.95) .. (266.5,123.46) -- cycle ;
	%Shape: Circle [id:dp543457889317305] 
	\draw  [draw opacity=0][fill={rgb, 255:red, 212; green, 212; blue, 212 }  ,fill opacity=1 ] (298.33,124.79) .. controls (298.33,120.3) and (301.97,116.67) .. (306.46,116.67) .. controls (310.95,116.67) and (314.58,120.3) .. (314.58,124.79) .. controls (314.58,129.28) and (310.95,132.92) .. (306.46,132.92) .. controls (301.97,132.92) and (298.33,129.28) .. (298.33,124.79) -- cycle ;
	%Shape: Circle [id:dp7341469396289795] 
	\draw  [draw opacity=0][fill={rgb, 255:red, 212; green, 212; blue, 212 }  ,fill opacity=1 ] (280.5,150.96) .. controls (280.5,146.47) and (284.14,142.83) .. (288.63,142.83) .. controls (293.11,142.83) and (296.75,146.47) .. (296.75,150.96) .. controls (296.75,155.45) and (293.11,159.08) .. (288.63,159.08) .. controls (284.14,159.08) and (280.5,155.45) .. (280.5,150.96) -- cycle ;
	%Shape: Circle [id:dp3081150642569459] 
	\draw  [draw opacity=0][fill={rgb, 255:red, 212; green, 212; blue, 212 }  ,fill opacity=1 ] (322,135.29) .. controls (322,130.8) and (325.64,127.17) .. (330.13,127.17) .. controls (334.61,127.17) and (338.25,130.8) .. (338.25,135.29) .. controls (338.25,139.78) and (334.61,143.42) .. (330.13,143.42) .. controls (325.64,143.42) and (322,139.78) .. (322,135.29) -- cycle ;
	%Shape: Circle [id:dp8597573232998028] 
	\draw  [draw opacity=0][fill={rgb, 255:red, 212; green, 212; blue, 212 }  ,fill opacity=1 ] (269,187.63) .. controls (269,183.14) and (272.64,179.5) .. (277.13,179.5) .. controls (281.61,179.5) and (285.25,183.14) .. (285.25,187.63) .. controls (285.25,192.11) and (281.61,195.75) .. (277.13,195.75) .. controls (272.64,195.75) and (269,192.11) .. (269,187.63) -- cycle ;
	%Shape: Circle [id:dp25298104539020194] 
	\draw  [draw opacity=0][fill={rgb, 255:red, 212; green, 212; blue, 212 }  ,fill opacity=1 ] (291.5,172.63) .. controls (291.5,168.14) and (295.14,164.5) .. (299.62,164.5) .. controls (304.11,164.5) and (307.75,168.14) .. (307.75,172.63) .. controls (307.75,177.11) and (304.11,180.75) .. (299.62,180.75) .. controls (295.14,180.75) and (291.5,177.11) .. (291.5,172.63) -- cycle ;
	%Shape: Circle [id:dp21127649154398997] 
	\draw  [draw opacity=0][fill={rgb, 255:red, 212; green, 212; blue, 212 }  ,fill opacity=1 ] (316.83,158.96) .. controls (316.83,154.47) and (320.47,150.83) .. (324.96,150.83) .. controls (329.45,150.83) and (333.08,154.47) .. (333.08,158.96) .. controls (333.08,163.45) and (329.45,167.08) .. (324.96,167.08) .. controls (320.47,167.08) and (316.83,163.45) .. (316.83,158.96) -- cycle ;
	%Shape: Circle [id:dp37293066534985364] 
	\draw  [draw opacity=0][fill={rgb, 255:red, 212; green, 212; blue, 212 }  ,fill opacity=1 ] (321,185.13) .. controls (321,180.64) and (324.64,177) .. (329.13,177) .. controls (333.61,177) and (337.25,180.64) .. (337.25,185.13) .. controls (337.25,189.61) and (333.61,193.25) .. (329.13,193.25) .. controls (324.64,193.25) and (321,189.61) .. (321,185.13) -- cycle ;
	%Rounded Rect [id:dp8417795589498158] 
	\draw  [pattern=_7ausj6b50,pattern size=6pt,pattern thickness=0.75pt,pattern radius=0pt, pattern color={rgb, 255:red, 155; green, 155; blue, 155}] (250.25,204) .. controls (250.25,201.79) and (252.04,200) .. (254.25,200) -- (356.75,200) .. controls (358.96,200) and (360.75,201.79) .. (360.75,204) -- (360.75,216) .. controls (360.75,218.21) and (358.96,220) .. (356.75,220) -- (254.25,220) .. controls (252.04,220) and (250.25,218.21) .. (250.25,216) -- cycle ;

	% Text Node
	\draw (175,230) node   [align=left] {serbatoio};
	% Text Node
	\draw (305,230) node   [align=left] {serbatoio};


	\end{tikzpicture}
\end{figure}
\FloatBarrier
Il volume sarà cambiato, il numero di moli è lo stesso, la temperatura è sempre $T_0$. Conoscendo lo stato iniziale e finale, bisogna capire se il sistema scambia energia con l'ambiente esterno. Si applica il primo principio della termodinamica:
\[
	\Delta U = Q_{\text{isoterma} }- \mathcal{L}_{\text{isoterma} } \quad \Delta U = 0 \implies \underbrace{Q_{\text{isoterma} }}_{>0} = \underbrace{\mathcal{L}_{\text{isoterma} }}_{>0}
\]
L'energia interna del sistema non varia perché dipende solo dalla temperatura. Il calore scambiato deve essere pari al lavoro compiuto. Durante la trasformazione, per potersi espandere, il gas deve assorbire calore dal serbatoio e convertirlo in lavoro meccanico di espansione. Sul calcolo del lavoro si distinguono i casi a seconda che la trasformazione venga fatta in maniera reversibile o irreversibile, perché il calore assorbito sarà diverso.
Si supponga che la trasformazione venga fatta in maniera reversibile, per passaggi continui dallo stato di equilibrio. Il metodo è eliminare tutti gli attriti in modo che quando il pistone scorre verso l'alto non dissipa energia. Bisogna poi immaginare la massa $m$ come un sacchetto di sabbia, dal quale si toglie un granello alla volta. Se si fa avvenire la trasformazione molto lentamente, il gas si assesta in condizioni di equilibrio molto vicine. Questo è un modo per far avvenire la trasformazione in maniera quai statica.
\[
	\mathcal{L}_{\text{isoterma} }= \int\rho_{\text{gas} }dV = \int \frac{nRT_0 }{V}dV = nRT_0\,\log \left( \frac{V_f}{V_i} \right)
\]
La trasformazione irreversibile si ha quando si rimuove il blocco in un colpo. L'istante in cui inizia la trasformazione, è già quello in cui il blocco è rimosso. Non si può conoscere la pressione del gas negli stati intermedi, quindi si calcola il lavoro come l'opposto di quello compiuto sull'ambiente esterno. Dall'istante iniziale la pressione atmosferica subisce una compressione da parte del gas. Si può calcolare il lavoro del gas come quello subito dall'ambiente esterno:
\[
	\mathcal{L}_{\text{isoterma} } =\int \rho_{\text{gas} }dV = -\mathcal{L}_{est} = p_{est}\Delta V = p_{atm}\Delta V_{\text{gas} }    	
\]

\paragraph{Lavoro in una isocora} Si hanno $n$ moli di gas, in un recipiente a contatto con un serbatoio il cui tappo è bloccato. Si cambia serbatoio con uno a temperatura maggiore. Il gas piano piano va in equilibrio termico con esso. Il volume non sarà cambiato (nessun lavoro compiuto) ma ci si aspetta che la pressione lo sia. Il gas aumenta la sua energia interna grazie al fatto che assorbe calore dal serbatoio.
\[Q-\mathcal{L}=Q=\Delta U\]
Lo scambio energetico non dipende dal fatto che la trasformazione sia fatta in modo reversibile o irreversibile, perché non viene compiuto lavoro, quindi il calore scambiato sarà sempre lo stesso.

\paragraph{Lavoro in un'isobara} Si riscalda un gas a pressione costante. Esso si trova in equilibrio con la pressione esterna $p_0$ e con la temperatura del serbatoio. Si vede quali sono i contributi energetici lungo la trasformazione. Si può scrivere che:
\begin{align*}
	Q_{\text{isobara}} &= \Delta U + \mathcal{L}_{\text{isobara}} \\
		&= \Delta U + p_0\Delta V \\
		&= nc_v\Delta T + \underbrace{p_0\Delta V}_{=nR\Delta T} = n\Delta T\,c_p \tag*{(in un'isobara)} \\
		&= nc_v\Delta T + nR\Delta T = n\Delta T c_p \\
		& \qquad \implies \boxed{c_v + R = c_p  } \\
\end{align*}
Si ottiene una relazione che mostra come il calore specifico a pressione costante sarà sempre maggiore di quello a volume costante. Infatti a pressione costante il gas usa parte del calore per compiere lavoro.

Da tale relazione si ricava un significato energetico per la costante dei gas ideali $R$: essa rappresenta il lavoro che a pressione costante compie $1 mol$.
\[
	\boxed{c_v + R = c_p  }  \qquad \text{relazione di Mayer}
\]

\paragraph{Trasformazioni adiabatiche reversibili} Si ha un gas perfetto contenuto in un recipiente in cui tutte le pareti sono adiabatiche, non permettono passaggio di calore all'esterno. Il gas si trova alla pressione $p_0$ e alla temperatura $T_0$. Si dovrà immaginare di poter rimuovere un ipotetica massa sopra il pistone molto lentamente. Alla fine si avrà che il gas è andato in equilibrio con la pressione atmosferica. Il volume e la temperatura saranno cambiati. Espandendosi si avrà una diminuzione della pressione e una variazione della temperatura. Tutte e tre le variabili termodinamiche cambiamo (caratteristica tipica delle adiabatiche). La pressione finale è tale da bilanciare quella esterna. Rimangono quindi due variabili termodinamiche da calcolare. Una la dà l'equazione di stato. Manca ancora una relazione per ricavare la terza variabile, si sfrutta quindi il primo principio della termodinamica:
\[
	\Delta U = \underbrace{Q}_{=0} - \mathcal{L}
\]
A fronte di un lavoro di espansione positivo fatto dal gas, ci si deve aspettare una diminuzione dell'energia interna accumulata e quindi un raffreddamento del gas. Si scrive il primo principio della termodinamica in forma infinitesima:
\[
	dU = dQ - d\mathcal{L}  \implies dU + d\mathcal{L} = 0
\]
La variazione infinitesima di energia interna sarà quindi $n c_v dT$. Il lavoro fatto dal gas sarà $p\,dV$. Si potrà inserire la relazione che lega la pressione con le altre variabili.
\begin{align*}
	n c_v dT + n\overbrace{R}^{c_p-c_v}T \frac{dV}{V} &= 0 \\
	\frac{c_v dT}{T} + (c_p-c_v)\frac{dV}{V} &= 0 \tag*{dividendo per $nT$} \\
	\frac{dT}{T} + \left(\frac{c_p}{c_v}-1\right)\frac{dV}{V} &= 0 \tag*{dividendo per $c_v$}\\
	\frac{dT}{T} &= \left( 1-\frac{c_p }{c_v} \right) \frac{dV}{V} \\
	\frac{dT}{T} &= \left( 1- \gamma \right) \frac{dV}{V} \\
	\int_{T_0 }^T \frac{dT}{T} &= \int_{V_0 }^V \left( 1- \gamma \right) \frac{dV}{V} \tag*{integrando}\\
\end{align*}
\begin{align*}
	\log \frac{T}{T_0} &= (1-\gamma) \log \frac{V}{V_0} \\
	\log \frac{T}{T_0} &= \log \left(\frac{V_0}{V}\right)^{\gamma-1} \\
	\frac{T}{T_0} &= \left( \frac{V_0}{V} \right)^{\gamma -1} \\
	TV^{\gamma -1} &= T_0 V_0^{\gamma -1}
\end{align*}
È stata trovata una relazione tra le variabili termodinamiche durante il processo adiabatico reversibile.
\[
	\boxed{TV^{\gamma -1} = \text{costante}}
\]
È una relazione caratteristica delle adiabatiche reversibili. A partire da questa relazione inseriamo la pressione per trovare un legame di questa con il volume.
\[
	\frac{pV}{nR}V^{\gamma -1} = \frac{p_0 V_0}{nR}V_0^{\gamma -1} \implies pV^{\gamma }=p_0 V_0^{\gamma } \implies \boxed{pV^{\gamma} = \text{costante}}
\]
Questo vuole fondamentalmente dire che tale è l'espansione che subirà il gas quando lo si fa espandere lungo un'adiabatica reversibile. Nell'isoterma, il gas parte dalla posizione $p_0$ e dal volume $V_0$, si espande fino alla pressione finale, quella atmosferica, e arriva a un certo volume finale. Nell'adiabatica reversibile, il gas si espande per arrivare alla stessa pressione finale e si raffredda. 
\begin{figure}[htpb]
	\centering
	

	\tikzset{every picture/.style={line width=0.75pt}} %set default line width to 0.75pt        

	\begin{tikzpicture}[x=0.75pt,y=0.75pt,yscale=-1,xscale=1]
	%uncomment if require: \path (0,300); %set diagram left start at 0, and has height of 300

	%Shape: Axis 2D [id:dp27504680673319837] 
	\draw  (128.5,213.85) -- (358.5,213.85)(139.52,73) -- (139.52,229) (351.5,208.85) -- (358.5,213.85) -- (351.5,218.85) (134.52,80) -- (139.52,73) -- (144.52,80)  ;
	%Curve Lines [id:da9113262903740069] 
	\draw    (164,119) .. controls (180.67,154.33) and (226,179) .. (274.67,180.33) ;
	%Curve Lines [id:da8659384784035273] 
	\draw    (177.33,121.67) .. controls (181.33,159.67) and (206,185) .. (254,189.67) ;
	%Straight Lines [id:da5714540065534941] 
	\draw  [dash pattern={on 0.84pt off 2.51pt}]  (140,144.67) -- (239.33,144.67) ;
	%Straight Lines [id:da8284301291991147] 
	\draw  [dash pattern={on 0.84pt off 2.51pt}]  (183.67,213.33) -- (183.67,103) ;

	% Text Node
	\draw (127,74.5) node    {$p$};
	% Text Node
	\draw (370.5,218) node    {$V$};
	% Text Node
	\draw (127,141.17) node    {$p_{0}$};
	% Text Node
	\draw (119.67,191.17) node    {$p_{atm}$};
	% Text Node
	\draw (189.17,228) node    {$V_{0}$};
	% Text Node
	\draw (275.33,166.33) node   [align=left] {isoterma};
	% Text Node
	\draw (251,199.67) node   [align=left] {adiabatica};


	\end{tikzpicture}
\end{figure}
\FloatBarrier
Essa passa sicuramente sotto all'isoterma reversibile perché $\gamma$ è maggiore di $1$, infatti il gas si raffredda quindi questo stato finale deve essere sull'adiabatica più in basso.
Il lavoro adiabatico può essere calcolato facilmente come l'opposto della variazione dell'energia interna. Si può anche calcolarlo come area sottesa dalla curva anche se il procedimento è più complicato.
Nella pratica l'adiabaticità perfetta non esiste: tutte le sostanze con cui si realizzano pareti adiabatiche permettono un certo scambio di calore. Si ammette che possa essere adiabatica una trasformazione che avviene rapidamente, così che non ci sia tempo per scambi di calore apprezzabili.

\paragraph{Trasformazioni adiabatiche irreversibili} Se si realizza lo stesso processo in maniera irreversibile, togliendo ad esempio la massa che comprime il gas in un solo istante, non si potrà arrivare allo stesso volume $V$. Questo perché per trovare $pV^{\gamma}=\text{costante}$, è stato imposto che le variabili termodinamiche del gas in ogni punto soddisfino l'equazione di stato. Se il processo non lo si fa avvenire in maniera quasi statica non può valere la relazione trovata. Se si arriva a un volume finale diverso, il lavoro fatto dal gas durante il processo irreversibile adiabatico è diverso da quello compiuto durante il processo reversibile adiabatico.
Si può comunque affermare che la variazione di energia interna sarà pari e opposta al lavoro fatto dal gas, ma si dovrà introdurre un'altra relazione per trovare entrambe le variabili incognite. Un'idea è quella di scrivere il lavoro come l'opposto di quello subito dall'ambiente esterno. Esso sarà dato da: $nc_v\Delta T = - p_{\text{fin}}\Delta V$.
Si noti che tutte le trasformazioni viste possono essere scritte come:
\[
	pV^x=\text{costante} \left\{ \begin{array}{l}
	 	x = 0 \quad \text{isobara}  \\
	 	x = 1 \quad \text{isoterma}  \\
	 	x = \frac{c_p }{c_v } \quad \text{adiabatica}  \\
	 	x \to +\infty  \quad \text{isocora} \quad (p^{1/x}V=\text{costante}) \\
	\end{array} \right.
\]
Più in generale,  qualunque trasformazione \emph{reversibile} segue la legge:
\[
	\boxed{ pV^x=\text{cost}} \quad \text{dove }x=\frac{c_p-c_x  }{c_v-c_x  }
\]
e prende il nome di \textbf{politropica}. La dimostrazione si opera sfruttando il primo principio della termodinamica in modo molto simile a quello che ha portato a trovare l'equazione di una adiabatica reversibile. L'unica differenza sta nel fatto che $\delta Q$ non è uguale a $0$ ma sarà il calore scambiato con un certo calore specifico $c_x$.







































\section{Macchine termodinamiche}

\subsection{Definizione e generalità}

Uno dei motivi per cui è nata la termodinamica è quello di realizzare macchine termiche, ossia dispositivi realizzanti processi che permettono di trasformare un energia termica in lavoro meccanico. Un esempio semplice può essere l'espansione isoterma di un gas contenuto in un recipiente chiuso da un pistone e a contatto con un serbatoio. Essa avviene senza accumulo di energia interna: il calore fornito dal serbatoio si trasforma in lavoro meccanico. In tal caso il pistone si solleva una volta sola ma in genere si vuole che il lavoro meccanico venga ripetuto più volte. Questo significa che non basta realizzare la trasformazione una sola volta. Si mantiene sempre tutto a contatto con il serbatoio, e una volta che il pistone si è espanso si rimette la massa sopra facendo comprimere il sistema. In tal modo, il lavoro ottenuto passando da $A$ a $B$ lo si devo rifornire completamente tornando indietro, al netto non si è ottenuto nulla. Per realizzare un processo che al netto dia lavoro positivo, bisogna passare su una isoterma più bassa in modo tale che il lavoro tornando indietro sia meno intenso. Per fare ciò si segue in andata un tratto a pressione più alta di quello in cui avviene la compressione del gas. Questo è proprio un ciclo termodinamico caratteristico di una macchina termica.
\begin{figure}[htpb]
	\centering
	

	\tikzset{every picture/.style={line width=0.75pt}} %set default line width to 0.75pt        

	\begin{tikzpicture}[x=0.75pt,y=0.75pt,yscale=-1,xscale=1]
	%uncomment if require: \path (0,300); %set diagram left start at 0, and has height of 300

	%Shape: Polygon Curved [id:ds23043235146641394] 
	\draw  [fill={rgb, 255:red, 212; green, 212; blue, 212 }  ,fill opacity=1 ] (196.5,110.5) .. controls (216.5,100.5) and (306.5,90.5) .. (286.5,110.5) .. controls (266.5,130.5) and (266.5,140.5) .. (286.5,170.5) .. controls (306.5,200.5) and (216.5,200.5) .. (196.5,170.5) .. controls (176.5,140.5) and (176.5,120.5) .. (196.5,110.5) -- cycle ;
	%Shape: Axis 2D [id:dp7762746332730381] 
	\draw  (128.5,213.85) -- (358.5,213.85)(139.52,73) -- (139.52,229) (351.5,208.85) -- (358.5,213.85) -- (351.5,218.85) (134.52,80) -- (139.52,73) -- (144.52,80)  ;
	%Shape: Circle [id:dp27074858222439246] 
	\draw  [fill={rgb, 255:red, 0; green, 0; blue, 0 }  ,fill opacity=1 ] (194.25,110.5) .. controls (194.25,109.26) and (195.26,108.25) .. (196.5,108.25) .. controls (197.74,108.25) and (198.75,109.26) .. (198.75,110.5) .. controls (198.75,111.74) and (197.74,112.75) .. (196.5,112.75) .. controls (195.26,112.75) and (194.25,111.74) .. (194.25,110.5) -- cycle ;
	%Shape: Circle [id:dp6385622176143919] 
	\draw  [fill={rgb, 255:red, 0; green, 0; blue, 0 }  ,fill opacity=1 ] (284,169.5) .. controls (284,168.26) and (285.01,167.25) .. (286.25,167.25) .. controls (287.49,167.25) and (288.5,168.26) .. (288.5,169.5) .. controls (288.5,170.74) and (287.49,171.75) .. (286.25,171.75) .. controls (285.01,171.75) and (284,170.74) .. (284,169.5) -- cycle ;
	\draw  [fill={rgb, 255:red, 0; green, 0; blue, 0 }  ,fill opacity=1 ] (281.71,123.07) -- (272.36,131.07) -- (271.57,118.79) -- (274.5,126) -- cycle ;
	\draw  [fill={rgb, 255:red, 0; green, 0; blue, 0 }  ,fill opacity=1 ] (191.82,172.6) -- (190.54,160.36) -- (201.1,166.68) -- (193.5,165) -- cycle ;

	% Text Node
	\draw (127,74.5) node    {$p$};
	% Text Node
	\draw (370.5,218) node    {$V$};
	% Text Node
	\draw (188.5,99.17) node    {$A$};
	% Text Node
	\draw (298.17,163.67) node    {$B$};


	\end{tikzpicture}
\end{figure}
\FloatBarrier
Per un ciclo termodinamico si può sempre scrivere che:
\begin{gather*}
	\Delta U_{\text{ciclo} } = Q_{\text{ciclo} } - \mathcal{L}_{\text{ciclo} } \\
	Q_{\text{ciclo} } = \mathcal{L}_{\text{ciclo} } > 0
\end{gather*}
La somma di tutti i calori scambiati durante il ciclo da luogo al lavoro netto compiuto durante il ciclo. Il lavoro totale in un ciclo termodinamico reversibile è allora l'area racchiusa da esso.

Si può osservare sperimentalmente che non è mai possibile realizzare un processo che comporta come unico risultato la trasformazione di tutto il calore fornito in lavoro. Una parte di calore fornito sarà sempre ceduto all'ambiente circostante o ad un serbatoio a temperatura più bassa.
\begin{figure}[htpb]
	\centering
	

	\tikzset{every picture/.style={line width=0.75pt}} %set default line width to 0.75pt        

	\begin{tikzpicture}[x=0.75pt,y=0.75pt,yscale=-1,xscale=1]
	%uncomment if require: \path (0,300); %set diagram left start at 0, and has height of 300

	%Right Arrow [id:dp038945600763613086] 
	\draw  [draw opacity=0][fill={rgb, 255:red, 184; green, 184; blue, 184 }  ,fill opacity=1 ] (175.06,108.44) -- (175.06,127.76) -- (180.8,127.76) -- (169.33,140.64) -- (157.87,127.76) -- (163.6,127.76) -- (163.6,108.44) -- cycle ;
	%Right Arrow [id:dp736310905270898] 
	\draw  [draw opacity=0][fill={rgb, 255:red, 184; green, 184; blue, 184 }  ,fill opacity=1 ] (184.67,152.24) -- (205.52,152.24) -- (205.52,146.51) -- (219.42,157.97) -- (205.52,169.44) -- (205.52,163.7) -- (184.67,163.7) -- cycle ;
	%Shape: Rectangle [id:dp49095902226996646] 
	\draw  [fill={rgb, 255:red, 155; green, 155; blue, 155 }  ,fill opacity=1 ] (83,75) -- (244.5,75) -- (244.5,112.27) -- (83,112.27) -- cycle ;
	%Shape: Rectangle [id:dp7012565668343214] 
	\draw  [fill={rgb, 255:red, 212; green, 212; blue, 212 }  ,fill opacity=1 ] (83,202.9) -- (244.5,202.9) -- (244.5,240.17) -- (83,240.17) -- cycle ;
	%Shape: Ellipse [id:dp5766499478882205] 
	\draw  [color={rgb, 255:red, 128; green, 128; blue, 128 }  ,draw opacity=1 ][fill={rgb, 255:red, 184; green, 184; blue, 184 }  ,fill opacity=1 ] (152,157.97) .. controls (152,148.4) and (159.76,140.64) .. (169.33,140.64) .. controls (178.91,140.64) and (186.67,148.4) .. (186.67,157.97) .. controls (186.67,167.55) and (178.91,175.31) .. (169.33,175.31) .. controls (159.76,175.31) and (152,167.55) .. (152,157.97) -- cycle ;
	%Right Arrow [id:dp6830856272799466] 
	\draw  [draw opacity=0][fill={rgb, 255:red, 184; green, 184; blue, 184 }  ,fill opacity=1 ] (175.06,175.31) -- (175.06,191.41) -- (180.8,191.41) -- (169.33,202.15) -- (157.87,191.41) -- (163.6,191.41) -- (163.6,175.31) -- cycle ;

	% Text Node
	\draw (122.67,128.17) node    {$Q_{\text{assorbito}}$};
	% Text Node
	\draw (121.67,183.83) node    {$Q_{\text{ceduto}}$};
	% Text Node
	\draw (104.12,92.39) node    {$T_{2}$};
	% Text Node
	\draw (104.12,221.53) node    {$T_{1}$};
	% Text Node
	\draw (231.8,158.27) node  [font=\small]  {$\mathcal{L}$};


	\end{tikzpicture}
\end{figure}
\FloatBarrier
È importante quindi che la frazione di calore trasformata in lavoro sia grande. Questo porta alla definizione di \textbf{rendimento} di una macchina termica, dato dal rapporto fra ciò che si ottiene e ciò che bisogna fornire per ottenerlo.
\begin{gather*}
	\eta = \text{rendimento} = \frac{\mathcal{L} }{Q_{\text{assorbito} } } > 0 \qquad 0 < \eta < 1 \quad \text{perché} \quad \mathcal{L} < Q_{\text{ass} } \\
	\underbrace{Q_{\text{ass}}}_{>0} + \underbrace{Q_{\text{ced}}}_{<0} = \mathcal{L} \quad \text{ossia} \quad Q_{\text{ass}} - |Q_{\text{ced}}| = \mathcal{L} \\
	\eta = \frac{Q_{\text{ass}} - |Q_{\text{ced}}|}{Q_{\text{ass}}} = 1 - \frac{|Q_{\text{ced}}|}{Q_{\text{ass}}}
\end{gather*}
Quest'ultimo modo di esprimere il rendimento, mette in evidenza che se non ci fosse calore ceduto (cosa irrealizzabile nella pratica), il rendimento sarebbe pari a $1$ e tutto il calore sarebbe trasformato in lavoro.

\subsection{Il ciclo di Carnot}

L'esempio concreto di macchina termica più importante che esiste è quella realizzabile facendo eseguire al sistema termodinamico un ciclo detto ciclo di Carnot.
\begin{figure}[htpb]
	\centering
	

	\tikzset{every picture/.style={line width=0.75pt}} %set default line width to 0.75pt        

	\begin{tikzpicture}[x=0.75pt,y=0.75pt,yscale=-1,xscale=1]
	%uncomment if require: \path (0,300); %set diagram left start at 0, and has height of 300

	%Shape: Axis 2D [id:dp48720272776316453] 
	\draw  (70,222.3) -- (317.5,222.3)(94.75,72) -- (94.75,239) (310.5,217.3) -- (317.5,222.3) -- (310.5,227.3) (89.75,79) -- (94.75,72) -- (99.75,79)  ;
	%Curve Lines [id:da6308193749582969] 
	\draw    (143.23,90.17) .. controls (160.08,113.9) and (193,132.28) .. (232.05,130.75) ;
	\draw [shift={(182.55,121.43)}, rotate = 205.42000000000002] [fill={rgb, 255:red, 0; green, 0; blue, 0 }  ][line width=0.08]  [draw opacity=0] (10.72,-5.15) -- (0,0) -- (10.72,5.15) -- (7.12,0) -- cycle    ;
	%Curve Lines [id:da6684877234044795] 
	\draw    (165.44,173.62) .. controls (188.4,189.7) and (224.64,202.2) .. (263.42,197.45) ;
	\draw [shift={(212.56,194.39)}, rotate = 14.74] [fill={rgb, 255:red, 0; green, 0; blue, 0 }  ][line width=0.08]  [draw opacity=0] (10.72,-5.15) -- (0,0) -- (10.72,5.15) -- (7.12,0) -- cycle    ;
	%Curve Lines [id:da4255798805877631] 
	\draw    (143.23,90.17) .. controls (146.29,119.26) and (144.76,146.92) .. (165.44,173.62) ;
	\draw [shift={(147.97,134.04)}, rotate = 80.14] [fill={rgb, 255:red, 0; green, 0; blue, 0 }  ][line width=0.08]  [draw opacity=0] (10.72,-5.15) -- (0,0) -- (10.72,5.15) -- (7.12,0) -- cycle    ;
	%Curve Lines [id:da2405649855739509] 
	\draw    (232.05,130.75) .. controls (235.11,159.84) and (242.75,170.75) .. (263.42,197.45) ;
	\draw [shift={(242.03,166.74)}, rotate = 244] [fill={rgb, 255:red, 0; green, 0; blue, 0 }  ][line width=0.08]  [draw opacity=0] (10.72,-5.15) -- (0,0) -- (10.72,5.15) -- (7.12,0) -- cycle    ;

	% Text Node
	\draw (81,74) node    {$p$};
	% Text Node
	\draw (331.33,221.33) node    {$V$};
	% Text Node
	\draw (142,76) node    {$A$};
	% Text Node
	\draw (241.33,117.33) node    {$B$};
	% Text Node
	\draw (277,194.33) node    {$C$};
	% Text Node
	\draw (156.33,183) node    {$D$};
	% Text Node
	\draw (202.67,177.33) node    {$T_{1}$};
	% Text Node
	\draw (194,107.33) node    {$T_{2}$};


	\end{tikzpicture}
\end{figure}
\FloatBarrier
Esso è costituito da:
\begin{itemize}
	\item Una espansione isoterma alla temperatura $T_2$ in cui il gas è a contatto con un serbatoio caldo. Esso espandendosi assorbe calore dal serbatoio e lo trasforma in lavoro ($AB$).
	\item Per abbassare la pressione si effettua un raffreddamento adiabatico ($BC$): si fa espandere il gas in maniera adiabatica reversibile.
	\item Poi si fa comprimere il gas in maniera isoterma, ma ad una temperatura più bassa di quella di partenza, perché esso si è raffreddato passando da $B$ a $C$. Il lavoro subito per questo sarà minore di quello compiuto nella fase di espansione ($CD$).
	\item Si esegue poi un riscaldamento adiabatico reversibile ($DA$).
\end{itemize}
Per calcolare il rendimento della macchina di Carnot, si mettono in evidenza le trasformazioni in cui c'è passaggio di calore con l'esterno (le due isoterme).
Si trova il calore assorbito:
\begin{align*}
	A\to B:\quad Q_{AB,\text{isoterma} } = \mathcal{L}_{AB} = \int p_{gas}dV &= \int_{V_A }^{V_B } \frac{nRT_2 }{V}dV \\
	&= nRT_2\log \left( \frac{V_B }{V_A } \right)
\end{align*}
E il calore ceduto:
\begin{gather*}
	C\to D:\quad Q_{CD,\text{isoterma}} = \mathcal{L}_{\text{isoterma} } = nRT_1\log \left( \frac{V_D }{V_C } \right) \\
	\eta = 1-\frac{|Q_{\text{ced} } |}{Q_{\text{ass} } } = 1 - \frac{nRT_1\log (V_c/V_D  ) }{nRT_2\log (V_B/V_A) }
\end{gather*}
Si aggiunge la relazione caratteristica che lega gli stati tra un adiabatica reversibile.
\begin{gather*}
	B\to C:\quad TV^{\gamma -1}=\text{costante} \qquad T_2 V_B^{\gamma -1}=T_1 V_C^{\gamma -1} \\
	D\to A:\quad T_2 V_A^{\gamma -1} = T_1 V_D^{\gamma -1} \\
	\left( \frac{V_B }{V_A } \right)^{\gamma -1} = \left( \frac{V_C }{V_D } \right)^{\gamma -1} \implies \frac{V_B }{V_A }=\frac{V_C }{V_D } \implies \boxed{\eta = 1-\frac{T_1 }{T_2 }}
\end{gather*}
Si vede che il rendimento della macchina di Carnot dipende solo dalla temperatura dei due serbatoi. Sfruttando il primo principio della termodinamica, cioè il fatto che nella macchia termica si segue un ciclo termodinamico, si può affermare che il lavoro fatto è pari alla somma dei calori scambiati con l'ambiente circostante.







































\section{Macchine frigorifere}

La termodinamica nasce anche per realizzare una seconda tipologia di macchine, che sono le macchine frigorifere. Si vuole mantenere l'ambiente all'interno del frigorifero ad una temperatura bassa $T_1$ e evitare che tenda a termalizzare con l'ambiente circostante. Si vuole invertire il processo secondo cui il calore passa dal corpo più caldo al corpo più freddo, fare in modo che avvenga il contrario. Non si tratta di un processo spontaneo, ecco perché per realizzarlo la macchina sfrutta il lavoro fornito dall'esterno.
\begin{figure}[htpb]
	\centering
	

	\tikzset{every picture/.style={line width=0.75pt}} %set default line width to 0.75pt        

	\begin{tikzpicture}[x=0.75pt,y=0.75pt,yscale=-1,xscale=1]
	%uncomment if require: \path (0,300); %set diagram left start at 0, and has height of 300

	%Right Arrow [id:dp6348625548042963] 
	\draw  [draw opacity=0][fill={rgb, 255:red, 184; green, 184; blue, 184 }  ,fill opacity=1 ] (222.6,154) -- (222.6,134.68) -- (216.87,134.68) -- (228.33,121.8) -- (239.8,134.68) -- (234.06,134.68) -- (234.06,154) -- cycle ;
	%Right Arrow [id:dp959084374703087] 
	\draw  [draw opacity=0][fill={rgb, 255:red, 184; green, 184; blue, 184 }  ,fill opacity=1 ] (280.42,172.7) -- (259.57,172.7) -- (259.57,178.44) -- (245.67,166.97) -- (259.57,155.51) -- (259.57,161.24) -- (280.42,161.24) -- cycle ;
	%Shape: Rectangle [id:dp8069606542485774] 
	\draw  [fill={rgb, 255:red, 155; green, 155; blue, 155 }  ,fill opacity=1 ] (158,84) -- (319.5,84) -- (319.5,121.27) -- (158,121.27) -- cycle ;
	%Shape: Rectangle [id:dp04641463651885647] 
	\draw  [fill={rgb, 255:red, 212; green, 212; blue, 212 }  ,fill opacity=1 ] (158,211.9) -- (319.5,211.9) -- (319.5,249.17) -- (158,249.17) -- cycle ;
	%Shape: Ellipse [id:dp8578905893334057] 
	\draw  [color={rgb, 255:red, 128; green, 128; blue, 128 }  ,draw opacity=1 ][fill={rgb, 255:red, 184; green, 184; blue, 184 }  ,fill opacity=1 ] (211,166.97) .. controls (211,157.4) and (218.76,149.64) .. (228.33,149.64) .. controls (237.91,149.64) and (245.67,157.4) .. (245.67,166.97) .. controls (245.67,176.55) and (237.91,184.31) .. (228.33,184.31) .. controls (218.76,184.31) and (211,176.55) .. (211,166.97) -- cycle ;
	%Right Arrow [id:dp9889661271891552] 
	\draw  [draw opacity=0][fill={rgb, 255:red, 184; green, 184; blue, 184 }  ,fill opacity=1 ] (222.6,211.15) -- (222.6,195.04) -- (216.87,195.04) -- (228.33,184.31) -- (239.8,195.04) -- (234.06,195.04) -- (234.06,211.15) -- cycle ;

	% Text Node
	\draw (179.12,101.39) node    {$T_{2}$};
	% Text Node
	\draw (179.12,230.53) node    {$T_{1}$};
	% Text Node
	\draw (202.37,137.03) node  [font=\small]  {$Q_{2}$};
	% Text Node
	\draw (204,196.87) node  [font=\small]  {$Q_{1}$};
	% Text Node
	\draw (290.8,167.27) node  [font=\small]  {$\mathcal{L}$};


	\end{tikzpicture}
\end{figure}
\FloatBarrier
Da un punto di vista pratico, per valutare la bontà di un frigorifero, si introduce un parametro di bontà che prende il nome di \textbf{efficienza frigorifera} o coefficiente di prestazione $\varepsilon$. Esso è il rapporto fra ciò che si vuole massimizzare, a scapito di ciò che si vuole fornire, il lavoro.
\[
	\varepsilon = \frac{Q_1 }{|\mathcal{L} |} < \infty
\]
È un lavoro fatto sul gas, quindi negativo: per avere un parametro positivo se ne utilizza allora il modulo.
Il processo non potrebbe mai avvenire senza immissione di lavoro, il denominatore non sarà mai zero, quindi l'efficienza non tenderà mai all'infinito. La si riscrive solo in termini di calori scambiati sfruttando il primo principio della termodinamica. Il ciclo termodinamico, visto che ci si aspetta un lavoro negativo, dovrà essere percorso in verso antiorario.
Si prende come esempio la macchina di Carnot, che si fa funzionare non come macchina termica ma come macchina frigorifera e quindi percorrendo il ciclo prima illustrato, ma in senso opposto.
Il gas per attuare questo processo deve ricevere un lavoro dall'ambiente circostante. Esso è rappresentato dall'area del ciclo, che in questo caso è negativa. Nella fase di espansione infatti viene compiuto un lavoro positivo ma nella fase di compressione il lavoro, che è negativo, è in modulo maggiore.
Poiché viene eseguito un ciclo, la somma dei calori scambiati è pari alla somma del lavoro fornito. La quantità $Q_1 - \abs{(Q_2)}$ è negativa. Si ha più calore ceduto all'ambiente cucina di quello che assorbito.
\[
	\Delta U_{\text{ciclo} } = Q - \mathcal{L}  = 0 \implies Q_1-|Q_2| = -|\mathcal{L}| < 0 \implies Q_1 < |Q_2|
\]
Andando a sostituire nell'espressione dell'efficienza frigorifera:
\[
	\varepsilon = \frac{Q_1 }{|Q_2| - Q_1 }
\]
Questa quantità è positiva e c'è sempre, non accadrà mai che tutto $Q_1$ diventi $Q_2$. Se si calcola il rendimento della particolare macchina frigorifera reversibile di Carnot (percorsa nell'altro senso) disegnata a lato, costituita da due isoterme e da due adiabatiche, si trova che:
\begin{align*}
	\varepsilon_{\text{Carnot} } &= \frac{Q_1 }{|Q_2| - |Q_1|} \\
	&= \frac{nRT_1\log (V_c/V_D  ) }{nRT_2\log (V_B/V_A) - nRT_1\log (V_C/V_D)} \\
	&=\frac{T_1 }{T_2-T_1  }
\end{align*}
Per questa particolare macchina termica frigorifera, il rendimento non dipende dal gas o dal numero di moli. Nel risultato del rendimento non compare $n$. Si è definito cosa sono una macchina termica e una macchina frigorifera. Esistono tanti esempi di macchine termiche e frigorifere che possono essere percorse con tanti tipi di cicli.

\paragraph{Ciclo di Stirling} È costituito da due isoterme e due isocore reversibili.
\begin{figure}[htpb]
	\centering
	

	\tikzset{every picture/.style={line width=0.75pt}} %set default line width to 0.75pt        

	\begin{tikzpicture}[x=0.75pt,y=0.75pt,yscale=-1,xscale=1]
	%uncomment if require: \path (0,300); %set diagram left start at 0, and has height of 300

	%Shape: Axis 2D [id:dp09005657817363133] 
	\draw  (70,219.06) -- (352.33,219.06)(98.23,39.56) -- (98.23,239) (345.33,214.06) -- (352.33,219.06) -- (345.33,224.06) (93.23,46.56) -- (98.23,39.56) -- (103.23,46.56)  ;
	%Curve Lines [id:da25665381324207437] 
	\draw    (176.57,57.92) .. controls (191.27,78.63) and (240.8,88.92) .. (274.87,87.58) ;
	%Straight Lines [id:da9344352155729672] 
	\draw    (176.57,57.92) -- (176.57,132.37) ;
	%Curve Lines [id:da17826742079624203] 
	\draw    (176.57,132.37) .. controls (191.27,153.08) and (240.8,163.37) .. (274.87,162.03) ;
	%Straight Lines [id:da9751411952155511] 
	\draw    (274.87,87.58) -- (274.87,162.03) ;
	%Straight Lines [id:da019457556764712836] 
	\draw    (227.53,54.11) .. controls (229.2,55.78) and (229.2,57.44) .. (227.53,59.11) .. controls (225.86,60.78) and (225.86,62.44) .. (227.53,64.11) .. controls (229.2,65.78) and (229.2,67.44) .. (227.53,69.11) .. controls (225.86,70.78) and (225.86,72.44) .. (227.53,74.11) .. controls (229.2,75.78) and (229.2,77.44) .. (227.53,79.11) .. controls (225.86,80.78) and (225.86,82.44) .. (227.53,84.11) .. controls (229.2,85.78) and (229.2,87.44) .. (227.53,89.11) -- (227.53,93.22) -- (227.53,96.22)(224.53,54.11) .. controls (226.2,55.78) and (226.2,57.44) .. (224.53,59.11) .. controls (222.86,60.78) and (222.86,62.44) .. (224.53,64.11) .. controls (226.2,65.78) and (226.2,67.44) .. (224.53,69.11) .. controls (222.86,70.78) and (222.86,72.44) .. (224.53,74.11) .. controls (226.2,75.78) and (226.2,77.44) .. (224.53,79.11) .. controls (222.86,80.78) and (222.86,82.44) .. (224.53,84.11) .. controls (226.2,85.78) and (226.2,87.44) .. (224.53,89.11) -- (224.53,93.22) -- (224.53,96.22) ;
	\draw [shift={(226.03,103.22)}, rotate = 270] [color={rgb, 255:red, 0; green, 0; blue, 0 }  ][line width=0.75]    (10.93,-4.9) .. controls (6.95,-2.3) and (3.31,-0.67) .. (0,0) .. controls (3.31,0.67) and (6.95,2.3) .. (10.93,4.9)   ;
	%Straight Lines [id:da37819583599228856] 
	\draw    (227.53,136.2) .. controls (229.2,137.87) and (229.2,139.53) .. (227.53,141.2) .. controls (225.86,142.87) and (225.86,144.53) .. (227.53,146.2) .. controls (229.2,147.87) and (229.2,149.53) .. (227.53,151.2) .. controls (225.86,152.87) and (225.86,154.53) .. (227.53,156.2) .. controls (229.2,157.87) and (229.2,159.53) .. (227.53,161.2) .. controls (225.86,162.87) and (225.86,164.53) .. (227.53,166.2) .. controls (229.2,167.87) and (229.2,169.53) .. (227.53,171.2) -- (227.53,175.3) -- (227.53,178.3)(224.53,136.2) .. controls (226.2,137.87) and (226.2,139.53) .. (224.53,141.2) .. controls (222.86,142.87) and (222.86,144.53) .. (224.53,146.2) .. controls (226.2,147.87) and (226.2,149.53) .. (224.53,151.2) .. controls (222.86,152.87) and (222.86,154.53) .. (224.53,156.2) .. controls (226.2,157.87) and (226.2,159.53) .. (224.53,161.2) .. controls (222.86,162.87) and (222.86,164.53) .. (224.53,166.2) .. controls (226.2,167.87) and (226.2,169.53) .. (224.53,171.2) -- (224.53,175.3) -- (224.53,178.3) ;
	\draw [shift={(226.03,185.3)}, rotate = 270] [color={rgb, 255:red, 0; green, 0; blue, 0 }  ][line width=0.75]    (10.93,-4.9) .. controls (6.95,-2.3) and (3.31,-0.67) .. (0,0) .. controls (3.31,0.67) and (6.95,2.3) .. (10.93,4.9)   ;
	%Straight Lines [id:da5937334242417012] 
	\draw    (251.69,123.31) .. controls (253.36,121.64) and (255.02,121.64) .. (256.69,123.31) .. controls (258.36,124.98) and (260.02,124.98) .. (261.69,123.31) .. controls (263.36,121.64) and (265.02,121.64) .. (266.69,123.31) .. controls (268.36,124.98) and (270.02,124.98) .. (271.69,123.31) .. controls (273.36,121.64) and (275.02,121.64) .. (276.69,123.31) .. controls (278.36,124.98) and (280.02,124.98) .. (281.69,123.31) .. controls (283.36,121.64) and (285.02,121.64) .. (286.69,123.31) -- (288.05,123.31) -- (291.05,123.31)(251.69,126.31) .. controls (253.36,124.64) and (255.02,124.64) .. (256.69,126.31) .. controls (258.36,127.98) and (260.02,127.98) .. (261.69,126.31) .. controls (263.36,124.64) and (265.02,124.64) .. (266.69,126.31) .. controls (268.36,127.98) and (270.02,127.98) .. (271.69,126.31) .. controls (273.36,124.64) and (275.02,124.64) .. (276.69,126.31) .. controls (278.36,127.98) and (280.02,127.98) .. (281.69,126.31) .. controls (283.36,124.64) and (285.02,124.64) .. (286.69,126.31) -- (288.05,126.31) -- (291.05,126.31) ;
	\draw [shift={(298.05,124.81)}, rotate = 180] [color={rgb, 255:red, 0; green, 0; blue, 0 }  ][line width=0.75]    (10.93,-4.9) .. controls (6.95,-2.3) and (3.31,-0.67) .. (0,0) .. controls (3.31,0.67) and (6.95,2.3) .. (10.93,4.9)   ;
	%Straight Lines [id:da5637070421962109] 
	\draw    (153.39,93.65) .. controls (155.06,91.98) and (156.72,91.98) .. (158.39,93.65) .. controls (160.06,95.32) and (161.72,95.32) .. (163.39,93.65) .. controls (165.06,91.98) and (166.72,91.98) .. (168.39,93.65) .. controls (170.06,95.32) and (171.72,95.32) .. (173.39,93.65) .. controls (175.06,91.98) and (176.72,91.98) .. (178.39,93.65) .. controls (180.06,95.32) and (181.72,95.32) .. (183.39,93.65) .. controls (185.06,91.98) and (186.72,91.98) .. (188.39,93.65) -- (189.76,93.65) -- (192.76,93.65)(153.39,96.65) .. controls (155.06,94.98) and (156.72,94.98) .. (158.39,96.65) .. controls (160.06,98.32) and (161.72,98.32) .. (163.39,96.65) .. controls (165.06,94.98) and (166.72,94.98) .. (168.39,96.65) .. controls (170.06,98.32) and (171.72,98.32) .. (173.39,96.65) .. controls (175.06,94.98) and (176.72,94.98) .. (178.39,96.65) .. controls (180.06,98.32) and (181.72,98.32) .. (183.39,96.65) .. controls (185.06,94.98) and (186.72,94.98) .. (188.39,96.65) -- (189.76,96.65) -- (192.76,96.65) ;
	\draw [shift={(199.76,95.15)}, rotate = 180] [color={rgb, 255:red, 0; green, 0; blue, 0 }  ][line width=0.75]    (10.93,-4.9) .. controls (6.95,-2.3) and (3.31,-0.67) .. (0,0) .. controls (3.31,0.67) and (6.95,2.3) .. (10.93,4.9)   ;

	% Text Node
	\draw (80.67,42) node    {$p$};
	% Text Node
	\draw (360,218) node    {$V$};
	% Text Node
	\draw (167.85,50.07) node    {$A$};
	% Text Node
	\draw (284.76,81.48) node    {$B$};
	% Text Node
	\draw (284.47,164.94) node    {$C$};
	% Text Node
	\draw (168.72,142.26) node    {$D$};
	% Text Node
	\draw (247.53,46.99) node    {$Q_{AB}$};
	% Text Node
	\draw (240.76,193.85) node    {$Q_{CD}$};
	% Text Node
	\draw (317.94,126.68) node    {$Q_{BC}$};
	% Text Node
	\draw (141.99,101.17) node    {$Q_{DA}$};


	\end{tikzpicture}
\end{figure}
\FloatBarrier
È una macchina termica che fa un lavoro meccanico, realizza un'espansione, seguita da una compressione più bassa
a scapito del calore assorbito. C'è del calore ceduto o assorbito in ogni trasformazione.
\begin{itemize}
	\item $AB$: espansione isoterma. Per fare espandere il gas mantenendo una temperatura costante, il sistema deve assorbire calore.
	\item $BC$: raffreddamento isocoro. Se il gas si raffredda a volume costante, dovrà cedere calore.
	\item $CD$: compressione isoterma. Se il gas viene compresso a temperatura costante, tende a scaldarsi, ma essendo costantemente a contatto con il serbatoio freddo, non lo fa perché cede calore ad esso.
	\item $DA$: riscaldamento isocoro. Si immette poi calore per scaldare il gas a volume costante.
\end{itemize}
Si avranno due frazioni di calore ceduto e due frazioni di calore assorbiti. Il rendimento della macchina di Stirlig è inferiore a quello della macchina di Carnot.







































\section{II principio della termodinamica}

Nel primo principio non viene posto nessun limite allo scambio di energia che avviene durante il processo. La situazione sperimentale tuttavia non appare simmetrica. Mentre è possibile sempre trasformare tutto il lavoro in calore, per esempio sfruttando l'attrito, la trasformazione di calore in lavoro sembra essere limitata indipendentemente dal primo principio. Quando una macchina scambia calore con una o più sorgenti, la somma dei calori assorbiti non si trasforma mai totalmente in lavoro, una parte viene sempre ceduta restando cioè sotto forma di calore scambiato. Un'altra impossibilità sperimentale è poi data dal fatto che il calore non passa mai spontaneamente da un corpo freddo a un corpo caldo. È possibile far avvenire questo passaggio, ma deve essere eseguito un lavoro sulla sostanza che compie il ciclo.
il secondo principio della termodinamica consiste nel prendere atto di queste impossibilità sperimentali e nel trasformarle in postulati, secondo i seguenti enunciati:

\textbf{Enunciato di Kelvin-Planck}

\noindent\fbox{%
	\parbox{\textwidth}{%
		\emph{Non è possibile realizzare un processo termico il cui unico risultato è la trasformazione del calore assorbito da un unico serbatoio in lavoro.}
	}%
}
\begin{figure}[htpb]
	\centering
	

	\tikzset{every picture/.style={line width=0.75pt}} %set default line width to 0.75pt        

	\begin{tikzpicture}[x=0.75pt,y=0.75pt,yscale=-1,xscale=1]
	%uncomment if require: \path (0,300); %set diagram left start at 0, and has height of 300

	%Right Arrow [id:dp8473176475298747] 
	\draw  [draw opacity=0][fill={rgb, 255:red, 184; green, 184; blue, 184 }  ,fill opacity=1 ] (180.06,84.5) -- (180.06,101.38) -- (185.8,101.38) -- (174.33,112.64) -- (162.87,101.38) -- (168.6,101.38) -- (168.6,84.5) -- cycle ;
	%Right Arrow [id:dp6689782123318382] 
	\draw  [draw opacity=0][fill={rgb, 255:red, 184; green, 184; blue, 184 }  ,fill opacity=1 ] (188.25,124.24) -- (209.1,124.24) -- (209.1,118.51) -- (223,129.97) -- (209.1,141.44) -- (209.1,135.7) -- (188.25,135.7) -- cycle ;
	%Shape: Rectangle [id:dp8418322748956599] 
	\draw  [fill={rgb, 255:red, 155; green, 155; blue, 155 }  ,fill opacity=1 ] (100,47) -- (261.5,47) -- (261.5,84.27) -- (100,84.27) -- cycle ;
	%Shape: Rectangle [id:dp6532511543396138] 
	\draw  [fill={rgb, 255:red, 212; green, 212; blue, 212 }  ,fill opacity=1 ] (100,174.9) -- (261.5,174.9) -- (261.5,212.17) -- (100,212.17) -- cycle ;
	%Shape: Ellipse [id:dp021022255460882988] 
	\draw  [color={rgb, 255:red, 128; green, 128; blue, 128 }  ,draw opacity=1 ][fill={rgb, 255:red, 184; green, 184; blue, 184 }  ,fill opacity=1 ] (157,129.97) .. controls (157,120.4) and (164.76,112.64) .. (174.33,112.64) .. controls (183.91,112.64) and (191.67,120.4) .. (191.67,129.97) .. controls (191.67,139.55) and (183.91,147.31) .. (174.33,147.31) .. controls (164.76,147.31) and (157,139.55) .. (157,129.97) -- cycle ;
	\draw  [line width=1.5]  (94.67,43) -- (267.33,215.67)(267.33,43) -- (94.67,215.67) ;

	% Text Node
	\draw (121.12,64.39) node    {$T_{2}$};
	% Text Node
	\draw (121.12,193.53) node    {$T_{1}$};
	% Text Node
	\draw (229.13,112.93) node  [font=\small]  {$\mathcal{L}$};
	% Text Node
	\draw (125.33,97.07) node  [font=\small]  {$Q_{2}$};
	% Text Node
	\draw (179.33,160.4) node  [font=\small]  {$Q_{1} =0$};


	\end{tikzpicture}
\end{figure}
\FloatBarrier
Non si può assorbire calore da un serbatoio per farlo diventare totalmente lavoro. Diretta conseguenza di questo risultato è che il rendimento della macchina termica deve essere strettamente minore di $1$. Si dovrà sempre avere che una frazione del calore assorbito non diventa lavoro ma rimane calore.

\textbf{Enunciato di Clausius}

\noindent\fbox{%
	\parbox{\textwidth}{%
		\emph{Non è possibile realizzare un processo il cui unico risultato sia il passaggio spontaneo di calore da un corpo più freddo a uno più caldo.}
	}%
}

Non si può avere un flusso di calore spontaneo che va da un corpo più freddo a uno più caldo.

Questi due enunciati hanno valenza molto pratica perché servono a dare una spiegazione ai limiti della macchina termica e all'efficienza della macchina frigorifera. Si può dimostrare che i due principi sono equivalenti. La dimostrazione procede per assurdo: negando la tesi enunciata da Kelvin e si ottiene la negazione dell'enunciato di Clausius. Si fa poi il viceversa. Se negando l'uno si nega anche l'altro in entrambi i sensi, vuol dire che i due concetti sono equivalenti.

\textbf{Dimostrazione}
\\

\textbf{Prima parte} Si immagini di poter realizzare un processo che va contro l'enunciato di Kelvin e quindi una macchina termica $NK$ che per assurdo trasforma tutto il calore $Q$ in lavoro $\mathcal{L}$.
\[
	Q = \mathcal{L}
\]
Si associ a questa macchina una frigorifera $F$ vera che lavora assorbendo il calore $Q_1$ da un serbatoio freddo $T_1$ e che, sfruttando un certo lavoro fornito dall'esterno, cede calore $Q_2$ al serbatoio $T_2$. Questa macchina frigorifera la si fa funzionare sfruttando il lavoro compiuto da $NK$.
\begin{figure}[htpb]
	\centering
	

	\tikzset{every picture/.style={line width=0.75pt}} %set default line width to 0.75pt        

	\begin{tikzpicture}[x=0.75pt,y=0.75pt,yscale=-1,xscale=1]
	%uncomment if require: \path (0,300); %set diagram left start at 0, and has height of 300

	%Right Arrow [id:dp692823662524541] 
	\draw  [draw opacity=0][fill={rgb, 255:red, 184; green, 184; blue, 184 }  ,fill opacity=1 ] (140.06,115.5) -- (140.06,132.38) -- (145.8,132.38) -- (134.33,143.64) -- (122.87,132.38) -- (128.6,132.38) -- (128.6,115.5) -- cycle ;
	%Right Arrow [id:dp9520394667646055] 
	\draw  [draw opacity=0][fill={rgb, 255:red, 184; green, 184; blue, 184 }  ,fill opacity=1 ] (194.6,148) -- (194.6,128.68) -- (188.87,128.68) -- (200.33,115.8) -- (211.8,128.68) -- (206.06,128.68) -- (206.06,148) -- cycle ;
	%Right Arrow [id:dp7108011204164864] 
	\draw  [draw opacity=0][fill={rgb, 255:red, 184; green, 184; blue, 184 }  ,fill opacity=1 ] (148.25,155.24) -- (169.1,155.24) -- (169.1,149.51) -- (183,160.97) -- (169.1,172.44) -- (169.1,166.7) -- (148.25,166.7) -- cycle ;
	%Shape: Rectangle [id:dp963014804190296] 
	\draw  [fill={rgb, 255:red, 155; green, 155; blue, 155 }  ,fill opacity=1 ] (90,78) -- (251.5,78) -- (251.5,115.27) -- (90,115.27) -- cycle ;
	%Shape: Rectangle [id:dp8078240816003164] 
	\draw  [fill={rgb, 255:red, 212; green, 212; blue, 212 }  ,fill opacity=1 ] (90,205.9) -- (251.5,205.9) -- (251.5,243.17) -- (90,243.17) -- cycle ;
	%Shape: Ellipse [id:dp6515968584185239] 
	\draw  [color={rgb, 255:red, 128; green, 128; blue, 128 }  ,draw opacity=1 ][fill={rgb, 255:red, 184; green, 184; blue, 184 }  ,fill opacity=1 ] (117,160.97) .. controls (117,151.4) and (124.76,143.64) .. (134.33,143.64) .. controls (143.91,143.64) and (151.67,151.4) .. (151.67,160.97) .. controls (151.67,170.55) and (143.91,178.31) .. (134.33,178.31) .. controls (124.76,178.31) and (117,170.55) .. (117,160.97) -- cycle ;
	%Shape: Ellipse [id:dp8113810604920477] 
	\draw  [color={rgb, 255:red, 128; green, 128; blue, 128 }  ,draw opacity=1 ][fill={rgb, 255:red, 184; green, 184; blue, 184 }  ,fill opacity=1 ] (183,160.97) .. controls (183,151.4) and (190.76,143.64) .. (200.33,143.64) .. controls (209.91,143.64) and (217.67,151.4) .. (217.67,160.97) .. controls (217.67,170.55) and (209.91,178.31) .. (200.33,178.31) .. controls (190.76,178.31) and (183,170.55) .. (183,160.97) -- cycle ;
	%Right Arrow [id:dp014211227584000286] 
	\draw  [draw opacity=0][fill={rgb, 255:red, 184; green, 184; blue, 184 }  ,fill opacity=1 ] (194.6,205.15) -- (194.6,189.04) -- (188.87,189.04) -- (200.33,178.31) -- (211.8,189.04) -- (206.06,189.04) -- (206.06,205.15) -- cycle ;

	% Text Node
	\draw (111.12,95.39) node    {$T_{2}$};
	% Text Node
	\draw (111.12,224.53) node    {$T_{1}$};
	% Text Node
	\draw (226.37,131.03) node  [font=\small]  {$Q_{2}$};
	% Text Node
	\draw (228,190.87) node  [font=\small]  {$Q_{1}$};
	% Text Node
	\draw (162.13,141.93) node  [font=\small]  {$\mathcal{L}$};
	% Text Node
	\draw (114,128.07) node  [font=\small]  {$Q$};


	\end{tikzpicture}
\end{figure}
\FloatBarrier
Il lavoro compiuto sulla macchina frigorifera, negativo, sarà uguale alla somma dei calori scambiati.
\[
	-\mathcal{L} = Q_1-|Q_2|
\]
Si consideri la macchina complessiva $NK+F$ e si mettano in evidenza i flussi energetici che entrano ed escono.
\begin{figure}[htpb]
	\centering
	

	\tikzset{every picture/.style={line width=0.75pt}} %set default line width to 0.75pt        

	\begin{tikzpicture}[x=0.75pt,y=0.75pt,yscale=-1,xscale=1]
	%uncomment if require: \path (0,300); %set diagram left start at 0, and has height of 300

	%Right Arrow [id:dp5710645086709427] 
	\draw  [draw opacity=0][fill={rgb, 255:red, 184; green, 184; blue, 184 }  ,fill opacity=1 ] (140.06,115.5) -- (140.06,132.38) -- (145.8,132.38) -- (134.33,143.64) -- (122.87,132.38) -- (128.6,132.38) -- (128.6,115.5) -- cycle ;
	%Right Arrow [id:dp1360586020769976] 
	\draw  [draw opacity=0][fill={rgb, 255:red, 184; green, 184; blue, 184 }  ,fill opacity=1 ] (194.6,148) -- (194.6,128.68) -- (188.87,128.68) -- (200.33,115.8) -- (211.8,128.68) -- (206.06,128.68) -- (206.06,148) -- cycle ;
	%Shape: Rectangle [id:dp49710259570510074] 
	\draw  [fill={rgb, 255:red, 155; green, 155; blue, 155 }  ,fill opacity=1 ] (90,78) -- (251.5,78) -- (251.5,115.27) -- (90,115.27) -- cycle ;
	%Shape: Rectangle [id:dp9725838477858029] 
	\draw  [fill={rgb, 255:red, 212; green, 212; blue, 212 }  ,fill opacity=1 ] (90,205.9) -- (251.5,205.9) -- (251.5,243.17) -- (90,243.17) -- cycle ;
	%Right Arrow [id:dp3142970613204923] 
	\draw  [draw opacity=0][fill={rgb, 255:red, 184; green, 184; blue, 184 }  ,fill opacity=1 ] (194.6,205.15) -- (194.6,189.04) -- (188.87,189.04) -- (200.33,178.31) -- (211.8,189.04) -- (206.06,189.04) -- (206.06,205.15) -- cycle ;
	%Rounded Rect [id:dp3923655510821018] 
	\draw  [color={rgb, 255:red, 128; green, 128; blue, 128 }  ,draw opacity=1 ][fill={rgb, 255:red, 184; green, 184; blue, 184 }  ,fill opacity=1 ] (113.5,150.4) .. controls (113.5,146.59) and (116.59,143.5) .. (120.4,143.5) -- (211.35,143.5) .. controls (215.16,143.5) and (218.25,146.59) .. (218.25,150.4) -- (218.25,171.1) .. controls (218.25,174.91) and (215.16,178) .. (211.35,178) -- (120.4,178) .. controls (116.59,178) and (113.5,174.91) .. (113.5,171.1) -- cycle ;

	% Text Node
	\draw (111.12,95.39) node    {$T_{2}$};
	% Text Node
	\draw (111.12,224.53) node    {$T_{1}$};
	% Text Node
	\draw (226.37,131.03) node  [font=\small]  {$Q_{2}$};
	% Text Node
	\draw (228,190.87) node  [font=\small]  {$Q_{1}$};
	% Text Node
	\draw (114,128.07) node  [font=\small]  {$Q$};


	\end{tikzpicture}
\end{figure}
\FloatBarrier
Si avranno un calore $Q$ e un calore $Q_1$ che entrano nella macchina complessiva. Essa non compie un lavoro netto, sull'ambiente esterno, perché tutto il lavoro compiuto da una macchina viene immesso nell'altra macchina. Indicando con il pedice ‘totale' i flussi energetici caratterizzati dalla macchina totale, si ha:
\[
	\mathcal{L}_{\text{tot} } = 0
\]
Il calore totale scambiato con il serbatoio freddo è $Q_1$, positivo. $Q_{1,\text{tot}} = Q_1$. Il calore scambiato con il serbatoio caldo è $Q_{2,\text{tot}} = Q - |Q_2|$.
\[
	\left\{ \begin{array}{l}
	 	\text{macchina 1:}\quad Q = \mathcal{L}   \\
	 	\text{macchina 2:}\quad \mathcal{L} = - Q_1+|Q_2|
	\end{array} \right.
						\implies Q_1 = |Q_2| - Q > 0
\]
Quindi
\[
	Q_{2,\text{tot} } = Q - |Q_2| = - Q_{1,\text{tot}}
\]
Nella macchina totale il calore scambiato con il serbatoio caldo è uguale in modulo a quello scambiato con il serbatoio freddo. Il flusso spontaneo di calore avviene dal serbatoio caldo a quello freddo: negando Kelvin si è ottenuta di conseguenza una macchina che nega l'enunciato di Clausius.

\textbf{Seconda parte.} Si verifica ora che negando Clausius si ottiene la negazione di Kelvin. Si immagini di poter assorbire il calore dal serbatoio freddo al serbatoio caldo con una macchina $NC$. Si affianca ad essa una macchina termica $K$ che lavora a contatto con gli stessi due serbatoi ma che non va a negare Kelvin. Si scelga $K$ in modo tale che assorba una quantità $Q_2$ e ceda una quantità $Q_1=-Q$.
\begin{figure}[htpb]
	\centering
	

	\tikzset{every picture/.style={line width=0.75pt}} %set default line width to 0.75pt        

	\begin{tikzpicture}[x=0.75pt,y=0.75pt,yscale=-1,xscale=1]
	%uncomment if require: \path (0,300); %set diagram left start at 0, and has height of 300

	%Right Arrow [id:dp268809622033948] 
	\draw  [draw opacity=0][fill={rgb, 255:red, 184; green, 184; blue, 184 }  ,fill opacity=1 ] (148.06,177.31) -- (148.06,194.19) -- (153.8,194.19) -- (142.33,205.45) -- (130.87,194.19) -- (136.6,194.19) -- (136.6,177.31) -- cycle ;
	%Right Arrow [id:dp05476736131166793] 
	\draw  [draw opacity=0][fill={rgb, 255:red, 184; green, 184; blue, 184 }  ,fill opacity=1 ] (148.06,115.5) -- (148.06,132.38) -- (153.8,132.38) -- (142.33,143.64) -- (130.87,132.38) -- (136.6,132.38) -- (136.6,115.5) -- cycle ;
	%Right Arrow [id:dp8241569512435416] 
	\draw  [draw opacity=0][fill={rgb, 255:red, 184; green, 184; blue, 184 }  ,fill opacity=1 ] (202.6,148) -- (202.6,128.68) -- (196.87,128.68) -- (208.33,115.8) -- (219.8,128.68) -- (214.06,128.68) -- (214.06,148) -- cycle ;
	%Right Arrow [id:dp7600028656618931] 
	\draw  [draw opacity=0][fill={rgb, 255:red, 184; green, 184; blue, 184 }  ,fill opacity=1 ] (126,166.7) -- (105.15,166.7) -- (105.15,172.44) -- (91.25,160.97) -- (105.15,149.51) -- (105.15,155.24) -- (126,155.24) -- cycle ;
	%Shape: Rectangle [id:dp5660737159746674] 
	\draw  [fill={rgb, 255:red, 155; green, 155; blue, 155 }  ,fill opacity=1 ] (90,78) -- (251.5,78) -- (251.5,115.27) -- (90,115.27) -- cycle ;
	%Shape: Rectangle [id:dp24441080398563697] 
	\draw  [fill={rgb, 255:red, 212; green, 212; blue, 212 }  ,fill opacity=1 ] (90,205.9) -- (251.5,205.9) -- (251.5,243.17) -- (90,243.17) -- cycle ;
	%Shape: Ellipse [id:dp7470281204567384] 
	\draw  [color={rgb, 255:red, 128; green, 128; blue, 128 }  ,draw opacity=1 ][fill={rgb, 255:red, 184; green, 184; blue, 184 }  ,fill opacity=1 ] (125,160.97) .. controls (125,151.4) and (132.76,143.64) .. (142.33,143.64) .. controls (151.91,143.64) and (159.67,151.4) .. (159.67,160.97) .. controls (159.67,170.55) and (151.91,178.31) .. (142.33,178.31) .. controls (132.76,178.31) and (125,170.55) .. (125,160.97) -- cycle ;
	%Shape: Ellipse [id:dp024660173797947715] 
	\draw  [color={rgb, 255:red, 128; green, 128; blue, 128 }  ,draw opacity=1 ][fill={rgb, 255:red, 184; green, 184; blue, 184 }  ,fill opacity=1 ] (191,160.97) .. controls (191,151.4) and (198.76,143.64) .. (208.33,143.64) .. controls (217.91,143.64) and (225.67,151.4) .. (225.67,160.97) .. controls (225.67,170.55) and (217.91,178.31) .. (208.33,178.31) .. controls (198.76,178.31) and (191,170.55) .. (191,160.97) -- cycle ;
	%Right Arrow [id:dp8055782873837063] 
	\draw  [draw opacity=0][fill={rgb, 255:red, 184; green, 184; blue, 184 }  ,fill opacity=1 ] (202.6,205.15) -- (202.6,189.04) -- (196.87,189.04) -- (208.33,178.31) -- (219.8,189.04) -- (214.06,189.04) -- (214.06,205.15) -- cycle ;

	% Text Node
	\draw (111.12,95.39) node    {$T_{2}$};
	% Text Node
	\draw (111.12,224.53) node    {$T_{1}$};
	% Text Node
	\draw (165.03,127.7) node  [font=\small]  {$Q_{2}$};
	% Text Node
	\draw (230,190.87) node  [font=\small]  {$Q$};
	% Text Node
	\draw (110.8,142.6) node  [font=\small]  {$\mathcal{L}$};
	% Text Node
	\draw (165.33,187.4) node  [font=\small]  {$-Q$};
	% Text Node
	\draw (232.67,124.73) node  [font=\small]  {$-Q$};
	% Text Node
	\draw (144.33,160.97) node  [font=\small]  {$2$};
	% Text Node
	\draw (209.33,160.97) node  [font=\small]  {$1$};


	\end{tikzpicture}
\end{figure}
\FloatBarrier
Si scrive il primo principio della termodinamica per la macchina $K$, la quale compie il lavoro $	\mathcal{L}$ positivo, frutto del fatto che va ad assorbire la quantità di calore $Q_2$ e a cedere la quantità di calore $Q$.
\[
	Q_2 - Q = \mathcal{L}
\]
Si consideri ora nuovamente la macchina complessiva, $NC+K$ e se ne mettano in evidenza i flussi energetici complessivi. Essa al netto compie un lavoro positivo.
\[
	Q_2 - Q = \mathcal{L}_\text{tot}
\]
Con il serbatoio freddo al netto non scambia calore perché $Q-Q=0$. Con il serbatoio caldo essa assorbe $Q_2$ e cede $-Q$.
\begin{align*}
	Q_{1,\text{tot}} &= 0 \\
	Q_{2,\text{tot} } &= Q_2 - Q
\end{align*}
Ma quest'ultima quantità è pari proprio al lavoro $\mathcal{L}_\text{tot}$ compiuto dalla macchina. Si è ottenuto che la macchina trasforma tutto il calore in lavoro.
\begin{figure}[htpb]
	\centering
	

	\tikzset{every picture/.style={line width=0.75pt}} %set default line width to 0.75pt        

	\begin{tikzpicture}[x=0.75pt,y=0.75pt,yscale=-1,xscale=1]
	%uncomment if require: \path (0,300); %set diagram left start at 0, and has height of 300

	%Right Arrow [id:dp5160310102780998] 
	\draw  [draw opacity=0][fill={rgb, 255:red, 184; green, 184; blue, 184 }  ,fill opacity=1 ] (147.06,115.5) -- (147.06,132.38) -- (152.8,132.38) -- (141.33,143.64) -- (129.87,132.38) -- (135.6,132.38) -- (135.6,115.5) -- cycle ;
	%Right Arrow [id:dp30761105478774775] 
	\draw  [draw opacity=0][fill={rgb, 255:red, 184; green, 184; blue, 184 }  ,fill opacity=1 ] (201.6,148) -- (201.6,128.68) -- (195.87,128.68) -- (207.33,115.8) -- (218.8,128.68) -- (213.06,128.68) -- (213.06,148) -- cycle ;
	%Right Arrow [id:dp37627454612607414] 
	\draw  [draw opacity=0][fill={rgb, 255:red, 184; green, 184; blue, 184 }  ,fill opacity=1 ] (125,166.7) -- (104.15,166.7) -- (104.15,172.44) -- (90.25,160.97) -- (104.15,149.51) -- (104.15,155.24) -- (125,155.24) -- cycle ;
	%Shape: Rectangle [id:dp007829210220858362] 
	\draw  [fill={rgb, 255:red, 155; green, 155; blue, 155 }  ,fill opacity=1 ] (90,78) -- (251.5,78) -- (251.5,115.27) -- (90,115.27) -- cycle ;
	%Shape: Rectangle [id:dp6803658960832004] 
	\draw  [fill={rgb, 255:red, 212; green, 212; blue, 212 }  ,fill opacity=1 ] (90,205.9) -- (251.5,205.9) -- (251.5,243.17) -- (90,243.17) -- cycle ;
	%Rounded Rect [id:dp9221191705387033] 
	\draw  [color={rgb, 255:red, 128; green, 128; blue, 128 }  ,draw opacity=1 ][fill={rgb, 255:red, 184; green, 184; blue, 184 }  ,fill opacity=1 ] (123.67,150.08) .. controls (123.67,146.15) and (126.85,142.97) .. (130.77,142.97) -- (218.14,142.97) .. controls (222.07,142.97) and (225.25,146.15) .. (225.25,150.08) -- (225.25,171.39) .. controls (225.25,175.32) and (222.07,178.5) .. (218.14,178.5) -- (130.77,178.5) .. controls (126.85,178.5) and (123.67,175.32) .. (123.67,171.39) -- cycle ;

	% Text Node
	\draw (111.12,95.39) node    {$T_{2}$};
	% Text Node
	\draw (111.12,224.53) node    {$T_{1}$};
	% Text Node
	\draw (164.03,127.7) node  [font=\small]  {$Q_{2}$};
	% Text Node
	\draw (113.8,142.6) node  [font=\small]  {$\mathcal{L}$};
	% Text Node
	\draw (231.67,124.73) node  [font=\small]  {$-Q$};
	% Text Node
	\draw (174.46,160.74) node    {$1+2$};


	\end{tikzpicture}
\end{figure}
\FloatBarrier
Negando Clausius si è arrivati alla negazione di Kelvin-Planck. Questi due enunciati che sembrano parlare di cose differenti, sono in realtà equivalenti, dicono qualcosa di molto simile.

\paragraph{Esempio} Si supponga di avere un carrellino che striscia su un piano d'appoggio grazie ad un energia cinetica. Si ha inizialmente un movimento ordinato di tutte le parti del corpo, che si muovono tutte solidali in una direzione. Il piano avrà un certo attrito, quindi dopo un certo tempo il corpo si ferma. L'energia meccanica si è dissipata nel momento in cui esso strisciando ha messo in agitazione le molecole che si trovano sulla superficie di contatto, e che hanno iniziato a vibrare in maniera più rapida. Il primo principio della termodinamica non vieta di pensare che si potrebbe prendere il calore e realizzare in qualche maniera una macchina che permetta di utilizzare il calore ceduto al tavolo per ritrasformarlo in movimento ordinato del carrellino, che ricomincia a muoversi magari indietro, realizzando un moto perpetuo. In realtà si sa che questo processo non è realizzabile. È il secondo principio della termodinamica a porre quindi dei limiti ai passaggi di energia, che possono avvenire solo in certe direzioni: il calore non può mai ritrasformarsi tutto in lavoro. Questo semplice esempio propone allora una visione più generale del secondo principio della termodinamica, secondo la quale vi è sempre una tendenza spontanea dell'energia a passare da una forma più ordinata a una forma più disordinata. Spontaneamente il moto ordinato di particelle si trasforma in moto disordinato.

\paragraph{Osservazione} Perché il termine \emph{unico} che figura nell'enunciato di Clausius è importante? Quando si fa espandere un gas, ad esempio, in maniera isoterma, esso parte da uno stato iniziale, si espande, aumenta il volume e quindi fornisce lavoro all'ambiente circostante, sollevando il pistone. Questo lavoro lo fa assorbendo calore dal serbatoio con cui è a contatto.
Questo è un processo in cui in realtà sfruttando un calore fornito da un serbatoio lo si trasforma tutto in lavoro. Esso tuttavia non viola, come sembrerebbe, il secondo principio della termodinamica. Alla fine del processo infatti non si è più nello stato iniziale, esso è cambiato e se si volesse tornare indietro lungo la stessa trasformazione si avrebbe il passaggio inverso e tutto il lavoro prodotto ritornerebbe in calore. Quindi c'è stato un altro risultato oltre alla trasformazione di calore in lavoro: il passaggio ad un altro stato termodinamico.

\paragraph{Reversibilità e irreversibilità} Finora si è posta l'attenzione, ai fini della reversibilità o irreversibilità di un processo, alle caratteristiche di equilibrio degli stati termodinamici attraversati dal sistema. Si possono estendere tali considerazioni all'ambiente e dire che in generale una trasformazione reversibile non comporta alterazioni permanenti, nel senso che è sempre possibile riportare nei rispettivi stati iniziali il sistema e l'ambiente che con esso interagisce. Al contrario, quando avviene una trasformazione irreversibile, non è più possibile ritornare allo stato di partenza senza modificare il resto dell'universo. Il sistema può anche essere riportato allo stato iniziale attraverso altre trasformazioni, ma l'ambiente subisce una modifica irreversibile. Nella pratica tutte le trasformazioni che avvengono in natura contengono fattori di irreversibilità. La rappresentazione di un fenomeno reale con una trasformazione reversibile costituisce quindi una idealizzazione del processo.







































\section{Il teorema di Carnot}

Il secondo principio della termodinamica, da un punto di vista pratico, assume il significato di affermare che il rendimento di una macchina termica è strettamente minore di uno. Si può affiancare ad esso un teorema che prende il nome di \textbf{teorema di Carnot} che quantifica ancora meglio qual è il limite massimo al rendimento di una certa macchina termica (si tratta di una precisazione quantitativa dell'enunciato di Kelvin-Planck). In particolare si concentra solo su una specifica tipologia di macchine termiche, quelle che scambiano calore con due serbatoi, sia reversibili che irreversibili. Esso afferma che:

\noindent\fbox{%
	\parbox{\textwidth}{%
		\emph{Tutte le macchine reversibili che lavorano tra le stesse sorgenti alle temperature $T_1$ e $T_2$ hanno rendimento eguale; qualsiasi altra macchina che lavori tra le stesse sorgenti non può avere rendimento maggiore. Il risultato è indipendente dal particolare sistema che compie il ciclo}.
	}%
}

Dal momento che la macchina reversibile di Carnot ha come rendimento $1-T_1/T_2$ e che tutte le macchine reversibili operanti fra le stesse sorgenti e quindi fra le stesse temperature hanno rendimento uguale, si avrà che il rendimento di qualsiasi macchina reversibile è dato da:
\[
	\eta_{\text{rev} } = \eta_{\text{Carnot} } = 1 - \frac{T_1 }{T_2 }
\]
Tutte le macchine irreversibili hanno rendimento strettamente inferiore a quello della macchina reversibile operante fra i due serbatoi. Il teorema sta dicendo che si può arrivare a ottenere come rendimento massimo $1-T_1/T_2$.
Insomma, a parità di lavoro fornito, la macchina reversibile è quella che assorbe meno calore. Il teorema si scrive in maniera compatta come:
\[
	\boxed{\eta \le 1-\frac{T_1}{T_2}}
\]

\textbf{Dimostrazione}

La dimostrazione procede per assurdo. Visto che il teorema di Carnot confronta il rendimento di diverse macchine termiche, bisogna immaginare di averne almeno due da confrontare. Si ha la macchina termica $X$ qualunque (non si sa se sia reversibile o irreversibile). La si confronta con una seconda macchina termica $R$ che si ipotizza invece reversibile. Per semplicità di dimostrazione si ipotizza che il lavoro compiuto dalle due macchine sia lo stesso.
\begin{figure}[htpb]
	\centering
	

	\tikzset{every picture/.style={line width=0.75pt}} %set default line width to 0.75pt        

	\begin{tikzpicture}[x=0.75pt,y=0.75pt,yscale=-1,xscale=1]
	%uncomment if require: \path (0,300); %set diagram left start at 0, and has height of 300

	%Right Arrow [id:dp912872186178135] 
	\draw  [draw opacity=0][fill={rgb, 255:red, 184; green, 184; blue, 184 }  ,fill opacity=1 ] (216.25,155.24) -- (237.1,155.24) -- (237.1,149.51) -- (251,160.97) -- (237.1,172.44) -- (237.1,166.7) -- (216.25,166.7) -- cycle ;
	%Right Arrow [id:dp6510841063831212] 
	\draw  [draw opacity=0][fill={rgb, 255:red, 184; green, 184; blue, 184 }  ,fill opacity=1 ] (140.06,177.31) -- (140.06,194.19) -- (145.8,194.19) -- (134.33,205.45) -- (122.87,194.19) -- (128.6,194.19) -- (128.6,177.31) -- cycle ;
	%Right Arrow [id:dp942922084825111] 
	\draw  [draw opacity=0][fill={rgb, 255:red, 184; green, 184; blue, 184 }  ,fill opacity=1 ] (140.06,115.5) -- (140.06,132.38) -- (145.8,132.38) -- (134.33,143.64) -- (122.87,132.38) -- (128.6,132.38) -- (128.6,115.5) -- cycle ;
	%Right Arrow [id:dp1581025718322091] 
	\draw  [draw opacity=0][fill={rgb, 255:red, 184; green, 184; blue, 184 }  ,fill opacity=1 ] (206.06,110.8) -- (206.06,130.12) -- (211.8,130.12) -- (200.33,143) -- (188.87,130.12) -- (194.6,130.12) -- (194.6,110.8) -- cycle ;
	%Right Arrow [id:dp9259895328868126] 
	\draw  [draw opacity=0][fill={rgb, 255:red, 184; green, 184; blue, 184 }  ,fill opacity=1 ] (118,166.7) -- (97.15,166.7) -- (97.15,172.44) -- (83.25,160.97) -- (97.15,149.51) -- (97.15,155.24) -- (118,155.24) -- cycle ;
	%Shape: Rectangle [id:dp6859292289839427] 
	\draw  [fill={rgb, 255:red, 155; green, 155; blue, 155 }  ,fill opacity=1 ] (90,78) -- (251.5,78) -- (251.5,115.27) -- (90,115.27) -- cycle ;
	%Shape: Rectangle [id:dp11183600044342779] 
	\draw  [fill={rgb, 255:red, 212; green, 212; blue, 212 }  ,fill opacity=1 ] (90,205.9) -- (251.5,205.9) -- (251.5,243.17) -- (90,243.17) -- cycle ;
	%Shape: Ellipse [id:dp25979482462928205] 
	\draw  [color={rgb, 255:red, 128; green, 128; blue, 128 }  ,draw opacity=1 ][fill={rgb, 255:red, 184; green, 184; blue, 184 }  ,fill opacity=1 ] (117,160.97) .. controls (117,151.4) and (124.76,143.64) .. (134.33,143.64) .. controls (143.91,143.64) and (151.67,151.4) .. (151.67,160.97) .. controls (151.67,170.55) and (143.91,178.31) .. (134.33,178.31) .. controls (124.76,178.31) and (117,170.55) .. (117,160.97) -- cycle ;
	%Right Arrow [id:dp6597263770006356] 
	\draw  [draw opacity=0][fill={rgb, 255:red, 184; green, 184; blue, 184 }  ,fill opacity=1 ] (206.06,176) -- (206.06,193.49) -- (211.8,193.49) -- (200.33,205.15) -- (188.87,193.49) -- (194.6,193.49) -- (194.6,176) -- cycle ;
	%Shape: Ellipse [id:dp1983125026578465] 
	\draw  [color={rgb, 255:red, 128; green, 128; blue, 128 }  ,draw opacity=1 ][fill={rgb, 255:red, 184; green, 184; blue, 184 }  ,fill opacity=1 ] (183,160.97) .. controls (183,151.4) and (190.76,143.64) .. (200.33,143.64) .. controls (209.91,143.64) and (217.67,151.4) .. (217.67,160.97) .. controls (217.67,170.55) and (209.91,178.31) .. (200.33,178.31) .. controls (190.76,178.31) and (183,170.55) .. (183,160.97) -- cycle ;

	% Text Node
	\draw (111.12,95.39) node    {$T_{2}$};
	% Text Node
	\draw (111.12,224.53) node    {$T_{1}$};
	% Text Node
	\draw (157.03,127.7) node  [font=\small]  {$Q_{2}$};
	% Text Node
	\draw (223,187.87) node  [font=\small]  {$Q'_{1}$};
	% Text Node
	\draw (104.8,144.6) node  [font=\small]  {$\mathcal{L}$};
	% Text Node
	\draw (157.33,187.4) node  [font=\small]  {$Q_{1}$};
	% Text Node
	\draw (222.53,127.2) node  [font=\small]  {$Q'_{2}$};
	% Text Node
	\draw (253.8,148.1) node  [font=\small]  {$\mathcal{L}$};
	% Text Node
	\draw (133.33,159.97) node  [font=\small]  {$X$};
	% Text Node
	\draw (200.33,160.97) node  [font=\small]  {$R$};


	\end{tikzpicture}
\end{figure}
\FloatBarrier
Si scrive per $X$ e $R$ il primo principio della termodinamica:
\[
	\left. \begin{array}{l}
	 	X:\quad \mathcal{L} = Q_2 - |Q_1|  \\
		R:\quad \mathcal{L} = Q_2' - |Q_1'|
	\end{array} \right\}
	\quad Q_2 - |Q_1| = Q_2' - |Q_1'|
\]
Si nega per assurdo che $\eta_x \le \eta_{\text{Carnot}}$. Quindi si ipotizza che $\eta_x > \eta_{\text{Carnot}}$. Ciò significa affermare, dato che $R$ è reversibile, che: $\eta_X > \eta_R$.
\[
	\frac{\mathcal{L} }{Q_2 } > \frac{\mathcal{L} }{Q_2'} \implies Q_2' > Q_2
\]
Se si afferma che $X$ ha rendimento maggiore di $R$ anche se fanno lo stesso lavoro, la prima assorbirà meno calore. Ciò comporta che:
\[
	Q_2-Q_2' < 0 \quad \text{ma} \quad Q_2-Q_2' = |Q_1| - |Q_1'|
\]
E allora:
\[
	|Q_1| - |Q_1'| < 0
\]
Dire che $X$ ha rendimento maggiore di $R$ significa che a pari lavoro fatto bisogna cedere al serbatoio freddo meno calore di quello che cede $R$. Poichè $R$ è reversibile, nessuno vieta di invertirne i flussi energetici, di farla funzionare come un frigo e l'unica cosa che cambia sono i segni di tali flussi. Si vuole realizzare una macchina termica complessiva che sfrutta il lavoro fatto da $X$, lo immette nella macchina $-R$ ribaltato di segno. Si analizza quello che succede alla macchina complessiva formata da $X - R$. Essa non compie un lavoro complessivo perché tutto quello compiuto da $X$ viene assorbito da $-R$. Il calore scambiato con il serbatoio freddo è, dal punto di vista di $X-R$, maggiore di zero. La macchina $X-R$ scambia con il serbatoio freddo un calore $Q_1$ totale che è positivo. La quantità $Q_2$ scambiata con il serbatoio caldo è negativa, quindi è ceduta.
\[
	\mathcal{L}_{\text{tot} } = 0 \qquad Q_{1,\text{tot} } = |Q_1'| - Q > 0 \qquad Q_{2,\text{tot} } = Q_2 - Q_2' < 0
\]
Questo processo non è realizzabile perché nega l'enunciato di Clausius.
\begin{figure}[htpb]
	\centering
	

	\tikzset{every picture/.style={line width=0.75pt}} %set default line width to 0.75pt        

	\begin{tikzpicture}[x=0.75pt,y=0.75pt,yscale=-1,xscale=1]
	%uncomment if require: \path (0,300); %set diagram left start at 0, and has height of 300

	%Right Arrow [id:dp6055110426069534] 
	\draw  [draw opacity=0][fill={rgb, 255:red, 184; green, 184; blue, 184 }  ,fill opacity=1 ] (194.6,208.15) -- (194.6,190.66) -- (188.87,190.66) -- (200.33,179) -- (211.8,190.66) -- (206.06,190.66) -- (206.06,208.15) -- cycle ;
	%Right Arrow [id:dp8251551468017688] 
	\draw  [draw opacity=0][fill={rgb, 255:red, 184; green, 184; blue, 184 }  ,fill opacity=1 ] (147.25,155.24) -- (168.1,155.24) -- (168.1,149.51) -- (182,160.97) -- (168.1,172.44) -- (168.1,166.7) -- (147.25,166.7) -- cycle ;
	%Right Arrow [id:dp05857966804666659] 
	\draw  [draw opacity=0][fill={rgb, 255:red, 184; green, 184; blue, 184 }  ,fill opacity=1 ] (140.06,177.31) -- (140.06,194.19) -- (145.8,194.19) -- (134.33,205.45) -- (122.87,194.19) -- (128.6,194.19) -- (128.6,177.31) -- cycle ;
	%Right Arrow [id:dp8971772094790116] 
	\draw  [draw opacity=0][fill={rgb, 255:red, 184; green, 184; blue, 184 }  ,fill opacity=1 ] (140.06,115.5) -- (140.06,132.38) -- (145.8,132.38) -- (134.33,143.64) -- (122.87,132.38) -- (128.6,132.38) -- (128.6,115.5) -- cycle ;
	%Right Arrow [id:dp3225172431020591] 
	\draw  [draw opacity=0][fill={rgb, 255:red, 184; green, 184; blue, 184 }  ,fill opacity=1 ] (194.6,146) -- (194.6,128) -- (188.87,128) -- (200.33,116) -- (211.8,128) -- (206.06,128) -- (206.06,146) -- cycle ;
	%Shape: Rectangle [id:dp152284570855713] 
	\draw  [fill={rgb, 255:red, 155; green, 155; blue, 155 }  ,fill opacity=1 ] (90,78) -- (251.5,78) -- (251.5,115.27) -- (90,115.27) -- cycle ;
	%Shape: Rectangle [id:dp0038282222062271387] 
	\draw  [fill={rgb, 255:red, 212; green, 212; blue, 212 }  ,fill opacity=1 ] (90,205.9) -- (251.5,205.9) -- (251.5,243.17) -- (90,243.17) -- cycle ;
	%Shape: Ellipse [id:dp07931735472955848] 
	\draw  [color={rgb, 255:red, 128; green, 128; blue, 128 }  ,draw opacity=1 ][fill={rgb, 255:red, 184; green, 184; blue, 184 }  ,fill opacity=1 ] (117,160.97) .. controls (117,151.4) and (124.76,143.64) .. (134.33,143.64) .. controls (143.91,143.64) and (151.67,151.4) .. (151.67,160.97) .. controls (151.67,170.55) and (143.91,178.31) .. (134.33,178.31) .. controls (124.76,178.31) and (117,170.55) .. (117,160.97) -- cycle ;
	%Shape: Ellipse [id:dp4662864673487097] 
	\draw  [color={rgb, 255:red, 128; green, 128; blue, 128 }  ,draw opacity=1 ][fill={rgb, 255:red, 184; green, 184; blue, 184 }  ,fill opacity=1 ] (183,160.97) .. controls (183,151.4) and (190.76,143.64) .. (200.33,143.64) .. controls (209.91,143.64) and (217.67,151.4) .. (217.67,160.97) .. controls (217.67,170.55) and (209.91,178.31) .. (200.33,178.31) .. controls (190.76,178.31) and (183,170.55) .. (183,160.97) -- cycle ;

	% Text Node
	\draw (111.12,95.39) node    {$T_{2}$};
	% Text Node
	\draw (111.12,224.53) node    {$T_{1}$};
	% Text Node
	\draw (112.03,127.7) node  [font=\small]  {$Q_{2}$};
	% Text Node
	\draw (230,189.87) node  [font=\small]  {$|Q'_{1} |$};
	% Text Node
	\draw (112.33,187.4) node  [font=\small]  {$Q_{1}$};
	% Text Node
	\draw (226.53,127.2) node  [font=\small]  {$-Q'_{2}$};
	% Text Node
	\draw (161.8,143.6) node  [font=\small]  {$\mathcal{L}$};
	% Text Node
	\draw (133.33,159.97) node  [font=\small]  {$X$};
	% Text Node
	\draw (200.33,160.97) node  [font=\small]  {$-R$};


	\end{tikzpicture}
\end{figure}
\FloatBarrier
Data una macchina $X$, avrà sempre rendimento minore o uguale di una qualunque macchina reversibile che lavora fra le stesse temperature.
\[
	\boxed{\eta_X \le \eta_{\text{Carnot} }}
\]
Si vuole anche dimostrare che se una macchina è reversibile, il suo rendimento è effettivamente uguale a quello della macchina di Carnot. Per farlo si ripete lo stesso procedimento tenendo conto del fatto che questa volta anche la macchina $X$ è reversibile. Si ipotizza per assurdo che esista una macchina $X$ reversibile che ha rendimento minore della macchina di Carnot.
\begin{figure}[htpb]
	\centering
	

	\tikzset{every picture/.style={line width=0.75pt}} %set default line width to 0.75pt        

	\begin{tikzpicture}[x=0.75pt,y=0.75pt,yscale=-1,xscale=1]
	%uncomment if require: \path (0,300); %set diagram left start at 0, and has height of 300

	%Right Arrow [id:dp06656680772679957] 
	\draw  [draw opacity=0][fill={rgb, 255:red, 184; green, 184; blue, 184 }  ,fill opacity=1 ] (236.25,175.24) -- (257.1,175.24) -- (257.1,169.51) -- (271,180.97) -- (257.1,192.44) -- (257.1,186.7) -- (236.25,186.7) -- cycle ;
	%Right Arrow [id:dp4295737856371209] 
	\draw  [draw opacity=0][fill={rgb, 255:red, 184; green, 184; blue, 184 }  ,fill opacity=1 ] (160.06,197.31) -- (160.06,214.19) -- (165.8,214.19) -- (154.33,225.45) -- (142.87,214.19) -- (148.6,214.19) -- (148.6,197.31) -- cycle ;
	%Right Arrow [id:dp004546700910957879] 
	\draw  [draw opacity=0][fill={rgb, 255:red, 184; green, 184; blue, 184 }  ,fill opacity=1 ] (160.06,135.5) -- (160.06,152.38) -- (165.8,152.38) -- (154.33,163.64) -- (142.87,152.38) -- (148.6,152.38) -- (148.6,135.5) -- cycle ;
	%Right Arrow [id:dp47307158531198845] 
	\draw  [draw opacity=0][fill={rgb, 255:red, 184; green, 184; blue, 184 }  ,fill opacity=1 ] (226.06,130.8) -- (226.06,150.12) -- (231.8,150.12) -- (220.33,163) -- (208.87,150.12) -- (214.6,150.12) -- (214.6,130.8) -- cycle ;
	%Right Arrow [id:dp7261559673782423] 
	\draw  [draw opacity=0][fill={rgb, 255:red, 184; green, 184; blue, 184 }  ,fill opacity=1 ] (138,186.7) -- (117.15,186.7) -- (117.15,192.44) -- (103.25,180.97) -- (117.15,169.51) -- (117.15,175.24) -- (138,175.24) -- cycle ;
	%Shape: Rectangle [id:dp25968617318137155] 
	\draw  [fill={rgb, 255:red, 155; green, 155; blue, 155 }  ,fill opacity=1 ] (110,98) -- (271.5,98) -- (271.5,135.27) -- (110,135.27) -- cycle ;
	%Shape: Rectangle [id:dp0412453343467456] 
	\draw  [fill={rgb, 255:red, 212; green, 212; blue, 212 }  ,fill opacity=1 ] (110,225.9) -- (271.5,225.9) -- (271.5,263.17) -- (110,263.17) -- cycle ;
	%Shape: Ellipse [id:dp0855071978115669] 
	\draw  [color={rgb, 255:red, 128; green, 128; blue, 128 }  ,draw opacity=1 ][fill={rgb, 255:red, 184; green, 184; blue, 184 }  ,fill opacity=1 ] (137,180.97) .. controls (137,171.4) and (144.76,163.64) .. (154.33,163.64) .. controls (163.91,163.64) and (171.67,171.4) .. (171.67,180.97) .. controls (171.67,190.55) and (163.91,198.31) .. (154.33,198.31) .. controls (144.76,198.31) and (137,190.55) .. (137,180.97) -- cycle ;
	%Right Arrow [id:dp5482255905886702] 
	\draw  [draw opacity=0][fill={rgb, 255:red, 184; green, 184; blue, 184 }  ,fill opacity=1 ] (226.06,196) -- (226.06,213.49) -- (231.8,213.49) -- (220.33,225.15) -- (208.87,213.49) -- (214.6,213.49) -- (214.6,196) -- cycle ;
	%Shape: Ellipse [id:dp7061906072557009] 
	\draw  [color={rgb, 255:red, 128; green, 128; blue, 128 }  ,draw opacity=1 ][fill={rgb, 255:red, 184; green, 184; blue, 184 }  ,fill opacity=1 ] (203,180.97) .. controls (203,171.4) and (210.76,163.64) .. (220.33,163.64) .. controls (229.91,163.64) and (237.67,171.4) .. (237.67,180.97) .. controls (237.67,190.55) and (229.91,198.31) .. (220.33,198.31) .. controls (210.76,198.31) and (203,190.55) .. (203,180.97) -- cycle ;

	% Text Node
	\draw (131.12,115.39) node    {$T_{2}$};
	% Text Node
	\draw (131.12,244.53) node    {$T_{1}$};
	% Text Node
	\draw (177.03,147.7) node  [font=\small]  {$Q_{2}$};
	% Text Node
	\draw (243,207.87) node  [font=\small]  {$Q'_{1}$};
	% Text Node
	\draw (124.8,164.6) node  [font=\small]  {$\mathcal{L}$};
	% Text Node
	\draw (177.33,207.4) node  [font=\small]  {$Q_{1}$};
	% Text Node
	\draw (242.53,147.2) node  [font=\small]  {$Q'_{2}$};
	% Text Node
	\draw (273.8,168.1) node  [font=\small]  {$\mathcal{L}$};
	% Text Node
	\draw (153.33,179.97) node  [font=\small]  {$X$};
	% Text Node
	\draw (220.33,180.97) node  [font=\small]  {$R$};


	\end{tikzpicture}
\end{figure}
\FloatBarrier
Si ha allora:
\begin{gather*}
	\frac{\mathcal{L} }{Q_2 } < \frac{\mathcal{L} }{Q_2'} \implies Q_2' < Q_2 \\
	\mathcal{L} = Q_2 - |Q_1| = Q_2' - |Q_1'| \implies \underbrace{Q_2 - Q_2'}_{>0} = \underbrace{|Q_1| - |Q_1'|}_{>0} \\
	|Q_1'| < |Q_1|
\end{gather*}
Visto che anche $X$ è reversibile, si può invertirne il funzionamento cambiando segno ai flussi energetici: $-X$. Si avrà la macchina $R+(-X)$.
\begin{figure}[htpb]
	\centering
	

	\tikzset{every picture/.style={line width=0.75pt}} %set default line width to 0.75pt        

	\begin{tikzpicture}[x=0.75pt,y=0.75pt,yscale=-1,xscale=1]
	%uncomment if require: \path (0,300); %set diagram left start at 0, and has height of 300

	%Right Arrow [id:dp5571594583875663] 
	\draw  [draw opacity=0][fill={rgb, 255:red, 184; green, 184; blue, 184 }  ,fill opacity=1 ] (194.6,208.15) -- (194.6,190.66) -- (188.87,190.66) -- (200.33,179) -- (211.8,190.66) -- (206.06,190.66) -- (206.06,208.15) -- cycle ;
	%Right Arrow [id:dp9701595242869454] 
	\draw  [draw opacity=0][fill={rgb, 255:red, 184; green, 184; blue, 184 }  ,fill opacity=1 ] (147.25,155.24) -- (168.1,155.24) -- (168.1,149.51) -- (182,160.97) -- (168.1,172.44) -- (168.1,166.7) -- (147.25,166.7) -- cycle ;
	%Right Arrow [id:dp5850403174906087] 
	\draw  [draw opacity=0][fill={rgb, 255:red, 184; green, 184; blue, 184 }  ,fill opacity=1 ] (140.06,177.31) -- (140.06,194.19) -- (145.8,194.19) -- (134.33,205.45) -- (122.87,194.19) -- (128.6,194.19) -- (128.6,177.31) -- cycle ;
	%Right Arrow [id:dp9065824176879271] 
	\draw  [draw opacity=0][fill={rgb, 255:red, 184; green, 184; blue, 184 }  ,fill opacity=1 ] (140.06,115.5) -- (140.06,132.38) -- (145.8,132.38) -- (134.33,143.64) -- (122.87,132.38) -- (128.6,132.38) -- (128.6,115.5) -- cycle ;
	%Right Arrow [id:dp09468071438501235] 
	\draw  [draw opacity=0][fill={rgb, 255:red, 184; green, 184; blue, 184 }  ,fill opacity=1 ] (194.6,146) -- (194.6,128) -- (188.87,128) -- (200.33,116) -- (211.8,128) -- (206.06,128) -- (206.06,146) -- cycle ;
	%Shape: Rectangle [id:dp19913563263937073] 
	\draw  [fill={rgb, 255:red, 155; green, 155; blue, 155 }  ,fill opacity=1 ] (90,78) -- (251.5,78) -- (251.5,115.27) -- (90,115.27) -- cycle ;
	%Shape: Rectangle [id:dp6157536168545994] 
	\draw  [fill={rgb, 255:red, 212; green, 212; blue, 212 }  ,fill opacity=1 ] (90,205.9) -- (251.5,205.9) -- (251.5,243.17) -- (90,243.17) -- cycle ;
	%Shape: Ellipse [id:dp9094909762729622] 
	\draw  [color={rgb, 255:red, 128; green, 128; blue, 128 }  ,draw opacity=1 ][fill={rgb, 255:red, 184; green, 184; blue, 184 }  ,fill opacity=1 ] (117,160.97) .. controls (117,151.4) and (124.76,143.64) .. (134.33,143.64) .. controls (143.91,143.64) and (151.67,151.4) .. (151.67,160.97) .. controls (151.67,170.55) and (143.91,178.31) .. (134.33,178.31) .. controls (124.76,178.31) and (117,170.55) .. (117,160.97) -- cycle ;
	%Shape: Ellipse [id:dp8941281519724829] 
	\draw  [color={rgb, 255:red, 128; green, 128; blue, 128 }  ,draw opacity=1 ][fill={rgb, 255:red, 184; green, 184; blue, 184 }  ,fill opacity=1 ] (183,160.97) .. controls (183,151.4) and (190.76,143.64) .. (200.33,143.64) .. controls (209.91,143.64) and (217.67,151.4) .. (217.67,160.97) .. controls (217.67,170.55) and (209.91,178.31) .. (200.33,178.31) .. controls (190.76,178.31) and (183,170.55) .. (183,160.97) -- cycle ;

	% Text Node
	\draw (111.12,95.39) node    {$T_{2}$};
	% Text Node
	\draw (111.12,224.53) node    {$T_{1}$};
	% Text Node
	\draw (112.03,127.7) node  [font=\small]  {$Q'_{2}$};
	% Text Node
	\draw (230,189.87) node  [font=\small]  {$|Q_{1} |$};
	% Text Node
	\draw (112.33,187.4) node  [font=\small]  {$Q'_{1}$};
	% Text Node
	\draw (226.53,127.2) node  [font=\small]  {$-Q_{2}$};
	% Text Node
	\draw (161.8,143.6) node  [font=\small]  {$\mathcal{L}$};
	% Text Node
	\draw (134.33,159.97) node  [font=\small]  {$R$};
	% Text Node
	\draw (200.33,160.97) node  [font=\small]  {$-X$};


	\end{tikzpicture}
\end{figure}
\FloatBarrier
Essa non compie lavoro perché quello realizzato dalla macchina $R$ è usato completamente dalla macchina $-X$. Le quantità
\[
	Q_{2,\text{tot} } = Q_2' - Q_2 < 0 \qquad Q_{1,\text{tot} } = |Q_1| - |Q_1'| > 0
\]
sono rispettivamente una negativa e l'altra positiva. Si sta realizzando una macchina in cui verso il serbatoio caldo arriva calore. La macchina viola l'enunciato di Clausius. Ciò significa che l'ipotesi di partenza $\eta_X< \eta_{\text{Carnot}}$ è assurda e quindi deve essere:
\[
	\eta_X \geq \eta_{\text{Carnot}}
\]
Ma per quanto dimostrato nella prima parte, si ha sempre:
\[
	\eta_X \leq \eta_{\text{Carnot}}
\]
Allora:
\[
	\left\{ \begin{array}{r}
	 	\eta_X \geq \eta_{\text{Carnot}} \\
		\eta_X \leq \eta_{\text{Carnot}}
	\end{array} \right.
	\implies \boxed{\eta_X = \eta_{\text{rev}} = \eta_{\text{Carnot}}}
\]

In sintesi:
\[
	\boxed{\eta_X \le \eta_{\text{Carnot} }\left\{ \begin{array}{l}
	 	\eta_X = \eta_{\text{Carnot} } = 1 - \frac{T_1 }{T_2 }\quad \text{se }X\text{ è reversibile} \\
		\eta_X < \eta_{\text{Carnot} } \quad \text{se }X\text{ è irreversibile}
	\end{array} \right.}
\]
Per dimostrare il teorema è stato usato uno dei due enunciati del secondo principio della termodinamica. Si potrebbe anche negare Clausius e arrivare alla negazione di Carnot. Questo indica che il teorema di Carnot è un altro modo per esprimere (un altro enunciato) il secondo principio della termodinamica ma ha lo stesso significato e porterà alla definizione della grandezza entropia.

\paragraph{Rendimento della macchina di Stirling} Si consideri la macchina di Stirling irreversibile. Si può sicuramente dire che il rendimento della macchina irreversibile è minore di $1-T_1/T_2$. Se lo stesso ciclo lo si esegue in maniera reversibile ci si può chiedere se questa macchina scambi effettivamente calore con solo due serbatoi. A occhio sembra di sì perché si hanno due isoterme a temperature diverse. Ma bisogna ragionare meglio su quello che accade nelle isocore.
\begin{figure}[htpb]
	\centering
	

	\tikzset{every picture/.style={line width=0.75pt}} %set default line width to 0.75pt        

	\begin{tikzpicture}[x=0.75pt,y=0.75pt,yscale=-1,xscale=1]
	%uncomment if require: \path (0,300); %set diagram left start at 0, and has height of 300

	%Shape: Axis 2D [id:dp5586747872942945] 
	\draw  (112,195.1) -- (343.5,195.1)(135.15,25) -- (135.15,214) (336.5,190.1) -- (343.5,195.1) -- (336.5,200.1) (130.15,32) -- (135.15,25) -- (140.15,32)  ;
	%Curve Lines [id:da9764359194008205] 
	\draw    (176.57,57.92) .. controls (191.27,78.63) and (240.8,88.92) .. (274.87,87.58) ;
	%Straight Lines [id:da42446695351819175] 
	\draw    (176.57,57.92) -- (176.57,132.37) ;
	%Curve Lines [id:da8065092271443957] 
	\draw    (176.57,132.37) .. controls (191.27,153.08) and (240.8,163.37) .. (274.87,162.03) ;
	%Straight Lines [id:da8598863477808285] 
	\draw    (274.87,87.58) -- (274.87,162.03) ;
	%Straight Lines [id:da8868732384234441] 
	\draw    (227.53,54.11) .. controls (229.2,55.78) and (229.2,57.44) .. (227.53,59.11) .. controls (225.86,60.78) and (225.86,62.44) .. (227.53,64.11) .. controls (229.2,65.78) and (229.2,67.44) .. (227.53,69.11) .. controls (225.86,70.78) and (225.86,72.44) .. (227.53,74.11) .. controls (229.2,75.78) and (229.2,77.44) .. (227.53,79.11) .. controls (225.86,80.78) and (225.86,82.44) .. (227.53,84.11) .. controls (229.2,85.78) and (229.2,87.44) .. (227.53,89.11) -- (227.53,93.22) -- (227.53,96.22)(224.53,54.11) .. controls (226.2,55.78) and (226.2,57.44) .. (224.53,59.11) .. controls (222.86,60.78) and (222.86,62.44) .. (224.53,64.11) .. controls (226.2,65.78) and (226.2,67.44) .. (224.53,69.11) .. controls (222.86,70.78) and (222.86,72.44) .. (224.53,74.11) .. controls (226.2,75.78) and (226.2,77.44) .. (224.53,79.11) .. controls (222.86,80.78) and (222.86,82.44) .. (224.53,84.11) .. controls (226.2,85.78) and (226.2,87.44) .. (224.53,89.11) -- (224.53,93.22) -- (224.53,96.22) ;
	\draw [shift={(226.03,103.22)}, rotate = 270] [color={rgb, 255:red, 0; green, 0; blue, 0 }  ][line width=0.75]    (10.93,-4.9) .. controls (6.95,-2.3) and (3.31,-0.67) .. (0,0) .. controls (3.31,0.67) and (6.95,2.3) .. (10.93,4.9)   ;

	% Text Node
	\draw (118.67,25) node    {$p$};
	% Text Node
	\draw (360,200) node    {$V$};
	% Text Node
	\draw (167.85,50.07) node    {$A$};
	% Text Node
	\draw (284.76,81.48) node    {$B$};
	% Text Node
	\draw (284.47,164.94) node    {$C$};
	% Text Node
	\draw (168.72,142.26) node    {$D$};
	% Text Node
	\draw (220.53,171.99) node    {$T_{1}$};
	% Text Node
	\draw (253.53,68.99) node    {$T_{2}$};


	\end{tikzpicture}
\end{figure}
\FloatBarrier
Si immagini di vedere come effettuare il raffreddamento isocoro da $B$ a $C$. Il gas è arrivato a espandersi a contatto con il serbatoio $T_2$. Non si può mettere il gas direttamente a contatto con il serbatoio $T_1$. Si immagini di effettuare il riscaldamento isocoro molto lentamente. Per fare ciò, invece che cambiare il serbatoio $T_2$ con il serbatoio $T_1$ in una sola volta, si mette il gas via via a contatto con un numero infinito di serbatoi a temperature via via sempre più basse, in modo che il passaggio da una temperatura all'altra sia continuo. Ecco allora che la macchina di Stirling reversibile non scambia calore con solo due serbatoi ma con un'infinità di essi. Di conseguenza, non si può confrontare il suo rendimento con quello della macchina di Carnot, che risulta essere l'unica macchina reversibile la quale scambia calore con due soli serbatoi. Il teorema di Carnot parla di macchine che operano fra \emph{due} sorgenti e, se il loro numero è maggiore, allora il rendimento è minore di $1-T_1/T_2$.







































\section{Il teorema di Clausius}

Ci sono altri tipi di macchine di Carnot che operano scambiando calore con tre serbatoi. Si tratta di macchine di Carnot costituite da tre isoterme e tre adiabatiche e il loro rendimento \emph{non} è pari a $1-T_\text{min}/T_\text{max}$. Si può estendere quindi il teorema di Carnot al caso di macchine che scambiano calore con un numero $n$ di serbatoi e poi a quelle che scambiano calore con un numero infinito di serbatoi.
\begin{figure}[htpb]
	\centering
	

	\tikzset{every picture/.style={line width=0.75pt}} %set default line width to 0.75pt        

	\begin{tikzpicture}[x=0.75pt,y=0.75pt,yscale=-1,xscale=1]
	%uncomment if require: \path (0,300); %set diagram left start at 0, and has height of 300

	%Right Arrow [id:dp06422156982176808] 
	\draw  [draw opacity=0][fill={rgb, 255:red, 184; green, 184; blue, 184 }  ,fill opacity=1 ] (177.25,155.24) -- (198.1,155.24) -- (198.1,149.51) -- (212,160.97) -- (198.1,172.44) -- (198.1,166.7) -- (177.25,166.7) -- cycle ;
	%Right Arrow [id:dp7080624294420244] 
	\draw  [draw opacity=0][fill={rgb, 255:red, 184; green, 184; blue, 184 }  ,fill opacity=1 ] (170.06,177.31) -- (170.06,194.19) -- (175.8,194.19) -- (164.33,205.45) -- (152.87,194.19) -- (158.6,194.19) -- (158.6,177.31) -- cycle ;
	%Right Arrow [id:dp5503612951569008] 
	\draw  [draw opacity=0][fill={rgb, 255:red, 184; green, 184; blue, 184 }  ,fill opacity=1 ] (170.06,115.5) -- (170.06,132.38) -- (175.8,132.38) -- (164.33,143.64) -- (152.87,132.38) -- (158.6,132.38) -- (158.6,115.5) -- cycle ;
	%Shape: Rectangle [id:dp41867474584744224] 
	\draw  [fill={rgb, 255:red, 155; green, 155; blue, 155 }  ,fill opacity=1 ] (90,78) -- (251.5,78) -- (251.5,115.27) -- (90,115.27) -- cycle ;
	%Shape: Rectangle [id:dp8775919710234561] 
	\draw  [fill={rgb, 255:red, 212; green, 212; blue, 212 }  ,fill opacity=1 ] (90,205.9) -- (251.5,205.9) -- (251.5,243.17) -- (90,243.17) -- cycle ;
	%Shape: Ellipse [id:dp41152320935778564] 
	\draw  [color={rgb, 255:red, 128; green, 128; blue, 128 }  ,draw opacity=1 ][fill={rgb, 255:red, 184; green, 184; blue, 184 }  ,fill opacity=1 ] (147,160.97) .. controls (147,151.4) and (154.76,143.64) .. (164.33,143.64) .. controls (173.91,143.64) and (181.67,151.4) .. (181.67,160.97) .. controls (181.67,170.55) and (173.91,178.31) .. (164.33,178.31) .. controls (154.76,178.31) and (147,170.55) .. (147,160.97) -- cycle ;

	% Text Node
	\draw (111.12,95.39) node    {$T_{2}$};
	% Text Node
	\draw (111.12,224.53) node    {$T_{1}$};
	% Text Node
	\draw (142.03,127.7) node  [font=\small]  {$Q_{2}$};
	% Text Node
	\draw (142.33,187.4) node  [font=\small]  {$Q_{1}$};
	% Text Node
	\draw (215.8,144.1) node  [font=\small]  {$\mathcal{L}$};


	\end{tikzpicture}
\end{figure}
\FloatBarrier
Il teorema di Carnot per $n=2$ serbatoi è dato da:
\[
	\eta_X = 1 - \frac{|Q_1|}{Q_2 } \le 1 - \frac{T_1 }{T_2 }
\]
Ricordando che $Q_1<0$:
\[
	\frac{Q_1 }{Q_2 } \le - \frac{T_1 }{T_2 }
\]
La relazione sta dicendo che, ricordandosi che le temperature sono sempre positive e che $Q_2$ è positivo, $Q_1$ è negativo, se la macchina $X$ è reversibile allora:
\begin{gather*}
	T_1>0 \implies \frac{Q_1 }{T_1 Q_2} \le - \frac{1}{T_2} \\
	Q_2>0 \implies \frac{Q_1 }{T_1 } \le - \frac{Q_2 }{T_2 } \\
	\frac{Q_1 }{T_1} + \frac{Q_2 }{T_2 } \le 0
\end{gather*}
Se la macchina $X$ è irreversibile il primo a termine è più negativo di quanto il secondo è positivo.
\[
	\frac{Q_1 }{T_1} + \frac{Q_2 }{T_2 } = 0 \quad (X\text{ rev} ) \qquad \frac{Q_1 }{T_1} + \frac{Q_2 }{T_2 } < 0 \quad (X\text{ irrev} )
\]
A questo punto tale risultato si estende al caso $n>2$ dicendo la sommatoria estesa a tutti gli $n$ serbatoi con cui la macchina ciclica sta scambiando calore, delle quantità $Q_i/T_i$ deve essere minore o al limite uguale a $0$. L'uguale vale se $X$ è reversibile, il minore se è irreversibile.
\begin{figure}[htpb]
	\centering
	

	\tikzset{every picture/.style={line width=0.75pt}} %set default line width to 0.75pt        

	\begin{tikzpicture}[x=0.75pt,y=0.75pt,yscale=-1,xscale=1]
	%uncomment if require: \path (0,300); %set diagram left start at 0, and has height of 300

	%Right Arrow [id:dp2833606865986553] 
	\draw  [draw opacity=0][fill={rgb, 255:red, 184; green, 184; blue, 184 }  ,fill opacity=1 ] (217,166.7) -- (196.15,166.7) -- (196.15,172.44) -- (182.25,160.97) -- (196.15,149.51) -- (196.15,155.24) -- (217,155.24) -- cycle ;
	%Right Arrow [id:dp24042892933630888] 
	\draw  [draw opacity=0][fill={rgb, 255:red, 184; green, 184; blue, 184 }  ,fill opacity=1 ] (158.6,205.45) -- (158.6,188.56) -- (152.87,188.56) -- (164.33,177.31) -- (175.8,188.56) -- (170.06,188.56) -- (170.06,205.45) -- cycle ;
	%Right Arrow [id:dp784876077397787] 
	\draw  [draw opacity=0][fill={rgb, 255:red, 184; green, 184; blue, 184 }  ,fill opacity=1 ] (170.06,115.5) -- (170.06,132.38) -- (175.8,132.38) -- (164.33,143.64) -- (152.87,132.38) -- (158.6,132.38) -- (158.6,115.5) -- cycle ;
	%Shape: Rectangle [id:dp6297204887164671] 
	\draw  [color={rgb, 255:red, 128; green, 128; blue, 128 }  ,draw opacity=1 ][fill={rgb, 255:red, 184; green, 184; blue, 184 }  ,fill opacity=1 ] (145,205.9) -- (184.75,205.9) -- (184.75,228.5) -- (145,228.5) -- cycle ;
	%Shape: Ellipse [id:dp16335549952306083] 
	\draw  [color={rgb, 255:red, 128; green, 128; blue, 128 }  ,draw opacity=1 ][fill={rgb, 255:red, 184; green, 184; blue, 184 }  ,fill opacity=1 ] (147,160.97) .. controls (147,151.4) and (154.76,143.64) .. (164.33,143.64) .. controls (173.91,143.64) and (181.67,151.4) .. (181.67,160.97) .. controls (181.67,170.55) and (173.91,178.31) .. (164.33,178.31) .. controls (154.76,178.31) and (147,170.55) .. (147,160.97) -- cycle ;
	%Right Arrow [id:dp4503207186581166] 
	\draw  [draw opacity=0][fill={rgb, 255:red, 184; green, 184; blue, 184 }  ,fill opacity=1 ] (112.25,155.24) -- (133.1,155.24) -- (133.1,149.51) -- (147,160.97) -- (133.1,172.44) -- (133.1,166.7) -- (112.25,166.7) -- cycle ;
	%Shape: Rectangle [id:dp7742211356142767] 
	\draw  [color={rgb, 255:red, 128; green, 128; blue, 128 }  ,draw opacity=1 ][fill={rgb, 255:red, 184; green, 184; blue, 184 }  ,fill opacity=1 ] (145,92.9) -- (184.75,92.9) -- (184.75,115.5) -- (145,115.5) -- cycle ;
	%Shape: Rectangle [id:dp2203924120705476] 
	\draw  [color={rgb, 255:red, 128; green, 128; blue, 128 }  ,draw opacity=1 ][fill={rgb, 255:red, 184; green, 184; blue, 184 }  ,fill opacity=1 ] (240.18,141.32) -- (240.18,181.07) -- (217.57,181.07) -- (217.57,141.32) -- cycle ;
	%Shape: Rectangle [id:dp2882355114293944] 
	\draw  [color={rgb, 255:red, 128; green, 128; blue, 128 }  ,draw opacity=1 ][fill={rgb, 255:red, 184; green, 184; blue, 184 }  ,fill opacity=1 ] (112.18,141.32) -- (112.18,181.07) -- (89.57,181.07) -- (89.57,141.32) -- cycle ;

	% Text Node
	\draw (164.88,216.7) node    {$T_{n-1}$};
	% Text Node
	\draw (196.3,137.1) node  [font=\small]  {$\mathcal{L}$};
	% Text Node
	\draw (164.88,103.7) node    {$T_{1}$};
	% Text Node
	\draw (228.88,160.7) node    {$T_{n}$};
	% Text Node
	\draw (100.88,160.7) node    {$T_{2}$};


	\end{tikzpicture}
\end{figure}
\FloatBarrier
Si arriva così al \textbf{teorema di Clausius} il quale afferma che, data una macchina $M$ qualsiasi che scambia calore con $n$ sorgenti, si ha:
\[
	\sum_{i=1}^N \frac{Q_i }{T_i } \le 0
\]
Si vuole scrivere la stessa relazione quando $n$ tende a infinito perché si passa per successivi e adiacenti stati di equilibrio termico. Si fa una somma continua estesa a tutti i serbatoi della quantità di calore infinitesima scambiata dal gas lungo i continui stati di equilibrio con il serbatoio alla generica temperatura $T_i$.
\[
	\oint \frac{dQ}{T} \le 0 \quad \text{per} \quad N\to +\infty
\]
L'integrale è uguale a zero se il ciclo è reversibile, minore se il ciclo è irreversibile.
\begin{gather*}
	\oint \frac{dQ_{\text{rev} } }{T} = 0 \quad \text{Teorema di Clausius per macchine reversibili} \\
	\oint \frac{dQ_{\text{irr} } }{T} < 0 \quad \text{Teorema di Clausius per macchine irreversibili} \\
\end{gather*}
Questa relazione dice che lungo un ciclo totalmente reversibile la somma dei calori scambiati su $T$ si bilanciano tutti. La disuguaglianza è nota come \textbf{disuguaglianza di Clausius}.







































\section{L'entropia}

Si supponga di avere una trasformazione ciclica $\Gamma$ costituita da due tratti $\Gamma_1$ e $\Gamma_2$.
\begin{figure}[htpb]
	\centering
	

	\tikzset{every picture/.style={line width=0.75pt}} %set default line width to 0.75pt        

	\begin{tikzpicture}[x=0.75pt,y=0.75pt,yscale=-1,xscale=1]
	%uncomment if require: \path (0,300); %set diagram left start at 0, and has height of 300

	%Shape: Axis 2D [id:dp1802588386155528] 
	\draw  (70,222.3) -- (317.5,222.3)(94.75,72) -- (94.75,239) (310.5,217.3) -- (317.5,222.3) -- (310.5,227.3) (89.75,79) -- (94.75,72) -- (99.75,79)  ;
	%Shape: Polygon Curved [id:ds7126263079707398] 
	\draw   (155.5,100.5) .. controls (175.5,90.5) and (265.5,80.5) .. (245.5,100.5) .. controls (225.5,120.5) and (229.75,144) .. (249.75,174) .. controls (269.75,204) and (175.5,190.5) .. (155.5,160.5) .. controls (135.5,130.5) and (135.5,110.5) .. (155.5,100.5) -- cycle ;
	%Shape: Circle [id:dp18230697087752912] 
	\draw  [fill={rgb, 255:red, 0; green, 0; blue, 0 }  ,fill opacity=1 ] (158.25,166.5) .. controls (158.25,165.53) and (159.03,164.75) .. (160,164.75) .. controls (160.97,164.75) and (161.75,165.53) .. (161.75,166.5) .. controls (161.75,167.47) and (160.97,168.25) .. (160,168.25) .. controls (159.03,168.25) and (158.25,167.47) .. (158.25,166.5) -- cycle ;
	%Shape: Circle [id:dp9179639156116408] 
	\draw  [fill={rgb, 255:red, 0; green, 0; blue, 0 }  ,fill opacity=1 ] (246.25,95) .. controls (246.25,94.03) and (247.03,93.25) .. (248,93.25) .. controls (248.97,93.25) and (249.75,94.03) .. (249.75,95) .. controls (249.75,95.97) and (248.97,96.75) .. (248,96.75) .. controls (247.03,96.75) and (246.25,95.97) .. (246.25,95) -- cycle ;

	% Text Node
	\draw (80,73) node    {$p$};
	% Text Node
	\draw (331.33,221.33) node    {$V$};
	% Text Node
	\draw (149,174) node    {$A$};
	% Text Node
	\draw (261.33,87.33) node    {$B$};
	% Text Node
	\draw (156.17,83.33) node    {$\Gamma _{1}$};
	% Text Node
	\draw (267.5,187.33) node    {$\Gamma _{2}$};


	\end{tikzpicture}
\end{figure}
\FloatBarrier
È un ciclo reversibile, per cui si può imporre la validità dell'uguaglianza di Clausius:
\[
	\int_{\Gamma_1, A\to B} \frac{dQ_{\text{rev} } }{T} + \int_{\Gamma_2, B\to A} \frac{dQ_{\text{rev} } }{T} = 0
\]
Poiché la trasformazione $\Gamma_2$ è reversibile, è possibile ribaltarla cambiandone il segno. Sono state realizzate così due trasformazioni diverse che partono dallo stesso stato $A$ iniziale e arrivano allo stesso stato $B$ finale.
\[
	\int_{\Gamma_1, A\to B} \frac{dQ_{\text{rev} } }{T} - \int_{-\Gamma_2, A\to B} \frac{dQ_{\text{rev} } }{T} = 0
\]
Allora:
\[
	\int_{\Gamma_1, A\to B} \frac{dQ_{\text{rev} } }{T} = \int_{-\Gamma_2, A\to B} \frac{dQ_{\text{rev} } }{T}
\]
Quindi date due trasformazioni diverse aventi stato iniziale e finale in comune, la quantità $\int dQ_\text{rev}/T$ è esattamente la stessa. Questo significa che essa non dipende dal particolare percorso, ma solo dallo stato iniziale e finale. Analogamente a quanto visto per l'energia potenziale, esiste allora sempre una funzione scalare, dello stato termodinamico in cui si trova il sistema, che dimensionalmente avrà la dimensione di un'energia su una temperatura, tale per cui:
\[
	\int_A^B \frac{dQ_{\text{rev} } }{T} = f(B) - f(A)
\]
A questa particolare funzione di stato viene dato il nome di \textbf{entropia} e là si indica con la lettera $S$.
La grandezza entropia è una funzione di stato esattamente come lo è l'energia interna di un certo sistema termodinamico, è definita a meno di una costante additiva, ma non ha importanza perché si è interessati alla sua variazione.
\[
	S(p,V,T,n) \quad \text{funzione di stato}
\]
Se si va da uno stato $A$ ad uno stato $B$ con una trasformazione irreversibile, si può comunque usare la definizione di entropia. $\Delta S$ sarà lo stesso di quello che si ottiene quando si fa una trasformazione reversibile. Si prende una qualunque trasformazione che porta da $A$ a $B$ e si considera il calore scambiato il maniera reversibile.
\[
	\int_A^B \frac{dQ_{\text{rev} } }{T} = S(B) - S(A)
\]

\paragraph{$\Delta S$ per una massa che varia temperatura} Si immagini di avere una certa massa caratterizzata dal calore specifico $c$ che inizialmente si trova alla temperatura $T_1$ e che, per contatto termico con un altro corpo, viene scaldata o raffreddata fino ad arrivare alla temperatura $T_2$. Si deve valutare la variazione di entropia di questo sistema termodinamico. Il processo è irreversibile, non passa per continui stati di equilibrio.
\begin{figure}[htpb]
	\centering
	

	\tikzset{every picture/.style={line width=0.75pt}} %set default line width to 0.75pt        

	\begin{tikzpicture}[x=0.75pt,y=0.75pt,yscale=-1,xscale=1]
	%uncomment if require: \path (0,300); %set diagram left start at 0, and has height of 300

	%Shape: Polygon [id:dp12394600880731876] 
	\draw   (142.86,124.23) -- (104.37,124.23) -- (92.48,87.62) -- (123.62,65) -- (154.75,87.62) -- cycle ;
	%Shape: Polygon [id:dp5204589102411801] 
	\draw   (247.63,124.23) -- (209.14,124.23) -- (197.25,87.62) -- (228.38,65) -- (259.52,87.62) -- cycle ;
	%Shape: Circle [id:dp7037865140831645] 
	\draw   (101,155) .. controls (101,147.27) and (107.27,141) .. (115,141) .. controls (122.73,141) and (129,147.27) .. (129,155) .. controls (129,162.73) and (122.73,169) .. (115,169) .. controls (107.27,169) and (101,162.73) .. (101,155) -- cycle ;
	%Shape: Circle [id:dp8431154371971894] 
	\draw   (161,155) .. controls (161,147.27) and (167.27,141) .. (175,141) .. controls (182.73,141) and (189,147.27) .. (189,155) .. controls (189,162.73) and (182.73,169) .. (175,169) .. controls (167.27,169) and (161,162.73) .. (161,155) -- cycle ;
	%Shape: Circle [id:dp8473592675610118] 
	\draw   (221,155) .. controls (221,147.27) and (227.27,141) .. (235,141) .. controls (242.73,141) and (249,147.27) .. (249,155) .. controls (249,162.73) and (242.73,169) .. (235,169) .. controls (227.27,169) and (221,162.73) .. (221,155) -- cycle ;
	%Straight Lines [id:da7807212811453503] 
	\draw    (129,155) -- (158,155) ;
	\draw [shift={(161,155)}, rotate = 180] [fill={rgb, 255:red, 0; green, 0; blue, 0 }  ][line width=0.08]  [draw opacity=0] (10.72,-5.15) -- (0,0) -- (10.72,5.15) -- (7.12,0) -- cycle    ;
	%Straight Lines [id:da16236224334225202] 
	\draw    (189,155) -- (218,155) ;
	\draw [shift={(221,155)}, rotate = 180] [fill={rgb, 255:red, 0; green, 0; blue, 0 }  ][line width=0.08]  [draw opacity=0] (10.72,-5.15) -- (0,0) -- (10.72,5.15) -- (7.12,0) -- cycle    ;

	% Text Node
	\draw (126,97) node    {$m\ c\ T_{1}$};
	% Text Node
	\draw (228,97) node    {$m\ c\ T_{2}$};
	% Text Node
	\draw (114.67,181.67) node    {$T_{1}$};
	% Text Node
	\draw (174.67,181.67) node    {$T_{1} -dT$};
	% Text Node
	\draw (236.67,181.67) node    {$T_{2}$};


	\end{tikzpicture}
\end{figure}
\FloatBarrier
Per il calcolo della variazione di entropia nelle trasformazioni irreversibili, basta scegliere una qualsiasi trasformazione reversibile che colleghi $A$ e $B$ e applicare a questa la definizione. Il risultato è valido in ogni caso. Si immagina quindi di raffreddare l'acqua mettendola a contatto con serbatoi di calore di temperature molto prossime tra di loro, anche se nella realtà il processo non sta avvenendo in questo modo, che piano piano abbassano la temperatura dell'acqua da $T_1$ a $T_2$.
Mentre si somma in maniera continua tutta la quantità, bisogna valutare cosa accade dallo stato iniziale allo stato finale. Si ottiene:
\begin{gather*}
	\underbrace{S(B) - S(A)}_{\Delta S} = \int_A^B \frac{dQ_{\text{rev} } }{T} = \int_{T_1 }^{T_2 } \frac{mcdT}{T} = mc\,\log \frac{T_2 }{T_1 } \\
	\Delta S \left\{ \begin{array}{r}
		>0 \quad \text{se} \quad T_2>T_1 \\
		<0 \quad \text{se} \quad T_2<T_1
	\end{array} \right.
\end{gather*}
Si noti che l'integrale non è il calore scambiato dalla massa su $T$.
\[
	Q_{\text{irrev}} = mc\Delta T = mc(T_2-T_1) \neq Q_{\text{rev}}
\]

\paragraph{$\Delta S$ per un serbatoio} Si supponga che la massa inizialmente alla temperatura $T_1$ sia stata portata a temperatura $T_2$ mettendola a contatto con un serbatoio freddo, che va ad assorbire calore dalla massa.
\begin{figure}[htpb]
	\centering
	

	\tikzset{every picture/.style={line width=0.75pt}} %set default line width to 0.75pt        

	\begin{tikzpicture}[x=0.75pt,y=0.75pt,yscale=-1,xscale=1]
	%uncomment if require: \path (0,300); %set diagram left start at 0, and has height of 300

	%Shape: Rectangle [id:dp7163828634385068] 
	\draw   (110.67,72.33) -- (237,72.33) -- (237,112.33) -- (110.67,112.33) -- cycle ;
	%Shape: Circle [id:dp38214896965195333] 
	\draw  [fill={rgb, 255:red, 184; green, 184; blue, 184 }  ,fill opacity=1 ] (155.17,72.33) .. controls (155.17,62.02) and (163.52,53.67) .. (173.83,53.67) .. controls (184.14,53.67) and (192.5,62.02) .. (192.5,72.33) .. controls (192.5,82.64) and (184.14,91) .. (173.83,91) .. controls (163.52,91) and (155.17,82.64) .. (155.17,72.33) -- cycle ;
	%Straight Lines [id:da7561408389051725] 
	\draw    (175.33,72.33) .. controls (177,74) and (177,75.66) .. (175.33,77.33) .. controls (173.66,79) and (173.66,80.66) .. (175.33,82.33) .. controls (177,84) and (177,85.66) .. (175.33,87.33) .. controls (173.66,89) and (173.66,90.66) .. (175.33,92.33) -- (175.33,94) -- (175.33,97)(172.33,72.33) .. controls (174,74) and (174,75.66) .. (172.33,77.33) .. controls (170.66,79) and (170.66,80.66) .. (172.33,82.33) .. controls (174,84) and (174,85.66) .. (172.33,87.33) .. controls (170.66,89) and (170.66,90.66) .. (172.33,92.33) -- (172.33,94) -- (172.33,97) ;
	\draw [shift={(173.83,106)}, rotate = 270] [fill={rgb, 255:red, 0; green, 0; blue, 0 }  ][line width=0.08]  [draw opacity=0] (10.72,-5.15) -- (0,0) -- (10.72,5.15) -- (7.12,0) -- cycle    ;
	%Shape: Rectangle [id:dp40895364106028165] 
	\draw   (270.67,72.33) -- (397,72.33) -- (397,112.33) -- (270.67,112.33) -- cycle ;
	%Shape: Circle [id:dp18749112210277175] 
	\draw  [fill={rgb, 255:red, 184; green, 184; blue, 184 }  ,fill opacity=1 ] (315.17,72.33) .. controls (315.17,62.02) and (323.52,53.67) .. (333.83,53.67) .. controls (344.14,53.67) and (352.5,62.02) .. (352.5,72.33) .. controls (352.5,82.64) and (344.14,91) .. (333.83,91) .. controls (323.52,91) and (315.17,82.64) .. (315.17,72.33) -- cycle ;
	%Straight Lines [id:da3176905019687204] 
	\draw    (335.33,72.33) .. controls (337,74) and (337,75.66) .. (335.33,77.33) .. controls (333.66,79) and (333.66,80.66) .. (335.33,82.33) .. controls (337,84) and (337,85.66) .. (335.33,87.33) .. controls (333.66,89) and (333.66,90.66) .. (335.33,92.33) -- (335.33,94) -- (335.33,97)(332.33,72.33) .. controls (334,74) and (334,75.66) .. (332.33,77.33) .. controls (330.66,79) and (330.66,80.66) .. (332.33,82.33) .. controls (334,84) and (334,85.66) .. (332.33,87.33) .. controls (330.66,89) and (330.66,90.66) .. (332.33,92.33) -- (332.33,94) -- (332.33,97) ;
	\draw [shift={(333.83,106)}, rotate = 270] [fill={rgb, 255:red, 0; green, 0; blue, 0 }  ][line width=0.08]  [draw opacity=0] (10.72,-5.15) -- (0,0) -- (10.72,5.15) -- (7.12,0) -- cycle    ;

	% Text Node
	\draw (173,120.67) node   [align=left] {serbatoio};
	% Text Node
	\draw (224.67,99.33) node    {$A$};
	% Text Node
	\draw (176.67,40) node    {$m\ c\ T_{1}$};
	% Text Node
	\draw (123.33,98) node    {$T_{2}$};
	% Text Node
	\draw (333,120.67) node   [align=left] {serbatoio};
	% Text Node
	\draw (384.67,99.33) node    {$B$};
	% Text Node
	\draw (336.67,40) node    {$m\ c\ T_{2}$};
	% Text Node
	\draw (283.33,98) node    {$T_{2}$};


	\end{tikzpicture}
\end{figure}
\FloatBarrier
Si calcola la variazione di entropia \emph{del serbatoio} mentre $m$ passa da questo stato iniziale a questo stato finale. Esso assorbe calore dalla massa molto rapidamente, quindi non in maniera reversibile. Quello che si può fare è immaginare di fare avvenire il processo molto lentamente. Questa volta però esso rimane alla temperatura $T_2$.
\[
	\Delta S_{\text{serbatoio}} = \int_A^B \frac{dQ_{\text{rev}} }{T_2} = \frac{\int_A^B dQ_{\text{rev} } }{T_2 } = \frac{Q_{\text{ass, serb }2 } }{T_2}
\]
Il calore assorbito è esattamente quello ceduto dalla massa: $mc\Delta T$. Per i serbatoi, la variazione di entropia è sempre il calore scambiato fratto la temperatura a cui opera, perché essa è costante.

Si immagini di avere questo esempio in cui il serbatoio assorbe calore e quindi esso aumenta la sua entropia.
\[
	\Delta S_{\text{serb} } = \frac{mc(T_1-T_2)}{T_2}
\]
Una parte del sistema aumenta la sua entropia, l'altra parte la diminuisce perché $T_1>T_2$. Le due variazioni di entropia non sono uguali. In un caso è $mc\log(T_2/T_1)$, nell'altro $mc\Delta T$. In realtà si arriverà a dire che se il processo è irreversibile, la variazione netta di entropia di tutte le parti del sistema è sempre positiva.
\[
	\frac{mc(T_1-T_2  )}{T_2 } \neq mc\,\log \frac{T_2 }{T_1}
\]

\paragraph{ $\Delta S$ di un gas perfetto} Si calcoli la variazione di entropia di un gas perfetto che esegue una trasformazione dallo stato $A$ allo stato $B$. Esso sarà caratterizzato da un certo numero di moli. Non bisogna specificare quale trasformazione segue, nè se è reversibile o meno. Si trova $\Delta S$ partendo dalla definizione.
\[
	\Delta S_{\text{gas} } = \int_A^B \frac{dQ_{\text{rev} } }{T}
\]
Utilizzando il primo principio della termodinamica, si può scrivere la variazione infinitesima di lavoro scambiato in maniera reversibile:
\[
	dQ_{\text{rev} } = dU + d\mathcal{L}_{\text{rev} }
\]
Bisogna immaginare che questa piccola trasformazione sia reversibile. Per l'energia interna non va specificato niente perché è una funzione di stato, ma per il lavoro bisogna sottolineare il fatto che si tratta di lavoro fatto in maniera reversibile.
\begin{equation*}
	\begin{aligned}
		\Delta S_{\text{gas} } &= \int_A^B \frac{dU + d\mathcal{L}_{\text{rev} } }{T} \\
		&= \int_A^B \frac{nc_v dT }{T} + \int_A^B \frac{\overbrace{d\mathcal{L}_{\text{rev} }}^{p_{\text{gas}}dV } }{T} \\
		&= \int_A^B \frac{nc_v dT }{T} + \int_A^B \frac{nRT/V}{T}dV \\
		&= \int_A^B \frac{nc_v dT }{T} + \int_A^B \frac{nR}{V}dV \\
		&=nc_v\log \frac{T_B }{T_A } + nR\log \frac{V_B }{V_A }
	\end{aligned}
\end{equation*}
Questa è la relazione che si può utilizzare sempre per calcolare la variazione di entropia di un gas perfetto, qualunque sia la trasformazione che segue.
\[
	\boxed{\Delta S_{\text{gas} } = nc_v\log \frac{T_B }{T_A } + nR\log \frac{V_B }{V_A }}
\]
La si applica a un caso particolare. Si consideri l'esperimento di espansione libera di un gas nel vuoto.
\begin{figure}[htpb]
	\centering
	

	\tikzset{every picture/.style={line width=0.75pt}} %set default line width to 0.75pt        

	\begin{tikzpicture}[x=0.75pt,y=0.75pt,yscale=-1,xscale=1]
	%uncomment if require: \path (0,300); %set diagram left start at 0, and has height of 300

	%Shape: Rectangle [id:dp7654859373784995] 
	\draw  [line width=1.5]  (132,71) -- (339.5,71) -- (339.5,158) -- (132,158) -- cycle ;
	%Straight Lines [id:da7473689178219229] 
	\draw [line width=3.75]    (235,70) -- (235,158) ;
	%Shape: Circle [id:dp23688450455889987] 
	\draw  [draw opacity=0][fill={rgb, 255:red, 212; green, 212; blue, 212 }  ,fill opacity=1 ] (144.67,86.83) .. controls (144.67,81.95) and (148.62,78) .. (153.5,78) .. controls (158.38,78) and (162.33,81.95) .. (162.33,86.83) .. controls (162.33,91.71) and (158.38,95.67) .. (153.5,95.67) .. controls (148.62,95.67) and (144.67,91.71) .. (144.67,86.83) -- cycle ;
	%Shape: Circle [id:dp22326990723298756] 
	\draw  [draw opacity=0][fill={rgb, 255:red, 212; green, 212; blue, 212 }  ,fill opacity=1 ] (184.5,86.33) .. controls (184.5,81.45) and (188.45,77.5) .. (193.33,77.5) .. controls (198.21,77.5) and (202.17,81.45) .. (202.17,86.33) .. controls (202.17,91.21) and (198.21,95.17) .. (193.33,95.17) .. controls (188.45,95.17) and (184.5,91.21) .. (184.5,86.33) -- cycle ;
	%Shape: Circle [id:dp39356421779783557] 
	\draw  [draw opacity=0][fill={rgb, 255:red, 212; green, 212; blue, 212 }  ,fill opacity=1 ] (208.5,107.5) .. controls (208.5,102.62) and (212.45,98.67) .. (217.33,98.67) .. controls (222.21,98.67) and (226.17,102.62) .. (226.17,107.5) .. controls (226.17,112.38) and (222.21,116.33) .. (217.33,116.33) .. controls (212.45,116.33) and (208.5,112.38) .. (208.5,107.5) -- cycle ;
	%Shape: Circle [id:dp9732115257870197] 
	\draw  [draw opacity=0][fill={rgb, 255:red, 212; green, 212; blue, 212 }  ,fill opacity=1 ] (150.5,112.67) .. controls (150.5,107.79) and (154.45,103.83) .. (159.33,103.83) .. controls (164.21,103.83) and (168.17,107.79) .. (168.17,112.67) .. controls (168.17,117.55) and (164.21,121.5) .. (159.33,121.5) .. controls (154.45,121.5) and (150.5,117.55) .. (150.5,112.67) -- cycle ;
	%Shape: Circle [id:dp14464327361910723] 
	\draw  [draw opacity=0][fill={rgb, 255:red, 212; green, 212; blue, 212 }  ,fill opacity=1 ] (167.17,100.33) .. controls (167.17,95.45) and (171.12,91.5) .. (176,91.5) .. controls (180.88,91.5) and (184.83,95.45) .. (184.83,100.33) .. controls (184.83,105.21) and (180.88,109.17) .. (176,109.17) .. controls (171.12,109.17) and (167.17,105.21) .. (167.17,100.33) -- cycle ;
	%Shape: Circle [id:dp6435399617117978] 
	\draw  [draw opacity=0][fill={rgb, 255:red, 212; green, 212; blue, 212 }  ,fill opacity=1 ] (185.17,115.33) .. controls (185.17,110.45) and (189.12,106.5) .. (194,106.5) .. controls (198.88,106.5) and (202.83,110.45) .. (202.83,115.33) .. controls (202.83,120.21) and (198.88,124.17) .. (194,124.17) .. controls (189.12,124.17) and (185.17,120.21) .. (185.17,115.33) -- cycle ;
	%Shape: Rectangle [id:dp9352467576178161] 
	\draw  [line width=1.5]  (132,171) -- (339.5,171) -- (339.5,258) -- (132,258) -- cycle ;
	%Shape: Circle [id:dp7581647083942251] 
	\draw  [draw opacity=0][fill={rgb, 255:red, 212; green, 212; blue, 212 }  ,fill opacity=1 ] (144.67,186.83) .. controls (144.67,181.95) and (148.62,178) .. (153.5,178) .. controls (158.38,178) and (162.33,181.95) .. (162.33,186.83) .. controls (162.33,191.71) and (158.38,195.67) .. (153.5,195.67) .. controls (148.62,195.67) and (144.67,191.71) .. (144.67,186.83) -- cycle ;
	%Shape: Circle [id:dp11922850645820748] 
	\draw  [draw opacity=0][fill={rgb, 255:red, 212; green, 212; blue, 212 }  ,fill opacity=1 ] (214,196.33) .. controls (214,191.45) and (217.95,187.5) .. (222.83,187.5) .. controls (227.71,187.5) and (231.67,191.45) .. (231.67,196.33) .. controls (231.67,201.21) and (227.71,205.17) .. (222.83,205.17) .. controls (217.95,205.17) and (214,201.21) .. (214,196.33) -- cycle ;
	%Shape: Circle [id:dp8219713042454879] 
	\draw  [draw opacity=0][fill={rgb, 255:red, 212; green, 212; blue, 212 }  ,fill opacity=1 ] (291,232) .. controls (291,227.12) and (294.95,223.17) .. (299.83,223.17) .. controls (304.71,223.17) and (308.67,227.12) .. (308.67,232) .. controls (308.67,236.88) and (304.71,240.83) .. (299.83,240.83) .. controls (294.95,240.83) and (291,236.88) .. (291,232) -- cycle ;
	%Shape: Circle [id:dp9516039190152856] 
	\draw  [draw opacity=0][fill={rgb, 255:red, 212; green, 212; blue, 212 }  ,fill opacity=1 ] (150.5,212.67) .. controls (150.5,207.79) and (154.45,203.83) .. (159.33,203.83) .. controls (164.21,203.83) and (168.17,207.79) .. (168.17,212.67) .. controls (168.17,217.55) and (164.21,221.5) .. (159.33,221.5) .. controls (154.45,221.5) and (150.5,217.55) .. (150.5,212.67) -- cycle ;
	%Shape: Circle [id:dp20810462008863717] 
	\draw  [draw opacity=0][fill={rgb, 255:red, 212; green, 212; blue, 212 }  ,fill opacity=1 ] (180.67,194.33) .. controls (180.67,189.45) and (184.62,185.5) .. (189.5,185.5) .. controls (194.38,185.5) and (198.33,189.45) .. (198.33,194.33) .. controls (198.33,199.21) and (194.38,203.17) .. (189.5,203.17) .. controls (184.62,203.17) and (180.67,199.21) .. (180.67,194.33) -- cycle ;
	%Shape: Circle [id:dp2765007615428927] 
	\draw  [draw opacity=0][fill={rgb, 255:red, 212; green, 212; blue, 212 }  ,fill opacity=1 ] (239.17,227.33) .. controls (239.17,222.45) and (243.12,218.5) .. (248,218.5) .. controls (252.88,218.5) and (256.83,222.45) .. (256.83,227.33) .. controls (256.83,232.21) and (252.88,236.17) .. (248,236.17) .. controls (243.12,236.17) and (239.17,232.21) .. (239.17,227.33) -- cycle ;
	%Shape: Circle [id:dp7090074746213548] 
	\draw  [draw opacity=0][fill={rgb, 255:red, 212; green, 212; blue, 212 }  ,fill opacity=1 ] (301.5,201.5) .. controls (301.5,196.62) and (305.45,192.67) .. (310.33,192.67) .. controls (315.21,192.67) and (319.17,196.62) .. (319.17,201.5) .. controls (319.17,206.38) and (315.21,210.33) .. (310.33,210.33) .. controls (305.45,210.33) and (301.5,206.38) .. (301.5,201.5) -- cycle ;
	%Shape: Circle [id:dp32636661863574723] 
	\draw  [draw opacity=0][fill={rgb, 255:red, 212; green, 212; blue, 212 }  ,fill opacity=1 ] (249.5,188.5) .. controls (249.5,183.62) and (253.45,179.67) .. (258.33,179.67) .. controls (263.21,179.67) and (267.17,183.62) .. (267.17,188.5) .. controls (267.17,193.38) and (263.21,197.33) .. (258.33,197.33) .. controls (253.45,197.33) and (249.5,193.38) .. (249.5,188.5) -- cycle ;
	%Shape: Circle [id:dp35918471123787743] 
	\draw  [draw opacity=0][fill={rgb, 255:red, 212; green, 212; blue, 212 }  ,fill opacity=1 ] (249.5,188.5) .. controls (249.5,183.62) and (253.45,179.67) .. (258.33,179.67) .. controls (263.21,179.67) and (267.17,183.62) .. (267.17,188.5) .. controls (267.17,193.38) and (263.21,197.33) .. (258.33,197.33) .. controls (253.45,197.33) and (249.5,193.38) .. (249.5,188.5) -- cycle ;

	% Text Node
	\draw (183.5,138.5) node    {$P_{A} ,V_{A} ,T_{A}$};
	% Text Node
	\draw (186.5,237.5) node    {$P_{B} ,V_{B} ,T_{B}$};


	\end{tikzpicture}
\end{figure}
\FloatBarrier
Si tratta di un processo irreversibile, il gas non tornerà mai indietro. Si tratta di un espansione adiabatica. Visto che il gas non scambia calore, si potrebbe pensare che allora nella relazione $dQ$ è pari a $0$ e quindi la variazione di entropia è nulla. Il calore scambiato invece non lo si può usare per calcolare la variazione di entropia perché è quello scambiato in maniera irreversibile. Per questo specifico caso, usando invece la relazione appena trovata:
\[
	nc_v\log \frac{T_B }{T_A } = 0 \implies \Delta S_{\text{gas} } = nR\log \frac{V_B }{V_A }
\]
In questo caso il gas aumenta sempre la sua entropia.

\paragraph{Osservazione} L'entropia può essere scritta anche in forma infinitesima, con riferimento ad una trasformazione reversibile infinitesima, che implica una variazione infinitesima delle coordinate termodinamiche:
\[
	dS = \left( \frac{dQ}{T} \right)_{\text{rev}}
\]
$dQ$ non è un differenziale esatto però se tale quantità è divisa per $T$ ed è considerata in una trasformazione reversibile, essa da luogo al differenziale esatto $dS$.







































\section{Principio di accrescimento dell'entropia dell'universo}

Si vede ora quali informazioni in più porta considerare la disuguaglianza di Clausius con il segno di minore stretto.  Per capirlo, si ripete la stessa dimostrazione che ha portato a definire l'entropia. Si consideri l'integrale definito per un ciclo. Tale ciclo è irreversibile, in particolare, si fa in modo che esso possa essere spezzato in due parti $A$ e $B$, tali per cui $AB$ è irreversibile, mentre $BA$ è reversibile.
\begin{figure}[htpb]
	\centering
	

	\tikzset{every picture/.style={line width=0.75pt}} %set default line width to 0.75pt        

	\begin{tikzpicture}[x=0.75pt,y=0.75pt,yscale=-1,xscale=1]
	%uncomment if require: \path (0,300); %set diagram left start at 0, and has height of 300

	%Shape: Axis 2D [id:dp5387427598927963] 
	\draw  (70,229.2) -- (302.33,229.2)(93.23,84) -- (93.23,245.33) (295.33,224.2) -- (302.33,229.2) -- (295.33,234.2) (88.23,91) -- (93.23,84) -- (98.23,91)  ;
	%Curve Lines [id:da06279909715521925] 
	\draw    (123,189.33) .. controls (120.69,188.11) and (120.06,186.33) .. (121.13,183.99) .. controls (122.54,182.42) and (122.36,180.83) .. (120.57,179.23) .. controls (118.94,177.74) and (119.07,176.14) .. (120.96,174.41) .. controls (122.98,173) and (123.4,171.38) .. (122.23,169.55) .. controls (121.22,167.52) and (121.92,165.9) .. (124.32,164.69) .. controls (126.51,164.14) and (127.31,162.75) .. (126.71,160.54) .. controls (126.28,158.2) and (127.25,156.82) .. (129.62,156.41) .. controls (131.99,156.11) and (133.12,154.75) .. (133.01,152.32) .. controls (132.6,150.29) and (133.65,149.17) .. (136.16,148.96) .. controls (138.67,148.83) and (139.81,147.73) .. (139.6,145.65) .. controls (139.97,143.08) and (141.2,142) .. (143.3,142.39) .. controls (145.9,142.4) and (147.21,141.34) .. (147.23,139.2) .. controls (147.88,136.61) and (149.26,135.57) .. (151.37,136.08) .. controls (154.04,136.23) and (155.49,135.22) .. (155.71,133.05) .. controls (155.98,130.86) and (157.48,129.88) .. (160.21,130.1) .. controls (162.31,130.75) and (163.54,129.99) .. (163.91,127.82) .. controls (164.33,125.64) and (165.91,124.72) .. (168.66,125.07) .. controls (170.74,125.82) and (172.03,125.11) .. (172.53,122.95) .. controls (173.72,120.44) and (175.36,119.6) .. (177.44,120.41) .. controls (179.49,121.26) and (180.82,120.61) .. (181.41,118.48) .. controls (182.72,116.03) and (184.39,115.27) .. (186.42,116.18) .. controls (188.42,117.13) and (189.76,116.55) .. (190.44,114.45) .. controls (191.83,112.08) and (193.51,111.41) .. (195.47,112.43) .. controls (197.39,113.48) and (199.06,112.86) .. (200.47,110.57) .. controls (201.26,108.53) and (202.59,108.08) .. (204.45,109.21) .. controls (206.92,110.17) and (208.55,109.66) .. (209.36,107.67) .. controls (210.86,105.51) and (212.47,105.06) .. (214.19,106.32) .. controls (216.48,107.46) and (218.06,107.07) .. (218.91,105.16) .. controls (220.43,103.13) and (222.26,102.76) .. (224.41,104.05) .. controls (225.86,105.5) and (227.33,105.27) .. (228.81,103.36) .. controls (230.34,101.5) and (232.03,101.32) .. (233.86,102.83) .. controls (235.54,104.4) and (237.38,104.34) .. (239.37,102.64) .. controls (240.94,101.02) and (242.39,101.09) .. (243.72,102.86) .. controls (245.24,104.74) and (246.97,105.03) .. (248.92,103.74) .. controls (251.02,102.63) and (252.49,103.16) .. (253.34,105.31) -- (255.67,106.67) ;
	%Curve Lines [id:da995829034171916] 
	\draw    (123,189.33) .. controls (161,208) and (259.67,151.33) .. (255.67,106.67) ;
	\draw [shift={(205.08,171.75)}, rotate = 329.9] [fill={rgb, 255:red, 0; green, 0; blue, 0 }  ][line width=0.08]  [draw opacity=0] (10.72,-5.15) -- (0,0) -- (10.72,5.15) -- (7.12,0) -- cycle    ;
	%Shape: Circle [id:dp006572357997259193] 
	\draw  [fill={rgb, 255:red, 0; green, 0; blue, 0 }  ,fill opacity=1 ] (120,187.17) .. controls (120,185.97) and (120.97,185) .. (122.17,185) .. controls (123.36,185) and (124.33,185.97) .. (124.33,187.17) .. controls (124.33,188.36) and (123.36,189.33) .. (122.17,189.33) .. controls (120.97,189.33) and (120,188.36) .. (120,187.17) -- cycle ;
	%Shape: Circle [id:dp7516795691016311] 
	\draw  [fill={rgb, 255:red, 0; green, 0; blue, 0 }  ,fill opacity=1 ] (252,105.17) .. controls (252,103.97) and (252.97,103) .. (254.17,103) .. controls (255.36,103) and (256.33,103.97) .. (256.33,105.17) .. controls (256.33,106.36) and (255.36,107.33) .. (254.17,107.33) .. controls (252.97,107.33) and (252,106.36) .. (252,105.17) -- cycle ;

	% Text Node
	\draw (78,84.33) node    {$p$};
	% Text Node
	\draw (316.67,228.33) node    {$V$};
	% Text Node
	\draw (111.67,197.67) node    {$A$};
	% Text Node
	\draw (269.17,100.17) node    {$B$};
	% Text Node
	\draw (147.67,109.67) node    {$T_{1}$};
	% Text Node
	\draw (230.67,181.17) node    {$T_{2}$};


	\end{tikzpicture}
\end{figure}
\FloatBarrier
\[
	\oint \frac{dQ_{\text{rev} } }{T} = \int_{\Gamma_1,A\to B } \frac{dQ_{\text{irr} } }{T} + \underbrace{\int_{\Gamma_2,B\to A} \frac{dQ_{\text{rev} } }{T}}_{S(A)-S(B)} < 0
\]
La quantità a secondo termine non è altro che la variazione di entropia da $B$ ad $A$. $S(A)-S(B)$ lo si sposta a destra cambiandolo di segno.
\[
	\int_{\Gamma_1,A\to B } \frac{dQ_{\text{irr} } }{T} < S(B)-S(A)
\]
Considerati uno stato iniziale $A$ e uno stato finale $B$, se si valuta la variazione di entropia fra questi due stati, quella che si misura è maggiore della somma continua del calore scambiato in maniera irreversibile su $T$.
\[
	S(B)-S(A) > \int_{\Gamma_1,A\to B } \frac{dQ_{\text{irr} } }{T}
\]
Si applica questo risultato alla valutazione della variazione di entropia di un sistema isolato. Si è visto che il sistema termodinamico entra in contatto con un ambiente circostante. In generale esso cede o assorbe calore dall'ambiente circostante. Il complesso sistema termodinamico-ambiente circostante viene detto universo termodinamico. Fondamentalmente gli scambi di calore e di lavoro sono pensati solo fra il sistema termodinamico e l'ambiente. Non c'è altro calore che esce fuori. È come se si immaginasse l'ambiente chiuso in un contenitore adiabatico. Per l'universo termodinamico si può fare una affermazione in termini di calore scambiato: esso è sempre uguale a $0$. C'è solo passaggio di calore fra sistema e l'ambiente ma non esce nulla fuori. Si applicano queste considerazioni ottenute al caso in cui si valuta ciò che accade per l'intero universo termodinamico. Si vuole capire quanto vale la variazione di entropia dell'universo.
\[
	\Delta S_{\text{universo} } = \int_A^B \frac{dQ_{\text{rev} } }{T}
\]
Se all'interno dell'universo termodinamico avvengono solo trasformazioni reversibili il calore totale scambiato è quello reversibile. Dal momento che l'universo è isolato, il secondo termine vale $0$ perché tutte le trasformazioni nell'ipotesi sono irreversibili. Ciò significa che in questo caso la variazione di entropia sarà quindi uguale a $0$ e l'entropia è costante. Non aumenta quando si fa passare il sistema da uno stato iniziale a un altro stato.
\[
	\Delta S_{\text{universo}} = S(B)-S(A) > \int \frac{dQ_{\text{irr} } }{T}
\]
Se invece nell'universo avvengono anche trasformazioni irreversibili, chi va a $0$ è il termine $dQ_\text{irr}/T$. Quando ci sono trasformazioni irreversibili, l'entropia dell'universo non può che aumentare. Questo risultato prende il nome di \text{principio di accrescimento dell'entropia dell'universo}. Esso fondamentalmente afferma che se si prende un universo termodinamico, o equivalentemente un sistema isolato dal resto del mondo,  quando esso passa da stato iniziale a stato finale, la variazione di entropia che subisce è sempre maggiore o al limite uguale a zero. L'entropia aumenta se c'è anche solo un processo irreversibile, si mantiene costante solo se tutte le trasformazioni sono reversibili.
\[
	\text{trasformazioni}
	\left\{ \begin{array}{l}
	 	\text{reversibili} \implies \Delta S_{\text{universo}} = 0 \\
		\text{irreversibili} \implies  \Delta S_{\text{universo}} > 0
	\end{array} \right.
\]
Riscrivendo il risultato in maniera più compatta:
\[
	\boxed{\Delta S_{\text{universo}} \geq 0}
\]
Si potrebbe avere una trasformazione in cui per una certa parte del sistema termodinamico l'entropia diminuisce, ma contemporaneamente ci deve essere un'altra parte del sistema per cui l'entropia aumenta. Se il processo è reversibile, la diminuzione di entropia in una certa parte del sistema è bilanciata dall'aumento di entropia di un'altra parte del sistema. Se il processo è irreversibile, chi aumenta l'entropia subirà un aumento maggiore di cui la diminuisce, così che al netto la variazione di entropia è strettamente positiva. Dal punto di vista più generale, il principio lo si può vedere come un altro modo di enunciare il secondo principio della termodinamica. Ci si è arrivati infatti a partire dal teorema di Carnot, dimostrato sfruttando l'enunciato di Clausius del secondo principio della termodinamica.

I processi che avvengono realmente, tendono sempre a generare una forma di energia che è più disordinata di quella avuta in partenza. Il principio di accrescimento dell'entropia dell'universo ha il significato di affermare che il disordine dell'universo tende sempre ad aumentare. Si è visto questo anche quando è stato fatto l'esempio dell'oggetto che si muove su un piano. L'energia cinetica ordinata si trasforma spontaneamente in calore, che è un'energia più disordinata legata al moto di agitazione termica delle molecole.
\begin{figure}[htpb]
	\centering
	

	\tikzset{every picture/.style={line width=0.75pt}} %set default line width to 0.75pt        

	\begin{tikzpicture}[x=0.75pt,y=0.75pt,yscale=-1,xscale=1]
	%uncomment if require: \path (0,300); %set diagram left start at 0, and has height of 300

	%Shape: Rectangle [id:dp1499911640223488] 
	\draw  [line width=1.5]  (132,71) -- (339.5,71) -- (339.5,158) -- (132,158) -- cycle ;
	%Straight Lines [id:da4030620174178323] 
	\draw [line width=3.75]    (235,70) -- (235,158) ;
	%Shape: Circle [id:dp24482697243748608] 
	\draw  [draw opacity=0][fill={rgb, 255:red, 212; green, 212; blue, 212 }  ,fill opacity=1 ] (144.67,86.83) .. controls (144.67,81.95) and (148.62,78) .. (153.5,78) .. controls (158.38,78) and (162.33,81.95) .. (162.33,86.83) .. controls (162.33,91.71) and (158.38,95.67) .. (153.5,95.67) .. controls (148.62,95.67) and (144.67,91.71) .. (144.67,86.83) -- cycle ;
	%Shape: Circle [id:dp5633133397032646] 
	\draw  [draw opacity=0][fill={rgb, 255:red, 212; green, 212; blue, 212 }  ,fill opacity=1 ] (208.5,135.67) .. controls (208.5,130.79) and (212.45,126.83) .. (217.33,126.83) .. controls (222.21,126.83) and (226.17,130.79) .. (226.17,135.67) .. controls (226.17,140.55) and (222.21,144.5) .. (217.33,144.5) .. controls (212.45,144.5) and (208.5,140.55) .. (208.5,135.67) -- cycle ;
	%Shape: Circle [id:dp10481425575013792] 
	\draw  [draw opacity=0][fill={rgb, 255:red, 212; green, 212; blue, 212 }  ,fill opacity=1 ] (187.17,114.17) .. controls (187.17,109.29) and (191.12,105.33) .. (196,105.33) .. controls (200.88,105.33) and (204.83,109.29) .. (204.83,114.17) .. controls (204.83,119.05) and (200.88,123) .. (196,123) .. controls (191.12,123) and (187.17,119.05) .. (187.17,114.17) -- cycle ;
	%Shape: Circle [id:dp46629536046966136] 
	\draw  [draw opacity=0][fill={rgb, 255:red, 212; green, 212; blue, 212 }  ,fill opacity=1 ] (141.83,134.67) .. controls (141.83,129.79) and (145.79,125.83) .. (150.67,125.83) .. controls (155.55,125.83) and (159.5,129.79) .. (159.5,134.67) .. controls (159.5,139.55) and (155.55,143.5) .. (150.67,143.5) .. controls (145.79,143.5) and (141.83,139.55) .. (141.83,134.67) -- cycle ;
	%Shape: Circle [id:dp7346336625541874] 
	\draw  [draw opacity=0][fill={rgb, 255:red, 212; green, 212; blue, 212 }  ,fill opacity=1 ] (158.5,106.33) .. controls (158.5,101.45) and (162.45,97.5) .. (167.33,97.5) .. controls (172.21,97.5) and (176.17,101.45) .. (176.17,106.33) .. controls (176.17,111.21) and (172.21,115.17) .. (167.33,115.17) .. controls (162.45,115.17) and (158.5,111.21) .. (158.5,106.33) -- cycle ;
	%Shape: Circle [id:dp768155105372726] 
	\draw  [draw opacity=0][fill={rgb, 255:red, 212; green, 212; blue, 212 }  ,fill opacity=1 ] (175.17,140) .. controls (175.17,135.12) and (179.12,131.17) .. (184,131.17) .. controls (188.88,131.17) and (192.83,135.12) .. (192.83,140) .. controls (192.83,144.88) and (188.88,148.83) .. (184,148.83) .. controls (179.12,148.83) and (175.17,144.88) .. (175.17,140) -- cycle ;
	%Shape: Rectangle [id:dp4130826075758587] 
	\draw  [line width=1.5]  (132,171) -- (339.5,171) -- (339.5,258) -- (132,258) -- cycle ;
	%Shape: Circle [id:dp2121341951790583] 
	\draw  [draw opacity=0][fill={rgb, 255:red, 212; green, 212; blue, 212 }  ,fill opacity=1 ] (201.17,86.83) .. controls (201.17,81.95) and (205.12,78) .. (210,78) .. controls (214.88,78) and (218.83,81.95) .. (218.83,86.83) .. controls (218.83,91.71) and (214.88,95.67) .. (210,95.67) .. controls (205.12,95.67) and (201.17,91.71) .. (201.17,86.83) -- cycle ;
	%Shape: Circle [id:dp9598242837355548] 
	\draw  [draw opacity=0][fill={rgb, 255:red, 155; green, 155; blue, 155 }  ,fill opacity=1 ] (249.17,89.5) .. controls (249.17,84.62) and (253.12,80.67) .. (258,80.67) .. controls (262.88,80.67) and (266.83,84.62) .. (266.83,89.5) .. controls (266.83,94.38) and (262.88,98.33) .. (258,98.33) .. controls (253.12,98.33) and (249.17,94.38) .. (249.17,89.5) -- cycle ;
	%Shape: Circle [id:dp7914645715580639] 
	\draw  [draw opacity=0][fill={rgb, 255:red, 155; green, 155; blue, 155 }  ,fill opacity=1 ] (286.5,88.83) .. controls (286.5,83.95) and (290.45,80) .. (295.33,80) .. controls (300.21,80) and (304.17,83.95) .. (304.17,88.83) .. controls (304.17,93.71) and (300.21,97.67) .. (295.33,97.67) .. controls (290.45,97.67) and (286.5,93.71) .. (286.5,88.83) -- cycle ;
	%Shape: Circle [id:dp714912835594077] 
	\draw  [draw opacity=0][fill={rgb, 255:red, 155; green, 155; blue, 155 }  ,fill opacity=1 ] (269.17,112.83) .. controls (269.17,107.95) and (273.12,104) .. (278,104) .. controls (282.88,104) and (286.83,107.95) .. (286.83,112.83) .. controls (286.83,117.71) and (282.88,121.67) .. (278,121.67) .. controls (273.12,121.67) and (269.17,117.71) .. (269.17,112.83) -- cycle ;
	%Shape: Circle [id:dp6674219180240206] 
	\draw  [draw opacity=0][fill={rgb, 255:red, 155; green, 155; blue, 155 }  ,fill opacity=1 ] (307.83,110.17) .. controls (307.83,105.29) and (311.79,101.33) .. (316.67,101.33) .. controls (321.55,101.33) and (325.5,105.29) .. (325.5,110.17) .. controls (325.5,115.05) and (321.55,119) .. (316.67,119) .. controls (311.79,119) and (307.83,115.05) .. (307.83,110.17) -- cycle ;
	%Shape: Circle [id:dp38802677325492496] 
	\draw  [draw opacity=0][fill={rgb, 255:red, 155; green, 155; blue, 155 }  ,fill opacity=1 ] (283.17,140.83) .. controls (283.17,135.95) and (287.12,132) .. (292,132) .. controls (296.88,132) and (300.83,135.95) .. (300.83,140.83) .. controls (300.83,145.71) and (296.88,149.67) .. (292,149.67) .. controls (287.12,149.67) and (283.17,145.71) .. (283.17,140.83) -- cycle ;
	%Shape: Circle [id:dp07135425312606158] 
	\draw  [draw opacity=0][fill={rgb, 255:red, 155; green, 155; blue, 155 }  ,fill opacity=1 ] (251.83,138.17) .. controls (251.83,133.29) and (255.79,129.33) .. (260.67,129.33) .. controls (265.55,129.33) and (269.5,133.29) .. (269.5,138.17) .. controls (269.5,143.05) and (265.55,147) .. (260.67,147) .. controls (255.79,147) and (251.83,143.05) .. (251.83,138.17) -- cycle ;
	%Shape: Circle [id:dp28812148833016016] 
	\draw  [draw opacity=0][fill={rgb, 255:red, 155; green, 155; blue, 155 }  ,fill opacity=1 ] (313.17,138.83) .. controls (313.17,133.95) and (317.12,130) .. (322,130) .. controls (326.88,130) and (330.83,133.95) .. (330.83,138.83) .. controls (330.83,143.71) and (326.88,147.67) .. (322,147.67) .. controls (317.12,147.67) and (313.17,143.71) .. (313.17,138.83) -- cycle ;
	%Shape: Circle [id:dp22984044315113872] 
	\draw  [draw opacity=0][fill={rgb, 255:red, 155; green, 155; blue, 155 }  ,fill opacity=1 ] (154.5,220.83) .. controls (154.5,215.95) and (158.45,212) .. (163.33,212) .. controls (168.21,212) and (172.17,215.95) .. (172.17,220.83) .. controls (172.17,225.71) and (168.21,229.67) .. (163.33,229.67) .. controls (158.45,229.67) and (154.5,225.71) .. (154.5,220.83) -- cycle ;
	%Shape: Circle [id:dp5312743164052502] 
	\draw  [draw opacity=0][fill={rgb, 255:red, 155; green, 155; blue, 155 }  ,fill opacity=1 ] (269.83,210.17) .. controls (269.83,205.29) and (273.79,201.33) .. (278.67,201.33) .. controls (283.55,201.33) and (287.5,205.29) .. (287.5,210.17) .. controls (287.5,215.05) and (283.55,219) .. (278.67,219) .. controls (273.79,219) and (269.83,215.05) .. (269.83,210.17) -- cycle ;
	%Shape: Circle [id:dp3499674700441038] 
	\draw  [draw opacity=0][fill={rgb, 255:red, 155; green, 155; blue, 155 }  ,fill opacity=1 ] (185.17,191.5) .. controls (185.17,186.62) and (189.12,182.67) .. (194,182.67) .. controls (198.88,182.67) and (202.83,186.62) .. (202.83,191.5) .. controls (202.83,196.38) and (198.88,200.33) .. (194,200.33) .. controls (189.12,200.33) and (185.17,196.38) .. (185.17,191.5) -- cycle ;
	%Shape: Circle [id:dp8378654755230381] 
	\draw  [draw opacity=0][fill={rgb, 255:red, 155; green, 155; blue, 155 }  ,fill opacity=1 ] (309.83,212.17) .. controls (309.83,207.29) and (313.79,203.33) .. (318.67,203.33) .. controls (323.55,203.33) and (327.5,207.29) .. (327.5,212.17) .. controls (327.5,217.05) and (323.55,221) .. (318.67,221) .. controls (313.79,221) and (309.83,217.05) .. (309.83,212.17) -- cycle ;
	%Shape: Circle [id:dp5028590035257423] 
	\draw  [draw opacity=0][fill={rgb, 255:red, 155; green, 155; blue, 155 }  ,fill opacity=1 ] (258.5,236.83) .. controls (258.5,231.95) and (262.45,228) .. (267.33,228) .. controls (272.21,228) and (276.17,231.95) .. (276.17,236.83) .. controls (276.17,241.71) and (272.21,245.67) .. (267.33,245.67) .. controls (262.45,245.67) and (258.5,241.71) .. (258.5,236.83) -- cycle ;
	%Shape: Circle [id:dp11373845294447271] 
	\draw  [draw opacity=0][fill={rgb, 255:red, 155; green, 155; blue, 155 }  ,fill opacity=1 ] (211.83,236.83) .. controls (211.83,231.95) and (215.79,228) .. (220.67,228) .. controls (225.55,228) and (229.5,231.95) .. (229.5,236.83) .. controls (229.5,241.71) and (225.55,245.67) .. (220.67,245.67) .. controls (215.79,245.67) and (211.83,241.71) .. (211.83,236.83) -- cycle ;
	%Shape: Circle [id:dp621786237948103] 
	\draw  [draw opacity=0][fill={rgb, 255:red, 155; green, 155; blue, 155 }  ,fill opacity=1 ] (221.17,190.83) .. controls (221.17,185.95) and (225.12,182) .. (230,182) .. controls (234.88,182) and (238.83,185.95) .. (238.83,190.83) .. controls (238.83,195.71) and (234.88,199.67) .. (230,199.67) .. controls (225.12,199.67) and (221.17,195.71) .. (221.17,190.83) -- cycle ;
	%Shape: Circle [id:dp10207076001389193] 
	\draw  [draw opacity=0][fill={rgb, 255:red, 212; green, 212; blue, 212 }  ,fill opacity=1 ] (145.33,188.83) .. controls (145.33,183.95) and (149.29,180) .. (154.17,180) .. controls (159.05,180) and (163,183.95) .. (163,188.83) .. controls (163,193.71) and (159.05,197.67) .. (154.17,197.67) .. controls (149.29,197.67) and (145.33,193.71) .. (145.33,188.83) -- cycle ;
	%Shape: Circle [id:dp4968942376874157] 
	\draw  [draw opacity=0][fill={rgb, 255:red, 212; green, 212; blue, 212 }  ,fill opacity=1 ] (291.17,237.67) .. controls (291.17,232.79) and (295.12,228.83) .. (300,228.83) .. controls (304.88,228.83) and (308.83,232.79) .. (308.83,237.67) .. controls (308.83,242.55) and (304.88,246.5) .. (300,246.5) .. controls (295.12,246.5) and (291.17,242.55) .. (291.17,237.67) -- cycle ;
	%Shape: Circle [id:dp3748523271121522] 
	\draw  [draw opacity=0][fill={rgb, 255:red, 212; green, 212; blue, 212 }  ,fill opacity=1 ] (187.83,216.17) .. controls (187.83,211.29) and (191.79,207.33) .. (196.67,207.33) .. controls (201.55,207.33) and (205.5,211.29) .. (205.5,216.17) .. controls (205.5,221.05) and (201.55,225) .. (196.67,225) .. controls (191.79,225) and (187.83,221.05) .. (187.83,216.17) -- cycle ;
	%Shape: Circle [id:dp1434364339701799] 
	\draw  [draw opacity=0][fill={rgb, 255:red, 212; green, 212; blue, 212 }  ,fill opacity=1 ] (173.83,242) .. controls (173.83,237.12) and (177.79,233.17) .. (182.67,233.17) .. controls (187.55,233.17) and (191.5,237.12) .. (191.5,242) .. controls (191.5,246.88) and (187.55,250.83) .. (182.67,250.83) .. controls (177.79,250.83) and (173.83,246.88) .. (173.83,242) -- cycle ;
	%Shape: Circle [id:dp11933510273696224] 
	\draw  [draw opacity=0][fill={rgb, 255:red, 212; green, 212; blue, 212 }  ,fill opacity=1 ] (297.17,188.33) .. controls (297.17,183.45) and (301.12,179.5) .. (306,179.5) .. controls (310.88,179.5) and (314.83,183.45) .. (314.83,188.33) .. controls (314.83,193.21) and (310.88,197.17) .. (306,197.17) .. controls (301.12,197.17) and (297.17,193.21) .. (297.17,188.33) -- cycle ;
	%Shape: Circle [id:dp6178267766318197] 
	\draw  [draw opacity=0][fill={rgb, 255:red, 212; green, 212; blue, 212 }  ,fill opacity=1 ] (236.5,218) .. controls (236.5,213.12) and (240.45,209.17) .. (245.33,209.17) .. controls (250.21,209.17) and (254.17,213.12) .. (254.17,218) .. controls (254.17,222.88) and (250.21,226.83) .. (245.33,226.83) .. controls (240.45,226.83) and (236.5,222.88) .. (236.5,218) -- cycle ;
	%Shape: Circle [id:dp24762806284995897] 
	\draw  [draw opacity=0][fill={rgb, 255:red, 212; green, 212; blue, 212 }  ,fill opacity=1 ] (257.17,188.83) .. controls (257.17,183.95) and (261.12,180) .. (266,180) .. controls (270.88,180) and (274.83,183.95) .. (274.83,188.83) .. controls (274.83,193.71) and (270.88,197.67) .. (266,197.67) .. controls (261.12,197.67) and (257.17,193.71) .. (257.17,188.83) -- cycle ;




	\end{tikzpicture}
\end{figure}
\FloatBarrier
C'è un esempio tipico sul fatto che il disordine molecolare aumenta sempre. Se si prende un contenitore diviso in due parti da una parete, in cui si mettono tante palline rosse da una parte e tante nere dall'altra, inizialmente le due parti del sistema sono molto ordinate. Se ad un certo punto si rimuove la parete e si agita il sistema, spontaneamente accade che le palline si mischiano e la probabilità che tutte le palline di un colore ritornino da una parte e tutte quelle dell'altro nell'altra, è assolutamente nulla. Questa è una visualizzazione del fatto che nell'universo ci sono sempre fenomeni che tendono a portare all'aumento del disordine molecolare, per cui spontaneamente si avrà un suo aumento.







































\section{Esempi di calcolo della variazione di entropia di un sistema isolato}

In seguito ci sono alcuni esempi in cui si verifica questo principio calcolando la variazione di entropia dell'universo.

\paragraph{Esempio 1} Si valuta cosa accade quando si ha una macchina termica che sta compiendo un ciclo termodinamico e si calcola la variazione di entropia dell'universo nei due casi in cui il ciclo è percorso in maniera reversibile o irreversibile.
\begin{figure}[htpb]
	\centering
	

	\tikzset{every picture/.style={line width=0.75pt}} %set default line width to 0.75pt        

	\begin{tikzpicture}[x=0.75pt,y=0.75pt,yscale=-1,xscale=1]
	%uncomment if require: \path (0,300); %set diagram left start at 0, and has height of 300

	%Right Arrow [id:dp29085795084499666] 
	\draw  [draw opacity=0][fill={rgb, 255:red, 184; green, 184; blue, 184 }  ,fill opacity=1 ] (178.06,177.31) -- (178.06,194.19) -- (183.8,194.19) -- (172.33,205.45) -- (160.87,194.19) -- (166.6,194.19) -- (166.6,177.31) -- cycle ;
	%Right Arrow [id:dp3212737354734736] 
	\draw  [draw opacity=0][fill={rgb, 255:red, 184; green, 184; blue, 184 }  ,fill opacity=1 ] (178.06,115.5) -- (178.06,132.38) -- (183.8,132.38) -- (172.33,143.64) -- (160.87,132.38) -- (166.6,132.38) -- (166.6,115.5) -- cycle ;
	%Shape: Rectangle [id:dp5029916281625293] 
	\draw  [fill={rgb, 255:red, 155; green, 155; blue, 155 }  ,fill opacity=1 ] (90,78) -- (251.5,78) -- (251.5,115.27) -- (90,115.27) -- cycle ;
	%Shape: Rectangle [id:dp3067577613932537] 
	\draw  [fill={rgb, 255:red, 212; green, 212; blue, 212 }  ,fill opacity=1 ] (90,205.9) -- (251.5,205.9) -- (251.5,243.17) -- (90,243.17) -- cycle ;
	%Shape: Ellipse [id:dp546617301961033] 
	\draw  [color={rgb, 255:red, 128; green, 128; blue, 128 }  ,draw opacity=1 ][fill={rgb, 255:red, 184; green, 184; blue, 184 }  ,fill opacity=1 ] (155,160.97) .. controls (155,151.4) and (162.76,143.64) .. (172.33,143.64) .. controls (181.91,143.64) and (189.67,151.4) .. (189.67,160.97) .. controls (189.67,170.55) and (181.91,178.31) .. (172.33,178.31) .. controls (162.76,178.31) and (155,170.55) .. (155,160.97) -- cycle ;
	%Shape: Rectangle [id:dp149197922715806] 
	\draw  [line width=2.25]  (69.5,65) -- (268.5,65) -- (268.5,258) -- (69.5,258) -- cycle ;

	% Text Node
	\draw (111.12,95.39) node    {$T_{2}$};
	% Text Node
	\draw (111.12,224.53) node    {$T_{1}$};
	% Text Node
	\draw (131.03,127.7) node  [font=\small]  {$Q_{2}  >0$};
	% Text Node
	\draw (150.33,187.4) node  [font=\small]  {$Q_{1}$};


	\end{tikzpicture}
\end{figure}
\FloatBarrier
\begin{itemize}
	\item Si prova a vedere cosa accade per un ciclo di Carnot. Se si guarda tutto dal punto di vista del gas perfetto, $Q_1$ è negativo e $Q_2$ è positivo. La variazione di entropia è una quantità additiva. Se l'universo è costituito da tante parti, la sua variazione di entropia totale è pari alla somma della variazione di entropia di ogni sua parte.
	\[
		\Delta S_{\text{universo}} = \underbrace{\Delta S_{\text{ciclo}}}_{=0} + \underbrace{\Delta S_1}_{\frac{Q_1 }{T_1 }} + \underbrace{\Delta S_2}_{-\frac{Q_2 }{T_2 }}
	\]
	La variazione di entropia durante il ciclo vale zero perché $S$ è una funzione stato e in quanto tale è nulla.
	
	Il serbatoio caldo diminuisce la sua entropia perché dal suo punto di vista il calore è ceduto. Il serbatoio freddo invece aumenta la sua entropia per motivi analoghi. Essendo l'intero processo reversibile, ci si aspetta che la variazione di entropia dell'universo faccia zero. Come si dimostra?  La macchina di Carnot ha rendimento per definizione:
	\[
		1 - \frac{|Q_1|}{Q_2 } = 1 - \frac{T_1 }{T_2 } \implies \frac{|Q_1|}{T_1 } = \frac{Q_2 }{T_2 }
	\]
	Si constata quindi che la variazione di entropia è uguale a $0$.
	\item Si ripete sempre lo stesso ragionamento nel caso in cui il ciclo percorso sia irreversibile. Abbiamo allora che il rendimento di una macchina irreversibile è sempre minore al rendimento della macchina di Carnot:
	\[
		1 - \frac{|Q_1|}{Q_2 } < 1 - \frac{T_1 }{T_2 } \implies \frac{|Q_1|}{T_1 } - \frac{Q_2 }{T_2 } > 0
	\]
	Essendo il ciclo irreversibile l'entropia dell'universo deve aumentare. Si è allora verificato il principio di accrescimento dell'entropia.
\end{itemize}

\paragraph{Esempio 2} Si hanno due corpi che scambiano calore fra di loro, ci si aspetta che l'entropia sia positiva, perché il processo è irreversibile.
\begin{figure}[htpb]
	\centering
	

	\tikzset{every picture/.style={line width=0.75pt}} %set default line width to 0.75pt        

	\begin{tikzpicture}[x=0.75pt,y=0.75pt,yscale=-1,xscale=1]
	%uncomment if require: \path (0,300); %set diagram left start at 0, and has height of 300

	%Rounded Rect [id:dp691740350311459] 
	\draw  [line width=2.25]  (61.67,82.6) .. controls (61.67,73.25) and (69.25,65.67) .. (78.6,65.67) -- (224.73,65.67) .. controls (234.09,65.67) and (241.67,73.25) .. (241.67,82.6) -- (241.67,133.4) .. controls (241.67,142.75) and (234.09,150.33) .. (224.73,150.33) -- (78.6,150.33) .. controls (69.25,150.33) and (61.67,142.75) .. (61.67,133.4) -- cycle ;
	%Rounded Rect [id:dp2640122403102567] 
	\draw  [line width=2.25]  (281.67,82.6) .. controls (281.67,73.25) and (289.25,65.67) .. (298.6,65.67) -- (444.73,65.67) .. controls (454.09,65.67) and (461.67,73.25) .. (461.67,82.6) -- (461.67,133.4) .. controls (461.67,142.75) and (454.09,150.33) .. (444.73,150.33) -- (298.6,150.33) .. controls (289.25,150.33) and (281.67,142.75) .. (281.67,133.4) -- cycle ;

	% Text Node
	\draw    (113.33, 107.33) circle [x radius= 32.31, y radius= 32.31]   ;
	\draw (113.33,107.33) node    {$m_{1} \ c\ T_{1}$};
	% Text Node
	\draw    (191.33, 107.33) circle [x radius= 32.31, y radius= 32.31]   ;
	\draw (191.33,107.33) node    {$m_{2} \ c\ T_{2}$};
	% Text Node
	\draw (153.33,163.67) node   [align=left] {prima};
	% Text Node
	\draw    (333.33, 107.33) circle [x radius= 34.18, y radius= 34.18]   ;
	\draw (333.33,107.33) node    {$m_{1} \ c\ T_{eq}$};
	% Text Node
	\draw    (411.33, 107.33) circle [x radius= 34.18, y radius= 34.18]   ;
	\draw (411.33,107.33) node    {$m_{2} \ c\ T_{eq}$};
	% Text Node
	\draw (373.33,163.67) node   [align=left] {dopo};


	\end{tikzpicture}
\end{figure}
\FloatBarrier
La temperatura finale sarà la media delle temperature iniziali pesate per la massa del corpo. Si calcola la variazione di entropia per il corpo che si sta scaldando. Bisogna valutare il calore scambiato in maniera reversibile fratto la temperatura, immaginando che il corpo inizialmente alla temperatura $T_1$, si porta fino alla temperatura finale di equilibrio:
\begin{gather*}
	\Delta S_{\text{universo}} = \Delta S_1 + \Delta S_2 \\
	\Delta S_1 = \int_{T_1 }^{T_{eq} } \frac{dQ_{\text{rev} } }{T } = \int_{T_1 }^{T_{eq} } \frac{m_1 c_1 dT}{T} = m_1 c_1\log \frac{T_{eq} }{T_1 } > 0
\end{gather*}
Per il primo corpo si ha un aumento di entropia, per il secondo corpo una diminuzione.
\[
	\Delta S_2 = m_2 c_2 \log \frac{T_2 }{T_{eq} }
\]
Si può dimostrare che la somma fra i due è sempre strettamente positiva.

\paragraph{Esempio 3} Si ha la trasformazione di un un gas perfetto, non ciclica, che lo porta da uno stato $A$ a uno stato $B$. L'espansione è isobara, il gas si trova in un recipiente con pistone mobile. Si valuta cosa accade all'intero universo quando il gas compie una trasformazione prima reversibile e poi irreversibile.
\begin{figure}[htpb]
	\centering
	

	% Pattern Info
	 
	\tikzset{
	pattern size/.store in=\mcSize, 
	pattern size = 5pt,
	pattern thickness/.store in=\mcThickness, 
	pattern thickness = 0.3pt,
	pattern radius/.store in=\mcRadius, 
	pattern radius = 1pt}
	\makeatletter
	\pgfutil@ifundefined{pgf@pattern@name@_swe55zz85}{
	\pgfdeclarepatternformonly[\mcThickness,\mcSize]{_swe55zz85}
	{\pgfqpoint{0pt}{-\mcThickness}}
	{\pgfpoint{\mcSize}{\mcSize}}
	{\pgfpoint{\mcSize}{\mcSize}}
	{
	\pgfsetcolor{\tikz@pattern@color}
	\pgfsetlinewidth{\mcThickness}
	\pgfpathmoveto{\pgfqpoint{0pt}{\mcSize}}
	\pgfpathlineto{\pgfpoint{\mcSize+\mcThickness}{-\mcThickness}}
	\pgfusepath{stroke}
	}}
	\makeatother

	% Pattern Info
	 
	\tikzset{
	pattern size/.store in=\mcSize, 
	pattern size = 5pt,
	pattern thickness/.store in=\mcThickness, 
	pattern thickness = 0.3pt,
	pattern radius/.store in=\mcRadius, 
	pattern radius = 1pt}
	\makeatletter
	\pgfutil@ifundefined{pgf@pattern@name@_kdvoi9lcg}{
	\pgfdeclarepatternformonly[\mcThickness,\mcSize]{_kdvoi9lcg}
	{\pgfqpoint{0pt}{-\mcThickness}}
	{\pgfpoint{\mcSize}{\mcSize}}
	{\pgfpoint{\mcSize}{\mcSize}}
	{
	\pgfsetcolor{\tikz@pattern@color}
	\pgfsetlinewidth{\mcThickness}
	\pgfpathmoveto{\pgfqpoint{0pt}{\mcSize}}
	\pgfpathlineto{\pgfpoint{\mcSize+\mcThickness}{-\mcThickness}}
	\pgfusepath{stroke}
	}}
	\makeatother
	\tikzset{every picture/.style={line width=0.75pt}} %set default line width to 0.75pt        

	\begin{tikzpicture}[x=0.75pt,y=0.75pt,yscale=-0.9,xscale=0.9]
	%uncomment if require: \path (0,300); %set diagram left start at 0, and has height of 300

	%Shape: Axis 2D [id:dp8058287341299237] 
	\draw  (70,229.2) -- (276.5,229.2)(90.65,84) -- (90.65,245.33) (269.5,224.2) -- (276.5,229.2) -- (269.5,234.2) (85.65,91) -- (90.65,84) -- (95.65,91)  ;
	%Shape: Circle [id:dp9545224062227355] 
	\draw  [fill={rgb, 255:red, 0; green, 0; blue, 0 }  ,fill opacity=1 ] (130.5,151.67) .. controls (130.5,150.47) and (131.47,149.5) .. (132.67,149.5) .. controls (133.86,149.5) and (134.83,150.47) .. (134.83,151.67) .. controls (134.83,152.86) and (133.86,153.83) .. (132.67,153.83) .. controls (131.47,153.83) and (130.5,152.86) .. (130.5,151.67) -- cycle ;
	%Shape: Circle [id:dp47566357433303774] 
	\draw  [fill={rgb, 255:red, 0; green, 0; blue, 0 }  ,fill opacity=1 ] (243.5,151.67) .. controls (243.5,150.47) and (244.47,149.5) .. (245.67,149.5) .. controls (246.86,149.5) and (247.83,150.47) .. (247.83,151.67) .. controls (247.83,152.86) and (246.86,153.83) .. (245.67,153.83) .. controls (244.47,153.83) and (243.5,152.86) .. (243.5,151.67) -- cycle ;
	%Straight Lines [id:da6627242733785048] 
	\draw  [dash pattern={on 4.5pt off 4.5pt}]  (132.67,151.67) -- (245.67,151.67) ;
	%Straight Lines [id:da3245537645654515] 
	\draw    (360,90) -- (360,190) ;
	%Straight Lines [id:da9971336049446691] 
	\draw    (450,90) -- (450,190) ;
	%Straight Lines [id:da7174015175372075] 
	\draw    (360,190) -- (450,190) ;
	%Straight Lines [id:da7830114963304644] 
	\draw [line width=2.25]    (360,131) -- (450,131) ;
	%Shape: Circle [id:dp1526687558537272] 
	\draw  [draw opacity=0][fill={rgb, 255:red, 212; green, 212; blue, 212 }  ,fill opacity=1 ] (388.5,146.13) .. controls (388.5,141.64) and (392.14,138) .. (396.63,138) .. controls (401.11,138) and (404.75,141.64) .. (404.75,146.13) .. controls (404.75,150.61) and (401.11,154.25) .. (396.63,154.25) .. controls (392.14,154.25) and (388.5,150.61) .. (388.5,146.13) -- cycle ;
	%Shape: Circle [id:dp11176800168068657] 
	\draw  [draw opacity=0][fill={rgb, 255:red, 212; green, 212; blue, 212 }  ,fill opacity=1 ] (414,143.13) .. controls (414,138.64) and (417.64,135) .. (422.13,135) .. controls (426.61,135) and (430.25,138.64) .. (430.25,143.13) .. controls (430.25,147.61) and (426.61,151.25) .. (422.13,151.25) .. controls (417.64,151.25) and (414,147.61) .. (414,143.13) -- cycle ;
	%Shape: Circle [id:dp1424335390311502] 
	\draw  [draw opacity=0][fill={rgb, 255:red, 212; green, 212; blue, 212 }  ,fill opacity=1 ] (370.5,157.63) .. controls (370.5,153.14) and (374.14,149.5) .. (378.63,149.5) .. controls (383.11,149.5) and (386.75,153.14) .. (386.75,157.63) .. controls (386.75,162.11) and (383.11,165.75) .. (378.63,165.75) .. controls (374.14,165.75) and (370.5,162.11) .. (370.5,157.63) -- cycle ;
	%Shape: Circle [id:dp6744602416932006] 
	\draw  [draw opacity=0][fill={rgb, 255:red, 212; green, 212; blue, 212 }  ,fill opacity=1 ] (404,160.63) .. controls (404,156.14) and (407.64,152.5) .. (412.13,152.5) .. controls (416.61,152.5) and (420.25,156.14) .. (420.25,160.63) .. controls (420.25,165.11) and (416.61,168.75) .. (412.13,168.75) .. controls (407.64,168.75) and (404,165.11) .. (404,160.63) -- cycle ;
	%Shape: Circle [id:dp5636972303361274] 
	\draw  [draw opacity=0][fill={rgb, 255:red, 212; green, 212; blue, 212 }  ,fill opacity=1 ] (369,177.63) .. controls (369,173.14) and (372.64,169.5) .. (377.13,169.5) .. controls (381.61,169.5) and (385.25,173.14) .. (385.25,177.63) .. controls (385.25,182.11) and (381.61,185.75) .. (377.13,185.75) .. controls (372.64,185.75) and (369,182.11) .. (369,177.63) -- cycle ;
	%Shape: Circle [id:dp7870849322523599] 
	\draw  [draw opacity=0][fill={rgb, 255:red, 212; green, 212; blue, 212 }  ,fill opacity=1 ] (429.5,154.63) .. controls (429.5,150.14) and (433.14,146.5) .. (437.63,146.5) .. controls (442.11,146.5) and (445.75,150.14) .. (445.75,154.63) .. controls (445.75,159.11) and (442.11,162.75) .. (437.63,162.75) .. controls (433.14,162.75) and (429.5,159.11) .. (429.5,154.63) -- cycle ;
	%Shape: Circle [id:dp33718622447106417] 
	\draw  [draw opacity=0][fill={rgb, 255:red, 212; green, 212; blue, 212 }  ,fill opacity=1 ] (389.5,177.63) .. controls (389.5,173.14) and (393.14,169.5) .. (397.63,169.5) .. controls (402.11,169.5) and (405.75,173.14) .. (405.75,177.63) .. controls (405.75,182.11) and (402.11,185.75) .. (397.63,185.75) .. controls (393.14,185.75) and (389.5,182.11) .. (389.5,177.63) -- cycle ;
	%Shape: Circle [id:dp8268977759879512] 
	\draw  [draw opacity=0][fill={rgb, 255:red, 212; green, 212; blue, 212 }  ,fill opacity=1 ] (421,175.13) .. controls (421,170.64) and (424.64,167) .. (429.13,167) .. controls (433.61,167) and (437.25,170.64) .. (437.25,175.13) .. controls (437.25,179.61) and (433.61,183.25) .. (429.13,183.25) .. controls (424.64,183.25) and (421,179.61) .. (421,175.13) -- cycle ;
	%Rounded Rect [id:dp8086986890864896] 
	\draw  [pattern=_swe55zz85,pattern size=6pt,pattern thickness=0.75pt,pattern radius=0pt, pattern color={rgb, 255:red, 155; green, 155; blue, 155}] (350.25,194) .. controls (350.25,191.79) and (352.04,190) .. (354.25,190) -- (456.75,190) .. controls (458.96,190) and (460.75,191.79) .. (460.75,194) -- (460.75,206) .. controls (460.75,208.21) and (458.96,210) .. (456.75,210) -- (354.25,210) .. controls (352.04,210) and (350.25,208.21) .. (350.25,206) -- cycle ;
	%Straight Lines [id:da2900764154820674] 
	\draw    (490,90) -- (490,190) ;
	%Straight Lines [id:da9993641947267269] 
	\draw    (580,90) -- (580,190) ;
	%Straight Lines [id:da3381628855234695] 
	\draw    (490,190) -- (580,190) ;
	%Straight Lines [id:da19064922329872536] 
	\draw [line width=2.25]    (490,97) -- (580,97) ;
	%Shape: Circle [id:dp5535222198453056] 
	\draw  [draw opacity=0][fill={rgb, 255:red, 212; green, 212; blue, 212 }  ,fill opacity=1 ] (496.5,113.46) .. controls (496.5,108.97) and (500.14,105.33) .. (504.63,105.33) .. controls (509.11,105.33) and (512.75,108.97) .. (512.75,113.46) .. controls (512.75,117.95) and (509.11,121.58) .. (504.63,121.58) .. controls (500.14,121.58) and (496.5,117.95) .. (496.5,113.46) -- cycle ;
	%Shape: Circle [id:dp9147284437607994] 
	\draw  [draw opacity=0][fill={rgb, 255:red, 212; green, 212; blue, 212 }  ,fill opacity=1 ] (528.33,114.79) .. controls (528.33,110.3) and (531.97,106.67) .. (536.46,106.67) .. controls (540.95,106.67) and (544.58,110.3) .. (544.58,114.79) .. controls (544.58,119.28) and (540.95,122.92) .. (536.46,122.92) .. controls (531.97,122.92) and (528.33,119.28) .. (528.33,114.79) -- cycle ;
	%Shape: Circle [id:dp9893912797861739] 
	\draw  [draw opacity=0][fill={rgb, 255:red, 212; green, 212; blue, 212 }  ,fill opacity=1 ] (512,130.96) .. controls (512,126.47) and (515.64,122.83) .. (520.13,122.83) .. controls (524.61,122.83) and (528.25,126.47) .. (528.25,130.96) .. controls (528.25,135.45) and (524.61,139.08) .. (520.13,139.08) .. controls (515.64,139.08) and (512,135.45) .. (512,130.96) -- cycle ;
	%Shape: Circle [id:dp8866417359566012] 
	\draw  [draw opacity=0][fill={rgb, 255:red, 212; green, 212; blue, 212 }  ,fill opacity=1 ] (552,125.29) .. controls (552,120.8) and (555.64,117.17) .. (560.13,117.17) .. controls (564.61,117.17) and (568.25,120.8) .. (568.25,125.29) .. controls (568.25,129.78) and (564.61,133.42) .. (560.13,133.42) .. controls (555.64,133.42) and (552,129.78) .. (552,125.29) -- cycle ;
	%Shape: Circle [id:dp4898906214425265] 
	\draw  [draw opacity=0][fill={rgb, 255:red, 212; green, 212; blue, 212 }  ,fill opacity=1 ] (499,177.63) .. controls (499,173.14) and (502.64,169.5) .. (507.13,169.5) .. controls (511.61,169.5) and (515.25,173.14) .. (515.25,177.63) .. controls (515.25,182.11) and (511.61,185.75) .. (507.13,185.75) .. controls (502.64,185.75) and (499,182.11) .. (499,177.63) -- cycle ;
	%Shape: Circle [id:dp4152153683103923] 
	\draw  [draw opacity=0][fill={rgb, 255:red, 212; green, 212; blue, 212 }  ,fill opacity=1 ] (503,152.63) .. controls (503,148.14) and (506.64,144.5) .. (511.13,144.5) .. controls (515.61,144.5) and (519.25,148.14) .. (519.25,152.63) .. controls (519.25,157.11) and (515.61,160.75) .. (511.13,160.75) .. controls (506.64,160.75) and (503,157.11) .. (503,152.63) -- cycle ;
	%Shape: Circle [id:dp32271698384904934] 
	\draw  [draw opacity=0][fill={rgb, 255:red, 212; green, 212; blue, 212 }  ,fill opacity=1 ] (546.83,148.96) .. controls (546.83,144.47) and (550.47,140.83) .. (554.96,140.83) .. controls (559.45,140.83) and (563.08,144.47) .. (563.08,148.96) .. controls (563.08,153.45) and (559.45,157.08) .. (554.96,157.08) .. controls (550.47,157.08) and (546.83,153.45) .. (546.83,148.96) -- cycle ;
	%Shape: Circle [id:dp5475742467603371] 
	\draw  [draw opacity=0][fill={rgb, 255:red, 212; green, 212; blue, 212 }  ,fill opacity=1 ] (551,175.13) .. controls (551,170.64) and (554.64,167) .. (559.13,167) .. controls (563.61,167) and (567.25,170.64) .. (567.25,175.13) .. controls (567.25,179.61) and (563.61,183.25) .. (559.13,183.25) .. controls (554.64,183.25) and (551,179.61) .. (551,175.13) -- cycle ;
	%Rounded Rect [id:dp7271810046379437] 
	\draw  [pattern=_kdvoi9lcg,pattern size=6pt,pattern thickness=0.75pt,pattern radius=0pt, pattern color={rgb, 255:red, 155; green, 155; blue, 155}] (480.25,194) .. controls (480.25,191.79) and (482.04,190) .. (484.25,190) -- (586.75,190) .. controls (588.96,190) and (590.75,191.79) .. (590.75,194) -- (590.75,206) .. controls (590.75,208.21) and (588.96,210) .. (586.75,210) -- (484.25,210) .. controls (482.04,210) and (480.25,208.21) .. (480.25,206) -- cycle ;
	%Straight Lines [id:da829616761495358] 
	\draw    (427.67,225) -- (514,225) ;
	\draw [shift={(517,225)}, rotate = 180] [fill={rgb, 255:red, 0; green, 0; blue, 0 }  ][line width=0.08]  [draw opacity=0] (10.72,-5.15) -- (0,0) -- (10.72,5.15) -- (7.12,0) -- cycle    ;
	%Straight Lines [id:da8315949696634524] 
	\draw    (534,200) .. controls (532.33,198.33) and (532.33,196.67) .. (534,195) .. controls (535.67,193.33) and (535.67,191.67) .. (534,190) .. controls (532.33,188.33) and (532.33,186.67) .. (534,185) .. controls (535.67,183.33) and (535.67,181.67) .. (534,180) .. controls (532.33,178.33) and (532.33,176.67) .. (534,175) .. controls (535.67,173.33) and (535.67,171.67) .. (534,170) -- (534,168.5) -- (534,165.5)(537,200) .. controls (535.33,198.33) and (535.33,196.67) .. (537,195) .. controls (538.67,193.33) and (538.67,191.67) .. (537,190) .. controls (535.33,188.33) and (535.33,186.67) .. (537,185) .. controls (538.67,183.33) and (538.67,181.67) .. (537,180) .. controls (535.33,178.33) and (535.33,176.67) .. (537,175) .. controls (538.67,173.33) and (538.67,171.67) .. (537,170) -- (537,168.5) -- (537,165.5) ;
	\draw [shift={(535.5,156.5)}, rotate = 450] [fill={rgb, 255:red, 0; green, 0; blue, 0 }  ][line width=0.08]  [draw opacity=0] (10.72,-5.15) -- (0,0) -- (10.72,5.15) -- (7.12,0) -- cycle    ;

	% Text Node
	\draw (78,84.33) node    {$p$};
	% Text Node
	\draw (296.67,228.33) node    {$V$};
	% Text Node
	\draw (120.33,162.33) node    {$A$};
	% Text Node
	\draw (256.5,154.83) node    {$B$};
	% Text Node
	\draw (407.33,226.33) node    {$T_{A}$};
	% Text Node
	\draw (537.33,226.33) node    {$T_{B}$};
	% Text Node
	\draw (407.33,111.33) node    {$p_{ext}$};
	% Text Node
	\draw (536.67,78) node    {$p_{ext}$};
	% Text Node
	\draw (345.33,176.33) node    {$T_{A}$};
	% Text Node
	\draw (597.33,176.33) node    {$T_{B}$};


	\end{tikzpicture}
\end{figure}
\FloatBarrier

\textbf{Processo reversibile}. Nell'espandersi il gas si sarà scaldato. Esso avrà bisogno di assorbire calore da un certo serbatoio. La pressione esterna viene mantenuta costante.
\begin{gather*}
	\Delta S_{\text{universo} } = \Delta S_{\text{gas} } + \Delta S_{\text{serb} } \\
	\Delta S_{\text{gas} } = nc_v \log \frac{T_B }{T_A } + nR\log \frac{V_B }{V_A }
\end{gather*}
Se la pressione è costante, volume e temperatura variano linearmente. Quindi:
\[
	pV=nRT \qquad \frac{V_B }{V_A } = \frac{T_B }{T_A } \qquad \Delta S_{\text{gas} } = \log \frac{T_B }{T_A }(c_v+R)n = nc_p\log \frac{T_B }{T_A }
\]
Si fa espandere in maniera molto lenta il gas mettendolo a contatto con infiniti serbatoi, fino a che non si passa dalla temperatura iniziale a quella finale. Bisogna valutare la variazione di entropia di tutti i serbatoi, facendo la somma continua di tutti i calori scambiati da ogni serbatoio. Essi dal punto di vista dei serbatoi sono ceduti quindi ci si aspetta una somma di tanti termini negativi. Il calore scambiato dal gas a pressione costante è:
\[
	\Delta S_{\text{serb} } = \int \frac{dQ_{\text{rev} } }{T} = \int_{T_A }^{T_B } \frac{n c_p dT }{T} = - nc_p\log \frac{T_B }{T_A }
\]
La variazione di entropia dell'universo è uguale a zero perché il processo è reversibile.

\textbf{Processo irreversibile}. Si suppone quindi che il gas venga scaldato alla temperatura $T_B$ mettendolo immediatamente a contatto con il serbatoio $T_B$. Il gas scambierà calore solo con lui. Nel valutare la variazione di entropia del gas, lo stato finale è sempre lo stesso perciò la variazione di entropia dal suo punto di vista sarà la stessa ottenuta per l'esempio precedente. È il serbatoio, che è uno solo, a diminuire l'entropia di una quantità minore di quella vista prima:
\begin{gather*}
	\Delta S_{\text{serb} } = - \frac{Q}{T_B } = \frac{-nc_v(T_B-T_A)}{T_A} \\
	\Delta S_{\text{universo}} = nc_p\log \frac{T_B }{T_A } - nc_p\frac{T_B - T_A  }{T_B } > 0
\end{gather*}

\paragraph{Esempio 4} Si consideri il caso in cui invece il gas segue una trasformazione adiabatica partendo da un certo stato $A$. Si ha un contenitore a pareti isolanti. Il gas in tal caso è isolato termicamente rispetto al resto del mondo. Nel caso di una trasformazione adiabatica il gas è già di suo l'universo termodinamico perché non può scambiare calore con nulla. Si studiano i due casi, reversibile o irreversibile.

\textbf{Processo reversibile}. Non c'è calore scambiato in maniera reversibile. Quindi, quando il gas si espande in maniera adiabatica, l'entropia non può aumentare perché si andrebbe contro il principio di accrescimento dell'entropia. Questo dà un'informazione molto interessante. Se si disegna nel piano di Clapeyron la funzione analitica che descrive la trasformazione reversibile, essa porterà a uno stato $B$ in cui il gas si è espanso, ha diminuito la sua pressione senza scambiare calore. Questa e tutte le curve del tipo $pV^{\gamma}=\text{costante}$ sono curve \textbf{isoentropiche}, dove cioè $S=\text{costante}=S(A)$.
\begin{figure}[htpb]
	\centering
	

	\tikzset{every picture/.style={line width=0.75pt}} %set default line width to 0.75pt        

	\begin{tikzpicture}[x=0.75pt,y=0.75pt,yscale=-1,xscale=1]
	%uncomment if require: \path (0,300); %set diagram left start at 0, and has height of 300

	%Shape: Axis 2D [id:dp6791519383518894] 
	\draw  (128.5,213.85) -- (334.75,213.85)(138.39,73) -- (138.39,229) (327.75,208.85) -- (334.75,213.85) -- (327.75,218.85) (133.39,80) -- (138.39,73) -- (143.39,80)  ;
	%Straight Lines [id:da040550923790030335] 
	\draw    (277.18,90.67) -- (249.75,119.5) ;
	\draw [shift={(279.25,88.5)}, rotate = 133.58] [fill={rgb, 255:red, 0; green, 0; blue, 0 }  ][line width=0.08]  [draw opacity=0] (10.72,-5.15) -- (0,0) -- (10.72,5.15) -- (7.12,0) -- cycle    ;
	%Curve Lines [id:da05652566709120643] 
	\draw    (169.67,133) .. controls (183.75,167.5) and (204.67,189.83) .. (249.25,195.5) ;
	%Curve Lines [id:da5603647881324592] 
	\draw    (196.67,117.17) .. controls (207.75,149) and (220.25,158.5) .. (253.33,165.83) ;
	%Curve Lines [id:da9493672186610809] 
	\draw    (223.33,107.83) .. controls (228.75,131) and (241.75,138.5) .. (259.33,143.83) ;
	%Straight Lines [id:da8585995024490722] 
	\draw    (202,130) -- (202,178) ;
	%Shape: Circle [id:dp6458738509454554] 
	\draw  [fill={rgb, 255:red, 0; green, 0; blue, 0 }  ,fill opacity=1 ] (200.25,130.75) .. controls (200.25,129.78) and (201.03,129) .. (202,129) .. controls (202.97,129) and (203.75,129.78) .. (203.75,130.75) .. controls (203.75,131.72) and (202.97,132.5) .. (202,132.5) .. controls (201.03,132.5) and (200.25,131.72) .. (200.25,130.75) -- cycle ;
	%Shape: Circle [id:dp08600776298855717] 
	\draw  [fill={rgb, 255:red, 0; green, 0; blue, 0 }  ,fill opacity=1 ] (200.25,177.75) .. controls (200.25,176.78) and (201.03,176) .. (202,176) .. controls (202.97,176) and (203.75,176.78) .. (203.75,177.75) .. controls (203.75,178.72) and (202.97,179.5) .. (202,179.5) .. controls (201.03,179.5) and (200.25,178.72) .. (200.25,177.75) -- cycle ;

	% Text Node
	\draw (127,84.5) node    {$p$};
	% Text Node
	\draw (348.5,214) node    {$V$};
	% Text Node
	\draw (277,193.5) node    {$S_{0}$};
	% Text Node
	\draw (277,166.5) node    {$S_{1}$};
	% Text Node
	\draw (277,143.5) node    {$S_{2}$};
	% Text Node
	\draw (194.5,191) node    {$A$};
	% Text Node
	\draw (209,115) node    {$B$};


	\end{tikzpicture}
\end{figure}
\FloatBarrier
Così come erano state disegnate tutte le isoterme e si era mostrato come la temperatura cresce in una certa direzione, analogamente nel piano di Clapeyron si può disegnare una famiglia di curve isoentropiche del tipo $pV^{\gamma}=\text{costante}$ e l'entropia aumenta nel verso in figura.

\textbf{Processo irreversibile}. Si lascia espandere il gas rimuovendo il peso che c'era sul pistone tutto in una volta. Si arriva in uno stato finale senza scambiare calore. Facendo considerazioni soltanto sulla variazione di entropia del processo, il sistema dovrà arrivare per forza in uno stato diverso, perché altrimenti se passasse per lo stato $B$ pur seguendo un processo irreversibile, si starebbe muovendo lungo una curva isoentropica e quindi arriverebbe ad avere la stessa entropia di $A$, ottenendo una variazione di entropia pari a $0$. Ma se il processo è irreversibile, bisogna aspettarsi che la variazione di entropia del gas sia maggiore di $0$. Si dovrà arrivare in un nuovo stato $B'$ più a sinistra, su una nuova isoentropica con entropia maggiore. Mentre con le isoterme, le isocore e le isobare, partendo da un certo stato si arriva sempre a un altro certo stato, nelle adiabatiche l'irreversiblità porta sempre il gas ad arrivare ad uno stato finale diverso.

Si osservi inoltre che quando si passa tramite un'adiabatica irreversibile da uno stato $A$ ad uno stato $B$, con $S_B$ necessariamente maggiore di $S_A$, non è più possibile ritornare in $A$ adiabaticamente, né in modo reversibile né irreversibile, perché in ogni caso ciò comporterebbe $\Delta S<0$. Si può allora dire che una trasformazione adiabatica irreversibile presenta una \emph{irreversibilità in senso stretto}. Questo non è il caso di altre trasformazioni irreversibili, perché in generale l'intervento dell'ambiente con le sue variazioni di entropia rende possibile il ritorno allo stato iniziale con lo stesso tipo di trasformazione senza violare il principio di aumento dell'entropia dell'universo termodinamico.
In altre parole, due stati collegati con un'adiabatica irreversibile hanno necessariamente entropia diversa e non possono quindi essere collegati anche con un'adiabatica reversibile, che è un'isoentropica.

La famiglia delle curve adiabatiche reversibili sono caratterizzate dalla relazione:
\[
	pV^{\gamma } = \text{costante}
\]
Tale famiglia è rappresentata in figura. Se ci si muove su una di queste curve, ci si sta muovendo su una trasformazione a entropia costante.

Come si fa a capire se spostandosi dalla curva a entropia $S_0$ a quella a entropia $S_1$ aumenta $S$?
Si supponga di prendere uno stato qualunque $A$ sulla curva $S_0$ e di muoversi con un' isocora su uno stato $B$ che è ad un certo punto della curva $S_1$ facendo fondamentalmente un riscaldamento a volume costante. Si scrive la variazione di entropia da stato $A$ a stato $B$ per la variazione di entropia di un gas perfetto.
\[
	\Delta S = S(B)-S(A) = nc_v\log \frac{T_2 }{T_1 } + nR \log \frac{V_2 }{V_1 }
\]
Nel caso che si sta considerando il secondo addendo non c'è, rimane solo il primo. $T_B$ è maggiore di $T_A$ perché ci si sposta su un'isoterma più a destra. È quindi un riscaldamento e $S(B)>S(A)$. Se si vanno a disegnare le varie isoentropiche nel piano di Clapeyron, si vede fondamentalmente che l'entropia cresce nel verso della freccia.
Mettendo le famiglie di curve, isoterme e isoentropiche, si possono fare delle considerazioni interessanti sulle porzioni del piano di Clapeyron che non sono permesse. Si sfrutta il principio di accrescimento dell'entropia dell'universo. Ci si pone in uno stato $A$ e in tale stato si vanno a disegnare la curva isoterma e quella isoentropica che passano per quello stato.
\begin{figure}[htpb]
	\centering
	

	\tikzset{every picture/.style={line width=0.75pt}} %set default line width to 0.75pt        

	\begin{tikzpicture}[x=0.75pt,y=0.75pt,yscale=-1,xscale=1]
	%uncomment if require: \path (0,300); %set diagram left start at 0, and has height of 300

	%Straight Lines [id:da0741049687567581] 
	\draw [color={rgb, 255:red, 214; green, 214; blue, 214 }  ,draw opacity=1 ]   (295,250.67) -- (270.67,275) ;
	%Straight Lines [id:da012561168898262531] 
	\draw [color={rgb, 255:red, 214; green, 214; blue, 214 }  ,draw opacity=1 ]   (167.67,138) -- (138.67,167) ;
	%Straight Lines [id:da6370578645029346] 
	\draw [color={rgb, 255:red, 214; green, 214; blue, 214 }  ,draw opacity=1 ]   (188,137.67) -- (138.67,187) ;
	%Straight Lines [id:da11423227056922736] 
	\draw [color={rgb, 255:red, 214; green, 214; blue, 214 }  ,draw opacity=1 ]   (207.67,138) -- (138.67,207) ;
	%Straight Lines [id:da808109097546619] 
	\draw [color={rgb, 255:red, 214; green, 214; blue, 214 }  ,draw opacity=1 ]   (228,137.67) -- (138.67,227) ;
	%Straight Lines [id:da5056432870620773] 
	\draw [color={rgb, 255:red, 214; green, 214; blue, 214 }  ,draw opacity=1 ]   (247.67,138) -- (138.67,247) ;
	%Straight Lines [id:da9799464593161868] 
	\draw [color={rgb, 255:red, 214; green, 214; blue, 214 }  ,draw opacity=1 ]   (207,198.67) -- (138.67,267) ;
	%Straight Lines [id:da7504849115865206] 
	\draw [color={rgb, 255:red, 214; green, 214; blue, 214 }  ,draw opacity=1 ]   (215,210.67) -- (150.67,275) ;
	%Straight Lines [id:da7433407923845881] 
	\draw [color={rgb, 255:red, 214; green, 214; blue, 214 }  ,draw opacity=1 ]   (225,220.67) -- (170.67,275) ;
	%Straight Lines [id:da38984121270333993] 
	\draw [color={rgb, 255:red, 214; green, 214; blue, 214 }  ,draw opacity=1 ]   (236,229.67) -- (190.67,275) ;
	%Straight Lines [id:da654830155031356] 
	\draw [color={rgb, 255:red, 214; green, 214; blue, 214 }  ,draw opacity=1 ]   (249,236.67) -- (210.67,275) ;
	%Straight Lines [id:da6109502725467904] 
	\draw [color={rgb, 255:red, 214; green, 214; blue, 214 }  ,draw opacity=1 ]   (263,242.67) -- (230.67,275) ;
	%Straight Lines [id:da73216921122118] 
	\draw [color={rgb, 255:red, 214; green, 214; blue, 214 }  ,draw opacity=1 ]   (278,247.67) -- (250.67,275) ;
	%Straight Lines [id:da23448718854163642] 
	\draw [color={rgb, 255:red, 214; green, 214; blue, 214 }  ,draw opacity=1 ]   (288,137.67) -- (225.67,200) ;
	%Straight Lines [id:da7315296214909419] 
	\draw [color={rgb, 255:red, 214; green, 214; blue, 214 }  ,draw opacity=1 ]   (308,137.67) -- (241.67,204) ;
	%Straight Lines [id:da34186492151441694] 
	\draw [color={rgb, 255:red, 214; green, 214; blue, 214 }  ,draw opacity=1 ]   (328,137.67) -- (258.67,207) ;
	%Straight Lines [id:da4290520340827755] 
	\draw [color={rgb, 255:red, 214; green, 214; blue, 214 }  ,draw opacity=1 ]   (331,154.67) -- (276.67,209) ;
	%Straight Lines [id:da5480858567328142] 
	\draw [color={rgb, 255:red, 214; green, 214; blue, 214 }  ,draw opacity=1 ]   (330,175.67) -- (295.67,210) ;
	%Shape: Axis 2D [id:dp9314597919243865] 
	\draw  (128.5,274.85) -- (334.75,274.85)(138.39,134) -- (138.39,290) (327.75,269.85) -- (334.75,274.85) -- (327.75,279.85) (133.39,141) -- (138.39,134) -- (143.39,141)  ;
	%Curve Lines [id:da49963022668914747] 
	\draw    (160.75,159) .. controls (179.25,194.5) and (241.75,211) .. (320.75,210.5) ;
	%Curve Lines [id:da2838240681359938] 
	\draw    (191.25,149.5) .. controls (200.75,215) and (239.25,246) .. (304.25,251.5) ;
	%Straight Lines [id:da9931014522236101] 
	\draw [color={rgb, 255:red, 155; green, 155; blue, 155 }  ,draw opacity=1 ] [dash pattern={on 0.84pt off 2.51pt}]  (204,141) -- (204,275) ;
	%Shape: Circle [id:dp8807672671591289] 
	\draw  [fill={rgb, 255:red, 0; green, 0; blue, 0 }  ,fill opacity=1 ] (202.25,192.75) .. controls (202.25,191.78) and (203.03,191) .. (204,191) .. controls (204.97,191) and (205.75,191.78) .. (205.75,192.75) .. controls (205.75,193.72) and (204.97,194.5) .. (204,194.5) .. controls (203.03,194.5) and (202.25,193.72) .. (202.25,192.75) -- cycle ;
	%Straight Lines [id:da5075107359701121] 
	\draw [color={rgb, 255:red, 214; green, 214; blue, 214 }  ,draw opacity=1 ]   (268,137.67) -- (210.67,195) ;

	% Text Node
	\draw (127,145.5) node    {$p$};
	% Text Node
	\draw (348.5,275) node    {$V$};
	% Text Node
	\draw (216,180) node    {$A$};
	% Text Node
	\draw (355.5,251) node   [align=left] {isoentropica};
	% Text Node
	\draw (363,209.5) node   [align=left] {isoterma};


	\end{tikzpicture}
\end{figure}
\FloatBarrier







































\section{Porzioni di piano permesse dal II principio}

Si vuol far espandere il gas con una trasformazione reale, quindi irreversibile, adiabatica, che porta da $A$ ad uno stato $B'$. Qual è sia là porzione di piano in cui può trovarsi $B'$?
È intanto permessa quella porzione di piano per cui $V>V_A$. Essendo adiabatica, la variazione di entropia del gas è uguale alla variazione di entropia dell'universo. Poiché è irreversibile la variazione di entropia del gas deve essere sicuramente positiva. Sono permesse solo le porzioni di spazio al di sopra della isoentropica passante per $A$.
Quando si fa espandere un gas si può fare una considerazione sulla temperatura. Infatti se lo si fa espandere non permettendogli di assorbire calore dall'ambiente costante, esso si raffredda. Quindi può stare solo in quella regione di piano al di sotto della isoterma passante per $A$.
Questo discorso fa capire che, mentre con il primo principio della termodinamica considerando questa situazione si sarebbero poste delle limitazioni solo riguardo al fatto che il volume deve aumentare e la temperatura diminuire, si capisce che ci sono regioni di spazio non permesse a causa del fatto che l'entropia dell'universo deve sempre aumentare. Il secondo principio della termodinamica pone dei limiti al modo in cui il sistema deve evolvere, alle possibili trasformazioni che si possono realizzare nella pratica. Ci si rende conto che la situazione $\mathcal{L} = -\Delta U$ è molto particolare, nel senso che la maggior parte delle trasformazioni di un sistema termodinamico avvengono solo se c'è scambio di calore.







































\section{Il piano di Gibbs}

Esiste un altro piano interessante noto come \textbf{piano di Gibbs}. In esso si va a disegnare lo stato di un sistema termodinamico in funzione delle variabili termodinamiche temperatura e entropia. L'entropia infatti, essendo funzione di stato, può essere scelta come variabile indipendente per descrivere, insieme ad una seconda variabile indipendente opportunamente scelta, lo stato termodinamico di un sistema. Nel piano di Clapeyron, se si prende il prodotto fra la variabile pressione e un infinitesimo della variabile sulle $x$, il volume, si ottiene un'energia, in particolare il lavoro compiuto da un sistema termodinamico.
Si riprenda allora la variazione di entropia finita fra due stati. Essa è uguale all'integrale del calore infinitesimo scambiato in maniera reversibile su $T$ integrato fra stato iniziale e finale. Questa relazione, scritta in termini finiti, si potrebbe anche esprimere come quantità infinitesima fra due stati molto vicini.
\[
	dS = \frac{dQ_{\text{rev} } }{T} \implies T\,dS = dQ_{\text{rev} }
\]
Così come si può rappresentare una trasformazione reversibile nel piano $pV$,  e valutarne l'area sottesa, che rappresenta il lavoro reversibile fatto dal gas, in maniera del tutto analoga, se si disegna nel piano $TS$ una qualunque trasformazione reversibile, l'area sottesa dalla curva è il calore scambiato durante la trasformazione in maniera reversibile. È interessante notare come si disegna il ciclo di Carnot nel piano di Gibbs.
\begin{figure}[htpb]
	\centering
	

	\tikzset{every picture/.style={line width=0.75pt}} %set default line width to 0.75pt        

	\begin{tikzpicture}[x=0.75pt,y=0.75pt,yscale=-1,xscale=1]
	%uncomment if require: \path (0,300); %set diagram left start at 0, and has height of 300

	%Shape: Axis 2D [id:dp6207852838282175] 
	\draw  (68.5,188.5) -- (255.5,188.5)(77.46,63) -- (77.46,202) (248.5,183.5) -- (255.5,188.5) -- (248.5,193.5) (72.46,70) -- (77.46,63) -- (82.46,70)  ;
	%Curve Lines [id:da17880604881557938] 
	\draw    (115,88) .. controls (132.33,108) and (159,120.67) .. (203,120) ;
	\draw [shift={(155.22,113.94)}, rotate = 199.15] [fill={rgb, 255:red, 0; green, 0; blue, 0 }  ][line width=0.08]  [draw opacity=0] (10.72,-5.15) -- (0,0) -- (10.72,5.15) -- (7.12,0) -- cycle    ;
	%Curve Lines [id:da7787902897460273] 
	\draw    (203,120) .. controls (205.67,138) and (214.33,156) .. (227.67,163.33) ;
	\draw [shift={(210.72,144.51)}, rotate = 245.12] [fill={rgb, 255:red, 0; green, 0; blue, 0 }  ][line width=0.08]  [draw opacity=0] (10.72,-5.15) -- (0,0) -- (10.72,5.15) -- (7.12,0) -- cycle    ;
	%Curve Lines [id:da5376663670104085] 
	\draw    (115,88) .. controls (117.67,106) and (121.67,128.67) .. (129,140.67) ;
	\draw [shift={(119.92,115.17)}, rotate = 77.79] [fill={rgb, 255:red, 0; green, 0; blue, 0 }  ][line width=0.08]  [draw opacity=0] (10.72,-5.15) -- (0,0) -- (10.72,5.15) -- (7.12,0) -- cycle    ;
	%Curve Lines [id:da706592795484827] 
	\draw    (129,140.67) .. controls (149.67,155.33) and (183.67,164) .. (227.67,163.33) ;
	\draw [shift={(176.72,159.28)}, rotate = 11.86] [fill={rgb, 255:red, 0; green, 0; blue, 0 }  ][line width=0.08]  [draw opacity=0] (10.72,-5.15) -- (0,0) -- (10.72,5.15) -- (7.12,0) -- cycle    ;
	%Shape: Axis 2D [id:dp4893064797014206] 
	\draw  (318.5,188.5) -- (505.5,188.5)(327.46,63) -- (327.46,202) (498.5,183.5) -- (505.5,188.5) -- (498.5,193.5) (322.46,70) -- (327.46,63) -- (332.46,70)  ;
	%Straight Lines [id:da6745812171453553] 
	\draw    (360.67,110.67) -- (468.33,110.67) ;
	%Straight Lines [id:da09160897438719262] 
	\draw    (360.67,110.67) -- (468.33,110.67) ;
	\draw [shift={(414.5,110.67)}, rotate = 180] [fill={rgb, 255:red, 0; green, 0; blue, 0 }  ][line width=0.08]  [draw opacity=0] (10.72,-5.15) -- (0,0) -- (10.72,5.15) -- (7.12,0) -- cycle    ;
	%Straight Lines [id:da3351262724181472] 
	\draw    (360.67,160.67) -- (468.33,160.67) ;
	\draw [shift={(414.5,160.67)}, rotate = 0] [fill={rgb, 255:red, 0; green, 0; blue, 0 }  ][line width=0.08]  [draw opacity=0] (10.72,-5.15) -- (0,0) -- (10.72,5.15) -- (7.12,0) -- cycle    ;
	%Straight Lines [id:da7493185984968482] 
	\draw    (468.33,160.67) -- (468.33,110.67) ;
	\draw [shift={(468.33,135.67)}, rotate = 270] [fill={rgb, 255:red, 0; green, 0; blue, 0 }  ][line width=0.08]  [draw opacity=0] (10.72,-5.15) -- (0,0) -- (10.72,5.15) -- (7.12,0) -- cycle    ;
	%Straight Lines [id:da37877542289534505] 
	\draw    (360.67,160.67) -- (360.67,110.67) ;
	\draw [shift={(360.67,135.67)}, rotate = 450] [fill={rgb, 255:red, 0; green, 0; blue, 0 }  ][line width=0.08]  [draw opacity=0] (10.72,-5.15) -- (0,0) -- (10.72,5.15) -- (7.12,0) -- cycle    ;

	% Text Node
	\draw (65,69.5) node    {$p$};
	% Text Node
	\draw (269.5,191) node    {$V$};
	% Text Node
	\draw (312,69.5) node    {$T$};
	% Text Node
	\draw (519.5,191) node    {$S$};
	% Text Node
	\draw (352.33,98.83) node    {$A$};
	% Text Node
	\draw (477,97.5) node    {$B$};
	% Text Node
	\draw (478.33,161.5) node    {$C$};
	% Text Node
	\draw (349,166.83) node    {$D$};
	% Text Node
	\draw (112.33,76.17) node    {$A$};
	% Text Node
	\draw (212.33,110.17) node    {$B$};
	% Text Node
	\draw (239.67,162.83) node    {$C$};
	% Text Node
	\draw (122.33,152.17) node    {$D$};
	% Text Node
	\draw (172.67,218.67) node   [align=left] {Piano di Clapeyron};
	% Text Node
	\draw (422.67,218.67) node   [align=left] {Piano di Gibbs};


	\end{tikzpicture}
\end{figure}
\FloatBarrier
Le isoterme si diagrammano come rette orizzontali alle quote rispettivamente $T_1$ e $T_2$. Le isoentropiche invece si rappresentano come rette verticali. Il piano di Gibbs è comodo perché si può calcolare direttamente dal grafico il rendimento di una macchina reversibile. Esso è dato dal rapporto tra l'area racchiusa dal ciclo (che rappresenta il lavoro per il primo principio della termodinamica, oltre che al calore) e quella sottesa dalla curva superiore.
Risulta così evidente che il rendimento della macchina è inferiore all'unità.

Si consideri ad esempio una macchina più complicata di Carnot che scambia calore con tre serbatoi. Il ciclo è rappresentato in figura su entrambi i piani.
\begin{figure}[htpb]
	\centering
	

	% Pattern Info
	 
	\tikzset{
	pattern size/.store in=\mcSize, 
	pattern size = 5pt,
	pattern thickness/.store in=\mcThickness, 
	pattern thickness = 0.3pt,
	pattern radius/.store in=\mcRadius, 
	pattern radius = 1pt}
	\makeatletter
	\pgfutil@ifundefined{pgf@pattern@name@_25zm4e1hf lines}{
	\pgfdeclarepatternformonly[\mcThickness,\mcSize]{_25zm4e1hf}
	{\pgfqpoint{0pt}{0pt}}
	{\pgfpoint{\mcSize+\mcThickness}{\mcSize+\mcThickness}}
	{\pgfpoint{\mcSize}{\mcSize}}
	{\pgfsetcolor{\tikz@pattern@color}
	\pgfsetlinewidth{\mcThickness}
	\pgfpathmoveto{\pgfpointorigin}
	\pgfpathlineto{\pgfpoint{\mcSize}{0}}
	\pgfusepath{stroke}}}
	\makeatother

	% Pattern Info
	 
	\tikzset{
	pattern size/.store in=\mcSize, 
	pattern size = 5pt,
	pattern thickness/.store in=\mcThickness, 
	pattern thickness = 0.3pt,
	pattern radius/.store in=\mcRadius, 
	pattern radius = 1pt}
	\makeatletter
	\pgfutil@ifundefined{pgf@pattern@name@_063m0qo2e}{
	\makeatletter
	\pgfdeclarepatternformonly[\mcRadius,\mcThickness,\mcSize]{_063m0qo2e}
	{\pgfpoint{-0.5*\mcSize}{-0.5*\mcSize}}
	{\pgfpoint{0.5*\mcSize}{0.5*\mcSize}}
	{\pgfpoint{\mcSize}{\mcSize}}
	{
	\pgfsetcolor{\tikz@pattern@color}
	\pgfsetlinewidth{\mcThickness}
	\pgfpathcircle\pgfpointorigin{\mcRadius}
	\pgfusepath{stroke}
	}}
	\makeatother

	% Pattern Info
	 
	\tikzset{
	pattern size/.store in=\mcSize, 
	pattern size = 5pt,
	pattern thickness/.store in=\mcThickness, 
	pattern thickness = 0.3pt,
	pattern radius/.store in=\mcRadius, 
	pattern radius = 1pt}
	\makeatletter
	\pgfutil@ifundefined{pgf@pattern@name@_u93ic1mtb}{
	\pgfdeclarepatternformonly[\mcThickness,\mcSize]{_u93ic1mtb}
	{\pgfqpoint{0pt}{0pt}}
	{\pgfpoint{\mcSize+\mcThickness}{\mcSize+\mcThickness}}
	{\pgfpoint{\mcSize}{\mcSize}}
	{
	\pgfsetcolor{\tikz@pattern@color}
	\pgfsetlinewidth{\mcThickness}
	\pgfpathmoveto{\pgfqpoint{0pt}{0pt}}
	\pgfpathlineto{\pgfpoint{\mcSize+\mcThickness}{\mcSize+\mcThickness}}
	\pgfusepath{stroke}
	}}
	\makeatother
	\tikzset{every picture/.style={line width=0.75pt}} %set default line width to 0.75pt        

	\begin{tikzpicture}[x=0.75pt,y=0.75pt,yscale=-1,xscale=1]
	%uncomment if require: \path (0,300); %set diagram left start at 0, and has height of 300

	%Shape: Rectangle [id:dp35068313855016964] 
	\draw  [draw opacity=0][pattern=_25zm4e1hf,pattern size=2.25pt,pattern thickness=0.75pt,pattern radius=0pt, pattern color={rgb, 255:red, 155; green, 155; blue, 155}] (360.67,160.67) -- (468.33,160.67) -- (468.33,186.8) -- (360.67,186.8) -- cycle ;
	%Shape: Rectangle [id:dp47426860666934667] 
	\draw  [draw opacity=0][pattern=_063m0qo2e,pattern size=6pt,pattern thickness=0.75pt,pattern radius=0.75pt, pattern color={rgb, 255:red, 155; green, 155; blue, 155}] (418.33,110.67) -- (468.33,110.67) -- (468.33,187.2) -- (418.33,187.2) -- cycle ;
	%Shape: Rectangle [id:dp593521786657115] 
	\draw  [draw opacity=0][pattern=_u93ic1mtb,pattern size=6pt,pattern thickness=0.75pt,pattern radius=0pt, pattern color={rgb, 255:red, 155; green, 155; blue, 155}] (360.67,60.67) -- (418.33,60.67) -- (418.33,186.8) -- (360.67,186.8) -- cycle ;
	%Shape: Axis 2D [id:dp42718451569737725] 
	\draw  (68.5,186.98) -- (260.67,186.98)(77.71,47.33) -- (77.71,202) (253.67,181.98) -- (260.67,186.98) -- (253.67,191.98) (72.71,54.33) -- (77.71,47.33) -- (82.71,54.33)  ;
	%Curve Lines [id:da6495179254879908] 
	\draw    (101.67,56.33) .. controls (123.33,69) and (130,76.33) .. (162,76.33) ;
	\draw [shift={(130.25,71.97)}, rotate = 202.63] [fill={rgb, 255:red, 0; green, 0; blue, 0 }  ][line width=0.08]  [draw opacity=0] (10.72,-5.15) -- (0,0) -- (10.72,5.15) -- (7.12,0) -- cycle    ;
	%Curve Lines [id:da8141226896387221] 
	\draw    (162,76.33) .. controls (164.67,94.33) and (171.33,109) .. (180.67,121) ;
	\draw [shift={(168.46,100.02)}, rotate = 249.3] [fill={rgb, 255:red, 0; green, 0; blue, 0 }  ][line width=0.08]  [draw opacity=0] (10.72,-5.15) -- (0,0) -- (10.72,5.15) -- (7.12,0) -- cycle    ;
	%Curve Lines [id:da13989607523506598] 
	\draw    (101.67,56.33) .. controls (104.33,74.33) and (110.67,125.33) .. (127.33,153.33) ;
	\draw [shift={(110.37,106.1)}, rotate = 77.43] [fill={rgb, 255:red, 0; green, 0; blue, 0 }  ][line width=0.08]  [draw opacity=0] (10.72,-5.15) -- (0,0) -- (10.72,5.15) -- (7.12,0) -- cycle    ;
	%Curve Lines [id:da5225465849499047] 
	\draw    (127.33,153.33) .. controls (144.67,164.67) and (191.33,173) .. (235.33,172.33) ;
	\draw [shift={(180.24,168.77)}, rotate = 8.85] [fill={rgb, 255:red, 0; green, 0; blue, 0 }  ][line width=0.08]  [draw opacity=0] (10.72,-5.15) -- (0,0) -- (10.72,5.15) -- (7.12,0) -- cycle    ;
	%Shape: Axis 2D [id:dp025738552451401775] 
	\draw  (318.5,187.3) -- (505.5,187.3)(327.46,50.67) -- (327.46,202) (498.5,182.3) -- (505.5,187.3) -- (498.5,192.3) (322.46,57.67) -- (327.46,50.67) -- (332.46,57.67)  ;
	%Straight Lines [id:da17184396492467124] 
	\draw    (360.67,60.67) -- (418.33,60.67) ;
	\draw [shift={(389.5,60.67)}, rotate = 180] [fill={rgb, 255:red, 0; green, 0; blue, 0 }  ][line width=0.08]  [draw opacity=0] (10.72,-5.15) -- (0,0) -- (10.72,5.15) -- (7.12,0) -- cycle    ;
	%Straight Lines [id:da3328308427577038] 
	\draw    (418.33,110.67) -- (468.33,110.67) ;
	\draw [shift={(443.33,110.67)}, rotate = 180] [fill={rgb, 255:red, 0; green, 0; blue, 0 }  ][line width=0.08]  [draw opacity=0] (10.72,-5.15) -- (0,0) -- (10.72,5.15) -- (7.12,0) -- cycle    ;
	%Straight Lines [id:da9505803052171997] 
	\draw    (360.67,160.67) -- (468.33,160.67) ;
	\draw [shift={(414.5,160.67)}, rotate = 0] [fill={rgb, 255:red, 0; green, 0; blue, 0 }  ][line width=0.08]  [draw opacity=0] (10.72,-5.15) -- (0,0) -- (10.72,5.15) -- (7.12,0) -- cycle    ;
	%Straight Lines [id:da5801133154586786] 
	\draw    (468.33,160.67) -- (468.33,110.67) ;
	\draw [shift={(468.33,135.67)}, rotate = 270] [fill={rgb, 255:red, 0; green, 0; blue, 0 }  ][line width=0.08]  [draw opacity=0] (10.72,-5.15) -- (0,0) -- (10.72,5.15) -- (7.12,0) -- cycle    ;
	%Straight Lines [id:da6975245276911515] 
	\draw    (360.67,160.67) -- (360.67,60.67) ;
	\draw [shift={(360.67,110.67)}, rotate = 450] [fill={rgb, 255:red, 0; green, 0; blue, 0 }  ][line width=0.08]  [draw opacity=0] (10.72,-5.15) -- (0,0) -- (10.72,5.15) -- (7.12,0) -- cycle    ;
	%Curve Lines [id:da6975020862261498] 
	\draw    (180.67,121) .. controls (196,129) and (202.67,131.67) .. (221.33,131.67) ;
	\draw [shift={(200.28,129.48)}, rotate = 198.23] [fill={rgb, 255:red, 0; green, 0; blue, 0 }  ][line width=0.08]  [draw opacity=0] (10.72,-5.15) -- (0,0) -- (10.72,5.15) -- (7.12,0) -- cycle    ;
	%Curve Lines [id:da8289401327108623] 
	\draw    (221.33,131.67) .. controls (224,149.67) and (226,160.33) .. (235.33,172.33) ;
	\draw [shift={(225.48,153.09)}, rotate = 255.79000000000002] [fill={rgb, 255:red, 0; green, 0; blue, 0 }  ][line width=0.08]  [draw opacity=0] (10.72,-5.15) -- (0,0) -- (10.72,5.15) -- (7.12,0) -- cycle    ;
	%Straight Lines [id:da3869861390769662] 
	\draw    (418.33,110.67) -- (418.33,60.67) ;
	\draw [shift={(418.33,85.67)}, rotate = 270] [fill={rgb, 255:red, 0; green, 0; blue, 0 }  ][line width=0.08]  [draw opacity=0] (10.72,-5.15) -- (0,0) -- (10.72,5.15) -- (7.12,0) -- cycle    ;
	%Straight Lines [id:da5224236485541403] 
	\draw    (139.17,50.67) .. controls (140.84,52.34) and (140.84,54) .. (139.17,55.67) .. controls (137.5,57.34) and (137.5,59) .. (139.17,60.67) .. controls (140.84,62.34) and (140.84,64) .. (139.17,65.67) .. controls (137.5,67.34) and (137.5,69) .. (139.17,70.67) .. controls (140.84,72.34) and (140.84,74) .. (139.17,75.67) .. controls (137.5,77.34) and (137.5,79) .. (139.17,80.67) -- (139.17,83) -- (139.17,86)(136.17,50.67) .. controls (137.84,52.34) and (137.84,54) .. (136.17,55.67) .. controls (134.5,57.34) and (134.5,59) .. (136.17,60.67) .. controls (137.84,62.34) and (137.84,64) .. (136.17,65.67) .. controls (134.5,67.34) and (134.5,69) .. (136.17,70.67) .. controls (137.84,72.34) and (137.84,74) .. (136.17,75.67) .. controls (134.5,77.34) and (134.5,79) .. (136.17,80.67) -- (136.17,83) -- (136.17,86) ;
	\draw [shift={(137.67,94)}, rotate = 270] [color={rgb, 255:red, 0; green, 0; blue, 0 }  ][line width=0.75]    (10.93,-3.29) .. controls (6.95,-1.4) and (3.31,-0.3) .. (0,0) .. controls (3.31,0.3) and (6.95,1.4) .. (10.93,3.29)   ;
	%Straight Lines [id:da491624189292325] 
	\draw    (207.17,106) .. controls (208.84,107.67) and (208.84,109.33) .. (207.17,111) .. controls (205.5,112.67) and (205.5,114.33) .. (207.17,116) .. controls (208.84,117.67) and (208.84,119.33) .. (207.17,121) .. controls (205.5,122.67) and (205.5,124.33) .. (207.17,126) .. controls (208.84,127.67) and (208.84,129.33) .. (207.17,131) .. controls (205.5,132.67) and (205.5,134.33) .. (207.17,136) -- (207.17,138.33) -- (207.17,141.33)(204.17,106) .. controls (205.84,107.67) and (205.84,109.33) .. (204.17,111) .. controls (202.5,112.67) and (202.5,114.33) .. (204.17,116) .. controls (205.84,117.67) and (205.84,119.33) .. (204.17,121) .. controls (202.5,122.67) and (202.5,124.33) .. (204.17,126) .. controls (205.84,127.67) and (205.84,129.33) .. (204.17,131) .. controls (202.5,132.67) and (202.5,134.33) .. (204.17,136) -- (204.17,138.33) -- (204.17,141.33) ;
	\draw [shift={(205.67,149.33)}, rotate = 270] [color={rgb, 255:red, 0; green, 0; blue, 0 }  ][line width=0.75]    (10.93,-3.29) .. controls (6.95,-1.4) and (3.31,-0.3) .. (0,0) .. controls (3.31,0.3) and (6.95,1.4) .. (10.93,3.29)   ;
	%Straight Lines [id:da8687331853501177] 
	\draw    (161.17,145.33) .. controls (162.84,147) and (162.84,148.66) .. (161.17,150.33) .. controls (159.5,152) and (159.5,153.66) .. (161.17,155.33) .. controls (162.84,157) and (162.84,158.66) .. (161.17,160.33) .. controls (159.5,162) and (159.5,163.66) .. (161.17,165.33) .. controls (162.84,167) and (162.84,168.66) .. (161.17,170.33) -- (161.17,171.67) -- (161.17,174.67)(158.17,145.33) .. controls (159.84,147) and (159.84,148.66) .. (158.17,150.33) .. controls (156.5,152) and (156.5,153.66) .. (158.17,155.33) .. controls (159.84,157) and (159.84,158.66) .. (158.17,160.33) .. controls (156.5,162) and (156.5,163.66) .. (158.17,165.33) .. controls (159.84,167) and (159.84,168.66) .. (158.17,170.33) -- (158.17,171.67) -- (158.17,174.67) ;
	\draw [shift={(159.67,182.67)}, rotate = 270] [color={rgb, 255:red, 0; green, 0; blue, 0 }  ][line width=0.75]    (10.93,-3.29) .. controls (6.95,-1.4) and (3.31,-0.3) .. (0,0) .. controls (3.31,0.3) and (6.95,1.4) .. (10.93,3.29)   ;
	%Straight Lines [id:da9329491022140817] 
	\draw    (397.17,43) .. controls (398.84,44.67) and (398.84,46.33) .. (397.17,48) .. controls (395.5,49.67) and (395.5,51.33) .. (397.17,53) .. controls (398.84,54.67) and (398.84,56.33) .. (397.17,58) .. controls (395.5,59.67) and (395.5,61.33) .. (397.17,63) .. controls (398.84,64.67) and (398.84,66.33) .. (397.17,68) -- (397.17,69.67) -- (397.17,72.67)(394.17,43) .. controls (395.84,44.67) and (395.84,46.33) .. (394.17,48) .. controls (392.5,49.67) and (392.5,51.33) .. (394.17,53) .. controls (395.84,54.67) and (395.84,56.33) .. (394.17,58) .. controls (392.5,59.67) and (392.5,61.33) .. (394.17,63) .. controls (395.84,64.67) and (395.84,66.33) .. (394.17,68) -- (394.17,69.67) -- (394.17,72.67) ;
	\draw [shift={(395.67,80.67)}, rotate = 270] [color={rgb, 255:red, 0; green, 0; blue, 0 }  ][line width=0.75]    (10.93,-3.29) .. controls (6.95,-1.4) and (3.31,-0.3) .. (0,0) .. controls (3.31,0.3) and (6.95,1.4) .. (10.93,3.29)   ;
	%Straight Lines [id:da35854165790435655] 
	\draw    (453.17,90.33) .. controls (454.84,92) and (454.84,93.66) .. (453.17,95.33) .. controls (451.5,97) and (451.5,98.66) .. (453.17,100.33) .. controls (454.84,102) and (454.84,103.66) .. (453.17,105.33) .. controls (451.5,107) and (451.5,108.66) .. (453.17,110.33) .. controls (454.84,112) and (454.84,113.66) .. (453.17,115.33) -- (453.17,117) -- (453.17,120)(450.17,90.33) .. controls (451.84,92) and (451.84,93.66) .. (450.17,95.33) .. controls (448.5,97) and (448.5,98.66) .. (450.17,100.33) .. controls (451.84,102) and (451.84,103.66) .. (450.17,105.33) .. controls (448.5,107) and (448.5,108.66) .. (450.17,110.33) .. controls (451.84,112) and (451.84,113.66) .. (450.17,115.33) -- (450.17,117) -- (450.17,120) ;
	\draw [shift={(451.67,128)}, rotate = 270] [color={rgb, 255:red, 0; green, 0; blue, 0 }  ][line width=0.75]    (10.93,-3.29) .. controls (6.95,-1.4) and (3.31,-0.3) .. (0,0) .. controls (3.31,0.3) and (6.95,1.4) .. (10.93,3.29)   ;
	%Straight Lines [id:da32349585359843824] 
	\draw    (400.5,141.67) .. controls (402.17,143.34) and (402.17,145) .. (400.5,146.67) .. controls (398.83,148.34) and (398.83,150) .. (400.5,151.67) .. controls (402.17,153.34) and (402.17,155) .. (400.5,156.67) .. controls (398.83,158.34) and (398.83,160) .. (400.5,161.67) .. controls (402.17,163.34) and (402.17,165) .. (400.5,166.67) -- (400.5,168.33) -- (400.5,171.33)(397.5,141.67) .. controls (399.17,143.34) and (399.17,145) .. (397.5,146.67) .. controls (395.83,148.34) and (395.83,150) .. (397.5,151.67) .. controls (399.17,153.34) and (399.17,155) .. (397.5,156.67) .. controls (395.83,158.34) and (395.83,160) .. (397.5,161.67) .. controls (399.17,163.34) and (399.17,165) .. (397.5,166.67) -- (397.5,168.33) -- (397.5,171.33) ;
	\draw [shift={(399,179.33)}, rotate = 270] [color={rgb, 255:red, 0; green, 0; blue, 0 }  ][line width=0.75]    (10.93,-3.29) .. controls (6.95,-1.4) and (3.31,-0.3) .. (0,0) .. controls (3.31,0.3) and (6.95,1.4) .. (10.93,3.29)   ;

	% Text Node
	\draw (65,46.5) node    {$p$};
	% Text Node
	\draw (269.5,191) node    {$V$};
	% Text Node
	\draw (312,49.5) node    {$T$};
	% Text Node
	\draw (519.5,191) node    {$S$};
	% Text Node
	\draw (352.33,49.5) node    {$A$};
	% Text Node
	\draw (429,50.17) node    {$B$};
	% Text Node
	\draw (431.67,91.5) node    {$C$};
	% Text Node
	\draw (349,166.83) node    {$F$};
	% Text Node
	\draw (99.67,42.5) node    {$A$};
	% Text Node
	\draw (172.67,218.67) node   [align=left] {Piano di Clapeyron};
	% Text Node
	\draw (422.67,218.67) node   [align=left] {Piano di Gibbs};
	% Text Node
	\draw (480.33,101.5) node    {$D$};
	% Text Node
	\draw (480.33,161.5) node    {$E$};
	% Text Node
	\draw (169,64.83) node    {$B$};
	% Text Node
	\draw (185.67,107.5) node    {$C$};
	% Text Node
	\draw (233,123.5) node    {$D$};
	% Text Node
	\draw (247,163.5) node    {$E$};
	% Text Node
	\draw (115.67,162.17) node    {$F$};


	\end{tikzpicture}
\end{figure}
\FloatBarrier
Si vede come calcolarne il rendimento.
\[
	\eta = \frac{\mathcal{L} }{Q_{\text{ass} } }
\]
Nel piano di Clapeyron si vede direttamente il lavoro compiuto dal ciclo ma non si riesce a valutare immediatamente quali sono i calori. Il rendimento può essere scritto come $1-|Q_\text{ced}|/Q_\text{ass}$. Il calore ceduto sarà quello scambiato durante la compressione isoterma a temperatura $T_1$. I calori assorbiti saranno quelli scambiati durante le espansioni isoterme. Quindi si avrà:
\[
	\eta = 1 - \frac{|Q_1|}{Q_2+Q_3}
\]
Tali calori li si riesce a vedere tutti nel piano di Gibbs perché $Q_3$ è l'area a linee diagonali, $Q_2$ quella a pallini, $Q_1$ quella a linee orizzontali.
Inoltre si legge chiaramente come si calcola la variazione di entropia di due importanti trasformazioni.
\begin{gather*}
	\Delta S = \frac{Q}{T} \quad \text{isoterma irreversibile} \\
	\Delta S = 0 \quad \text{adiabatica reversibile}
\end{gather*}
Un'ultima osservazione su piano di Gibbs e piano di Clapeyron: In entrambi i casi si fa il prodotto fra due variabili che dimensionalmente danno sempre un energia. Il prodotto fra queste due variabili nel piano di Clapeyron è $pV$ ed è infatti il frutto di una variabile estensiva e una intensiva. Nel piano di Gibbs si ha il prodotto fra la temperatura, che è una variabile intensiva, e l'entropia, che è invece una variabile estensiva. Essa infatti è un valore che caratterizza tutto il sistema termodinamico ed è tanto più grande quanto più questo è esteso. Esiste sempre quindi tale analogia, il fatto che l'energia è sempre il prodotto fra una variabile intensiva e una variabile intensiva.







































\section{Teoria cinetica dei gas}

\subsection{Le ipotesi della teoria}

È stata data una visione della termodinamica descrivendo un gas in termini di variabili macroscopiche. Un gas ha una certa pressione, occupa un certo volume, ha una certa temperatura. La teoria cinetica dei gas descrive tali variabili con un modello microscopico che si va a legare al comportamento del gas inteso come dato da particelle molecolari che hanno un certo comportamento.

L'obbiettivo della sezione è quello di arrivare a dare un'interpretazione microscopica della temperatura e della pressione di un certo gas.
Si suppone di avere a che fare con un gas ideale contenuto in un recipiente cubico di lato $A$. Nella sua descrizione, si utilizzeranno le seguenti ipotesi:
\begin{itemize}
	\item Il gas è costituito da particelle molecolari tutte uguali e distribuite in maniera caotica.
	\item Il gas è rarefatto. Ciò significa che le particelle hanno dimensioni molto piccole rispetto allo spazio che c'è fra di loro. C'è molto più vuoto rispetto allo spazio occupato dalle particelle.
	\item Durante il loro movimento caotico, le particelle si urtano fra di loro e urtano le pareti del recipiente. L'ipotesi è che tali urti siano ideali. Per ideali si intende elastici, quindi in assenza di forze di attrito, e centrali.
	\item Non ci siano altre forze di interazione tra le molecole del gas. Esse non si attraggono per interazione elettrostatica.
\end{itemize}
\begin{figure}[htpb]
	\centering
	

	\tikzset{every picture/.style={line width=0.75pt}} %set default line width to 0.75pt        

	\begin{tikzpicture}[x=0.75pt,y=0.75pt,yscale=-1,xscale=1]
	%uncomment if require: \path (0,300); %set diagram left start at 0, and has height of 300

	%Straight Lines [id:da45879241397607173] 
	\draw  [dash pattern={on 0.84pt off 2.51pt}]  (217.33,94.33) -- (217.33,142) ;
	%Straight Lines [id:da9742441093997318] 
	\draw  [dash pattern={on 0.84pt off 2.51pt}]  (267,142) -- (217.13,142) ;
	%Straight Lines [id:da6927479790989322] 
	\draw  [dash pattern={on 0.84pt off 2.51pt}]  (217.33,142) -- (201.63,157.7) ;
	%Straight Lines [id:da4354392473018709] 
	\draw    (169,170) -- (169,97.67) ;
	\draw [shift={(169,94.67)}, rotate = 450] [fill={rgb, 255:red, 0; green, 0; blue, 0 }  ][line width=0.08]  [draw opacity=0] (10.72,-5.15) -- (0,0) -- (10.72,5.15) -- (7.12,0) -- cycle    ;
	%Straight Lines [id:da9740500633855294] 
	\draw    (169,170) -- (146.37,192.63) ;
	\draw [shift={(144.25,194.75)}, rotate = 315] [fill={rgb, 255:red, 0; green, 0; blue, 0 }  ][line width=0.08]  [draw opacity=0] (10.72,-5.15) -- (0,0) -- (10.72,5.15) -- (7.12,0) -- cycle    ;
	%Straight Lines [id:da17051109556175215] 
	\draw    (169,170) -- (322.67,170) ;
	\draw [shift={(325.67,170)}, rotate = 180] [fill={rgb, 255:red, 0; green, 0; blue, 0 }  ][line width=0.08]  [draw opacity=0] (10.72,-5.15) -- (0,0) -- (10.72,5.15) -- (7.12,0) -- cycle    ;
	%Shape: Cube [id:dp559325738726933] 
	\draw   (201.33,110.33) -- (217.33,94.33) -- (267,94.33) -- (267,142) -- (251,158) -- (201.33,158) -- cycle ; \draw   (267,94.33) -- (251,110.33) -- (201.33,110.33) ; \draw   (251,110.33) -- (251,158) ;
	%Shape: Circle [id:dp3295026946934616] 
	\draw  [draw opacity=0][fill={rgb, 255:red, 128; green, 128; blue, 128 }  ,fill opacity=1 ] (213.33,120.83) .. controls (213.33,119.27) and (214.6,118) .. (216.17,118) .. controls (217.73,118) and (219,119.27) .. (219,120.83) .. controls (219,122.4) and (217.73,123.67) .. (216.17,123.67) .. controls (214.6,123.67) and (213.33,122.4) .. (213.33,120.83) -- cycle ;
	%Shape: Circle [id:dp8704491879909637] 
	\draw  [draw opacity=0][fill={rgb, 255:red, 128; green, 128; blue, 128 }  ,fill opacity=1 ] (238.67,148.17) .. controls (238.67,146.6) and (239.94,145.33) .. (241.5,145.33) .. controls (243.06,145.33) and (244.33,146.6) .. (244.33,148.17) .. controls (244.33,149.73) and (243.06,151) .. (241.5,151) .. controls (239.94,151) and (238.67,149.73) .. (238.67,148.17) -- cycle ;
	%Shape: Circle [id:dp9279661658238063] 
	\draw  [draw opacity=0][fill={rgb, 255:red, 128; green, 128; blue, 128 }  ,fill opacity=1 ] (256.33,113.83) .. controls (256.33,112.27) and (257.6,111) .. (259.17,111) .. controls (260.73,111) and (262,112.27) .. (262,113.83) .. controls (262,115.4) and (260.73,116.67) .. (259.17,116.67) .. controls (257.6,116.67) and (256.33,115.4) .. (256.33,113.83) -- cycle ;
	%Shape: Circle [id:dp6636119609330704] 
	\draw  [draw opacity=0][fill={rgb, 255:red, 128; green, 128; blue, 128 }  ,fill opacity=1 ] (253.33,128.83) .. controls (253.33,127.27) and (254.6,126) .. (256.17,126) .. controls (257.73,126) and (259,127.27) .. (259,128.83) .. controls (259,130.4) and (257.73,131.67) .. (256.17,131.67) .. controls (254.6,131.67) and (253.33,130.4) .. (253.33,128.83) -- cycle ;
	%Shape: Circle [id:dp7894888587590783] 
	\draw  [draw opacity=0][fill={rgb, 255:red, 128; green, 128; blue, 128 }  ,fill opacity=1 ] (257.33,139.83) .. controls (257.33,138.27) and (258.6,137) .. (260.17,137) .. controls (261.73,137) and (263,138.27) .. (263,139.83) .. controls (263,141.4) and (261.73,142.67) .. (260.17,142.67) .. controls (258.6,142.67) and (257.33,141.4) .. (257.33,139.83) -- cycle ;
	%Shape: Circle [id:dp21747914903436993] 
	\draw  [draw opacity=0][fill={rgb, 255:red, 128; green, 128; blue, 128 }  ,fill opacity=1 ] (240.67,117.5) .. controls (240.67,115.94) and (241.94,114.67) .. (243.5,114.67) .. controls (245.06,114.67) and (246.33,115.94) .. (246.33,117.5) .. controls (246.33,119.06) and (245.06,120.33) .. (243.5,120.33) .. controls (241.94,120.33) and (240.67,119.06) .. (240.67,117.5) -- cycle ;
	%Shape: Circle [id:dp7272550201057293] 
	\draw  [draw opacity=0][fill={rgb, 255:red, 128; green, 128; blue, 128 }  ,fill opacity=1 ] (209,146.83) .. controls (209,145.27) and (210.27,144) .. (211.83,144) .. controls (213.4,144) and (214.67,145.27) .. (214.67,146.83) .. controls (214.67,148.4) and (213.4,149.67) .. (211.83,149.67) .. controls (210.27,149.67) and (209,148.4) .. (209,146.83) -- cycle ;
	%Shape: Circle [id:dp6688552083133636] 
	\draw  [draw opacity=0][fill={rgb, 255:red, 128; green, 128; blue, 128 }  ,fill opacity=1 ] (230.67,137.5) .. controls (230.67,135.94) and (231.94,134.67) .. (233.5,134.67) .. controls (235.06,134.67) and (236.33,135.94) .. (236.33,137.5) .. controls (236.33,139.06) and (235.06,140.33) .. (233.5,140.33) .. controls (231.94,140.33) and (230.67,139.06) .. (230.67,137.5) -- cycle ;
	%Shape: Circle [id:dp4611735824001546] 
	\draw  [draw opacity=0][fill={rgb, 255:red, 128; green, 128; blue, 128 }  ,fill opacity=1 ] (225.67,124.5) .. controls (225.67,122.94) and (226.94,121.67) .. (228.5,121.67) .. controls (230.06,121.67) and (231.33,122.94) .. (231.33,124.5) .. controls (231.33,126.06) and (230.06,127.33) .. (228.5,127.33) .. controls (226.94,127.33) and (225.67,126.06) .. (225.67,124.5) -- cycle ;
	%Shape: Circle [id:dp8099918451310808] 
	\draw  [draw opacity=0][fill={rgb, 255:red, 128; green, 128; blue, 128 }  ,fill opacity=1 ] (213.67,134.5) .. controls (213.67,132.94) and (214.94,131.67) .. (216.5,131.67) .. controls (218.06,131.67) and (219.33,132.94) .. (219.33,134.5) .. controls (219.33,136.06) and (218.06,137.33) .. (216.5,137.33) .. controls (214.94,137.33) and (213.67,136.06) .. (213.67,134.5) -- cycle ;
	%Shape: Circle [id:dp6693692291433728] 
	\draw  [draw opacity=0][fill={rgb, 255:red, 128; green, 128; blue, 128 }  ,fill opacity=1 ] (223.67,150.5) .. controls (223.67,148.94) and (224.94,147.67) .. (226.5,147.67) .. controls (228.06,147.67) and (229.33,148.94) .. (229.33,150.5) .. controls (229.33,152.06) and (228.06,153.33) .. (226.5,153.33) .. controls (224.94,153.33) and (223.67,152.06) .. (223.67,150.5) -- cycle ;
	%Shape: Circle [id:dp5611121559728784] 
	\draw  [draw opacity=0][fill={rgb, 255:red, 128; green, 128; blue, 128 }  ,fill opacity=1 ] (239.67,130.5) .. controls (239.67,128.94) and (240.94,127.67) .. (242.5,127.67) .. controls (244.06,127.67) and (245.33,128.94) .. (245.33,130.5) .. controls (245.33,132.06) and (244.06,133.33) .. (242.5,133.33) .. controls (240.94,133.33) and (239.67,132.06) .. (239.67,130.5) -- cycle ;

	% Text Node
	\draw (161.67,87.17) node    {$y$};
	% Text Node
	\draw (338.17,173.33) node    {$x$};
	% Text Node
	\draw (136.67,194.33) node    {$z$};


	\end{tikzpicture}
\end{figure}
\FloatBarrier
Come si vedrà più in dettaglio, la prima ipotesi implica che la velocità media vettoriale sia nulla. La terza che negli urti tra molecole si conservano quantità di moto ed energia, mentre nell'urto di una molecola contro una parete si conserva solo l'energia. Dalla quarta ipotesi deriva che l'energia potenziale interna è nulla e quindi la sola forma di energia è quella cinetica.
Sulla base del modello cinetico, è stata sviluppata, nella seconda metà dell'800, la \textbf{teoria cinetica dei gas}, che permette di arrivare a previsioni sul comportamento dei gas, verificabili sperimentalmente.

\subsection{La visione microscopica della pressione}

Si ha un contenitore cubico con $n$ molecole che si muovono caoticamente e ogni molecola $i$-esima possiede la velocità vettoriale $\vec{v}_i$. Se si vuole calcolare il valore medio della velocità vettoriale di tutte le particelle si trova che esso è pari a $0$ perché il moto è caotico. Se $n$ particelle vanno a sinistra ce ne saranno altrettante che vanno a destra. Le forze esterne sono trascurabili e quindi il centro di massa ha velocità costante, in particolare pari a $0$. La particella $i$-esima avrà una componente lungo $x,y$ e $z$. L'obbiettivo che ci si pone è di legare la pressione al moto microscopico delle particelle. Si parte dalla pressione che si sente sulla parete $yz$ del recipiente. Essa sarà dovuta agli urti delle particelle contro la parete, sarà la forza totale esercitata dalle molecole, divisa per l'area della parete. La pressione è frutto soltanto delle forze ortogonali alla superficie, quindi conta soltanto la forza che sta agendo in direzione $x$. La forza totale è la somma estesa a tutte le molecole della forza generata dalla singola molecola in direzione $x$.
\[
	p = \frac{\norma{\vec{F}_{tot,x}} }{a^2} = \frac{\sum_{i=1}^N F_{x,i}}{a^2}
\]
Per comodità si trasforma il disegno da tridimensionale a bidimensionale, si ha la parete su cui vanno a urtare le varie molecole. Si consideri la molecola $i$-esima che sta arrivando con velocità qualunque.
\begin{figure}[htpb]
	\centering
	

	\tikzset{every picture/.style={line width=0.75pt}} %set default line width to 0.75pt        

	\begin{tikzpicture}[x=0.75pt,y=0.75pt,yscale=-1,xscale=1]
	%uncomment if require: \path (0,300); %set diagram left start at 0, and has height of 300

	%Shape: Axis 2D [id:dp18325872511828445] 
	\draw  (68.5,190) -- (260.67,190)(237.5,47.33) -- (237.5,202) (253.67,185) -- (260.67,190) -- (253.67,195) (232.5,54.33) -- (237.5,47.33) -- (242.5,54.33)  ;
	%Straight Lines [id:da46888513249136965] 
	\draw    (120,160) -- (197.32,121.34) ;
	\draw [shift={(200,120)}, rotate = 513.4300000000001] [fill={rgb, 255:red, 0; green, 0; blue, 0 }  ][line width=0.08]  [draw opacity=0] (10.72,-5.15) -- (0,0) -- (10.72,5.15) -- (7.12,0) -- cycle    ;
	%Straight Lines [id:da752842138682754] 
	\draw    (122.68,81.34) -- (200,120) ;
	\draw [shift={(120,80)}, rotate = 26.57] [fill={rgb, 255:red, 0; green, 0; blue, 0 }  ][line width=0.08]  [draw opacity=0] (10.72,-5.15) -- (0,0) -- (10.72,5.15) -- (7.12,0) -- cycle    ;
	%Shape: Rectangle [id:dp8263956807721704] 
	\draw  [draw opacity=0][fill={rgb, 255:red, 155; green, 155; blue, 155 }  ,fill opacity=1 ] (210,80) -- (200,80) -- (200,160) -- (210,160) -- cycle ;
	%Shape: Arc [id:dp42415565235130726] 
	\draw  [draw opacity=0] (160,120.49) .. controls (160,120.33) and (160,120.16) .. (160,120) .. controls (160,113.42) and (161.59,107.22) .. (164.4,101.74) -- (200,120) -- cycle ; \draw   (160,120.49) .. controls (160,120.33) and (160,120.16) .. (160,120) .. controls (160,113.42) and (161.59,107.22) .. (164.4,101.74) ;
	%Shape: Circle [id:dp9782826031645953] 
	\draw  [draw opacity=0][fill={rgb, 255:red, 155; green, 155; blue, 155 }  ,fill opacity=1 ] (110,160) .. controls (110,154.48) and (114.48,150) .. (120,150) .. controls (125.52,150) and (130,154.48) .. (130,160) .. controls (130,165.52) and (125.52,170) .. (120,170) .. controls (114.48,170) and (110,165.52) .. (110,160) -- cycle ;
	%Shape: Arc [id:dp33191213110278794] 
	\draw  [draw opacity=0] (156.04,141.66) .. controls (152.83,135.16) and (151.02,127.85) .. (151,120.11) -- (200,120) -- cycle ; \draw   (156.04,141.66) .. controls (152.83,135.16) and (151.02,127.85) .. (151,120.11) ;
	%Straight Lines [id:da6672059317477186] 
	\draw [color={rgb, 255:red, 155; green, 155; blue, 155 }  ,draw opacity=1 ]   (100,120) -- (200,120) ;

	% Text Node
	\draw (254,45.5) node    {$z$};
	% Text Node
	\draw (269.5,191) node    {$x$};
	% Text Node
	\draw (150.5,105) node    {$\alpha $};
	% Text Node
	\draw (141,131.5) node    {$\alpha $};


	\end{tikzpicture}
\end{figure}
\FloatBarrier
Gli urti sono elastici. Quindi la velocità dopo l'urto sarà la stessa perché l'energia cinetica si conserva. Visto che non c'è alcuna forza di attrito, la componente della velocità in direzione parallela alla parete non cambierà. Ciò significa che l'unica cosa che succede è che la componente $v_{i,x}$ semplicemente si ribalta e diventa $-v_{i,x}$. Quindi l'unica componente della velocità che varia è quella in $x$. Quando una particella urta una parete in modo perfettamente elastico, l'urto è speculare. Rispetto alla normale i due angoli sono uguali. La quantità di moto in direzione $x$ della particella è cambiata perché si è ribaltata di segno. Ciò è accaduto perché ha ricevuto una forza impulsiva $\vec{F}_{i,x}$ dalla parete. Si va a calcolarla. La variazione della quantità di moto in un urto è pari all'impulso della forza $\vec{F}_{i,x}$.
\[
	\Delta p = \int F_{x,i} dt = F_{x,i}\Delta t
\]
Ora il problema rimane il calcolo di $\vec{F}_{i,x}$, ossia la forza generata dalla particella $i$-esima sulla parete. Si ragioni sull'evoluzione temporale di quello che accade a questa parete a causa del fatto che la particella va a sbattere.
Si va a dare un grafico dell'andamento nel tempo della forza esercitata dalla parete sulla particella $i$-esima considerata.
\begin{figure}[htpb]
	\centering
	

	\tikzset{every picture/.style={line width=0.75pt}} %set default line width to 0.75pt        

	\begin{tikzpicture}[x=0.75pt,y=0.75pt,yscale=-1,xscale=1]
	%uncomment if require: \path (0,300); %set diagram left start at 0, and has height of 300

	% Plotting does not support converting to Tikz
	%Straight Lines [id:da5012780936511261] 
	\draw    (82.36,182.71) -- (286.36,182.71) ;
	\draw [shift={(289.36,182.71)}, rotate = 180] [fill={rgb, 255:red, 0; green, 0; blue, 0 }  ][line width=0.08]  [draw opacity=0] (10.72,-5.15) -- (0,0) -- (10.72,5.15) -- (7.12,0) -- cycle    ;
	%Curve Lines [id:da31123642176672517] 
	\draw [line width=1.5]    (104.5,104.25) .. controls (114.71,104) and (114.14,182) .. (124.5,182.25) .. controls (134.86,182.5) and (135.86,105.14) .. (145,105.25) .. controls (154.14,105.36) and (154.43,182) .. (165.5,181.75) .. controls (176.57,181.5) and (175.86,103.71) .. (186,103.75) .. controls (196.14,103.79) and (196.43,181.71) .. (206.5,181.75) .. controls (216.57,181.79) and (216.14,104.57) .. (226.5,104.75) .. controls (236.86,104.93) and (236.43,182) .. (247,182.25) .. controls (257.57,182.5) and (257.29,105.14) .. (267.5,105.25) ;




	\end{tikzpicture}
\end{figure}
\FloatBarrier
La particella rimbalza indietro e si mette a muoversi nell'altra direzione. Dopo un certo intervallo di tempo va a risbattere contro una stessa parete. Quindi essa urta le pareti continuamente. C'è un continuo di forze impulsive esercitate dalla particella su quella parete di intensità $\vec{F}_{i,x}$. Per valutare l'intervallo di tempo che intercorre fra un urto e il successivo, conta il tempo medio che ci mette la particella a fare avanti e indietro e quindi conta spazio / velocità. Lo spazio è $2a$, la velocità della particella è quella nella direzione $x$.
\[
	\Delta t = \frac{2a}{v_x }
\]
Nella realtà la particella è inclinata, ma la componente in $y$ non ha alcun effetto sul moto nella direzione considerata. Nella realtà inoltre essa muovendosi non continua a fare avanti indietro. Andrà a sbattere contro un'altra particella che si mette a muoversi. Per tenere conto di ciò si sfrutta il fatto che gli urti sono elastici. Quando un corpo urta in maniera elastica un corpo inizialmente fermo, il primo che era in moto si ferma e trasferisce tutta la sua quantità di moto al secondo. La particella urta una seconda particella, la prima si ferma e va avanti la seconda. Questo fa sì che il moto delle particelle possa essere pensato come quello di un'unica particella che si muove. Quello in figura è l'andamento nel tempo della forza, lo si vuole integrare nel tempo. Il numero di urti al secondo, ossia la frequenza degli urti, è l'inverso di $\Delta t$. Se tra un urto e l'altro passano $100 ms$, in un secondo ci sono 10 urti. Si può prendere il valore medio della forza fra un urto e il successivo e moltiplicarlo per l'intervallo di tempo per ottenere la variazione della quantità di moto in un intervallo di tempo. La relazione permette di calcolare quanto vale $\vec{F}_{i,x}$, quindi la forza che la particella $i$-esima imprime alla parete. $\vec{F}_{i,x}$ sarà quindi la variazione della quantità di moto in un urto fratto l'intervallo di tempo che intercorre fra un urto e il successivo.
Visto che che $\Delta p = 2m\,v_{x,i}$:
\[
	F_{x,i} = \frac{\Delta p_{\text{urto} } }{\Delta t} = \frac{2m\,v_{x,i} }{2a/v_{x,i} } = \frac{m\,v_{x,i}^2}{a} \qquad \text{impulso comunicato in un secondo}
\]
A questo punto si può sostituire il risultato nell'espressione della pressione vista come rapporto fra la forza esercitata da tutte le particelle sulla parere e l'area della parete stessa.
\[
	p = \frac{F_{tot,x}}{a^2} = \frac{\sum_{i=1}^N F_{i,x}  }{a^2} = \frac{\sum_{i=1}^N v_{x,i}^2\cdot m}{\underbrace{a^3}_V}
\]
La pressione dipende dal volume del recipiente e dal quadrato della velocità lungo $x$. Si attua un ultimo passaggio. Visto che la velocità vettoriale media non porta alcuna informazione, bisogna considerare un altro valore medio che dica quanto rapidamente si muovono le particelle. Si calcola il valore medio del quadrato della velocità, detto appunto velocità quadratica media.
\[
	\langle v^2 \rangle = \frac{1}{N} \sum_{i=1}^N (v_{x,i}^2 +v_{y,i}^2 + v_{z,i}^2)
\]
Esso si può esprimere anche in termini delle tre componenti.
\[
	\langle \vec{v} \rangle = 0 \qquad \langle v^2 \rangle = \langle v_x^2 +v_y^2 +v_z^2 \rangle = \langle v_x^2 \rangle + \langle v_y^2 \rangle + \langle v_z^2 \rangle
\]
Il valore medio di tre addendi si può anche scrivere come la somma dei valori medi degli addendi. Visto che le particelle sono tutte uguali e si muovono caoticamente in tutte le direzioni, ci si deve aspettare che, considerato il quadrato delle componenti, questi tre valori medi siano uguali. In media tutte le particelle si muovono rapide ugualmente in tutte tre le direzione. Allora questo valore medio al quadrato si può riscrivere come tre volte il valore medio della velocità quadratica media.
\[
	\langle v^2 \rangle = 3 \langle v_x^2 \rangle = 3 \frac{\sum_{i=1}^N v_{x,i}^2}{N} = \frac{3}{N} \sum_{i=1}^N v_{x,i}^2
\]
Andando a sostituirla nell'espressione per la pressione trovata in precedenza:
\[
	\boxed{p = \frac{mN\langle v^2 \rangle}{3V}}
\]
Si è arrivati a dire che la pressione che si va a misurare con un sensore sulla parete è fondamentalmente legata alla velocità quadratica media con cui si muovono le particelle e a quante sono. Più le particelle si muovono con velocità quadratica media elevata, più la pressione è elevata. Questa è la visione microscopica della pressione. Volendo là si può legare invece che alla velocità quadratica media, all'\textbf{energia cinetica media} delle molecole. Si definisce energia cinetica media di una molecola del gas, la quantità: $E_K = \frac{1}{2} m\langle v^2 \rangle$. Questa relazione si può riscrivere in modo tale che la pressione diventi:
\[
	\boxed{p = \frac{2}{3} \frac{N}{V} \langle E_K \rangle}
\]

\subsection{La visione microscopica della temperatura}

È noto che il prodotto pressione e volume è proporzionale a una certa quantità. Ci si aggiunge il risultato dato dell'equazione di stato dei gas perfetti. Il tutto perché si vuole avere una visione microscopica della temperatura.
Si ottiene che $pV = \frac{2}{3} N \langle E_K \rangle = nRT$. Attuando la sostituzione: $n=N/N_A$, si ha:
\[
	\frac{N}{N_A } RT = \frac{2}{3} N \langle E_K \rangle
\]
e il rapporto
\[
	\frac{R}{N_A } = k_B = \frac{8.314}{6.022 \times 10^{23} } = 1.3807 \times 10^{-23}\,J/K
\]
$R/N_A$ prende il nome di \emph{costante di Boltzmann}. Si ottiene che l'energia cinetica media delle particelle è:
\[
	k_B T = \frac{2}{3} \langle E_K \rangle \implies \boxed{\langle E_K \rangle = \frac{3}{2} k_B T}
\]
Questa è l'interpretazione microscopica della temperatura, o se si vuole, del moto delle particelle. Più un gas è caldo in temperatura, più le particelle si muovono con energia cinetica media elevata. L'agitazione termica è data dalla temperatura.

\emph{L'energia cinetica media traslazionale di una molecola di un gas ideale è proporzionale alla temperatura di un gas}.






































































% APPENDICE
\appendix

\chapter{Costanti ed unità di misura}

\section*{Costanti fisiche}

\begin{equation*}
	\begin{array}{ l l }
		\text{Numero di Avogadro} & N_A =6.022\times 10^{23} \ mol^{-1}\\
		\text{Costante dei gas perfetti} & R=8.314\ J/( K\ mol)\\
		\text{Costante di Boltzmann} & k_B =R/N_A =1.381\times 10^{-23} \ J/K\\
		\text{Costante gravitazionale} & G=6.672\times 10^{-11} \ N\ m^2 /kg^2
	\end{array}
\end{equation*}

% FIS2
% \begin{equation*}
% \begin{array}{ l l }
% \text{Carica elementare} & e=1.602\times 10^{-19} \ C\\
% \text{Massa dell'elettrone} & m_e =9.109\times 10^{-31} \ kg\\
% \text{Costante dielettrica del vuoto} & \varepsilon_0 =8.854\times 10^{-12} \ C^2 /\left( N\ m^2\right)\\
% \text{Permeabilita magnetica del vuoto} & \mu_0 =4\pi \times 10^{-7} \ N/A^2\\
% \text{Velocita della luce nel vuoto} & c=2.998\times 10^8 \ m/s\\
% \text{Raggio di Bohr} & a_0 =5.292\times 10^{-11} \ m\\
% \text{Magnetone di Bohr} & \mu_B =9.274\times 10^{-24} \ J/T
% \end{array}
% \end{equation*}







































\section*{Prefissi per le potenze di dieci}

\begin{center}

	\begin{table}[!h]
	\centering
	\begin{tabular}{ccc|ccc}
		\textbf{Potenza} & \textbf{Prefisso} & \textbf{Abbreviazione} & \textbf{Potenza} & \textbf{Prefisso} & \textbf{Abbreviazione} \\
		\hline
		$\displaystyle 10^{-18}$ & atto & a & $\displaystyle 10^1$ & deca & da \\
		$\displaystyle 10^{-15}$ & femto & f & $\displaystyle 10^2$ & etto & h \\
		$\displaystyle 10^{-12}$ & pico & p & $\displaystyle 10^3$ & chilo & k \\
		$\displaystyle 10^{-9}$ & nano & n & $\displaystyle 10^6$ & mega & M \\
		$\displaystyle 10^{-6}$ & micro & $\displaystyle \mu $ & $\displaystyle 10^9$ & giga & G \\
		$\displaystyle 10^{-3}$ & milli & m & $\displaystyle 10^{12}$ & tera & T \\
		$\displaystyle 10^{-2}$ & centi & c & $\displaystyle 10^{15}$ & peta & P \\
		$\displaystyle 10^{-1}$ & deci & d & $\displaystyle 10^{18}$ & exa & E \\
	\end{tabular}
	\end{table}
\end{center}







































\newpage\section*{Grandezze ed unità di misura impiegate nel testo}


\begin{center}
	\textbf{Grandezze fondamentali nel Sistema Internazionale}
\end{center}

\begin{table}[!h]
	\centering
	\begin{tabular}{l|l|l}
		\hline
		\textbf{Grandezza} & \textbf{Unità di misura} & \textbf{Simbolo} \\
		\hline
		Lunghezza & metro & $m$ \\
		Massa & chilogrammo & $kg$ \\
		Intervalli di tempo & secondo & $s$ \\
		Temperatura assoluta & kelvin & $K$ \\
		Quantità di sostanza & mole & $mol$ \\
		Angolo & radiante & $rad$ \\
	\end{tabular}
\end{table}

\begin{center}
	\textbf{Grandezze derivate}
\end{center}

\begin{table}[!h]
	\centering
	\begin{tabular}{l|c|c|c}
		\hline
		\textbf{Grandezza} & \textbf{Unità} & \textbf{Simbolo} & \textbf{Espressioni equivalenti} \\
		\hline
		Velocità $\displaystyle \vec{v}$ & - & $\displaystyle m/s$ & - \\
		Velocità angolare $\displaystyle \vec{\omega }$ & - & $\displaystyle rad/s$ & - \\
		Frequenza $\displaystyle \nu $ & hertz & $\displaystyle Hz$ & $\displaystyle 1/s$ \\
		Accelerazione $\displaystyle \vec{a}$ & - & $\displaystyle m/s^2$ & - \\
		Accelerazione angolare $\displaystyle \vec{\alpha }$ & - & $\displaystyle rad/s^2$ & - \\
		Forza $\displaystyle \vec{F}$ & newton & $\displaystyle N$ & $\displaystyle kg\ m/s^2 ;\ J/m$ \\
		Momento meccanico $\displaystyle \vec{M}$ & - & $\displaystyle N\ m$ & $\displaystyle kg\ m^2 /s^2$ \\
		Momento angolare $\displaystyle \vec{L}$ & - & $\displaystyle kg\ m^2 /s$ & - \\
		Quantità di moto $\displaystyle \vec{p}$ & - & $\displaystyle kg\ m /s$ & - \\
		Momento d'inerzia $\displaystyle I$ & - & $\displaystyle kg\ m^2$ & - \\
		Energia $\displaystyle E,U$; lavoro $\displaystyle L$; calore $\displaystyle Q$ & joule & $\displaystyle J$ & $\displaystyle N\ m;\ kg\ m^2 /s^2$ \\
		Potenza $\displaystyle P$ & watt & $\displaystyle W$ & $\displaystyle J/s;\ kg\ m^2 /s^3$ \\
		Pressione $\displaystyle p$ & pascal & $\displaystyle Pa$ & $\displaystyle N/m^2 ;\ kg/m\ s^2$ \\
		Densità per unità di volume $\displaystyle \rho $ & - & $\displaystyle kg/m^3$ & - \\
		Entropia $\displaystyle S$ & - & $\displaystyle J/K$ & - \\
	\end{tabular}
\end{table}

\newpage

\begin{center}
	\textbf{Altre unità di misura}
\end{center}

\begin{table}[!h]
	\centering
	\begin{tabular}{l|l|l}
		\hline
		\textbf{Grandezza} & \textbf{Unità di misura} & \textbf{Equivalenza nel S.I.} \\
		\hline
		Velocità & chilometro-ora ($\displaystyle km/h$) & $\displaystyle 0.2778\ m/s$ \\
		Forza & chilogrammo-forza ($\displaystyle kg_f$) & $\displaystyle 9.81\ N$ \\
		Energia & chilowatt-ora ($\displaystyle kWh$) & $\displaystyle 3.6\times 10^6 \ J$ \\
		Volume & litro ($\displaystyle L$) & $\displaystyle 10^{-3} \ m^3$ \\
		Pressione & $\displaystyle bar$ & $\displaystyle 10^5 \ Pa$ \\
		'' & $\displaystyle mm\ Hg$ & $\displaystyle 133.3\ Pa$ \\
		'' & atmosfera ($\displaystyle atm$) & $\displaystyle 1.013\times 10^5 \ Pa$ \\
		Calore & caloria ($\displaystyle cal$) & $\displaystyle 4.18\ J$ \\
	\end{tabular}
\end{table}

\begin{center}
	\textbf{Equivalenze utili}
\end{center}

Detta $\displaystyle t$ la temperatura di un oggetto misurata nella scala Celsius e $\displaystyle T$ la corrispondente temperatura assoluta, vale la relazione:
\begin{equation*}
	t( \degree \text{C}) =T(\text{K}) +273.15
\end{equation*}















\chapter{Richiami di trigonometria}

\section*{Funzioni trigonometriche}

Dato il triangolo rettangolo mostrato in figura di cateti $\displaystyle a$ e $\displaystyle b$, ipotenusa $\displaystyle c$ ed angolo $\displaystyle \vartheta $ opposto al cateto $\displaystyle a$, si definiscono:

\begin{equation*}
	\begin{array}{ l l }
		\sin \vartheta =\frac{a}{c} & \cos \vartheta =\frac{b}{c}\\
		[4mm]
		\tan \vartheta =\frac{a}{b} =\frac{\sin \vartheta }{\cos \vartheta } \ \ \ \  & \cot \vartheta =\frac{b}{a} =\frac{\cos \vartheta }{\sin \vartheta }
	\end{array}
\end{equation*}


\begin{figure}[htpb]
	\centering
	\tikzset{every picture/.style={line width=0.75pt}} %set default line width to 0.75pt

	\begin{tikzpicture}[x=0.75pt,y=0.75pt,yscale=-1,xscale=1]
	%uncomment if require: \path (0,300); %set diagram left start at 0, and has height of 300

	%Shape: Right Triangle [id:dp9340776108358415]
	\draw  [line width=1.5]  (162,57) -- (444.5,211.44) -- (162,211.44) -- cycle ;
	%Shape: Arc [id:dp8374391809216872]
	\draw  [draw opacity=0] (394.5,211.86) .. controls (394.5,211.72) and (394.5,211.58) .. (394.5,211.44) .. controls (394.5,202.5) and (396.84,194.12) .. (400.95,186.86) -- (444.5,211.44) -- cycle ; \draw   (394.5,211.86) .. controls (394.5,211.72) and (394.5,211.58) .. (394.5,211.44) .. controls (394.5,202.5) and (396.84,194.12) .. (400.95,186.86) ;

	% Text Node
	\draw (384.8,196) node    {$\vartheta $};
	% Text Node
	\draw (149.6,134.8) node    {$a$};
	% Text Node
	\draw (288,224.8) node    {$b$};
	% Text Node
	\draw (306,118) node    {$c$};

	\end{tikzpicture}
\end{figure}







































\section*{Identità trigonometriche}

Risultano le seguenti relazioni

\begin{equation*}
	\begin{array}{ c c }
		\sin \vartheta =\cos\left(\frac{\pi }{2} -\vartheta \right) & \cos \vartheta =\sin\left(\frac{\pi }{2} -\vartheta \right)\\
		[4mm]
		\cot \vartheta =\tan\left(\frac{\pi }{2} -\vartheta \right) & \sin^2 \vartheta +\cos^2 \vartheta =1\\
		[4mm]
		\cos^2 \vartheta =\frac{1}{1+\tan^2 \vartheta } & \sin^2 \vartheta =\frac{1}{1+\cot^2 \vartheta }\\
		[4mm]
		\cos 2\vartheta =\cos^2 \vartheta -\sin^2 \vartheta  & \sin 2\vartheta =2\sin \vartheta \cos \vartheta \\
		[4mm]
		\cos^2\frac{\vartheta }{2} =\frac{1+\cos \vartheta }{2} & \sin^2\frac{\vartheta }{2} =\frac{1-\cos \vartheta }{2}\\
		[4mm]
		\tan 2\vartheta =\frac{2\tan \vartheta }{1-\tan^2 \vartheta } & \tan\frac{\vartheta }{2} =\sqrt{\frac{1-\cos \vartheta }{1+\cos \vartheta }}
	\end{array}
\end{equation*}

Inoltre valgono le seguenti regole di somma:

\begin{equation*}
	\begin{array}{ c }
		\sin( \alpha \pm \beta ) =\sin \alpha \cos \beta \pm \cos \alpha \sin \beta \\
		[4mm]
		\cos( \alpha \pm \beta ) =\cos \alpha \cos \beta \mp \sin \alpha \sin \beta \\
		[4mm]
		\tan( \alpha \pm \beta ) =\frac{\tan \alpha \pm \tan \beta }{1\mp \tan \alpha \tan \beta }\\
		[4mm]
		\sin \alpha \pm \sin \beta =2\sin\left[\frac{1}{2}( \alpha \pm \beta )\right]\cos\left[\frac{1}{2}( \alpha \mp \beta )\right]\\
		[4mm]
		\cos \alpha +\cos \beta =2\cos\left[\frac{1}{2}( \alpha +\beta )\right]\cos\left[\frac{1}{2}( \alpha -\beta )\right]\\
		[4mm]
		\cos \alpha -\cos \beta =2\sin\left[\frac{1}{2}( \alpha +\beta )\right]\sin\left[\frac{1}{2}( \alpha -\beta )\right]\\
		[4mm]
		\sin \alpha \cos \beta =\frac{1}{2}[\sin( \alpha +\beta ) +\sin( \alpha -\beta )]\\
		[4mm]
		\cos \alpha \cos \beta =\frac{1}{2}[\cos( \alpha +\beta ) +\cos( \alpha -\beta )]\\
		[4mm]
		\sin^2 \alpha -\sin^2 \beta =\sin( \alpha +\beta )\sin( \alpha -\beta )\\
		[4mm]
		\cos^2 \alpha -\cos^2 \beta =\sin( \alpha +\beta )\sin( \beta -\alpha )
	\end{array}
\end{equation*}

Nel campo complesso valgono le seguenti relazioni

\begin{equation*}
	\begin{array}{ c }
		\sin z=\frac{1}{2i}\left( e^{iz} -e^{-iz}\right)\\
		[4mm]
		\cos z=\frac{1}{2}\left( e^{iz} +e^{-iz}\right)
	\end{array}
\end{equation*}

dette formule di Eulero; inoltre

\begin{equation*}
	\begin{array}{ c }
		e^{iz} =\cos z+i\sin z\\
		[4mm]
		e^{x+iy} =e^x(\cos y+i\sin y)
	\end{array}
\end{equation*}







































\section*{Formule notevoli per un triangolo}

Dato il triangolo mostrato in figura di lati $\displaystyle a$, $\displaystyle b$ e $\displaystyle c$ ed angoli $\displaystyle \alpha$, $\displaystyle \beta $ e $\displaystyle \gamma$, valgono le seguenti relazioni:

\begin{equation*}
	\begin{array}{ c }
		\alpha +\beta +\gamma =\pi \\
		[4mm]
		\frac{a}{\sin \alpha } =\frac{b}{\sin \beta } =\frac{c}{\sin \gamma }\\
		[4mm]
		a^2 =b^2 +c^2 -2bc\cos \alpha
	\end{array}
\end{equation*}




\begin{figure}[htpb]
	\centering
	\tikzset{every picture/.style={line width=0.75pt}} %set default line width to 0.75pt

	\begin{tikzpicture}[x=0.75pt,y=0.75pt,yscale=-1,xscale=1]
	%uncomment if require: \path (0,300); %set diagram left start at 0, and has height of 300

	%Shape: Boxed Polygon [id:dp5667907928913438]
	\draw  [line width=1.5]  (459.41,171.13) -- (162.59,172.5) -- (343.14,53.5) -- cycle ;
	%Shape: Arc [id:dp1523693032409319]
	\draw  [draw opacity=0] (187.57,155.88) .. controls (190.73,160.62) and (192.58,166.31) .. (192.59,172.43) -- (162.59,172.5) -- cycle ; \draw   (187.57,155.88) .. controls (190.73,160.62) and (192.58,166.31) .. (192.59,172.43) ;
	%Shape: Arc [id:dp8340291029684717]
	\draw  [draw opacity=0] (364.27,74.8) .. controls (358.85,80.18) and (351.38,83.5) .. (343.14,83.5) .. controls (332.66,83.5) and (323.44,78.13) .. (318.07,69.98) -- (343.14,53.5) -- cycle ; \draw   (364.27,74.8) .. controls (358.85,80.18) and (351.38,83.5) .. (343.14,83.5) .. controls (332.66,83.5) and (323.44,78.13) .. (318.07,69.98) ;
	%Shape: Arc [id:dp5702801543421352]
	\draw  [draw opacity=0] (429.41,171.18) .. controls (429.41,171.17) and (429.41,171.15) .. (429.41,171.13) .. controls (429.41,162.73) and (432.86,155.14) .. (438.42,149.7) -- (459.41,171.13) -- cycle ; \draw   (429.41,171.18) .. controls (429.41,171.17) and (429.41,171.15) .. (429.41,171.13) .. controls (429.41,162.73) and (432.86,155.14) .. (438.42,149.7) ;

	% Text Node
	\draw (209.6,155.2) node    {$\gamma $};
	% Text Node
	\draw (338.8,92.8) node    {$\beta $};
	% Text Node
	\draw (420.4,153.6) node    {$\alpha $};
	% Text Node
	\draw (247.2,97.6) node    {$a$};
	% Text Node
	\draw (413.2,100.8) node    {$c$};
	% Text Node
	\draw (316.4,182.4) node    {$b$};

	\end{tikzpicture}
\end{figure}















\chapter{Calcolo differenziale ed integrale}

\section{Regole di derivazione di una funzione}

\subsection*{Regole generali}

Sia $\displaystyle k$ una costante ed $\displaystyle f( x)$, $\displaystyle g( x)$ due funzioni continue e derivabili (per la nozione di derivabilità si rimanda ad un testo di Analisi Matematica). Indicando con $\displaystyle f'( x) =\frac{df}{dx}$ e $\displaystyle g'( x) =\frac{dg}{dx}$ le derivate prime di $\displaystyle f( x)$ e $\displaystyle g( x)$, si hanno le seguenti relazioni
\begin{equation*}
	\begin{array}{ c }
		\frac{d}{dx}[ k\ f( x)] =k\ f'( x)\\
		[4mm]
		\frac{d}{dx}[ g( x) +f( x)] =g'( x) +f'( x)\\
		[4mm]
		\frac{d}{dx}[ g( x) \cdot f( x)] =g'( x) f( x) +g( x) f( x) '\\
		[4mm]
		\frac{d}{dx}\left[\frac{g( x)}{f( x)}\right] =\frac{g'( x) f( x) -g( x) f( x) '}{f^2( x)}
	\end{array}
\end{equation*}
Se $\displaystyle g=g( z)$ e $\displaystyle z=f( x)$, la funzione $\displaystyle g=g[ f( x)]$ si dice composita. In tal caso la derivata di $\displaystyle g( x)$ risulta
\begin{equation*}
	\frac{d}{dx} g( z) =\frac{dg}{dz} \cdot \frac{dz}{dx} =\frac{dg}{dz} \cdot \frac{df}{dx}
\end{equation*}



\subsection*{Tabella delle principali derivate}

\begin{equation*}
	\begin{array}{ l l }
		\frac{d}{dx} k=0 & \frac{d}{dx} x^n =n\ x^{n-1}\\
		[4mm]
		\frac{d}{dx} e^{kx} =k\ e^{kx} & \frac{d}{dx} a^{kx} =k\ a^{kx}\ln a\\
		[4mm]
		\frac{d}{dx}\sin x=\cos x & \frac{d}{dx}\cos x=-\sin x\\
		[4mm]
		\frac{d}{dx}\tan x=\frac{1}{\cos^2 x} \ \ \ \  & \frac{d}{dx}\cot x=-\frac{1}{\sin^2 x}\\
		[4mm]
		\frac{d}{dx}\ln x=\frac{1}{x} & \frac{d}{dx}\ln[ f( x)] =\frac{f'( x)}{f( x)}
	\end{array}
\end{equation*}







































\section{Regole di integrazione di una funzione}

\subsection*{Regole generali}

Date due funzioni $\displaystyle f( x)$ ed $\displaystyle F( x)$, diremo che $\displaystyle F$ è una primitiva di $\displaystyle f$ se:
\begin{equation*}
	\frac{d}{dx} F(x) =f(x)
\end{equation*}
La primitiva di una funzione viene anche detta integrale indefinito ed indicata con la notazione:
\begin{equation*}
	F(x) =\int f(x) \ dx
\end{equation*}
In virtù delle regole di derivazione, la primitiva è sempre definita a meno di una costante arbitraria. Per il teorema del calcolo integrale, l'integrale definito di una funzione $\displaystyle f( x)$ valutato fra gli estremi $\displaystyle a$ e $\displaystyle b$ risulta pari alla differenza dei valori assunti dalla primitiva nei due estremi:
\begin{equation*}
	\int^b_a f( x) \ dx\ =[ F( x)]^b_a =F( b) -F( a)
\end{equation*}
Vi sono due metodi importanti di risoluzione di un integrale definito:
\begin{enumerate}
	\item \textit{Integrazione per sostituzione}. Posto $\displaystyle x=y( t)$ e detta $\displaystyle t=g( x)$ la sua funzione inversa, risulta:
	\[
		\int^b_a f( x) \ dx=\int^{g( b)}_{g( a)} f[ y( t)]\frac{dy}{dt} dt
	\]
	\item \textit{Integrazione per parti}. Se in un integrale appare il prodotto di una funzione $\displaystyle f( x)$ per la derivata di una funzione $\displaystyle g( x)$, possiamo allora porre:
	\[
		\int^b_a f( x)\frac{dg}{dx} dx=[ f( x) \ g( x)]^b_a -\int^b_a\frac{df}{dx} g( x) \ dx
	\]
\end{enumerate}



\subsection*{Tabella dei principali integrali indefiniti}

Si noti che tutti gli integrali qui riportati sono sempre assegnati a meno di una costante arbitraria $\displaystyle C$.
\begin{equation*}
	\begin{array}{ l l }
		\int k\ dx=kx & \int x^n \ dx=\frac{x^{n+1}}{n+1}\\
		[4mm]
		\int e^{kx} \ dx=\frac{e^{kx}}{k} & \int a^{kx} \ dx=\frac{a^{kx}}{k\ln a}\\
		[4mm]
		\int \sin x\ dx=-\cos x & \int \cos x\ dx=\sin x\\
		[4mm]
		\int \tan x\ dx=-\ln(\cos x) \ \ \ \  & \int \cot x\ dx=\ln(\sin x)\\
		[4mm]
		\int \frac{1}{x} \ dx=\ln x & \int \ln x\ dx=x\ln x-x
	\end{array}
\end{equation*}







































\section{Espansione in serie di una funzione}

Si consideri una funzione $\displaystyle f( x)$ continua assieme alle sue derivate in un intorno del punto $\displaystyle x_0 ;\ f( x)$ può essere rappresentata in tale intorno mediante una serie di potenze di $\displaystyle x$:
\begin{equation*}
	f( x) =\sum^{\infty }_{n=0}\frac{1}{n!}\frac{d^n f}{dx^n}( x_0)( x-x_0)^n
\end{equation*}
dove $\displaystyle n!=n\cdot ( n-1) \cdot ( n-2) ...3\cdot 2\cdot 1;\ 0!=1$ e le derivate di ordine $\displaystyle n$-esimo della funzione $\displaystyle f( x)$ sono tutte valutate nel punto $\displaystyle x_0$. Se la precedente sommatoria viene arrestata ai primi addendi, si ottiene una espressione che approssima la funzione $\displaystyle f( x)$. Di seguito si riportano le approssimazioni di alcune funzioni in un intorno di $\displaystyle x_0 =0$.







\subsection*{Approssimazione di $\displaystyle f( x)$ nell'intorno di $\displaystyle x_0 =0$}


\begin{equation*}
	\begin{array}{ l l }
		\sin x\approx x & \cos x\approx 1-\frac{x^2}{2}\\
		[4mm]
		e^x \approx 1+x & \tan x\approx x\\
		[4mm]
		\frac{1}{1+x} \approx 1-x & \frac{1}{1-x} \approx 1+x\\
		[4mm]
		( 1+x)^2 \approx 1+2x\ \ \ \  & ( 1-x)^2 \approx 1-2x\\
		[4mm]
		\sqrt{1+x} \approx 1+\frac{x}{2} & \sqrt{1-x} \approx 1-\frac{x}{2}\\
		[4mm]
		\ln( 1+x) \approx x &
	\end{array}
\end{equation*}















\chapter{Identità vettoriali}

Si indicherà col simbolo $\mathbf{A}$ il vettore $\vec{A}$ per non appesantire la notazione. Si utilizzerà il simbolo $\nabla$ per indicare la divergenza ($\nabla\cdot\mathbf{A}$), il rotore ($\nabla\times\mathbf{A}$) di un campo vettoriale $\mathbf{A}$ e il gradiente ($\nabla f$) di un campo scalare $f$.

\section{Identità vettoriali generiche}

\textbf{Triplo prodotto}
\begin{align*}
	&\mathbf{A} \times (\mathbf{B} \times \mathbf{C}) = \mathbf{B} (\mathbf{A} \cdot \mathbf{C}) - \mathbf{C} (\mathbf{A} \cdot \mathbf{B}) \\
	&\mathbf{A} \cdot (\mathbf{B} \times \mathbf{C}) = \mathbf{B} \cdot (\mathbf{C} \times \mathbf{A}) = \mathbf{C} \cdot (\mathbf{A} \times \mathbf{B})
\end{align*}
da cui si ha
\begin{align*}
	(\mathbf{A} \times \mathbf{B}) \cdot (\mathbf{C} \times \mathbf{D}) = (\mathbf{A} \cdot \mathbf{C}) (\mathbf{B} \cdot \mathbf{D}) - (\mathbf{A} \cdot \mathbf{D}) (\mathbf{B} \cdot \mathbf{C})
\end{align*}
ed in particolare
\begin{align*}
	|\mathbf{A} \times \mathbf{B}|^2 = |\mathbf{A}|^2 |\mathbf{B}|^2 - (\mathbf{A} \cdot \mathbf{B})^2
\end{align*}







































\section{Proprietà degli operatori vettoriali}

\textbf{Proprietà distributiva}
\begin{align*}
	& \nabla (f+g) = \nabla f + \nabla g \\
	& \nabla \cdot ( \mathbf{A} + \mathbf{B} ) = \nabla \cdot \mathbf{A} + \nabla \cdot \mathbf{B} \\
	& \nabla \times ( \mathbf{A} + \mathbf{B} ) = \nabla \times \mathbf{A} + \nabla \times \mathbf{B}
\end{align*}
\textbf{Proprietà del prodotto scalare}
\begin{align*}
	&\nabla(\mathbf{A} \cdot \mathbf{B}) = (\mathbf{A} \cdot \nabla)\mathbf{B} + (\mathbf{B} \cdot \nabla)\mathbf{A} + \mathbf{A} \times (\nabla \times \mathbf{B}) + \mathbf{B} \times (\nabla \times \mathbf{A})
\end{align*}
\textbf{Proprietà del prodotto vettoriale}
\begin{align*}
	&\nabla \cdot (\mathbf{A} \times \mathbf{B}) = \mathbf{B} \cdot \nabla \times \mathbf{A} - \mathbf{A} \cdot \nabla \times \mathbf{B} \\
	&\nabla \times (\mathbf{A} \times \mathbf{B}) = \mathbf{A} (\nabla \cdot \mathbf{B}) -\mathbf{B}  (\nabla \cdot \mathbf{A})+( \mathbf{B}\cdot \nabla)\mathbf{A}-( \mathbf{A}\cdot \nabla)\mathbf{B}
\end{align*}
\textbf{Prodotto tra scalari e vettori}
\begin{align*}
	&\nabla (fg) = f \nabla g + g \nabla f \\
	&\nabla \cdot (f\mathbf{A}) = \nabla f \cdot \mathbf{A} + f \nabla \cdot \mathbf{A}\\
	&\nabla \times (f \mathbf{A}) = \nabla f \times \mathbf{A} + f \nabla \times \mathbf{A}
\end{align*}







































\section{Combinazione di operatori vettoriali}

\textbf{Divergenza del gradiente}
\begin{align*}
	\nabla \cdot \nabla f = \nabla^2 f   = \sum_{i=1}^n \frac {\partial^2f}{\partial x^2_i}
\end{align*}
L'operatore $ \nabla^2$ viene detto operatore di Laplace (o laplaciano) e viene anche indicato con $ \Delta $. \\\\
\textbf{Rotore del gradiente}
\begin{align*}
	\nabla \times \nabla f = 0
\end{align*}
\textbf{Divergenza del rotore}
\begin{align*}
	\nabla \cdot \nabla \times \mathbf{A} = 0
\end{align*}
\textbf{Rotore del rotore}
\begin{align*}
	\nabla \times \left( \mathbf{\nabla \times F} \right) = \mathbf{\nabla} (\mathbf{\nabla \cdot F}) - \nabla^2 \mathbf{F}
\end{align*}
\textbf{Altre identità}
\begin{align*}
	\frac{1}{2} \nabla \mathbf{A}^2 = \mathbf{A} \times (\nabla \times \mathbf{A}) + (\mathbf{A} \cdot \nabla) \mathbf{A}
\end{align*}























\end{document}
